\documentclass[11pt]{article}
 
\usepackage[top=0.75in, bottom=1.25in, left=1in, right=1in]{geometry} 
\usepackage{amsmath,amsthm,amssymb} %this is THE math package
\usepackage{mathtools}
\usepackage{tikz}
\usepackage{graphicx}
\usepackage{fancybox}
\usepackage{hyperref}
\usepackage{varwidth}
\usepackage{mdframed}
\usepackage{mathrsfs}
\usepackage[most]{tcolorbox}
%------------------------
%Fonts I use, uncomment if you like to use them.
%The first is the general font, and the second a math font
\usepackage{mathpazo}
\usepackage{eulervm}
%------------------------
%This is so that we have standard fonts for the double-stroked symbols
%for reals, naturals etc. regardless of what font you use.
%Don't comment
\AtBeginDocument{
  \DeclareSymbolFont{AMSb}{U}{msb}{m}{n}
  \DeclareSymbolFontAlphabet{\mathbb}{AMSb}}
%------------------------

%----------------------------------------------
%User-defined environments
%Commented because we're not using them in this document
%The only uncommented ones are the Problem and Solution environment

% \newenvironment{theorem}[2][Theorem]{\begin{trivlist}
% \item[\hskip \labelsep {\bfseries #1}\hskip \labelsep {\bfseries #2.}]}{\end{trivlist}}
% \newenvironment{lemma}[2][Lemma]{\begin{trivlist}
% \item[\hskip \labelsep {\bfseries #1}\hskip \labelsep {\bfseries #2.}]}{\end{trivlist}}
% \newenvironment{exercise}[2][Exercise]{\begin{trivlist}
% \item[\hskip \labelsep {\bfseries #1}\hskip \labelsep {\bfseries #2.}]}{\end{trivlist}}
% \newenvironment{question}[2][Question]{\begin{trivlist}
% \item[\hskip \labelsep {\bfseries #1}\hskip \labelsep {\bfseries #2.}]}{\end{trivlist}}
% \newenvironment{corollary}[2][Corollary]{\begin{trivlist}
% \item[\hskip \labelsep {\bfseries #1}\hskip \labelsep {\bfseries #2.}]}{\end{trivlist}}
\newenvironment{problem}[2][Problem\!]{\begin{trivlist}
\item[\hskip \labelsep {\bfseries #1}\hskip \labelsep {\bfseries #2}]}{\end{trivlist}}
%\newenvironment{sub-problem}[2][]{\begin{trivlist}
%\item[\hskip \labelsep {\bfseries #1}\hskip \labelsep {\bfseries #2}]}{\end{trivlist}}
\newenvironment{solution}{\begin{proof}[\textbf{\textit{Solution}}] }{\end{proof}}
%----------------------------------------------

%----------------------------
%User-defined notations
\newcommand{\zz}{\mathbb Z}   %blackboard bold Z
\newcommand{\qq}{\mathbb Q}   %blackboard bold Q
\newcommand{\ff}{\mathbb F}   %blackboard bold F
\newcommand{\rr}{\mathbb R}   %blackboard bold R
\newcommand{\nn}{\mathbb N}   %blackboard bold N
\newcommand{\cc}{\mathbb C}   %blackboard bold C
\newcommand{\af}{\mathbb A}   %blackboard bold A
\newcommand{\pp}{\mathbb P}   %blackboard bold P
\newcommand{\id}{\operatorname{id}} %for identity map
\newcommand{\im}{\operatorname{im}} %for image of a function
\newcommand{\dom}{\operatorname{dom}} %for domain of a function
\newcommand{\cat}[1]{\mathscr{#1}}   %calligraphic category
\newcommand{\abs}[1]{\left\lvert#1\right\rvert} %for absolute value
\newcommand{\norm}[1]{\left\lVert#1\right\rVert} %for norm
\newcommand{\modar}[1]{\text{ mod }{#1}} %for modular arithmetic
\newcommand{\set}[1]{\left\{#1\right\}} %for set
\newcommand{\setp}[2]{\left\{#1\ \middle|\ #2\right\}} %for set with a property
\newcommand{\card}[1]{\#\,{#1}} %for cardinality of a set
\newcommand\m[1]{\begin{pmatrix}#1\end{pmatrix}} 

%Re-defined notations
\renewcommand{\epsilon}{\varepsilon}
\renewcommand{\phi}{\varphi}
\renewcommand{\emptyset}{\varnothing}
\renewcommand{\geq}{\geqslant}
\renewcommand{\leq}{\leqslant}
\renewcommand{\Re}{\operatorname{Re}}
\renewcommand{\Im}{\operatorname{Im}}
%----------------------------

\allowdisplaybreaks
 
 
\begin{document}
 
\title{Homework 3}
\author{Kevin Guillen\\[0.5em]
MATH 201 | Algebra II | Winter 2022}
\date{} 
\maketitle

%Use \[...\] instead of $$...$$

\begin{tcolorbox}
  \begin{problem} {1}
    Let $M = \zz^2$ and let $N$ be the $\zz$-submodule generated by the 2 elements 
    \[\begin{bmatrix}
        120 \\ 240
    \end{bmatrix} \ \text{ and } \ \begin{bmatrix}
        360 \\ -300
    \end{bmatrix}\]
    In (a) and (b) we use the notation and terminology adopted in class on the 3rd of February.

    The terminology in (c) and (d) will be defined on the 8th of February.
    \begin{itemize}
      \item[(a)]
        Find an element $v \in Hom_\zz(M,\zz)$ such that $v(N)$ is maximal in $\Sigma_{M,N}$: prove that it is indeed maximal. 
      \item[(b)]
        Find an element $y_1 \in M$ such that $M = \zz{y_1} \bigoplus \ker(v)$ and $N = \zz a_v y_1 \bigoplus (\ker(v) \cap N).$ 
      \item[(c)]
        Find the invariant factors of the quotient $\zz$-module $M/N$. 
      \item[(d)]   
        Find the elementary divisors of the quotient $\zz$-module $M/N$ 
    \end{itemize}
  \end{problem}
\end{tcolorbox}

\begin{itemize}
  \item[(a)]
  \begin{proof}
    Any $v_A\in Hom_\zz(M,\zz)$ is of the form,
    \[v_A\left(\begin{bmatrix}
      x \\ y
    \end{bmatrix}\right) = ax + by = \begin{bmatrix}
      a & b
    \end{bmatrix} \begin{bmatrix}
      x \\ y
    \end{bmatrix}\]
    where $A = \begin{bmatrix}
      a & b
    \end{bmatrix}$
    Now let's consider the following,
    \begin{align*}
      v_A\left(\begin{bmatrix}
        120 \\ 240
      \end{bmatrix}\right) = 120a + 240b &= 60(2a + 4b) \\
      v_A\left(\begin{bmatrix}
        360 \\ -300
      \end{bmatrix}\right) = 360a -300b &= 60(6a-5b)
    \end{align*}
    Let $A = \begin{bmatrix}
      1 & 1
    \end{bmatrix}$. We see then that $v_A(N) = (60)$ and is indeed maximal. This is because there is no element $y\in N$ and $v_A\in Hom(M,\zz)$ that can map $y$ to a factor of 60 that is not 60. This is because 60 is the GCD of all 4 numbers in the given matrices. Meaning $a_1 = a_v = 60$ 
  \end{proof} 
  \item[(b)]
  \begin{proof}
    First we want a $y \in N$ that under $v_A$ maps to the generator of $v_A(N)$. In other words $v_A(y) = 60$. So we can simply let $y = \begin{bmatrix}
      360 \\ -300
    \end{bmatrix}$ we see this holds since,
    \[v_A\left(\begin{bmatrix}
      360 \\ -300
    \end{bmatrix}\right) = 360 - 300 = 60.\] Now we can let our $y_1$ be the divisor of $y$ such that $a_vy_1 = y$,
    \[60y_1 = \begin{bmatrix}
      360 \\ -300
    \end{bmatrix} \rightarrow y_1 = \begin{bmatrix}
      6 \\ -5
    \end{bmatrix} \]    

    \[\ker(v_A) = \set{\begin{bmatrix}
      x \\ y
    \end{bmatrix} \in \zz^{2} \mid x + y = 0} = \zz \begin{bmatrix}
      1 \\ -1
    \end{bmatrix}\] We also have $\ker(v_A) \cap N$ to be,
    \[\ker(v_A) \cap N= \set{a \begin{bmatrix}
      120 \\ 240
    \end{bmatrix} + b \begin{bmatrix}
      360 \\ -300
    \end{bmatrix} \mid a,b \in \zz}\]
    Writing $a$ in terms of $b$ we get $a = 5.5b$ so,
    \begin{align*}
      \ker(v_A) \cap N &= \set{(5.5b)\begin{bmatrix}
        120 \\ 240
      \end{bmatrix}  + b \begin{bmatrix}
        360 \\ -300
      \end{bmatrix} \mid b \in \zz} \\
      \ker(v_A) \cap N &=\set{\begin{bmatrix}
        1020b \\ 1020b
      \end{bmatrix} \mid b \in \zz}
    \end{align*}
    Now let $a_2 = gcd(1020,1020) = 1020$ and $y_2 = \begin{bmatrix}
      1 \\ -1
    \end{bmatrix}$ we have then that,
    \begin{align*}
      M &= \zz\begin{bmatrix}
        360 \\ -300
      \end{bmatrix} \bigoplus \zz \begin{bmatrix}
        1 \\ -1
      \end{bmatrix} \\
      N &= \zz 60 \begin{bmatrix}
        360 \\ -300
      \end{bmatrix} \bigoplus \zz 1020 \begin{bmatrix}
        1 \\ -1
      \end{bmatrix}
    \end{align*}
  \end{proof} 
  \item[(c)]
  \item[(d)]   
\end{itemize}

\end{document}