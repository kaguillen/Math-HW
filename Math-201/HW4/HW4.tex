\documentclass[11pt]{article}
 
\usepackage[top=0.75in, bottom=1.25in, left=1in, right=1in]{geometry} 
\usepackage{amsmath,amsthm,amssymb} %this is THE math package
\usepackage{mathtools}
\usepackage{tikz}
\usepackage{graphicx}
\usepackage{fancybox}
\usepackage{hyperref}
\usepackage{varwidth}
\usepackage{mdframed}
\usepackage{mathrsfs}
\usepackage[most]{tcolorbox}
%------------------------
%Fonts I use, uncomment if you like to use them.
%The first is the general font, and the second a math font
\usepackage{mathpazo}
\usepackage{eulervm}
%------------------------
%This is so that we have standard fonts for the double-stroked symbols
%for reals, naturals etc. regardless of what font you use.
%Don't comment
\AtBeginDocument{
  \DeclareSymbolFont{AMSb}{U}{msb}{m}{n}
  \DeclareSymbolFontAlphabet{\mathbb}{AMSb}}
%------------------------

%----------------------------------------------
%User-defined environments
%Commented because we're not using them in this document
%The only uncommented ones are the Problem and Solution environment

% \newenvironment{theorem}[2][Theorem]{\begin{trivlist}
% \item[\hskip \labelsep {\bfseries #1}\hskip \labelsep {\bfseries #2.}]}{\end{trivlist}}
% \newenvironment{lemma}[2][Lemma]{\begin{trivlist}
% \item[\hskip \labelsep {\bfseries #1}\hskip \labelsep {\bfseries #2.}]}{\end{trivlist}}
% \newenvironment{exercise}[2][Exercise]{\begin{trivlist}
% \item[\hskip \labelsep {\bfseries #1}\hskip \labelsep {\bfseries #2.}]}{\end{trivlist}}
% \newenvironment{question}[2][Question]{\begin{trivlist}
% \item[\hskip \labelsep {\bfseries #1}\hskip \labelsep {\bfseries #2.}]}{\end{trivlist}}
% \newenvironment{corollary}[2][Corollary]{\begin{trivlist}
% \item[\hskip \labelsep {\bfseries #1}\hskip \labelsep {\bfseries #2.}]}{\end{trivlist}}
\newenvironment{problem}[2][Problem\!]{\begin{trivlist}
\item[\hskip \labelsep {\bfseries #1}\hskip \labelsep {\bfseries #2}]}{\end{trivlist}}
%\newenvironment{sub-problem}[2][]{\begin{trivlist}
%\item[\hskip \labelsep {\bfseries #1}\hskip \labelsep {\bfseries #2}]}{\end{trivlist}}
\newenvironment{solution}{\begin{proof}[\textbf{\textit{Solution}}] }{\end{proof}}
%----------------------------------------------

%----------------------------
%User-defined notations
\newcommand{\zz}{\mathbb Z}   %blackboard bold Z
\newcommand{\qq}{\mathbb Q}   %blackboard bold Q
\newcommand{\ff}{\mathbb F}   %blackboard bold F
\newcommand{\rr}{\mathbb R}   %blackboard bold R
\newcommand{\nn}{\mathbb N}   %blackboard bold N
\newcommand{\cc}{\mathbb C}   %blackboard bold C
\newcommand{\af}{\mathbb A}   %blackboard bold A
\newcommand{\pp}{\mathbb P}   %blackboard bold P
\newcommand{\id}{\operatorname{id}} %for identity map
\newcommand{\im}{\operatorname{im}} %for image of a function
\newcommand{\dom}{\operatorname{dom}} %for domain of a function
\newcommand{\cat}[1]{\mathscr{#1}}   %calligraphic category
\newcommand{\abs}[1]{\left\lvert#1\right\rvert} %for absolute value
\newcommand{\norm}[1]{\left\lVert#1\right\rVert} %for norm
\newcommand{\modar}[1]{\text{ mod }{#1}} %for modular arithmetic
\newcommand{\set}[1]{\left\{#1\right\}} %for set
\newcommand{\setp}[2]{\left\{#1\ \middle|\ #2\right\}} %for set with a property
\newcommand{\card}[1]{\#\,{#1}} %for cardinality of a set
\newcommand{\lrp}[1]{\left(#1\right)}
\newcommand{\lrb}[1]{\left[#1\right]}
\newcommand{\lrc}[1]{\left\{#1\right\}}
\newcommand\m[1]{\begin{pmatrix}#1\end{pmatrix}} 

%Re-defined notations
\renewcommand{\epsilon}{\varepsilon}
\renewcommand{\phi}{\varphi}
\renewcommand{\emptyset}{\varnothing}
\renewcommand{\geq}{\geqslant}
\renewcommand{\leq}{\leqslant}
\renewcommand{\Re}{\operatorname{Re}}
\renewcommand{\Im}{\operatorname{Im}}
%----------------------------

\allowdisplaybreaks
 
 
\begin{document}
 
\title{Homework 4}
\author{Kevin Guillen\\[0.5em]
MATH 201 | Algebra II | Winter 2022}
\date{} 
\maketitle

%Use \[...\] instead of $$...$$

\begin{tcolorbox}
  \begin{problem} {1}
    Consider the complex matrix
    \[A = \begin{pmatrix}
        1 & 2 & 0 \\
        2 & 1 & 2 \\
        0 & -2 & 1
    \end{pmatrix}\]
    \begin{itemize}
        \item[(1)] Compute the characteristic polynomial of $A$: Show your work.
        \item[(2)] Determine the Jordan form of $A$: Show your work.
    \end{itemize}
  \end{problem}
\end{tcolorbox}

\begin{itemize}
    \item[(1)]
        \begin{solution}
            To find the characteristic polynomial we must calculate $det\lrp{A - \lambda I_3}$. This works out to be,
            \begin{align*}
                det\lrp{\begin{pmatrix}
                    1- \lambda & 2 & 0 \\
                    2 & 1 - \lambda & 2 \\
                    0 & -2 & 1 - \lambda
                \end{pmatrix}} &= (1-\lambda)((1-\lambda)^{2} + 4) -2(2-2\lambda) + 0 \\
                &= (1-\lambda)(\lambda^{2} -2\lambda + 5) - 4 + 4\lambda \\
                &= -\lambda^{3} + 2\lambda -5\lambda + \lambda^{2} + 5 \\
                &= -\lambda^{3} +3\lambda^{2} - 3\lambda  + 1.
            \end{align*}
            So we have $-\lambda^{3} +3\lambda^{2} - 3\lambda  + 1$ to be the characteristic polynomial of $A$.
        \end{solution}
    \item[(2)] 
        \begin{solution}
            First we will find the roots of the characteristic polynomial to determine the eigenvalues for $A$,
            \[-\lambda^{3} +3\lambda^{2} - 3\lambda  + 1 = -(\lambda -1 )^{3}\]
            it is clear that 1 is the eigenvalue of $A$ with multiplicity 3. 

            We have 3 scenarios for the potential Jordan from of $A$ that is, one $3\times3$ block, one $2\times 2$ block with one $1\times 1$ block, or three $1\times 1$ blocks. Let us consider the dimension of the eigenspace for $\lambda = 1$,
            \begin{align*}
                (A - 1I_3) = \begin{pmatrix}
                    0 & 2 & 0 \\
                    2 & 0 & 1 \\
                    0 & -2 & 0 
                \end{pmatrix}
            \end{align*}
            so we solve the following,
            \begin{align*}
                \begin{pmatrix}
                    0 & 2 & 0 \\
                    2 & 0 & 2 \\
                    0 & -2 & 0 
                \end{pmatrix} \begin{pmatrix}
                    x \\ y \\ z
                \end{pmatrix} = \begin{pmatrix}0 \\ 0 \\ 0 \end{pmatrix} \implies \begin{pmatrix}
                    x \\ y \\ z
                \end{pmatrix} = \alpha\begin{pmatrix}
                    1 \\ 0 \\ -1
                \end{pmatrix} 
            \end{align*}
            where $\alpha$ is a scalar. Therefore the eigenspace for $\lambda = 1$ is spanned by 1 vector meaning it has dimension 1. So we have that the number of Jordan blocks for $\lambda =1$ to be 1. So we must have one $3\times 3$ jordan block. Meaning the Jordan form of $A$ has to be,
            \[\begin{pmatrix}
                1 & 0 & 0 \\
                1 & 1 & 0 \\
                0 & 1 & 1
            \end{pmatrix}\]
            
        \end{solution}
\end{itemize}

\begin{tcolorbox}
    \begin{problem}{2}
        Consider the complex matrix \[B = \begin{pmatrix}
            4 & 5-5i \\ 
            5 + 5i & -1
        \end{pmatrix}\]
        Find a basis $(v_1, v_2)$ of $\cc^{2}$ such that 
        \begin{itemize}
            \item[(a)] Each of $v_1, v_2$ is an eigenvector of $B$.
            \item[(b)] $<v_1\mid v_1>_{std} = 1 = <v_2 \mid v_2>_{std}$ and $<v_1 \mid v_2>_{std} = 0$ at the same time.
        \end{itemize}
    \end{problem}
\end{tcolorbox}
        \begin{solution}
            Let us first find the eigenvalues of the given matrix $B$,
            \begin{align*}
                det\lrp{\begin{pmatrix}
                    4-\lambda & 5-5i \\ 
                    5 + 5i & -1 - \lambda
                \end{pmatrix} } &= (4-\lambda)(-1-\lambda) - (5-5i)(5 +5i) \\
                &= -4 + \lambda -4\lambda +\lambda^{2} + 25 -25i +25i + 25 \\
                &= \lambda^{2} - 3\lambda - 54
            \end{align*}
            solving for $\lambda$,
            \begin{align*}
                \lambda = \dfrac{3 \pm \sqrt{9 - 4(-54)}}{2} = \dfrac{3 \pm \sqrt{225}}{2} = \dfrac{3 \pm 15}{2}
            \end{align*}
            so $\lambda_1 = 9$ and $\lambda_2 = -6$

            Now solving for the eigenvector for $\lambda_1$ first we get,
            \begin{align*}
                \begin{pmatrix}
                    -5 & 5-5i \\
                    5+5i & -10
                \end{pmatrix} \begin{pmatrix}
                    x \\ y
                \end{pmatrix} = \begin{pmatrix} 0 \\ 0 \end{pmatrix} \implies \begin{pmatrix}
                    x \\ y
                \end{pmatrix} = \begin{pmatrix}
                    1 - i \\ 1
                \end{pmatrix}
            \end{align*}
            so $w_1 = \begin{pmatrix}
                1 - i \\ 1
            \end{pmatrix}$ next we need to get the eigenvector for $\lambda_2$,
            \begin{align*}
                \begin{pmatrix}
                    10 & 5-5i \\
                    5+5i & 5
                \end{pmatrix} \begin{pmatrix}
                    x \\ y
                \end{pmatrix} = \begin{pmatrix}
                    0 \\ 0
                \end{pmatrix} \implies \begin{pmatrix}
                    x \\ y
                \end{pmatrix} =\begin{pmatrix}
                    -1 + i \\ 2
                \end{pmatrix}
            \end{align*}
            and we have $w_2 = \begin{pmatrix}
                -1 + i \\ 2
            \end{pmatrix}$. Now we must scale these eigenvectors with some $\alpha$ and $\beta$in $\cc$, so that $<\alpha w_1 \mid \alpha w_1> = 1 = <\beta w_2 \mid \beta w_2>$ and $<\alpha w_1 \mid \beta w_2> = 0$. Once that is met we can simply set $v_1 = \alpha w_1$ and $v_2 = \beta w_2$

            To solve for $\alpha$ to satisfy $<\alpha w_1 \mid \alpha w_1> = 1$ we simply solve for $\alpha$ in the following,
            \begin{align*}
                (\alpha+\alpha i)(\alpha - \alpha i) + \alpha^{2} &= 1 \\
                2\alpha^{2} + \alpha^{2} &= 1 \\
                3\alpha^{2} &= 1 \\
                \alpha &= \sqrt{\frac{1}{3}}
            \end{align*}

            Now verifying $<\alpha w_1 \mid \alpha w_1> = 1$,
            \begin{align*}
                <\alpha w_1 \mid \alpha w_1> = \begin{pmatrix}
                    \sqrt{\frac{1}{3}}(1+i) & \sqrt{\frac{1}{3}}
                \end{pmatrix} \begin{pmatrix}
                    \sqrt{\frac{1}{3}}(1-i) \\ \sqrt{\frac{1}{3}}
                \end{pmatrix} = \frac{1}{3} + \frac{1}{3} + \frac{1}{3} = 1
            \end{align*}

            Now solving for $\beta$ to satisfy $<\beta w_2 \mid \beta w_2> = 1$ we do like before and solve for $\beta$ in the following,
            \begin{align*}
                (-\beta -\beta i)(-\beta + \beta i) + 4\beta &= 1 \\
                2\beta^{2} + 4\beta^{2} &= 1 \\
                6\beta^{2} &= 1 \\
                \beta &=\sqrt{\frac{1}{6}} 
            \end{align*}
            Now verifying $<\beta w_2 \mid \beta w_2> = 1$,
            \begin{align*}
                <\beta w_2 \mid \beta w_2> = \begin{pmatrix}
                    \sqrt{\frac{1}{6}(-1 - i)} & 2\sqrt{\frac{1}{6}} 
                \end{pmatrix}
                \begin{pmatrix}
                    \sqrt{\frac{1}{6}}(-1 + i) \\ 2\sqrt{\frac{1}{6}} 
                \end{pmatrix}
                = \frac{1}{6} + \frac{1}{6} + \frac{4}{6} = 1
            \end{align*}

            Finally verifying $<\alpha w_1 \mid \beta w_2> = 0$,
            \begin{align*}
                <\alpha w_1 \mid \beta w_2> = \begin{pmatrix}
                    \sqrt{\frac{1}{3}}(1+i) & \sqrt{\frac{1}{3}}
                \end{pmatrix}\begin{pmatrix}
                    \sqrt{\frac{1}{6}}(-1 + i) \\ 2\sqrt{\frac{1}{6}} 
                \end{pmatrix} = -\frac{\sqrt{2}}{3} + \frac{\sqrt{2}}{3} = 0
            \end{align*}
            as desired. Therefore we have,
            \[(v_1, v_2) = \lrp{\sqrt\frac{1}{3}\begin{pmatrix}
                (1-i) \\ 1
            \end{pmatrix}), \sqrt{\frac{1}{6}}\begin{pmatrix}-1 + i \\ 2 \end{pmatrix}}\]

        \end{solution}
\end{document}