    \documentclass[11pt]{article}
 
\usepackage[top=0.75in, bottom=1.25in, left=1in, right=1in]{geometry} 
\usepackage{amsmath,amsthm,amssymb} %this is THE math package
\usepackage{mathtools}
\usepackage{tikz}
\usepackage{graphicx}
\usepackage{fancybox}
\usepackage{hyperref}
\usepackage{varwidth}
\usepackage{mdframed}
\usepackage{mathrsfs}
\usepackage[most]{tcolorbox}
%------------------------
%Fonts I use, uncomment if you like to use them.
%The first is the general font, and the second a math font
\usepackage{mathpazo}
\usepackage{eulervm}
%------------------------
%This is so that we have standard fonts for the double-stroked symbols
%for reals, naturals etc. regardless of what font you use.
%Don't comment
\AtBeginDocument{
  \DeclareSymbolFont{AMSb}{U}{msb}{m}{n}
  \DeclareSymbolFontAlphabet{\mathbb}{AMSb}}
%------------------------

%----------------------------------------------
%User-defined environments
%Commented because we're not using them in this document
%The only uncommented ones are the Problem and Solution environment

% \newenvironment{theorem}[2][Theorem]{\begin{trivlist}
% \item[\hskip \labelsep {\bfseries #1}\hskip \labelsep {\bfseries #2.}]}{\end{trivlist}}
% \newenvironment{lemma}[2][Lemma]{\begin{trivlist}
% \item[\hskip \labelsep {\bfseries #1}\hskip \labelsep {\bfseries #2.}]}{\end{trivlist}}
% \newenvironment{exercise}[2][Exercise]{\begin{trivlist}
% \item[\hskip \labelsep {\bfseries #1}\hskip \labelsep {\bfseries #2.}]}{\end{trivlist}}
% \newenvironment{question}[2][Question]{\begin{trivlist}
% \item[\hskip \labelsep {\bfseries #1}\hskip \labelsep {\bfseries #2.}]}{\end{trivlist}}
% \newenvironment{corollary}[2][Corollary]{\begin{trivlist}
% \item[\hskip \labelsep {\bfseries #1}\hskip \labelsep {\bfseries #2.}]}{\end{trivlist}}
\newenvironment{problem}[2][Problem\!]{\begin{trivlist}
\item[\hskip \labelsep {\bfseries #1}\hskip \labelsep {\bfseries #2}]}{\end{trivlist}}
%\newenvironment{sub-problem}[2][]{\begin{trivlist}
%\item[\hskip \labelsep {\bfseries #1}\hskip \labelsep {\bfseries #2}]}{\end{trivlist}}
\newenvironment{solution}{\begin{proof}[\textbf{\textit{Solution}}] }{\end{proof}}
%----------------------------------------------

%----------------------------
%User-defined notations
\newcommand{\zz}{\mathbb Z}   %blackboard bold Z
\newcommand{\qq}{\mathbb Q}   %blackboard bold Q
\newcommand{\ff}{\mathbb F}   %blackboard bold F
\newcommand{\rr}{\mathbb R}   %blackboard bold R
\newcommand{\nn}{\mathbb N}   %blackboard bold N
\newcommand{\cc}{\mathbb C}   %blackboard bold C
\newcommand{\af}{\mathbb A}   %blackboard bold A
\newcommand{\pp}{\mathbb P}   %blackboard bold P
\newcommand{\id}{\operatorname{id}} %for identity map
\newcommand{\im}{\operatorname{im}} %for image of a function
\newcommand{\dom}{\operatorname{dom}} %for domain of a function
\newcommand{\cat}[1]{\mathscr{#1}}   %calligraphic category
\newcommand{\abs}[1]{\left\lvert#1\right\rvert} %for absolute value
\newcommand{\norm}[1]{\left\lVert#1\right\rVert} %for norm
\newcommand{\modar}[1]{\text{ mod }{#1}} %for modular arithmetic
\newcommand{\set}[1]{\left\{#1\right\}} %for set
\newcommand{\setp}[2]{\left\{#1\ \middle|\ #2\right\}} %for set with a property
\newcommand{\card}[1]{\#\,{#1}} %for cardinality of a set
\newcommand\m[1]{\begin{pmatrix}#1\end{pmatrix}} 

%Re-defined notations
\renewcommand{\epsilon}{\varepsilon}
\renewcommand{\phi}{\varphi}
\renewcommand{\emptyset}{\varnothing}
\renewcommand{\geq}{\geqslant}
\renewcommand{\leq}{\leqslant}
\renewcommand{\Re}{\operatorname{Re}}
\renewcommand{\Im}{\operatorname{Im}}
%----------------------------

\allowdisplaybreaks
 
 
\begin{document}
 
\title{Homework 1}
\author{Kevin Guillen\\[0.5em]
MATH 201 | Algebra II | Winter 2022}
\date{} 
\maketitle

%Use \[...\] instead of $$...$$

\begin{tcolorbox}
  \begin{problem} {P1}
    Let $M$ be a left $R-$module.
    \begin{itemize}
        \item[(a)] Let $N_1 \subseteq N_2 \subseteq N_3 \subseteq \dots$ be an ascending chan of $R-$submodules in $M$. Prove that the union $\bigcup_{j =1 }^{\infty}N_j$ is an $R-$submodule of $M$.
        \item[(b)] Let $R = \mathcal{C}(\rr)$ denote the ring of (real-valued) continuous functions on $\rr$, with pointwise addition and multiplication (as in class). defined
        \begin{align*}
            \mathcal{C}_c(\rr) = \set{f \in \mathcal{C}(\rr) : \exists N = N(f) \in \nn \text{ such that }f(x) = 0 \text{ for all } |x| > N}
        \end{align*} 
        Prove that $\mathcal{C}_C(\rr)$ is an $R-$submodule of $R$. Is it a subring?
    \end{itemize}
  \end{problem}
\end{tcolorbox}
\begin{itemize}
    \item [(a)]
    \begin{proof}
        First we will define $N$ to be the following,
        \begin{align*}
            N = \bigcup_{j =1}^{\infty}N_j.
        \end{align*}
        Now we must show that $N$ is a subgroup of $M$ under addition and that it is closed under scalar operation for it to be a submodule. So we will show that it is non-empty, closed under addition, and closed under scalars. Inverses is handled through the proof of scalars since $-1_R \in R$.

        We know $N$ is non-empty since it is a union of non-empty sets (because $N_j$ is given to be a submodule). So let $x,y \in N$ we know then there exists some $a, b \in \nn$ such that $x\in N_{a}$ and $y \in N_{b}$. We can then let $k = \text{max}\set{a,b}$. Which means $N_a \subseteq N_k$ and $N_b \subseteq N_k$ and therefore $x,y \in N_k$. 

        Now because we know $N_k$ to be a submodule of $M$ (since $k$ is either equal to $a$ or $b$), we have the following
        \[x + y \in N_k\]
        and because $N_k \subseteq N$ we have,
        \[x + y \in N\]
        as desired.

        Now let $r \in R$ and $x \in N$. Like before this means that there exists some $a \in \nn$ such that $x \in N_a$. Where $N_a$ is a submodule of $M$. So we know the following
        \[rx \in N_a\]
        and because $N_a \subseteq N$ we have,
        \[rx \in N\]
        as desired.

        All this together then means that $\bigcup_{j = 1}^{\infty}N_j$ is indeed a submodule of $M$.
    \end{proof}

    \item [(b)]
    \begin{proof}
        In order to show that $\mathcal{C}_C(\rr)$ is an $R-$submodule we will show that it is non-empty, closed under addition, and closed under scalars. 

        First consider the zero function, which we will denote as $o$, that maps everything to $0_\rr$. Since we know 
        \[o(x) = 0, \ \forall x \in \rr\]
        then we know it must be in $\mathcal{C}_C(\rr)$ since for $N = 1$ we have 
        \[o(x) = 0, \ \forall \abs{x} > 1.\]

        Now let $f,g \in \mathcal{C}_C(\rr)$. Then we know there exists $N_f, N_g \in \nn$ such that the following hold,
        \begin{align}
            f(x) &= 0, \ \forall \abs{x} > N_f \\
            g(x) &= 0, \ \forall \abs{x} > N_g
        \end{align}
        Now we want to show that $(f + g) \in \mathcal{C}_C(\rr)$. We know addition is defined pointwise so,
        \[(f+g)(x) = f(x) + g(x)\]
        And we know the addition of continuous function is again continuous. Now let $N_{f+g} = \text{max}\set{N_f, N_g}$, we know then that the following holds,
        \begin{align}
            f(x) + g(x) = 0,\ \forall \abs{x} > N_{f+g}.
        \end{align}
        This is because,
        \begin{align*}
            f(x) &= 0, \ \forall \abs{x} > N_{f+g} \geq N_f && \text{by (1)} \\
            g(x) &= 0, \ \forall \abs{x} > N_{f+g} \geq N_g && \text{by (2)}\\
        \end{align*}
        and $0 + 0 = 0$. Since $N_{f+g} \in \nn$ we have then that $(f+g) \in \mathcal{C}_C(\rr)$ as desired.

        Now we will show that $\mathcal{C}_C(\rr)$ is closed under scalars. Let $r \in R = \mathcal{C}(\rr)$ and $f \in \mathcal{C}_C(\rr)$. We know then there exists $N_f \in \nn$ such that,
        \[f(x) = 0, \ \forall \abs{x} > N_f.\]
        We know then that the following holds,
        \[(rf)(x) = r(x)f(x) = 0, \ \forall \abs{x} > N_f.\]
        This is because $r(x)$ will evaluate to some real number and we know $f(x) = 0$ for all $\abs{x} > N_f$. Any real number times $0$ will again be $0$, and the product of continuous functions is again continuous, meaning $(rf) \in \mathcal{C}_C(\rr)$ as desired.

        From this we can quickly see that $\mathcal{C}_C(\rr)$ is not a subring of $\mathcal{C}(\rr)$. This is because $\mathcal{C}(\rr)$ contains a multiplicative identity which is the constant function that maps everything to $1$. This function is not in the set $\mathcal{C}_C(\rr)$ and therefore $\mathcal{C}_C(\rr)$ cannot be a subring of $\mathcal{C}(\rr)$ since it can't share the same multiplicative identity. 
        
    \end{proof}

    \begin{tcolorbox}
        \begin{problem} {P2} Let $M$ be a left $R-$module. The $\textit{annihlator}$ of $M$ in $R$ is defined as:
            \[\textit{Ann}_R(M) = \set{r \in R: rm = 0 \text{ for all } m\in M}\]
            \begin{itemize}
                \item [(a)] Prove that $\textit{Ann}_R(M)$ is a bilateral ideal of $R$.
                \item [(b)] If $M_1$ and $M_2$ are two left $R-$modules, prove that
                \[\textit{Ann}_R(M_1 \times M_2) = \textit{Ann}_R(M_1)\cap \textit{Ann}_R(M_2)\]
                \item [(c)] Compute $\textit{Ann}_R(M)$ when $R=\zz$ and $M = (\zz/112\zz)^{\times}$ is the multiplicative abelian group of units in $\zz/112\zz$
            \end{itemize}
        \end{problem}
    \end{tcolorbox}

    \begin{itemize}
        \item[(a)]
        \begin{proof}
            Let $\textit{Ann}_R(M)$ be denoted as $I$. We know this $I$ is non-empty since $0\in R$ and $0m = 0 $ for all $m \in M$. So now let $a,b \in I$, we will show that $a+b \in I$. Let $m \in M$. Consider the following,
            \begin{align*}
                (a+b)m &= am + bm && \text{by definition} \\
                &= 0 + 0 && \text{since $a,b \in I$}
            \end{align*}
            and since $m$ was arbitrary we have then that $(a+b)m = 0$ for all $m \in M$, meaning $(a+b) \in I $.

            Let $r \in R$ and $a \in I$. We want to show $ra \in I$, so let $m \in M$. We see through the following,
            \begin{align*}
                (ra)m &= r(am) && a \in I \\
                &= r0 \\
                &= 0
            \end{align*}
            $ra$ is in $I$. Now we want to show that $ar \in I$,
            \begin{align*}
                (ar)m &= a(rm) && \text{$M$ is closed under scalars so $rm \in M$, and $a \in I$} \\
                &= 0
            \end{align*}
            Now with all this together we have that $\textit{Ann}_R(M)$ is a bilateral ideal of $R$
        \end{proof}  
        \item[(b)]
        \begin{proof}
            Let $r \in \textit{Ann}_R(M_1 \times M_2)$. That means then for all $(m_1,m_2) \in M_1 \times M_2$,
            \[r(m_1, m_2) = (rm_1, rm_2) = (0,0)\]
             since scalar multiplication is done component wise when working with the cross product of $R$-modules, $rm_1 = 0$ and $rm_2 = 0$ for all $m_1\in M_1$ and for all $m_2 \in M_2 $ therefore, $r \in \textit{Ann}_R(M_1) \cap \textit{Ann}_R(M_2)$. \\ Because $r$ was arbitrary, $\textit{Ann}_R(M_1 \times M_2 )\subseteq \textit{Ann}_R(M_1) \cap \textit{Ann}_R(M_2) $

            Now consider $r\in \textit{Ann}_R(M_1)\cap \textit{Ann}_R(M_2)$. Let $(m_1, m_2) \in M_1 \times M_2$, we have the following,
            \begin{align*}
                r(m_1,m_2) &= (rm_1, rm_2) && r \in \textit{Ann}_R(M_1)  \text{ and } r\in \textit{Ann}_R(M_2) \\
                &= (0,0)
            \end{align*}
            which means $r \in \textit{Ann}_R(M_1 \times M_2)$ and since $r$ was arbitrary we have $\textit{Ann}_R(M_1)\cap \textit{Ann}_R(M_2) \subseteq \textit{Ann}_R(M_1 \times M_2)$.

            Together we then have that $\textit{Ann}_R(M_1 \times M_2) = \textit{Ann}_R(M_1)\cap \textit{Ann}_R(M_2)$ as desired. 
        \end{proof} 
        \item[(c)]  
        Let $z\in \zz$ and $\overline{m}\in (\zz/112\zz)^{\times}$. For $z$ to be an element of the annihilator of $M$ in $\zz$ we must have
        \[z\overline{m} = 0\]
        for all $\overline{m}\in (\zz/112\zz)^{\times}$. This means then that $112\mid z$, because $\overline{m} \neq 0$, but 
        \[112\mid z \Longleftrightarrow 7\mid z \land  2 \mid z\]
        giving us the following congruencies,
        \begin{align*}
            z \equiv 0 \text{ mod 7} \\
            z \equiv 0 \text{ mod 2}
        \end{align*}
        and by CRT the solution is $z \equiv 0 \text{ mod 14}$ which is to say $z \in 14\zz$. Therefore the annihilator of $(\zz/112\zz)^{\times}$ in $\zz$ is $14\zz$.
    \end{itemize}

\end{itemize}

\end{document}