\documentclass[12pt]{article}
 
\addtolength{\textheight}{2.7in}
\addtolength{\topmargin}{-1.25in}
\addtolength{\textwidth}{1.0in}
\addtolength{\evensidemargin}{-0.5in}
\addtolength{\oddsidemargin}{-0.65in}
\setlength{\parskip}{0.1in}
\setlength{\parindent}{0.0in}

\usepackage{ multirow }
\usepackage[top=0.75in, bottom=1.25in, left=1in, right=1in]{geometry} 
\usepackage{amsmath,amsthm,amssymb} %this is THE math package
\usepackage{mathtools}
\usepackage{tikz}
\usepackage{graphicx}
\usepackage{fancybox}
\usepackage{hyperref}
\usepackage{varwidth}
\usepackage{mdframed}
\usepackage{mathrsfs}
\usepackage[most]{tcolorbox}
\usepackage{pagecolor}

\newcommand{\given}{\, | \,}

\pagecolor{black}
\color{white}
\newenvironment{solution}{\begin{proof}[\textbf{\textit{Solution}}] }{\end{proof}}

\pagestyle{empty}

\raggedbottom
 
\begin{document}

\vspace*{-0.3in}

\begin{flushleft}

Prof.~David Draper \\
Department of Statistics \\
University of California, Santa Cruz

\end{flushleft}

\begin{center}

\textbf{\large STAT 206: Quiz 1 \textit{[90 total points]}}

\end{center}

\begin{flushleft}

Name: Kevin Guillen \underline{\hspace*{5.85in}}

\end{flushleft}

Here is Your background information, translatable into $\mathcal{ B }$, for this problem.

\begin{itemize}

\item

\textit{(Fact 1)} As a broad generalization (which you can verify empirically), statisticians tend to have shy personalities more often than economists do --- let's quantify this observation by assuming (based on previous psychological studies) that 84\% of statisticians are shy but the corresponding percentage among economists is only 10\%. 

\item

\textit{(Fact 2)} Conferences on the topic of \textit{econometrics} are almost exclusively attended by economists and statisticians, with the majority of participants being economists --- let's approximately quantify this fact by assuming (based on data from previous conferences) that 95\% of the attendees are economists (and the rest statisticians, except for a tiny proportion of people from other professions, which can be ignored). 

\end{itemize}

Suppose that you (a physicist, say) go to an econometrics conference --- you strike up a conversation with the first person you (haphazardly) meet, and find that this person is shy. The point of this problem is to show that the (conditional) probability $p$ that you're talking to a statistician, given this data and the above background information, is only about 31\%, which most people find surprisingly low, and to understand why this is the right answer. Let $St$ = (person is statistician), $E$ = (person is economist), and $Sh$ = (person is shy).

\begin{itemize}

% -----------------------PART A---------------------------

\item[(a)]

Identify (in the form of a proposition $B_1$, one of the elements of $\mathcal{ B }$) the most important assumption needed in this problem to permit  its solution to be probabilistic; explain briefly. \textit{[5 points]}
\begin{solution}
    Our most important assumption is that we haphazardly meet a person. Since translating this requirement to math language our $B_1$ is taking a random sample of 1 from the population, which is all conference attendees.
\end{solution}

\vspace*{0.75in}

% -----------------------PART B---------------------------
\item[(b)]
 
Using the $St$, $E$ and $Sh$ notation, express the three numbers (84\%, 10\%, 95\%) above, and the probability we're solving for, in conditional probability terms, remembering to condition appropriately on $\mathcal{ B }$. \textit{[5 points]}
\begin{solution}
    \begin{align}
        84\% &= P(Sh \given St, \mathcal{ B}) \\
        10\% &= P(Sh \given  E, \mathcal{B}) \\
        95\% &= P(E \given \mathcal{B})
    \end{align}
    we also obtain the following,
    \begin{align}
        5\% = P(St \given \mathcal{B})
    \end{align}
\end{solution}

\vspace*{0.75in}

% -----------------------PART C---------------------------
\item[(c)]

Briefly explain why calculating the desired probability is a good job for Bayes's Theorem. \textit{[5 points]}
\begin{solution}
    Well, we are trying to calculate $P(St \given Sh, \mathcal{B} )$ and we see that probability (1) from above is just the reverse ordering of conditioning of what we are trying to solve for, which is perfect for Bayes' Theorem. 
    \begin{align*}
        P(St \given Sh, \mathcal{B}) = \dfrac{P(St \given \mathcal{B}) P(Sh \given St, \mathcal{B})}{P(Sh \given \mathcal{B})}
    \end{align*}
\end{solution}

\vspace*{0.7in}

\end{itemize}

\newpage

The goal in the rest of the problem is for you to use all three of the methods developed in class --- the 2 by 2 table cross-tabulating truth against data, Bayes's Theorem in odds ratio form, and calculating the denominator using the \textit{Law of Total Probability}, by partitioning over the unknown truth --- to compute $P ( St \given Sh, \mathcal{ B } )$, the posterior probability that the haphazard person is a statistician given that this person is shy (and given $\mathcal{ B }$).

\begin{table}[t!]

\centering

\caption{\textit{2 by 2 table cross-tabulating truth (statistician, economist) against data (shy, not shy) for the people at the conference, assuming a total number of attendees of 1,000.}}

\label{t:basic-table-1}

\bigskip

\begin{tabular}{cc|c|c|c}

& \multicolumn{1}{c}{} & \multicolumn{2}{c}{\textbf{Truth}} \\

& \multicolumn{1}{c}{} & \multicolumn{1}{c}{Statistician} & \multicolumn{1}{c}{Economist}  & Total \\ \cline{3-4}

\multirow{2}{*}{\textbf{Data}} & Shy & 42 & 95 & 137 \\ \cline{3-4}

& Not Shy & 8 & 855 & 863 \\ \cline{3-4}

& \multicolumn{1}{c}{Total} & \multicolumn{1}{c}{50} & \multicolumn{1}{c}{950} & 1,000

\end{tabular}

\end{table}

\begin{itemize}

\item[(d)]

Use the three numerical facts (84\%, 10\%, 95\%) given at the beginning of the quiz to fill in all 8 of the entries marked `---' in Table \ref{t:basic-table-1}, taking the total number of attendees at the conference to be 1,000 (\textit{Hint:} All of these numbers are integers), thereby showing that $P ( St \given Sh, \mathcal{ B } ) = \frac{ 42 }{ 137 } \doteq$ 30.7\%; show your work \textit{[20 points]}.

\begin{solution}
    First we can obtain the total number of economists and statisticians using (3) and (4) from part (b),
    \begin{align*}
        |E| &= P(E \given \mathcal{B}) \cdot 1000 = 0.95 \cdot 1000 = 950 \\
        |St| &= P(St \given \mathcal{B}) \cdot 1000 = 0.05 \cdot 1000 = 50.
    \end{align*}
    Now with (1) and (2) we can get how many of these statisticians and economists are shy,
    \begin{align*}
        |Sh, St| &= P(Sh \given St, \mathcal{B}) \cdot |St| = .84 \cdot 50 = 42 \\
        |Sh, E| &= P(E \given E, \mathcal{B}) \cdot |E| = .10 \cdot 950 = 95
    \end{align*}
    now knowing the total of economists and statisticians alongside of many of each are shy we can obtain how many of them are not shy,
    \begin{align*}
        |\neg Sh, St| &= |St| - |Sh, St| = 50 - 42 = 8 \\
        |\neg Sh, E| &= |E| - |Sh, E| =  950 - 95  = 855.
    \end{align*}

    Which means given that the person we haphazardly meet at this convention is shy, the probability that they are a statistician is,
    \begin{align*}
        P( St \given Sh, \mathcal{B}) = \dfrac{42}{137} = 30.7\%
    \end{align*}
\end{solution}

\newpage

\item[(e)]

Briefly explain why the following expression is a correct use of Bayes's Theorem on the odds ratio scale in this problem. \textit{[5 points]} \vspace*{-0.1in}

\begin{center}

\Large

\[
\begin{array}{ccccc}
\left[ \frac{ P ( St \given Sh, \, \mathcal{ B } ) }{ P ( E \given Sh, \, \mathcal{ B } ) } \right] & = & \left[ \frac{ P ( St \given \mathcal{ B } ) }{ P ( E \given \mathcal{ B } ) } \right] & \cdot & \left[ \frac{ P ( Sh \given St, \, \mathcal{ B } ) }{ P ( Sh \given E, \, \mathcal{ B } ) } \right] \\( 1 ) & = & ( 2 ) & \cdot & ( 3 )
\end{array}
\]

\normalsize

\end{center}

\vspace*{0.5in}

\item[(f)]

Here are three terms that are relevant to the quantities in part (e) above:

\begin{itemize}

\item

(Prior odds ratio in favor of $St$ over $E$, given $\mathcal{ B }$)

\item

(Bayes factor in favor of $St$ over $E$, given $\mathcal{ B }$)

\item

(Posterior odds ratio in favor of $St$ over $E$, given $\mathcal{ B }$)

\end{itemize}

Match these three terms with the numbers $( 1 ), ( 2 ), ( 3 )$ in the second line of the equation in part (e). \textit{[5 points]}

\vspace*{0.5in}

\item[(g)]

Compute the three ratios in part (e), briefly explaining your reasoning, thereby demonstrating that the posterior odds ratio $o$ in favor of $St$ over $E$ (given $\mathcal{ B }$) is $o = \frac{ 42 }{ 95 } \doteq 0.442$. \textit{[15 points]}

\vspace*{1.0in}

\item[(h)]

Use the expression $p = \frac{ o }{ 1 + o }$ to show that the desired probability in this problem --- the conditional probability that you're talking to a statistician (given $\mathcal{ B }$) --- is $p = \frac{ 42 }{ 137 } \doteq 0.307$. \textit{[5 points]}

\vspace*{0.75in}

\item[(i)]

Briefly explain why the following expression is a correct use of Bayes's Theorem on the probability scale in this problem. \textit{[5 points]}
\begin{equation} \label{e:bayes-1}
P ( St \given Sh, \, \mathcal{ B } ) = \frac{ P ( St \given \mathcal{ B } ) \, P ( Sh \given St, \, \mathcal{ B } ) }{ P ( Sh \given \mathcal{ B } ) } \, .
\end{equation}

\vspace*{0.5in}

\item[(j)]

Notice as usual that you know both of the numerator probabilities in equation (\ref{e:bayes-2}) but you don't (yet) know the denominator $P ( Sh \given \mathcal{ B } )$. Use the \textit{Law of Total Probability}, partitioning over the unknown truth, to show that
\begin{equation} \label{e:bayes-2}
P ( Sh \given \mathcal{ B } ) = \frac{ 137 }{ 1000 } = 0.137 \, ,
\end{equation}
and use this to show that
\begin{equation} \label{e:bayes-3}
P ( St \given Sh, \, \mathcal{ B } ) = \frac{ ( 0.05 ) ( 0.84 ) }{ 0.137 } = \frac{ 42 }{ 137 } \doteq 0.307 \, .
\end{equation}
\textit{[15 points]}

\vspace*{1.0in}

\item[(k)]

Someone says, ``That 30.7\% probability can't be right: 84\% of statisticians are shy, versus 10\% for economists, so your probability $p$ of talking to a statistician has to be over 50\%.'' Briefly explain why this line of reasoning is wrong, and why $p$ should indeed be less than 50\%. \textit{[5 points]}

\end{itemize}

\end{document}
