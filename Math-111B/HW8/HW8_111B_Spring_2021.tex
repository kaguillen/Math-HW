\documentclass[12pt,twoside,a4paper]{article}

\usepackage{amsmath,amsthm,amssymb}

\newtheorem{theorem}{Theorem}[section]					%Theorem environments
\newtheorem{lemma}[theorem]{Lemma}
\newtheorem{corollary}[theorem]{Corollary}
\newtheorem{proposition}[theorem]{Proposition}
\theoremstyle{definition}
\newtheorem{definition}[theorem]{Definition}
\theoremstyle{definition}
\newtheorem{example}[theorem]{Example}

\newcommand{\N}{\mathbb{N}}								%Sets of numbers
\newcommand{\Z}{\mathbb{Z}}
\newcommand{\Q}{\mathbb{Q}}
\newcommand{\R}{\mathbb{R}}
\newcommand{\C}{\mathbb{C}}
\newcommand{\F}{\mathbb{F}}

\newcommand{\catname}[1]{{\normalfont\textbf{#1}}}		%Category names
\newcommand{\Ab}{\catname{Ab}}
\newcommand{\Alg}[1]{{}_{#1}\catname{Alg}}
\newcommand{\bimod}[2]{{}_{#1}\catname{mod}_{#2}}
\newcommand{\biMod}[2]{{}_{#1}\catname{Mod}_{#2}}
\newcommand{\Grp}{\catname{Grp}}
\newcommand{\lmod}[1]{{}_{#1}\catname{mod}}
\newcommand{\lMod}[1]{{}_{#1}\catname{Mod}}
\newcommand{\Ring}{\catname{Ring}}
\newcommand{\rmod}[1]{\catname{mod}_{#1}}
\newcommand{\rMod}[1]{\catname{Mod}_{#1}}
\newcommand{\Set}{\catname{Set}}
\newcommand{\Top}{\catname{Top}}

\newcommand{\Add}{\operatorname{Add}}
\newcommand{\Ann}{\operatorname{Ann}}
\newcommand{\Aut}{\operatorname{Aut}}
\newcommand{\by}{\times}
\newcommand{\ch}{\operatorname{char}}
\newcommand{\coker}{\operatorname{coker}}
\newcommand{\Defres}{\operatorname{Defres}}
\newcommand{\divides}{\mid}
\newcommand{\dsum}{\oplus}
\newcommand{\End}{\operatorname{End}}
\newcommand{\Ext}{\operatorname{Ext}}
\newcommand{\fgmod}{\operatorname{mod}}
\newcommand{\Frac}{\operatorname{Frac}}
\newcommand{\Fun}{\operatorname{Fun}}
\newcommand{\Gal}{\operatorname{Gal}}
\newcommand{\GL}{\operatorname{GL}}
\newcommand{\gp}[1]{\langle#1\rangle}
\newcommand{\Hom}{\operatorname{Hom}}
\newcommand{\Id}{\operatorname{Id}}
\newcommand{\im}{\operatorname{im}}
\newcommand{\Ind}{\operatorname{Ind}}
\newcommand{\Indinf}{\operatorname{Indinf}}
\newcommand{\into}{\hookrightarrow}
\newcommand{\Irr}{\operatorname{Irr}}
\newcommand{\iso}{\cong}
\newcommand{\lat}{\operatorname{lat}}
\newcommand{\longto}{\longrightarrow}
\newcommand{\MaxSpec}{\operatorname{MaxSpec}}
\newcommand{\mc}[1]{\mathcal{#1}}
\newcommand{\Mod}{\operatorname{Mod}}
\newcommand{\ndivides}{\nmid}
\newcommand{\nor}{\trianglelefteq}
\newcommand{\norm}[1]{\left\lVert#1\right\rVert}
\newcommand{\Ob}{\operatorname{Ob}}
\newcommand{\Obj}{\operatorname{Obj}}
\newcommand{\onto}{\twoheadrightarrow}
\newcommand{\op}[1]{#1^{\text{op}}}
\newcommand{\Out}{\operatorname{Out}}
\newcommand{\rank}{\operatorname{rank}}
\newcommand{\Res}{\operatorname{Res}}
\newcommand{\semil}{\rtimes}
\newcommand{\semir}{\ltimes}
\newcommand{\set}[1]{\{#1\}}
\newcommand{\Sets}{\textbf{Sets}}
\newcommand{\Spec}{\operatorname{Spec}}
\newcommand{\spn}{\operatorname{span}}
\newcommand{\subgp}{\leq}
\newcommand{\Sym}{\operatorname{Sym}}
\newcommand{\tensor}{\otimes}
\newcommand{\To}{\Rightarrow}
\newcommand{\Tor}{\operatorname{Tor}}
\newcommand{\tr}{\operatorname{tr}}
\newcommand{\transv}{\mathrel{\text{\tpitchfork}}}
	\makeatletter
	\newcommand{\tpitchfork}{%
		\vbox{
			\baselineskip\z@skip
			\lineskip-.52ex
			\lineskiplimit\maxdimen
			\m@th
			\ialign{##\crcr\hidewidth\smash{$-$}\hidewidth\crcr$\pitchfork$\crcr}
		}%
	}
	\makeatother

\begin{document}
	
\begin{center}
	\textbf{Homework 8}\\
	Due: Friday, May 28th
\end{center}

\noindent Mandatory:\\
\\
(1) If $n,m\in\N$, let us say that $n\leq m$ if and only if $n$ divides $m$. Show that the relation $\leq$ is a partial order on $\N$. Is the set $\set{2,3,4,5}$ a chain?\\
\\
(2) Let $R$ be a commutative ring with $1\neq 0$. The \textit{Jacobson radical} of $R$, denoted $J(R)$, is defined as the intersection of all maximal ideals of $R$. In symbols:
\begin{equation*}
	J(R):=\bigcap_{M=\text{maximal ideal of }R} M.
\end{equation*}
Prove that $J(R)$ is a proper ideal of $R$, and determine $J(\Z)$.\\
\\
(3) Recall that the \textit{Gaussian integers} $\Z[i]$ is the following subring of $\C$:
\begin{equation*}
	\Z[i]:=\set{a+bi:a,b\in\Z}.
\end{equation*}
Use the fact that $(1+i)(1-i)=2$ to show that the ideal of $\Z[i]$ generated by 2 is not a prime ideal.\\
\\
(4) Let $A$ and $B$ be commutative rings both with 1 and let $f:A\to B$ be a unital ring homomorphism. If $P$ is a prime ideal of $B$, prove that
\begin{equation*}
	f^{-1}(P)=\set{a\in A:f(a)\in P}
\end{equation*}
is a prime ideal of $A$.\\
\\
Optional:\\
\\
(5) Give an example to show that the result of problem (4) is not true if ``prime ideal'' is replaced with ``maximal ideal.''\\
\\
(6) Let $p\in\N$ be a prime. Consider the following subset of the rationals:
\begin{equation*}
	\Z_{(p)}:=\set{\frac{a}{b}\in\Q:\gcd(a,b)=1\text{ and }p\ndivides b}.
\end{equation*}
In words, $\Z_{(p)}$ is the set of rational numbers $x\in\Q$ such that when $x$ is written in lowest terms, say $x=\frac{a}{b}$, then $p$ does not divide $b$.\\
(a) Show that $\Z_{(p)}$ is a subring of $\Q$. (The ring $\Z_{(p)}$ is called the \textit{localization of }$\Z$\textit{ at }$(p)$.)\\
(b) Use the usual bar notation for the elements of $\Z/p\Z$; that is, write $\overline{k}=k+p\Z$ for each $k\in\Z$. Define a map
\begin{align*}
	\phi:\Z_{(p)}	&\to\Z/p\Z\\
	x		&\mapsto \overline{a}\cdot(\overline{b})^{-1}\text{ if }x=\frac{a}{b}.
\end{align*}
Prove that $\phi$ is a well-defined, unital, and surjective ring homomorphism. Deduce that
\begin{equation*}
	p\Z_{(p)}=\set{\frac{a}{b}\in\Q:\gcd(a,b)=1\text{, }p\ndivides b\text{, and }p\divides a}
\end{equation*}
is a maximal ideal of $\Z_{(p)}$.\\
(c) Show that $p\Z_{(p)}$ is the \textit{unique} maximal ideal of $\Z_{(p)}$.
\end{document}