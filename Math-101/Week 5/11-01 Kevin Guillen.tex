\documentclass[11pt]{article}
 
\usepackage[top=0.75in, bottom=1.25in, left=1in, right=1in]{geometry} 
\usepackage{amsmath,amsthm,amssymb} %this is THE math package
\usepackage{mathtools}
\usepackage{tikz}
\usepackage{graphicx}
\usepackage{fancybox}
\usepackage{hyperref}
\usepackage{varwidth}
\usepackage{mdframed}
\usepackage{mathrsfs}
\usepackage[most]{tcolorbox}
%------------------------
%Fonts I use, uncomment if you like to use them.
%The first is the general font, and the second a math font
\usepackage{mathpazo}
\usepackage{eulervm}

%------------------------
%This is so that we have standard fonts for the double-stroked symbols
%for reals, naturals etc. regardless of what font you use.
%Don't comment
\AtBeginDocument{
  \DeclareSymbolFont{AMSb}{U}{msb}{m}{n}
  \DeclareSymbolFontAlphabet{\mathbb}{AMSb}}
%------------------------

%----------------------------------------------
%User-defined environments
%Commented because we're not using them in this document
%The only uncommented ones are the Problem and Solution environment

% \newenvironment{theorem}[2][Theorem]{\begin{trivlist}
% \item[\hskip \labelsep {\bfseries #1}\hskip \labelsep {\bfseries #2.}]}{\end{trivlist}}
% \newenvironment{lemma}[2][Lemma]{\begin{trivlist}
% \item[\hskip \labelsep {\bfseries #1}\hskip \labelsep {\bfseries #2.}]}{\end{trivlist}}
% \newenvironment{exercise}[2][Exercise]{\begin{trivlist}
% \item[\hskip \labelsep {\bfseries #1}\hskip \labelsep {\bfseries #2.}]}{\end{trivlist}}
% \newenvironment{question}[2][Question]{\begin{trivlist}
% \item[\hskip \labelsep {\bfseries #1}\hskip \labelsep {\bfseries #2.}]}{\end{trivlist}}
% \newenvironment{corollary}[2][Corollary]{\begin{trivlist}
% \item[\hskip \labelsep {\bfseries #1}\hskip \labelsep {\bfseries #2.}]}{\end{trivlist}}
\newenvironment{problem}[2][Problem\!]{\begin{trivlist}
\item[\hskip \labelsep {\bfseries #1}\hskip \labelsep {\bfseries #2}]}{\end{trivlist}}
%\newenvironment{sub-problem}[2][]{\begin{trivlist}
%\item[\hskip \labelsep {\bfseries #1}\hskip \labelsep {\bfseries #2}]}{\end{trivlist}}
\newenvironment{solution}{\begin{proof}[\textbf{\textit{Solution}}] }{\end{proof}}
%----------------------------------------------

%----------------------------
%User-defined notations
\newcommand{\zz}{\mathbb Z}   %blackboard bold Z
\newcommand{\qq}{\mathbb Q}   %blackboard bold Q
\newcommand{\ff}{\mathbb F}   %blackboard bold F
\newcommand{\rr}{\mathbb R}   %blackboard bold R
\newcommand{\nn}{\mathbb N}   %blackboard bold N
\newcommand{\cc}{\mathbb C}   %blackboard bold C
\newcommand{\af}{\mathbb A}   %blackboard bold A
\newcommand{\pp}{\mathbb P}   %blackboard bold P
\newcommand{\id}{\operatorname{id}} %for identity map
\newcommand{\im}{\operatorname{im}} %for image of a function
\newcommand{\dom}{\operatorname{dom}} %for domain of a function
\newcommand{\cat}[1]{\mathscr{#1}}   %calligraphic category
\newcommand{\abs}[1]{\left\lvert#1\right\rvert} %for absolute value
\newcommand{\norm}[1]{\left\lVert#1\right\rVert} %for norm
\newcommand{\modar}[1]{\text{ mod }{#1}} %for modular arithmetic
\newcommand{\set}[1]{\left\{#1\right\}} %for set
\newcommand{\setp}[2]{\left\{#1\ \middle|\ #2\right\}} %for set with a property
\newcommand{\card}[1]{\#\,{#1}} %for cardinality of a set
\newcommand\m[1]{\begin{pmatrix}#1\end{pmatrix}} 

%Re-defined notations
\renewcommand{\epsilon}{\varepsilon}
\renewcommand{\phi}{\varphi}
\renewcommand{\emptyset}{\varnothing}
\renewcommand{\geq}{\geqslant}
\renewcommand{\leq}{\leqslant}
\renewcommand{\Re}{\operatorname{Re}}
\renewcommand{\Im}{\operatorname{Im}}
%----------------------------

\allowdisplaybreaks
 
 
\begin{document}
 
\title{11/01 Submission}
\author{Kevin Guillen\\[0.5em]
MATH 101  | Problem Solving | Fall 2021}
\date{} 
\maketitle

%Use \[...\] instead of $$...$$

\begin{tcolorbox}
    \begin{problem} {IC - 10/22 - 81.}
        Find a formula for the sum \[\sum_{k=1}^{n}\frac{k}{(k+1)!}.\]
    \end{problem}
\end{tcolorbox}
\begin{proof}
    The formula for the given sum will be the following,
    \[\frac{(n+1)!-1}{(n+1)!}.\]
    We will prove this using induction.

    Case $n=1:$
    \begin{align*}
        \sum_{k=1}^{1}\frac{1}{(1+1)! } = \frac{1}{2} = \frac{(1+1)! -1}{(1+1)!}.
    \end{align*}

    Induction step, assume it holds for $n \leq p$.

    Case $n = p+1$:
    \begin{align*}
        \sum_{k =1 }^{p+1}\frac{k}{(k+1)!} &= \underbrace{\sum_{k =1}^{p}\frac{k}{(k+1)!}}_{\text{true for }n=p} + \frac{p+1}{(p+1+1)!} \\
        &= \frac{(p+1)!-1}{(p+1)!} + \frac{p+1}{(p+2)!} \\
        &= \left(\frac{p+2}{p+2}\right)\frac{(p+1)!-1}{(p+1)!} + \frac{p+1}{(p+2)!} \\
        &= \frac{(p+2)! -p -2}{(p+2)!} + \frac{p+1}{(p+2)!} \\
        &= \frac{(p+2)! -1}{(p+2)!}\\
        &= \frac{(n+1)!-1}{(n+1)!}
    \end{align*}
    as desired. 
\end{proof}

\newpage
\begin{tcolorbox}
    \begin{problem} {OC - 10/25 - 61.}
        Show that $(n+1)^{n} \geq 2^{n}n!$
    \end{problem}
\end{tcolorbox}
\begin{proof}
    We will prove this using induction. 

    Case $n = 1$: \[(1+1)^{1} \geq 2^{1}1! \rightarrow 2 \geq 2\]

    Induction step, assume it holds for $n \geq k$.

    Case $n = k+1$:
    \begin{align*}
        (k+1)^{k} &\geq 2^{k}k! && \text{multiply by } 2(k+1) \\
        2(k+1)^{k+1} &\geq (k+1)!2^{k+1}
    \end{align*}
    The last thing we need is that we want to show $2(k+1)^{k+1} \leq (k+2)^{k+1}$ but this is of the form
    \begin{align*}
        2n^{n} &\leq (n+1)^{n} \\
        2 &\leq (1 + \frac{1}{n})^{n}
    \end{align*} 
    We know by binomial expansion that this does indeed hold though since the first two terms will be $1 + \frac{n}{n}$.
    Therefore it holds for $n = k+1$.
\end{proof}
\begin{tcolorbox}
    \begin{problem} {IC - 10/27 - 99.}
        Prove for all real numbers $x,y,z$ that $x^{2} + y^{2} + z^{2} \geq xy + yz + zx$
    \end{problem}
\end{tcolorbox}
\begin{proof}
    We that for any real numbers $x,y,z$ that the following holds, \[(x-y)^{2} + (y-z)^{2} + (x-z)^{2}\geq 0.\]
    Expanding this inequality will yield the desired result.
    \begin{align*}
        (x-y)^{2} + (y-z)^{2} + (x-z)^{2}&\geq 0 \\
        x^{2} + y^{2} -2xy + y^{2} + z^{2} -2yz + x^{2} + z^{2} -2xz &\geq 0 \\
        2x^{2} + 2y^{2} + 2z^{2} -2xy -2yz -2xz &\geq 0 \\
        2(x^{2} + y^{2} + z^{2}) &\geq 2(xy + yz + xz) \\
        x^{2} + y^{2} + z^{2} &\geq xy + yz + xz
    \end{align*}
\end{proof}

\newpage
\begin{tcolorbox}
    \begin{problem} {IC - 10/29 - 105.}
        Verify algebraically the identity $\binom{n}{r} \binom{r}{k} = \binom{n}{k}\binom{n-k}{r-k}$ and then give a combinatorial proof. 
    \end{problem}
\end{tcolorbox}
\begin{proof}
    Algebraically: Recall the formula for $n$ choose $k$ is simply $\dfrac{n!}{k!(n-k)!}$. Applying this to the LHS,
    \begin{align*}
        \binom{n}{r} \binom{r}{k} &= \frac{n!}{r!(n-r)!}\frac{r!}{k!(r-k)!} \\
        &= \frac{n!r!}{r!k!(n-r)!(r-k)!} \\
        &= \frac{n!}{k!(n-r)!(r-k)!}
    \end{align*}

    Applying to the RHS,
    \begin{align*}
        \binom{n}{k}\binom{n-k}{r-k} &= \frac{n!}{k!(n-k)!}\frac{(n-k)!}{(r-k)!(n-k - (r-k))!} \\
        &=\frac{n!}{k!(n-k)!}\frac{(n-k)!}{(r-k)!(n-r)!} \\
        &= \frac{n! (n-k)!}{k!(n-k)!(r-k)!(n-r)!} \\
        &= \frac{n!}{k!(n-r)!(r-k)!}
    \end{align*}
    
    We see that the RHS does indeed equal the LHS, meaning the identity is true.

    Not sure how to show combinatorially.
\end{proof}

\end{document}
