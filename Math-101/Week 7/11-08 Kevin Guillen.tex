\documentclass[11pt]{article}
 
\usepackage[top=0.75in, bottom=1.25in, left=1in, right=1in]{geometry} 
\usepackage{amsmath,amsthm,amssymb} %this is THE math package
\usepackage{mathtools}
\usepackage{tikz}
\usepackage{graphicx}
\usepackage{fancybox}
\usepackage{hyperref}
\usepackage{varwidth}
\usepackage{mdframed}
\usepackage{mathrsfs}
\usepackage[most]{tcolorbox}
%------------------------
%Fonts I use, uncomment if you like to use them.
%The first is the general font, and the second a math font
\usepackage{mathpazo}
\usepackage{eulervm}

%------------------------
%This is so that we have standard fonts for the double-stroked symbols
%for reals, naturals etc. regardless of what font you use.
%Don't comment
\AtBeginDocument{
  \DeclareSymbolFont{AMSb}{U}{msb}{m}{n}
  \DeclareSymbolFontAlphabet{\mathbb}{AMSb}}
%------------------------

%----------------------------------------------
%User-defined environments
%Commented because we're not using them in this document
%The only uncommented ones are the Problem and Solution environment

% \newenvironment{theorem}[2][Theorem]{\begin{trivlist}
% \item[\hskip \labelsep {\bfseries #1}\hskip \labelsep {\bfseries #2.}]}{\end{trivlist}}
% \newenvironment{lemma}[2][Lemma]{\begin{trivlist}
% \item[\hskip \labelsep {\bfseries #1}\hskip \labelsep {\bfseries #2.}]}{\end{trivlist}}
% \newenvironment{exercise}[2][Exercise]{\begin{trivlist}
% \item[\hskip \labelsep {\bfseries #1}\hskip \labelsep {\bfseries #2.}]}{\end{trivlist}}
% \newenvironment{question}[2][Question]{\begin{trivlist}
% \item[\hskip \labelsep {\bfseries #1}\hskip \labelsep {\bfseries #2.}]}{\end{trivlist}}
% \newenvironment{corollary}[2][Corollary]{\begin{trivlist}
% \item[\hskip \labelsep {\bfseries #1}\hskip \labelsep {\bfseries #2.}]}{\end{trivlist}}
\newenvironment{problem}[2][Problem\!]{\begin{trivlist}
\item[\hskip \labelsep {\bfseries #1}\hskip \labelsep {\bfseries #2}]}{\end{trivlist}}
%\newenvironment{sub-problem}[2][]{\begin{trivlist}
%\item[\hskip \labelsep {\bfseries #1}\hskip \labelsep {\bfseries #2}]}{\end{trivlist}}
\newenvironment{solution}{\begin{proof}[\textbf{\textit{Solution}}] }{\end{proof}}
%----------------------------------------------

%----------------------------
%User-defined notations
\newcommand{\zz}{\mathbb Z}   %blackboard bold Z
\newcommand{\qq}{\mathbb Q}   %blackboard bold Q
\newcommand{\ff}{\mathbb F}   %blackboard bold F
\newcommand{\rr}{\mathbb R}   %blackboard bold R
\newcommand{\nn}{\mathbb N}   %blackboard bold N
\newcommand{\cc}{\mathbb C}   %blackboard bold C
\newcommand{\af}{\mathbb A}   %blackboard bold A
\newcommand{\pp}{\mathbb P}   %blackboard bold P
\newcommand{\id}{\operatorname{id}} %for identity map
\newcommand{\im}{\operatorname{im}} %for image of a function
\newcommand{\dom}{\operatorname{dom}} %for domain of a function
\newcommand{\cat}[1]{\mathscr{#1}}   %calligraphic category
\newcommand{\abs}[1]{\left\lvert#1\right\rvert} %for absolute value
\newcommand{\norm}[1]{\left\lVert#1\right\rVert} %for norm
\newcommand{\modar}[1]{\text{ mod }{#1}} %for modular arithmetic
\newcommand{\set}[1]{\left\{#1\right\}} %for set
\newcommand{\setp}[2]{\left\{#1\ \middle|\ #2\right\}} %for set with a property
\newcommand{\card}[1]{\#\,{#1}} %for cardinality of a set
\newcommand\m[1]{\begin{pmatrix}#1\end{pmatrix}} 

%Re-defined notations
\renewcommand{\epsilon}{\varepsilon}
\renewcommand{\phi}{\varphi}
\renewcommand{\emptyset}{\varnothing}
\renewcommand{\geq}{\geqslant}
\renewcommand{\leq}{\leqslant}
\renewcommand{\Re}{\operatorname{Re}}
\renewcommand{\Im}{\operatorname{Im}}
%----------------------------

\allowdisplaybreaks
 
 
\begin{document}
 
\title{Assignment}
\author{Kevin Guillen\\[0.5em]
MATH  | Class | Quarter}
\date{} 
\maketitle

%Use \[...\] instead of $$...$$

\begin{tcolorbox}
    \begin{problem}{ IC | 11/05 | 124 (Putnam)}
        Define a selfish set to be a set which has its own cardinality as an element. FInd, with a proof, the number of subsets of $\set{1,2,\dots, n}$ which are minimal selfish sets, that is, selfish sets none of whose proper subsets is selfish. 
    \end{problem}
\end{tcolorbox}
\begin{proof}
    Consider an arbitrary set $A$. If $A$ were to be a minimal selfish set, then by definition every element of $A$ would need to be greater than or equal to the cardinality of $A$. This observation will be needed. 

    Let $A_n$ be defined as follows,
    \[A_n := \set{1,2,\dots, n}.\]
    Also let $S(A_n)$ denote the number of subsets where are minimal selfish sets in $A_n$.


    Let $B \subseteq A$ and be a minimal selfish set. There are 2 cases here, the first is that $n$ is in $B$ and the second is that $n$ is not in $B$.
    
    If $n$ is not contained in $B$ then we know know $B$ will also have to be a minimal selfish set of $A_{n-1}$ which there are $S(A_{n-1})$ of. 

    If $n$ is indeed contained in $B$ then we know that the element 1 cannot be in $B$ because then $B$ would not be a minimal selfish set. This is because $\set{1}$ is a minimal selfish set. So this subset must also be a subset of $\set{2, \dots, n}$. If we remove the element $n$ from this set then we know it must be a subset of $\set{2, \dots, n-1}$. Next if we were to subtract 1 from every element this new set, let's refer to it as $C$, has got to be a subset of $\set{1,\dots,n-2}$. Meaning if $B$ was a minimal selfish set of $A_n$ that contained $n$, then our derived set $C$, must be a minimal selfish set of $A_{n-2}$. Which we know there are $S(A_{n-2})$ of. 
    
    Putting this together we get,
    \[S(A_n) = S(A_{n-1} + S(A_{n-2})).\]

    We see for $A_1 = \set{1}$. That it only has 1 subset that is a minimal selfish set, that it $\set{1}$. Next we see for $A_2 = \set{1,2}$ that it only contains 1 subset that is a minimal selfish set at that is $\set{1}$ again. So we have $A_1 = A_2 = 1$. Notice though that these are the same starting values as the fibonacci sequence, and that we define the value of $S(A_n)$ for $n > 2$ as the sum of the previous 2 terms. Thus $S(A_n) = \text{Fib}(n)$, where $\text{Fib}(n)$ is the $n$th term in the fibonacci sequence. 
\end{proof}

\newpage
\begin{tcolorbox}
    \begin{problem} {IC | 11/03 | 122}
        If eight dies are rolled what is the probability that all six numbers appear?
    \end{problem}
\end{tcolorbox}
\begin{proof}
    There are 2 scenarios that result in all six numbers appearing. The first is that four numbers appear once and two numbers appear two times. 

    The second scenario is that every number appears once, and that one of the numbers appears another 2 times. In other words five numbers appear once and one number appears three times. 

    Let's begin by calculating the first scenario. We have 6 choose 2 ways of 
\end{proof}


\end{document}
