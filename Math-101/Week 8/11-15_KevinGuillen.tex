\documentclass[11pt]{article}
 
\usepackage[top=0.75in, bottom=1.25in, left=1in, right=1in]{geometry} 
\usepackage{amsmath,amsthm,amssymb} %this is THE math package
\usepackage{mathtools}
\usepackage{tikz}
\usepackage{graphicx}
\usepackage{fancybox}
\usepackage{hyperref}
\usepackage{varwidth}
\usepackage{mdframed}
\usepackage{mathrsfs}
\usepackage[most]{tcolorbox}
%------------------------
%Fonts I use, uncomment if you like to use them.
%The first is the general font, and the second a math font
\usepackage{mathpazo}
\usepackage{eulervm}
%------------------------
%This is so that we have standard fonts for the double-stroked symbols
%for reals, naturals etc. regardless of what font you use.
%Don't comment
\AtBeginDocument{
  \DeclareSymbolFont{AMSb}{U}{msb}{m}{n}
  \DeclareSymbolFontAlphabet{\mathbb}{AMSb}}
%------------------------

%----------------------------------------------
%User-defined environments
%Commented because we're not using them in this document
%The only uncommented ones are the Problem and Solution environment

% \newenvironment{theorem}[2][Theorem]{\begin{trivlist}
% \item[\hskip \labelsep {\bfseries #1}\hskip \labelsep {\bfseries #2.}]}{\end{trivlist}}
% \newenvironment{lemma}[2][Lemma]{\begin{trivlist}
% \item[\hskip \labelsep {\bfseries #1}\hskip \labelsep {\bfseries #2.}]}{\end{trivlist}}
% \newenvironment{exercise}[2][Exercise]{\begin{trivlist}
% \item[\hskip \labelsep {\bfseries #1}\hskip \labelsep {\bfseries #2.}]}{\end{trivlist}}
% \newenvironment{question}[2][Question]{\begin{trivlist}
% \item[\hskip \labelsep {\bfseries #1}\hskip \labelsep {\bfseries #2.}]}{\end{trivlist}}
% \newenvironment{corollary}[2][Corollary]{\begin{trivlist}
% \item[\hskip \labelsep {\bfseries #1}\hskip \labelsep {\bfseries #2.}]}{\end{trivlist}}
\newenvironment{problem}[2][Problem\!]{\begin{trivlist}
\item[\hskip \labelsep {\bfseries #1}\hskip \labelsep {\bfseries #2}]}{\end{trivlist}}
%\newenvironment{sub-problem}[2][]{\begin{trivlist}
%\item[\hskip \labelsep {\bfseries #1}\hskip \labelsep {\bfseries #2}]}{\end{trivlist}}
\newenvironment{solution}{\begin{proof}[\textbf{\textit{Solution}}] }{\end{proof}}
%----------------------------------------------

%----------------------------
%User-defined notations
\newcommand{\zz}{\mathbb Z}   %blackboard bold Z
\newcommand{\qq}{\mathbb Q}   %blackboard bold Q
\newcommand{\ff}{\mathbb F}   %blackboard bold F
\newcommand{\rr}{\mathbb R}   %blackboard bold R
\newcommand{\nn}{\mathbb N}   %blackboard bold N
\newcommand{\cc}{\mathbb C}   %blackboard bold C
\newcommand{\af}{\mathbb A}   %blackboard bold A
\newcommand{\pp}{\mathbb P}   %blackboard bold P
\newcommand{\id}{\operatorname{id}} %for identity map
\newcommand{\im}{\operatorname{im}} %for image of a function
\newcommand{\dom}{\operatorname{dom}} %for domain of a function
\newcommand{\cat}[1]{\mathscr{#1}}   %calligraphic category
\newcommand{\abs}[1]{\left\lvert#1\right\rvert} %for absolute value
\newcommand{\norm}[1]{\left\lVert#1\right\rVert} %for norm
\newcommand{\modar}[1]{\text{ mod }{#1}} %for modular arithmetic
\newcommand{\set}[1]{\left\{#1\right\}} %for set
\newcommand{\setp}[2]{\left\{#1\ \middle|\ #2\right\}} %for set with a property
\newcommand{\card}[1]{\#\,{#1}} %for cardinality of a set
\newcommand\m[1]{\begin{pmatrix}#1\end{pmatrix}} 

%Re-defined notations
\renewcommand{\epsilon}{\varepsilon}
\renewcommand{\phi}{\varphi}
\renewcommand{\emptyset}{\varnothing}
\renewcommand{\geq}{\geqslant}
\renewcommand{\leq}{\leqslant}
\renewcommand{\Re}{\operatorname{Re}}
\renewcommand{\Im}{\operatorname{Im}}
%----------------------------

\allowdisplaybreaks
 
 
\begin{document}
 
\title{Nov 15 re-Submission}
\author{Kevin Guillen\\[0.5em]
MATH 101 | Problem Solving | Fall 2021}
\date{} 
\maketitle

%Use \[...\] instead of $$...$$
Feedback:

IC 135.This would be $C^{}3$ if you did no include unnecessary statements, e.g
Fermat’s little theorem.

IC 136. See IC 135.

\begin{tcolorbox}
  \begin{problem} {IC | 11/10 | 135.} Show if $a^{2} + b^{2} = c^{2}$ then $3| ab$
  \end{problem}
\end{tcolorbox}
\begin{proof}
  We have $n^{3} \equiv n \text{ mod }3$, meaning $3|(n^{3}-n)\rightarrow 3|n(n^{2}-1)$. Because 3 is a prime that means it must divide one of these factors. In the case that 3 divides $n$, then it must also divide $n^{2}$. In the case that it divides $(n^{2}-1)$ that means $n^{2}\equiv 1 \text{ mod }3$. Meaning the only possible remainders are $0$ and $1$. 

  If $3 \nmid ab$ that would mean neither $a$ or $b$ are divisible by $3$. Implying they are of the form $a^{2} \equiv 1\text{ mod }3$ and $b^{2} \equiv 1\text{ mod }3$. Therefore $c^{2} \equiv 2 \text{ mod } 3$, but that is impossible since the only possible remainders for a square mod 3 are 0 and 1. Therefore if the equation holds then $3|ab$.
\end{proof}

\begin{tcolorbox}
  \begin{problem} {IC | 11/10 | 136.} If $x^{3} + y^{3} = z^{3}$ show that at least 1 of $x,y,z$ is divisible by $7$.    
  \end{problem}
\end{tcolorbox}
\begin{proof}
  We have, $n^{7} \equiv n \text{ mod }$ which means $7| (n^{7}-n) \rightarrow 7|n(n^{3-1})(n^{3} + 1)$. Because 7 is a prime it must divide one of these factors. In the case that 7 divides $n$ then it must divide $n^{3}$, implying $n^{3} \equiv 0 \text{ mod }7$. In the case that 7 divides $(n^{3}-1)$ then that means $n^{3} \equiv 1 \text{ mod }7 $. Finally if 7 divides $(n^{3} + 1)$ that means $n^{3} \equiv -1 \text{ mod }7$. 

  Now in the case that neither $x^{3}$ or $y^{3}$ are divisible by 7. That means they have a remainder of $\pm 1$ when dividing by 7. Without loss of generality say $x^{3}$ has remainder $-1$ and $y^{3}$ has remainder $1$. Then their sum has to have reaminder 0 meaning $z^{3}$ will be divisible by 7. In the case they both have remainder 1 that would result in a contradiction because $z^{3} \equiv 2 \text{ mod }7$ is not possible. Therefore at least one of these integers is disvisble by 7 if the given equation holds. 
\end{proof}





\newpage
\begin{tcolorbox}
  \begin{problem} {IC | 11/10 | 139.}
    For what values of $n$ can $\set{1,2,\dots,n}$ be partitioned into three subsets with equal sums? 
  \end{problem}
\end{tcolorbox}
\begin{proof}
    If we are able to partition the set into 3 subsets that all have the same sum that would be the sum of all the terms is divisble by 3. This gives us the following requirement, \[3|\sum_{k =1}^{n}k\]

    We can obtain a formula for the summation through the following,
    \begin{align*}
        1 + 2 + \dots + (n-1) + n = n + (n-1) + \dots + 2 + 1
    \end{align*}
    adding both sides to each other we get 
    \begin{align*}
        \underbrace{(n+1) + (n+1) + \dots + (n+1)}_{n} = n(n+1)
    \end{align*}
    now we have to divide by 2 to undo our addition and we get the following,
    \begin{align*}
        \sum_{k = 1}^{n}k = \dfrac{n(n+1)}{2}.
    \end{align*}

    This means in order to get 3 paritions that have equal sum, 3 must divide $\dfrac{n(n+1)}{2}$. Therfore $n$ must satisfy either $n \equiv 0 \text{ mod }3$ or $n \equiv 2 \text{ mod }3$. We see though in the case that $n = 3$, such a partition is not possible. Therefore there is also a lower bound for $n$. We see this lower bound is simply $n = 5$. We see this throught the following, 
    \begin{align}
        \set{1,4}, \set{2,3}, \set{5}.
    \end{align}

    Therfore $n >=5$ and either $n \equiv 0 \text{ mod }3$ or $n \equiv 2 \text{ mod }3$
\end{proof}

\begin{tcolorbox}
    \begin{problem} {IC | 11/12 | 143.}
        Find all positive integers $n$ such that $2^{4} + 2^{7} + 2^{n}$ is a perfect square. 
    \end{problem}
\end{tcolorbox}
\begin{proof}
    This is same as finding $n$ such that $n^{2} + 144 = k^{2}$. Consider the following though,
    \begin{align*}
        2^{n} + 144 &= k^{2} \\
        2^{n} &= k^{2} - 144 \\
        2^{n} &= = (k + 12)(k-12)
    \end{align*}

    Therefore we have that $(k+12)$ and $(k-12)$ must both be powers of 2 and that they must differ by 24. We can see that 8 and 32 differ by 24 and are both powers of 2. This gives us,
    \[8 \cdot 32 = 2^{3}2^{5} = 2^{8}.\]

    Thus the only integer $n$ that can satisfy this is $n = 8$. This is because the distance between powers of two is always increasing there will never be another pair of powers of 2 such that their difference is 24. 
\end{proof}
\newpage

\begin{tcolorbox}
    \begin{problem} {OC | 11/10 | 88.}
        Prove that there does not exist a natural number $n$ such that $n(n+1)$ is a perfect square.
    \end{problem}
\end{tcolorbox}
\begin{proof}
    Assume $n(n+1)$ is indeed a perfect square. That means it can expressed as, $n(n+1)= k^{2}$ for some $k \in \zz$. Consider the following though,
    \begin{align*}
        n(n+1) &= k^{2} \\
        n^{2} + n &= k^{2} \\
        n^{2} + k^{2} &= -n \\
        (n+k)(n-k) &= -n
    \end{align*}

    But either $(n+k)$ or $(n-k)$ is greater than $\abs{n}$, so this is a contradiction. Therfore $n(n+1)$ cannot be a perfect square. 
\end{proof}

\begin{tcolorbox}
    \begin{problem} {OC | 11/12 | 90.}
        Prove there is a unique integer $n$ such that $2^{8} + 2^{11} + 2^{n}$ is a perfect square.
    \end{problem}
\end{tcolorbox}
\begin{proof}
    This is similair to our IC class problem 143. First we see we are looking to satisfy the following,
    \begin{align*}
        2^{8} + 2^{11} + 2^{n} &= k^{2} \\
        2^{n} + 2304 &= k^{2} \\
        2^{n} &= k^{2} - 2304 \\
        2^{n} &= (k - 48)(k+48)
    \end{align*}

    Thus there has to be two powers of 2 such that their difference is 96. Consider 128, we see $128-96 = 32$, and botb $128$ and $32$ are powers of 2. This gives us the following,
    \[32 \cdot 128 = 2^{5} 2^{7} = 2^{12}.\]
    Therefore $n =12$ meaning there does exist indeed exist an $n$ such that the sum given is a perfect square.
\end{proof}

\end{document}
