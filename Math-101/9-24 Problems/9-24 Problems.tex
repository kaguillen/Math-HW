\documentclass[12pt]{article}
 
\usepackage[top=0.5in, bottom=1.25in, left=.8in, right=.8in]{geometry} 
\usepackage{amsmath,amsthm,amssymb} %this is THE math package
\usepackage{mathtools}
\usepackage{tikz}
\usepackage{graphicx}
\usepackage{fancybox}
\usepackage{hyperref}
\usepackage{varwidth}
\usepackage{mdframed}
\usepackage{mathrsfs}
\usepackage{xcolor}

%\pagecolor[rgb]{0.1,0.1,0.1}
%\color[rgb]{.8,.8,.8}
%------------------------
%Fonts I use, uncomment if you like to use them.
%The first is the general font, and the second a math font
\usepackage{mathpazo}
\usepackage{eulervm}

%------------------------
%This is so that we have standard fonts for the doublestroked symbols
%for reals, naturals etc. regardless of what font you use.
%Don't comment
\AtBeginDocument{
  \DeclareSymbolFont{AMSb}{U}{msb}{m}{n}
  \DeclareSymbolFontAlphabet{\mathbb}{AMSb}}
%------------------------

%----------------------------------------------
%User-defined environments
%Commented because we're not using them in this document
%The only uncommented ones are the Problem and Solution environment

% \newenvironment{theorem}[2][Theorem]{\begin{trivlist}
% \item[\hskip \labelsep {\bfseries #1}\hskip \labelsep {\bfseries #2.}]}{\end{trivlist}}
% \newenvironment{lemma}[2][Lemma]{\begin{trivlist}
% \item[\hskip \labelsep {\bfseries #1}\hskip \labelsep {\bfseries #2.}]}{\end{trivlist}}
% \newenvironment{exercise}[2][Exercise]{\begin{trivlist}
% \item[\hskip \labelsep {\bfseries #1}\hskip \labelsep {\bfseries #2.}]}{\end{trivlist}}
% \newenvironment{question}[2][Question]{\begin{trivlist}
% \item[\hskip \labelsep {\bfseries #1}\hskip \labelsep {\bfseries #2.}]}{\end{trivlist}}
% \newenvironment{corollary}[2][Corollary]{\begin{trivlist}
% \item[\hskip \labelsep {\bfseries #1}\hskip \labelsep {\bfseries #2.}]}{\end{trivlist}}
\newenvironment{problem}[2][Problem\!]{\begin{trivlist}
\item[\hskip \labelsep {\bfseries #1}\hskip \labelsep {\bfseries #2.}]}{\end{trivlist}}
%\newenvironment{sub-problem}[2][]{\begin{trivlist}
%\item[\hskip \labelsep {\bfseries #1}\hskip \labelsep {\bfseries #2}]}{\end{trivlist}}
\newenvironment{solution}{\begin{proof}[\textbf{\textit{Solution}}]}{\end{proof}}
%----------------------------------------------

%----------------------------
%User-defined notations
\newcommand{\zz}{\mathbb Z}   %blackboard bold Z
\newcommand{\qq}{\mathbb Q}   %blackboard bold Q
\newcommand{\ff}{\mathbb F}   %blackboard bold F
\newcommand{\rr}{\mathbb R}   %blackboard bold R
\newcommand{\nn}{\mathbb N}   %blackboard bold N
\newcommand{\cc}{\mathbb C}   %blackboard bold C
\newcommand{\af}{\mathbb A}   %blackboard bold A
\newcommand{\pp}{\mathbb P}   %blackboard bold P
\newcommand{\id}{\operatorname{id}} %for identity map
\newcommand{\im}{\operatorname{im}} %for image of a function
\newcommand{\dom}{\operatorname{dom}} %for domain of a function
\newcommand{\cat}[1]{\mathscr{#1}}   %calligraphic category
\newcommand{\abs}[1]{\left\lvert#1\right\rvert} %for absolute value
\newcommand{\norm}[1]{\left\lVert#1\right\rVert} %for norm
\newcommand{\modar}[1]{\text{ mod }{#1}} %for modular arithmetic
\newcommand{\set}[1]{\left\{#1\right\}} %for set
\newcommand{\setp}[2]{\left\{#1\ \middle|\ #2\right\}} %for set with a property
\newcommand{\card}[1]{\#\,{#1}} %for cardinality of a set

%Re-defined notations
\renewcommand{\epsilon}{\varepsilon}
\renewcommand{\phi}{\varphi}
\renewcommand{\emptyset}{\varnothing}
\renewcommand{\geq}{\geqslant}
\renewcommand{\leq}{\leqslant}
\renewcommand{\Re}{\operatorname{Re}}
\renewcommand{\Im}{\operatorname{Im}}
%----------------------------

\allowdisplaybreaks
 
\begin{document}
 
\title{9/24 Problems}
\author{Kevin Guillen\\[0.5cm]
MATH 101 | Problem Solving | Fall 2021}
\date{} 
\maketitle

%Use \[...\] instead of $$...$$

\begin{problem}{1}
Find the sum of:
\[\frac{1}{1 \cdot 2\cdot 3} + \frac{1}{2 \cdot 3 \cdot 4} + \dots + \frac{1}{100 \cdot 101 \cdot 102}\]

\begin{solution}
    If we look at this as a summation we can describe each term as, 
    \[\frac{1}{i \cdot (i +1) \cdot (i+2)} \] 
    for $i\in \set{1, \dots, n}$. Giving us,
    \[\sum_{i=1}^n \frac{1}{i \cdot (i +1) \cdot (i+2)}\]
    Doing some algebra we can obtain the following,
    \begin{align*}
        \sum_{i=1}^n \frac{1}{i \cdot (i +1) \cdot (i+2)} &= \sum_{i=1}^n \frac{1}{2}\cdot \frac{2}{i \cdot (i +1) \cdot (i+2)} \\
        &= \frac{1}{2} \cdot \sum_{i=1}^n \frac{i + 2 - i}{i \cdot (i +1) \cdot (i+2)} \\
        &= \frac{1}{2} \cdot\sum_{i=1}^n  \frac{(i + 2)}{i \cdot (i +1) \cdot (i+2)} - \frac{i}{i \cdot (i +1) \cdot (i+2)} \\
        &= \frac{1}{2} \cdot\sum_{i=1}^n  \frac{1}{i \cdot (i +1)} - \frac{1}{(i +1) \cdot (i+2)}
    \end{align*}

    \noindent Finally when we expand the summation out we can see something important,
    \[\frac{1}{2}\left(\frac{1}{1 \cdot (2)} - \frac{1}{(2)\cdot(3)} + \frac{1}{2\cdot(3)} - \frac{1}{(3)\cdot(4)}+ \dots + \frac{1}{n \cdot(n+1)}-\frac{1}{(n +1)\cdot(n + 2)}\right)\]
    that the middle terms cancel out giving us,
    \[\frac{1}{2} \cdot \left(\frac{1}{1\cdot(1+1)} - \frac{1}{(n+1)\cdot(n+2)}\right)\]
    Now plugging in our desired value for the problem, $n = 100$, we get,
    \begin{align*}
        \frac{1}{2} \cdot \left(\frac{1}{2} - \frac{1}{101 \cdot 102} \right) &= \frac{1}{4} - \frac{1}{20604} \\ 
        &= \frac{5151}{20604} - \frac{1}{20604}  \\
        &= \frac{5150}{20604}
    \end{align*}
    \end{solution}
\end{problem}


\end{document}
