\documentclass{beamer}
\usepackage{amsmath,amsthm,amssymb} %this is THE math package
\usepackage{mathtools}
\usepackage{tikz}
\usepackage{graphicx}
\usepackage{fancybox}
\usepackage{hyperref}
\usepackage{varwidth}
\usepackage{mdframed}
\usepackage{mathrsfs}
\usepackage[most]{tcolorbox}
\usepackage{xcolor}
\newcommand{\zz}{\mathbb Z}   %blackboard bold Z
\newcommand{\qq}{\mathbb Q}   %blackboard bold Q
\newcommand{\ff}{\mathbb F}   %blackboard bold F
\newcommand{\rr}{\mathbb R}   %blackboard bold R
\newcommand{\nn}{\mathbb N}   %blackboard bold N
\newcommand{\cc}{\mathbb C}   %blackboard bold C
\newcommand{\af}{\mathbb A}   %blackboard bold A
\newcommand{\pp}{\mathbb P}   %blackboard bold P
\newcommand{\id}{\operatorname{id}} %for identity map
\newcommand{\im}{\operatorname{im}} %for image of a function
\newcommand{\dom}{\operatorname{dom}} %for domain of a function
\newcommand{\cat}[1]{\mathscr{#1}}   %calligraphic category
\newcommand{\abs}[1]{\left\lvert#1\right\rvert} %for absolute value
\newcommand{\norm}[1]{\left\lVert#1\right\rVert} %for norm
\newcommand{\modar}[1]{\text{ mod }{#1}} %for modular arithmetic
\newcommand{\set}[1]{\left\{#1\right\}} %for set
\newcommand{\setp}[2]{\left\{#1\ \middle|\ #2\right\}} %for set with a property
\newcommand{\card}[1]{\#\,{#1}} %for cardinality of a set
\newcommand\m[1]{\begin{pmatrix}#1\end{pmatrix}} 

%Re-defined notations
\renewcommand{\epsilon}{\varepsilon}
\renewcommand{\phi}{\varphi}
\renewcommand{\emptyset}{\varnothing}
\renewcommand{\geq}{\geqslant}
\renewcommand{\leq}{\leqslant}
\renewcommand{\Re}{\operatorname{Re}}
\renewcommand{\Im}{\operatorname{Im}}
%----------------------------

\newcommand{\lrp}[1]{\left(#1\right)}
\newcommand{\lrb}[1]{\left[#1\right]}
\newcommand{\lrc}[1]{\left\{#1\right\}}

\newcommand{\tcr}[1]{\textcolor{red}{#1}}
\title{P-adic Numbers}
\author{Kevin Guillen}
\institute{UC Santa Cruz}
\date{March 10th 2022}

\begin{document}
    \frame{\titlepage}

    \begin{frame}
        \frametitle{Number Systems}
        A review of what we know:
        \begin{itemize}
            \item<1-> $\nn = \lrc{1, 2, \dots }$
            \item<2-> $\zz = \lrc{-2, -1, 0, 1, 2}$ through $-$
            \item<3-> $\qq = \lrc{\dots, -\frac{4}{7}, \frac{1}{2}, \dots}$ through $/$
        \end{itemize}
    \end{frame}

    \begin{frame}
        \frametitle{Number Systems}
        What now? We need new machinery. 
        Bring in limits of sequences $(x_n)$ where $x_n \in \qq$ for all $n$. 
        \\
        \begin{itemize}
            \item<1-> $\forall \epsilon > 0$
            \item<2-> $\exists N$
            \item<3-> $\forall n > N$
            \item<4-> $\abs{x_n - x} < \epsilon$
        \end{itemize} 
    \end{frame}

    \begin{frame}
        \frametitle{Number Systems}
        This leads to the idea of Cauchy Sequences.\pause 
        
        Which is when terms in a sequence become arbitrarily close to one another and in a sense settle on something. $\lrb{4}$ \pause
        
        When we consider a sequence like:
        \[(x_n) = \lrp{1 + \dfrac{1}{n}}^{n}\] \pause

        Settles on something that is not in $\qq$\pause,
        it goes to $e$. 

        Which then gives us $\rr$.
    \end{frame}

    \begin{frame}
        \frametitle{Distance (Aboslute Value)}
        Limit was dependent on $\abs{\cdot}$, which is our measure of distance. \pause

        An \textbf{absolute value} on a field $\ff$ is a function $\abs{\cdot}$ from $\ff$ to $\rr_{\geq 0}$ that satisfies the following properties for all $a,b \in \ff$:
        \begin{itemize}
        \item[(1)]Positive-definiteness: $\abs{a} = 0 \Longleftrightarrow a = 0$  
        \item[(2)]Multiplicativity: $\abs{ab} = \abs{a} \abs{b}$ 
        \item[(3)]Triangle Inequality: $\abs{a + b} \leq \abs{a} + \abs{y}$
        \end{itemize} 
    \end{frame}

    \begin{frame}
        \frametitle{Metric/Cauchy Space}
        A \textbf{Metric Space} is a non-empty set equipped with a metric (distance function).$\lrb{5}$ \pause

        Previous examples our metric space was $(\qq, \abs{\cdot})$. \pause 

        A metric space is \textbf{Complete} if every Cauchy Sequence's limit point is in metric space, also called a \textbf{Cauchy Space}. $\lrb{5}$ \pause 

        $\rr$ completes $\qq$.
    \end{frame}
    
    \begin{frame}
        \frametitle{$P-$adic valuation}
            Defined by $v_p: \qq \to \zz \cup \lrc{\infty}$ \pause

            $x \in \zz$ then $v_p(x)$ is the unique positive integer satisfying,
            \[x = p^{v_p(x)} x' \ \text{where }p \nmid x'\] \pause

            EX: $x = 90$ and $p = 3$ the $3-$adic valuation of $x$ to be $2$ since, 
            \[90 = 3^{2}\cdot 10\]

            
    \end{frame}

    \begin{frame}
        \frametitle{$P-$adic valuation}
        $x \in \qq - \zz$ then $v_p(x)$, we have $x = \dfrac{a}{b}$ and $a,b \in \zz$. So we define,
            \[v_p(x) = v_p(a) - v_p(b)\] \pause

        EX $x = \dfrac{13}{24}$ and $p = 2$ we have then that $a = 13$ and $b = 24$ \pause
        \[v_2(13) = 0 \ v_2(24) = 3\] \pause
        therefore $v_2(x) = 0 -3 = -3$ \pause

        Finally $v_p(0) = +\infty$
    
        
    
    \end{frame}

    \begin{frame}
        \frametitle{$P-$adic absolute value}
        A function $\abs{\cdot}_p: \qq \to \rr_{\geq 0}$ defined as,
        \[\abs{x}_p = \begin{cases}
            \dfrac{1}{p^{v_p(x)}} & x \neq 0 \\
            0 & x = 0
        \end{cases}\] \pause

        inducing the $p-$adic metric $d_p$
    \end{frame}

    \begin{frame}
        \frametitle{$\qq_p$}
        Field of $p-$adic numbers, $\qq_p$. \pause

        Completion of $\qq$ with respect to the $p-adic$ metric. \pause

        
    \end{frame}

    \begin{frame}
        \frametitle{Examples ($\zz$ and $\zz_p$)}
    \end{frame}

    \begin{frame}
        \frametitle{What is a $p-$adic number?}
        An example of a $p-adic$ number that is not  already a rational/integer number is $\dots$ \pause difficult to show.  \pause

        $p-$adic expansion of an integer is simply that number written in base $p$ \pause
        
        $p-$adic expansion of a rational number is the formal power series, $r \in \qq$,
        \[r = \sum_{i=k}^{\infty}a_ip^{i}\]

    \end{frame}

    \begin{frame}
        \frametitle{Wrangling with $\infty$}
        Closeness is measured by the right expansion\pause
        \begin{align*}
            2.71828182\dots \\
            2.71827071\dots
        \end{align*} \pause

        Difference is $0.0001111\dots$ \pause

        Divided by $10^{-4}$ is $1.111\dots$ \pause

        Note that $10^{-4} = \dfrac{1}{10^{4}}$


        
    
    \end{frame}
    \begin{frame}
        \frametitle{Wrangling with $\infty$}
        Consider working in the $3-adic$ system we have,
        \begin{align*}
            \dots 21120 \\
            \dots 10120 
        \end{align*}    
        where this number continues expanding to the left (infinite). \pause
        
        The difference will be $\dots 11000$ \pause,
        \[\dots \tcr{1}\cdot 3^{4} + \tcr{1} \cdot 3^{3} + \tcr{0}\cdot 3^{2} + \tcr{0}\cdot 3^{1} + \tcr{0}\cdot 3^{0}\] \pause
        \[(\dots \tcr{1}\cdot 3^{1} + 1)3^{3}\]
    
    \end{frame}
    \begin{frame}   
        \frametitle{Number Line?}


    \end{frame}

    \begin{frame}
        \frametitle{References}
        \begin{thebibliography}{9}
            \bibitem{texbook}
            Fernando Gouvea (2003) \emph{p-adic Numbers: An Introduction}, Springer Science \& Business Media, 2003.
            
            \bibitem{texbook}
            Alain M. Robert (2000) \emph{A Course in p-adic Analysis}, Springer; 2000th edition
        
            \bibitem{texbook}
            Svetlana Katok (2007) \emph{P-adic Analysis Compared With Real (Student Mathematical Library)}, American Mathematical Society
            
            \bibitem{texbook}
            Terence Tao (2016) \emph{Analysis I: Third Edition}, Hindustan Book Agency, 1st ed. 2016 edition
        
            \bibitem{texbook}
            James Munkres (2000) \emph{Topology}, Pearson College Div; 2nd edition (January 7, 2000)
          \end{thebibliography}
        
    
    \end{frame}
\end{document}