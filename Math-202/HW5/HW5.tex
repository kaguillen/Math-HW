\documentclass[11pt]{article}
 
\usepackage[top=0.75in, bottom=1.25in, left=1.5in, right=1.5in]{geometry} 
\usepackage{amsmath,amsthm,amssymb} %this is THE math package
\usepackage{mathtools}
\usepackage{tikz}
\usepackage{graphicx}
\usepackage{fancybox}
\usepackage{enumitem}
\usepackage{hyperref}
\usepackage{varwidth}
\usepackage{mdframed}
\usepackage{mathrsfs}
\usepackage[most]{tcolorbox}
%------------------------
%Fonts I use, uncomment if you like to use them.
%The first is the general font, and the second a math font
\usepackage{mathpazo}
\usepackage{eulervm}
%------------------------
%This is so that we have standard fonts for the double-stroked symbols
%for reals, naturals etc. regardless of what font you use.
%Don't comment
\AtBeginDocument{
  \DeclareSymbolFont{AMSb}{U}{msb}{m}{n}
  \DeclareSymbolFontAlphabet{\mathbb}{AMSb}}
%------------------------

%----------------------------------------------
%User-defined environments
%Commented because we're not using them in this document
%The only uncommented ones are the Problem and Solution environment

% \newenvironment{theorem}[2][Theorem]{\begin{trivlist}
% \item[\hskip \labelsep {\bfseries #1}\hskip \labelsep {\bfseries #2.}]}{\end{trivlist}}
% \newenvironment{lemma}[2][Lemma]{\begin{trivlist}
% \item[\hskip \labelsep {\bfseries #1}\hskip \labelsep {\bfseries #2.}]}{\end{trivlist}}
% \newenvironment{exercise}[2][Exercise]{\begin{trivlist}
% \item[\hskip \labelsep {\bfseries #1}\hskip \labelsep {\bfseries #2.}]}{\end{trivlist}}
% \newenvironment{question}[2][Question]{\begin{trivlist}
% \item[\hskip \labelsep {\bfseries #1}\hskip \labelsep {\bfseries #2.}]}{\end{trivlist}}
% \newenvironment{corollary}[2][Corollary]{\begin{trivlist}
% \item[\hskip \labelsep {\bfseries #1}\hskip \labelsep {\bfseries #2.}]}{\end{trivlist}}
\newenvironment{problem}[2][Problem\!]{\begin{tcolorbox}\begin{trivlist}
\item[\hskip \labelsep {\bfseries #1}\hskip \labelsep {\bfseries #2}]}{\end{trivlist}\end{tcolorbox}}
%\newenvironment{sub-problem}[2][]{\begin{trivlist}
%\item[\hskip \labelsep {\bfseries #1}\hskip \labelsep {\bfseries #2}]}{\end{trivlist}}
\newenvironment{solution}{\begin{proof}[\textbf{\textit{Solution}}] }{\end{proof}}
%----------------------------------------------

%----------------------------
%User-defined notations
\newcommand{\zz}{\mathbb Z}   %blackboard bold Z
\newcommand{\qq}{\mathbb Q}   %blackboard bold Q
\newcommand{\ff}{\mathbb F}   %blackboard bold F
\newcommand{\rr}{\mathbb R}   %blackboard bold R
\newcommand{\nn}{\mathbb N}   %blackboard bold N
\newcommand{\cc}{\mathbb C}   %blackboard bold C
\newcommand{\af}{\mathbb A}   %blackboard bold A
\newcommand{\pp}{\mathbb P}   %blackboard bold P
\newcommand{\id}{\operatorname{id}} %for identity map
\newcommand{\im}{\operatorname{im}} %for image of a function
\newcommand{\dom}{\operatorname{dom}} %for domain of a function
\newcommand{\cat}[1]{\mathscr{#1}}   %calligraphic category
\newcommand{\abs}[1]{\left\lvert#1\right\rvert} %for absolute value
\newcommand{\norm}[1]{\left\lVert#1\right\rVert} %for norm
\newcommand{\modar}[1]{\text{ mod }{#1}} %for modular arithmetic
\newcommand{\set}[1]{\left\{#1\right\}} %for set
\newcommand{\setp}[2]{\left\{#1\ \middle|\ #2\right\}} %for set with a property
\newcommand{\card}[1]{\#\,{#1}} %for cardinality of a set
\newcommand\m[1]{\begin{pmatrix}#1\end{pmatrix}} 

%Re-defined notations
\renewcommand{\epsilon}{\varepsilon}
\renewcommand{\phi}{\varphi}
\renewcommand{\emptyset}{\varnothing}
\renewcommand{\geq}{\geqslant}
\renewcommand{\leq}{\leqslant}
\renewcommand{\Re}{\operatorname{Re}}
\renewcommand{\Im}{\operatorname{Im}}
%----------------------------

\allowdisplaybreaks

\newcommand{\tcr}[1]{\textcolor{red}{#1}}
\newcommand{\tcb}[1]{\textcolor{blue}{#1}}
\newcommand{\tco}[1]{\textcolor{orange}{#1}}

\newcommand{\lrp}[1]{\left(#1\right)}
\newcommand{\lrb}[1]{\left[#1\right]}
\newcommand{\lrc}[1]{\left\{#1\right\}}
\newcommand{\lrw}[1]{\left<#1\right>}
 
 
\begin{document}
 
\title{Homework 5}
\author{Kevin Guillen\\[0.5em]
MATH 202  | Algebra III | Spring 2022}
\date{} 
\maketitle

%Use \[...\] instead of $$...$$

\begin{problem}{14.2.7}
    Determine all the subfields of the splitting field of $x^{8} - 2$ which are Galois over $\qq$.
\end{problem}
\begin{proof}
    Let $K$ refer to the splitting field of the given polynomial. We know by the Fundamental Theorem of Galois Theory that for every subfield $E$ of $K$, there is a corresponding subgroup $H$ of the Galois group which fixes $E$. Which also gives us that $E$ will be Galois over $\qq$ if and only if $H$ is normal. To assist with this problem we can look at the last 2 lattice diagrams given at the end of this section chapter. 

    We know the subgroups of index 2 are normal and that the normality of subgroups of order 2 can be easily checked,
    \[\sigma \left<\tau\sigma^{2k} \right>\sigma^{-1} = \left<\tau\sigma^{2k + 2}\right> \neq \left<\tau\sigma^{2k} \right>\]
    meaning that these subgroups are not normal, but $\left<\sigma^{4}\right> $ is the center so it is normal. 

    Working with the subgroups of order $4$ and conjugating them all by $\sigma$,
    \begin{align*}
        \sigma \left<\tau \sigma^{3}\right>\sigma^{-1} &= \sigma \left<\tau \sigma\right>\sigma^{-1} \\
        \sigma\left<\tau \sigma\right>\sigma^{-1} &= \sigma\left<\tau \sigma^{3}\right>\sigma^{-1} \\
        \sigma\left<\sigma^{4}, \tau\right>\sigma^{-1} &= \left< \sigma^{4}, \tau \sigma^{2} \right>\\
        \sigma\left<\sigma^{4}, \tau\sigma^{2}\right>\sigma^{-1} &=\left<\sigma^{4},\tau \right> 
    \end{align*}
    we see that none of them are normal. Giving us that the only subgroup of order 4 that is invariant to conjugation by $\sigma$ is $\left<\sigma^{2} \right>$. Also note that $\tau \sigma^{2}\tau = \sigma^{6} \in \left< \sigma^{2}\right>$, therefore $\left< \sigma^{2}\right>$ is normal. 

    So in all we have the normal subgroups of the Galois group to be \[Gal(K/F), 1, \lrw{\sigma^{2}}, \lrw{\sigma^{4}}, \lrw{\sigma^{2}, \tau\sigma^{3}}, \lrw{\sigma^{2}, \tau}, \text{ and }\lrw{\sigma}.\]

    Using the lattice diagrams given to us at the end of the chapter we see the corresponding subfields of $K$ which are Galois over $\qq$ are precisely (on the left),
    \begin{align*}
        \qq &\longleftrightarrow Gal(K/F) \\
        K &\longleftrightarrow 1 \\
        \qq(i, \sqrt[4]{2}) &\longleftrightarrow \lrw{\sigma^{2}} \\
        \qq(i, \sqrt{2}) &\longleftrightarrow \lrw{\sigma^{4}} \\
        \qq(\sqrt{-2}) &\longleftrightarrow \lrw{\sigma^{2}, \tau\sigma^{3}} \\
        \qq(\sqrt{2}) &\longleftrightarrow \lrw{\sigma^{2}, \tau}\\
        \qq(i) &\longleftrightarrow \lrw{\sigma}.
    \end{align*}

\end{proof}

\vspace*{15pt}

\begin{problem}{14.2.8}
    Suppose $K$ is Galois extension of $F$ of degree $p^{n}$ for some prime $p$ and some $n\geq 1 $. Show there are Galois extensions of $F$ contained in $K$ of degrees $p$ and $p^{n-1}$.
\end{problem}
\begin{proof}
    Because $K$ is a Galois extension of $F$ of degree $p^{n}$ we know that its Galois group will be of order $p^{n}$. We know from Math 200 / Group Theory, that a group of order $p^{n}$ will have normal subgroups of order $p^{k}$ where $k = 0, \dots, n$. Using this we know then there are normal subgroups of the Galois group of order $p^{n-1}$ and $p$, and by the correspondence given to us from the Fundamental Theorem of Galois Theory, we have that there must exist corresponding subfields of $K$ which are Galois extensions of $F$ and of degree $p$ and $p^{n-1}$. 
\end{proof}

\vspace*{15pt}

\begin{problem}{14.2.13}
    Prove that if the Galois group of the splitting field of a cubic over $\qq$ is the cyclic group of order 3 then all the roots of the cubic are real. 
\end{problem}
\begin{proof}
    Let $p(x) \in \qq[x]$ be of degree 3 (cubic) and for let us assume that is has a complex root. The Galois group has a subgroup generated by complex conjugation due to this complex root. This subgroup is $\zz/2\zz$ which isn't $\zz/3\zz$, meaning then that the Galois group of $p(x)$ can't be $\zz/3\zz$. Therefore if the Galois group is the cyclic group of order $3$ then all the roots of $p(x)$ must be real. 
\end{proof}

\vspace*{15pt}

\begin{problem}{14.3.3}
    Prove that an algebraically closed field must be infinite. 
\end{problem}
\begin{proof}
    Let $F$ be a finite field. Let us consider a polynomial $p(x) \in F[x]$ specifically,
    \begin{align*}
        p(x) = 1 + \prod_{\alpha \in F}(x - \alpha). 
    \end{align*}
    It is clear that this $p(x)$ is indeed in $F[x]$ since all its coefficients are in $F$. 
    We have by Ring Product with zero that $\prod_{\alpha \in F}(x -\alpha) = 0$ for all $x \in F$. Meaning that $p(x) = 1$ for all $x \in F$. Meaning $F$ is not algebraically closed. Therefore an algebraically closed field must be an infinite field. 

\end{proof}

\vspace*{15pt}

\begin{problem}{14.3.4}
    Construct the finite field of 16 elements and find a generator for the multiplicative group. How many generators are there?
\end{problem}
\begin{proof}
    Consider $\ff_2$. We want to find an irreducible polynomial of degree 4 in order to help construct this field of 16 elements. We know polynomials like these though must be factors of $x^{2^{4}} - x$, giving us $x, x-1, \text{ and } x^{2} + x + 1$ which are degree less than 4. Therefore the polynomial of degree 4 needed can't be divided by the linear polynomials above (0 and 1 are not roots), and must not be divisible by $x^{2} + x + 1$. It is clear a degree 4 polynomial meeting these requirements is \[p(x) = x^{4} + x^{3} + x^{2}+ x + 1\], this let's us then construct,
    \begin{align*}
        \ff_{16} \cong \ff_2[x]/(p(x))
    \end{align*}

    We see that $x$ unfortunately does not generate the multiplicative group of $\ff_{16}$ since,
    \[(x^{5} - 1) = (x+1)(x^{4} + x^{3} + x^{2} + x + 1) = 0\] giving us that $x^{5} = 1$. Let us consider $x+1$ though, we note that,
    \[(x+1)^{3} = x^{3} + x^{2}+ x + 1 = x^{4}\]
    therefore $\lrw{x} =\lrw{x^{4}} \subseteq \lrw{x + 1}$. Now note that $\lrw{x} = x^{k}$ for $k = 0,\dots, 4$, meaning $x+1 \notin \lrw{x}$, therefore $\lrw{x+1}$ is a strictly large subgroup than $\lrw{x}$, but the only bigger subgroup is $\ff_{16}^{\times}$. Giving us that $x+1$ is a generator of the multiplicative group. 

    From here we know all the other generators are simply $(x+1)^{k}$ where $k$ is relatively prime to $15$, this is given by the Euler's Totient function, $\phi(15) = 8$. Meaning we have $8$ generators in this multiplicative group. 
\end{proof}

\vspace*{15pt}

\begin{problem}{14.3.8}
    Determine the splitting field of the polynomial $x^{p} - x -a$ over $\ff_p$ where $a\neq 0 $, $a\in \ff_p$. Show explicitly that the Galois group is cyclic. [Show $\alpha \mapsto \alpha +1$ is an automorphism.] Such an extension is called an Artin-Schreier extension. 
\end{problem}
\begin{proof}
    Let $p(x) = x^{p} - x -a$, and let $\alpha$ be a root of $p(x)$. Notice now though,
    \begin{align*}
        p(\alpha + 1) &= (\alpha + 1)^{p} - (\alpha + 1) - a && \text{Apply Prop. 35 from 13.5} \\
        &=\alpha^{p } + 1^{p} - \alpha -1 -a  \\
        &= \alpha^{p} -\alpha - a \\
        &= 0.
    \end{align*}
    Therefore $\alpha +1$ is also a root. ($\star$) Meaning the $p$ roots of $p(x)$ are just $\alpha + k$ where $k = 1, \dots , p$. We also have then that $\alpha \notin \ff_p$, because if that were the case we would have $a= 0$ since $a = \alpha^{p} -\alpha = 0$, which would be a contradiction. So we have that $\ff_p(\alpha)$ is the splitting field of the separable polynomial $p(x)$ (we know this from $\star$) over $\ff_p$, thus $\ff_p(\alpha)/\ff_p$ is a Galois extension. 

    Now let us consider
    \begin{align*}
        \sigma: \ff_p(\alpha) &\to \ff_p(\alpha) \\
        \alpha &\mapsto \alpha + 1
    \end{align*}
    which fixes $\ff_p$. Note that $\sigma$ has a two sided inverse defined by $\alpha \mapsto \alpha - 1$ which also fixed $\ff_p$ meaning then that $\sigma\in Gal(\ff_p(\alpha)/\ff_p)$. 

    We know by properties of the Galois group, that any other map $\pi$ in \\$ Gal(\ff_p(\alpha)/ \ff_p)$ must not only fix $\ff_p$, but it must send $\alpha$ to a root of $p(x)$, meaning $\pi$ is of the form $\alpha \mapsto \alpha +k$ for some $k \in \ff_p$ (which we showed to be the roots from $\star$). Notice though we if we take the powers of $\sigma$ that, $\sigma^{k}(\alpha) = \alpha + k = \pi(\alpha)$ and fixes $\ff_p$, therefore $\sigma^{k} = \pi$. Meaning every element of the Galois group is simply a power of $\sigma$, and because $\sigma^{p} = 1$ we have that the Galois group is cyclic.  

\end{proof}

\vspace*{15pt}

\begin{problem}{14.3.11}
    Prove that $(x^{p})^{n} - x + 1$ is irreducible over $\ff_p$, only when $n = 1 = p = 2$. [Note that if $\alpha$ is a root, then so is $\alpha + \alpha $ for any $\alpha \in \ff_{p^{n}}$. Show that this implies $\ff_p(\alpha )$ contains $\ff_{p^{n}}$ and that $[\ff_p(\alpha): \ff_{p^{n}}] = p$]
\end{problem}
\begin{proof}
    Let $p(x) = (x^{p})^{n} - x + 1$. If $\alpha$ is a root of $p(x)$ then so is $\alpha + k$ for any $k \in \ff_p$. Therefore $\ff_{p^{n}} \subseteq \ff_p(\alpha)$ because if not, the latter would contain all the roots of $x^{p^{n}} - x + 1$, which would lead to a contradiction. Now any automorphism of the Galois group of $\ff_p(\alpha)/\ff_{p^{n}}$ must be defined by $\alpha \mapsto \alpha + k$ for some $k \in \ff_{p^{n}}$, and because they fix $\ff_{p^{n}}$ they are all of order $p$. Because a Galois group of degree $d$ is always cyclic over $\ff_p$, with a generator $\sigma_p$, and thus cyclic over $\ff_{p^{n}}$, we have $[\ff_p(\alpha): \ff_{p^{n}}] = p.$

    So we must have that $p^{n} = pn$. Now we write $n = p^{k}$ for $k \geq 0$. If we have that $k > 1$, then we have $k + 1 = p^{k}$, but $p\geq 2$, so $k + 1 < 2^{k} \leq p^{k}$ when $k = 2$ and through induction we have that,
    \[(k + 1) +1 < 2^{k} + 1 \leq 2^{k} + (2^{k + 1} - 2^{k}) = 2^{k + 1} \leq p^{k + 1}\]
    $k$ must be 0 and $n = 1$, or $k =1$ and $n = p$ with $p^{2} = p^{n}$ and $n = p = 2$. Therefore $x^{p} - x + 1$ is irreducible and we see that $x^{4}- x + 1 \in \ff_2[x]$ has no roots and is not divisible the only irreducible quadratic over $\ff_2[x]$, $x^{2} + x + 1$.
\end{proof}

\newpage

\begin{problem}{14.4.1}
    Determine the Galois closure of the field $\qq(\sqrt{1 + \sqrt{2}})$ over $\qq$.
\end{problem}
\begin{proof}
    The Galois closure is simply the splitting field for the minimal polynomial of $\sqrt{1 + \sqrt{2}}$ over $\qq$. We see that the minimal polynomial is just $p(x) = (x^{2} - 1)^{2} -2 = x^{4} -2x^{2} -1$, which has roots,
    \begin{align*}
        \alpha = \pm i \sqrt{-1 + \sqrt{2}}&& \alpha = \pm\sqrt{1 + \sqrt{2}}
    \end{align*}
    the splitting field is therefore $\qq(i, \sqrt{1 + \sqrt{2}})$, which is then the Galois closure. 
\end{proof}

\vspace*{15pt}

\begin{problem}{14.4.2}
    Find a primitive generator for $\qq(\sqrt{2}, \sqrt{3}, \sqrt{5})$ over $\qq$.
\end{problem}
\begin{proof}
    A member of the extension is $\alpha = \sqrt{2} + \sqrt{3} + \sqrt{5}$ so we have that $\qq(\alpha)$ is a subset of  $\qq(\sqrt{2}, \sqrt{3}, \sqrt{5})$. We note that $\alpha$ is not fixed by any of the nontrivial Galois automorphisms of $\qq(\sqrt{2}, \sqrt{3}, \sqrt{5})$ so $\qq(\sqrt{2}, \sqrt{3}, \sqrt{5})$ is also a subset of $\qq(\alpha)$. Together we have containment in both directions so, $\alpha$ is a primitive generator of $\qq(\sqrt{2}, \sqrt{3}, \sqrt{5})$.
\end{proof}

\vspace*{15pt}

\begin{problem}{14.4.6}
    Prove that $\ff_p(x,y)/\ff_p(x^{p},y^{p})$ is not a simple extension by explicitly exhibiting an infinite number of intermediate subfields. 
\end{problem}
\begin{proof}
    We see that $x$ is a root of $A^{p} - x^{p}$ in $\ff_p(x^{p},y^{p})[A]$ which is a polynomial of degree $p$ in $A$. We have a similar property for $y$ in that it is a root of $A^{p} - y^{p} $ in $\ff_p(x^{p}, y^{p})[A]$. As indeterminants we have $\ff_p(x) \cap \ff_p(y) = \ff_p$, and therefore $[\ff_p(x,y) : \ff_p(x^{p}, y^{p})] = p^{2}$.

    Now our goal from here is to show that for every $\delta \in \ff_(x^{p})$, we have the fields $\ff_p(x^{p}, y^{p})(x + \delta y) = \ff(x^{p},y^{p}, x + \delta y)$ to be distinct. Now take $\delta$, $\gamma$ from $\ff_p(x^{p})$ where $\delta \neq \gamma$, and suppose $\ff(x^{p}, y^{p}, x + \delta y) = \ff(x^{p}, y^{p}, x + \gamma y)$. We have then that 
    \[(x + \delta y) - (x + \gamma y) = (\delta - \gamma)y\] 
    to be in $\ff(x^{p}, y^{p}, x + \delta y)$, which implies then that both $x$ and $y$ are in our extension. Meaning that $\ff_p(x,y) = \ff_p(x^{p}, y^{p}, x + \delta y) $ is an extension with degree $p^{2}$ over $\ff_p(x^{p}, y^{p})$. Note though that $(x + \delta y)^{p} = x^{p} + \delta^{p} y^{p}$ (Prop 35 from 13.5), which comes as the solution of $x^{p} - (x^{p} + \delta^{p} y^{p})$ which is of degree $p$. This is a contradiction though since we just stated that this is degree $p^{2}$. All together then, since there is an infinite amount of elements in $\ff_p(x^{p})$ there is an infinite number of subfields of $\ff_p(x,y)$. 
\end{proof}

\vspace*{15pt}

\end{document}