\documentclass[11pt]{article}
 
\usepackage[top=0.75in, bottom=1.25in, left=1in, right=1in]{geometry} 
\usepackage{amsmath,amsthm,amssymb} %this is THE math package
\usepackage{mathtools}
\usepackage{tikz}
\usepackage{graphicx}
\usepackage{fancybox}
\usepackage{enumitem}
\usepackage{hyperref}
\usepackage{varwidth}
\usepackage{mdframed}
\usepackage{mathrsfs}
\usepackage[most]{tcolorbox}
%------------------------
%Fonts I use, uncomment if you like to use them.
%The first is the general font, and the second a math font
\usepackage{mathpazo}
\usepackage{eulervm}
%------------------------
%This is so that we have standard fonts for the double-stroked symbols
%for reals, naturals etc. regardless of what font you use.
%Don't comment
\AtBeginDocument{
  \DeclareSymbolFont{AMSb}{U}{msb}{m}{n}
  \DeclareSymbolFontAlphabet{\mathbb}{AMSb}}
%------------------------

%----------------------------------------------
%User-defined environments
%Commented because we're not using them in this document
%The only uncommented ones are the Problem and Solution environment

% \newenvironment{theorem}[2][Theorem]{\begin{trivlist}
% \item[\hskip \labelsep {\bfseries #1}\hskip \labelsep {\bfseries #2.}]}{\end{trivlist}}
% \newenvironment{lemma}[2][Lemma]{\begin{trivlist}
% \item[\hskip \labelsep {\bfseries #1}\hskip \labelsep {\bfseries #2.}]}{\end{trivlist}}
% \newenvironment{exercise}[2][Exercise]{\begin{trivlist}
% \item[\hskip \labelsep {\bfseries #1}\hskip \labelsep {\bfseries #2.}]}{\end{trivlist}}
% \newenvironment{question}[2][Question]{\begin{trivlist}
% \item[\hskip \labelsep {\bfseries #1}\hskip \labelsep {\bfseries #2.}]}{\end{trivlist}}
% \newenvironment{corollary}[2][Corollary]{\begin{trivlist}
% \item[\hskip \labelsep {\bfseries #1}\hskip \labelsep {\bfseries #2.}]}{\end{trivlist}}
\newenvironment{problem}[2][Problem\!]{\begin{tcolorbox}\begin{trivlist}
\item[\hskip \labelsep {\bfseries #1}\hskip \labelsep {\bfseries #2}]}{\end{trivlist}\end{tcolorbox}}
%\newenvironment{sub-problem}[2][]{\begin{trivlist}
%\item[\hskip \labelsep {\bfseries #1}\hskip \labelsep {\bfseries #2}]}{\end{trivlist}}
\newenvironment{solution}{\begin{proof}[\textbf{\textit{Solution}}] }{\end{proof}}
%----------------------------------------------

%----------------------------
%User-defined notations
\newcommand{\zz}{\mathbb Z}   %blackboard bold Z
\newcommand{\qq}{\mathbb Q}   %blackboard bold Q
\newcommand{\ff}{\mathbb F}   %blackboard bold F
\newcommand{\rr}{\mathbb R}   %blackboard bold R
\newcommand{\nn}{\mathbb N}   %blackboard bold N
\newcommand{\cc}{\mathbb C}   %blackboard bold C
\newcommand{\af}{\mathbb A}   %blackboard bold A
\newcommand{\pp}{\mathbb P}   %blackboard bold P
\newcommand{\id}{\operatorname{id}} %for identity map
\newcommand{\im}{\operatorname{im}} %for image of a function
\newcommand{\dom}{\operatorname{dom}} %for domain of a function
\newcommand{\cat}[1]{\mathscr{#1}}   %calligraphic category
\newcommand{\abs}[1]{\left\lvert#1\right\rvert} %for absolute value
\newcommand{\norm}[1]{\left\lVert#1\right\rVert} %for norm
\newcommand{\modar}[1]{\text{ mod }{#1}} %for modular arithmetic
\newcommand{\set}[1]{\left\{#1\right\}} %for set
\newcommand{\setp}[2]{\left\{#1\ \middle|\ #2\right\}} %for set with a property
\newcommand{\card}[1]{\#\,{#1}} %for cardinality of a set
\newcommand\m[1]{\begin{pmatrix}#1\end{pmatrix}} 

%Re-defined notations
\renewcommand{\epsilon}{\varepsilon}
\renewcommand{\phi}{\varphi}
\renewcommand{\emptyset}{\varnothing}
\renewcommand{\geq}{\geqslant}
\renewcommand{\leq}{\leqslant}
\renewcommand{\Re}{\operatorname{Re}}
\renewcommand{\Im}{\operatorname{Im}}
%----------------------------

\allowdisplaybreaks

\newcommand{\tcr}[1]{\textcolor{red}{#1}}
\newcommand{\tcb}[1]{\textcolor{blue}{#1}}
\newcommand{\tco}[1]{\textcolor{orange}{#1}}

\newcommand{\lrp}[1]{\left(#1\right)}
\newcommand{\lrb}[1]{\left[#1\right]}
\newcommand{\lrc}[1]{\left\{#1\right\}}
\newcommand{\lrw}[1]{\left<#1\right>}
 
 
\begin{document}
 
\title{Homework 6}
\author{Kevin Guillen\\[0.5em]
MATH 202 | Algebra III | Spring 2022}
\date{} 
\maketitle

%Use \[...\] instead of $$...$$

\begin{problem}{14.2.17}
    Let $K/F$ be any finite extension and let $\alpha \in K$. Let $L$ be a Galois extension of $F$ containing $K$ and let $H \leq Gal(L/F)$ be the subgroup corresponding to $K$. Define the norm of $\alpha$ from $K$ to $F$ be
    \[N_{K/F}(\alpha) = \prod_{\sigma}\sigma(\alpha)\]
    where the product is taken over all the embeddings of $K$ into an algebraic closure of $F$ (so over a set of coset representatives for $H$ in $Gal(L/F)$ by the Fundamental Theorem of Galois Theory). This is a product of Galois conjugates of $\alpha$. In particular, if $K/F$ is Galois this is $\prod_{\sigma \in Gal(K/F)}\sigma(\alpha)$. 
\end{problem}
\begin{proof}
    We see that the product of this norm is well defined since $K$ is the fixed field of $H$, and the elements of a coset $\sigma H \subset Gal(L/F)$ all correspond to the same embedding of $\sigma$. This means then that if $I$ and $J$ were to be two sets of coset representatives of $H$,
    \[\prod_{\sigma \in J}\sigma(\alpha) = \prod_{\sigma\in J}\sigma(\alpha).\]

    Next, if $J$ is a set of coset representatives for $H$, we see that for any $\pi \in Gal(L/F)$ that $\pi J$ is also a complete set of representatives, which we will refer to as $M$. Meaning then that,
    \begin{align*}
        \pi N_{K/F}(\alpha) &= \pi \prod_{\sigma\in  J}\sigma(\alpha) \\
        &= \prod_{\sigma \in J}\pi \sigma(\alpha) \\
        &= \prod_{\sigma \in M}\sigma(\alpha) \\
        &= N_{K/F}(\alpha).
    \end{align*}
    Showing us that $N_{K/F}(\alpha)$ lies in $F$, since it is fixed by $Gal(L/F)$. 

    We see through the following that the norm is multiplicative, let $\alpha, \beta \in K$,
    \begin{align*}
        N_{K/F}(\alpha \beta) &= \prod_{\sigma}\sigma(\alpha\beta) \\
        &= \prod_\sigma \sigma(\alpha)\sigma(\beta) \\
        &= \prod_\sigma \sigma(\alpha)\prod_\sigma \sigma(\beta) = N_{K/F}(\alpha)N_{K/F}(\beta). 
    \end{align*}

    Now if $K = F(\sqrt{D})$ is a quadratic extension of $F$, then we'd have that $K/F$ is Galois. In this scenario the only non-identity element of $Gal(K/F)$ is the map $\sqrt{D} \mapsto -\sqrt{D}$, and therefore ($\alpha \in K$),
    \begin{align*}
        N_{K/F}(\alpha) &= N_{K/F}(a +b \sqrt{D}) && a,b \in F \\ 
        &= (a+b\sqrt{D})(a-b\sqrt{D}) \\
        &= a^{2} - Db^{2}
    \end{align*}

    Let $d = [F(\alpha): F]$ and $n = [K:F]$, then it is clear that $d\mid n$ since $F\subseteq F(\alpha) \subseteq K$. We have $F\subseteq K \subseteq L$ and since $L$ is Galois over $F$, we have $L$ is separable over $F$, therefore $K$ must also be separable over $F$. Recall that the roots of the minimal polynomials must precisely be the Galois conjugates of $\alpha$, and $m_\alpha$ doesn't have multiple roots ($m_\alpha$ being the minimal polynomial). We know there must $d$ of them since $\deg(m_\alpha) = d$. We also have that there are $n$ embeddings of $K$ into an algebraic closure of $F$, and that each of these embeddings sends $\alpha$ to a Galois conjugate, therefore each conjugate appears $n/d$ times in the product of the norm. Let $\set{\alpha,\dots ,\alpha_d}$ be the roots of $m_\alpha$ then we have,
    \begin{align*}
        N_{K/F} (\alpha) = \prod_\sigma \sigma(\alpha) = \lrp{\prod_{i = 1}^{d}\alpha_i}^{n/d}.
    \end{align*} 
    Consider that $a_0 = (-1)^{d}\prod_{i = 1}^{d}\alpha_i$ we have,
    \[N_{K/F}(\alpha) = (-1)^{n}\alpha_0^{n/d}\]
    as desired. 

\end{proof}

\vspace*{15pt}

\begin{problem}{14.5.5}
    Let $p$ be a prime and let $\epsilon_1, \epsilon_2, \dots , \epsilon_{p-1}$ denote the primitive $p^{th}$ roots of unity. Set $p_n = \epsilon_1^{n} + \epsilon_2^{n} + \dots + \epsilon_{p-1}^{n}$, the sum of the $n^{th}$ powers of the $\epsilon_i$. Prove that $p_n = -1$ if $p$ does not divide $n$ and that $p_n = p-1$ if $p$ does not divide $n$. [One approach: $p_1 = -1$ from $\phi_{p}(x)$; show that $p_n$ is a Galois conjugate of $p_1$ for $p$ not dividing $n$, hence is also $-1$.]
\end{problem}
\begin{proof}
    Because $\phi_p = x^{p-1} + x^{p-2} + \dots + 1$ we have $\phi(\zeta_p) = 0 = p_1 + 1 \implies p_1 = -1$. Recall though that the elements of the Cyclotomic Galois group are defined by $\sigma_a(\zeta_p) = \zeta_p^{a}$ where $p \nmid a$, therefore we have $\sigma_a(p_1) = p_a$ and so for $p\nmid a$ we have that $p_a = -1$.

    In the case that $p\mid a$ we have $\epsilon_i^{a} = (\epsilon_i^{p})^{m} = 1^{m} = 1 \implies p_a = p-1$. 
\end{proof}

\vspace*{15pt}

\begin{problem}{14.5.10}
    Prove that $\qq(\sqrt[3]{2})$ is not a subfield of any cyclotomic field over $\qq$. 
\end{problem}
\begin{proof}
    We know from the text that the Cyclotomic fields $\qq(\zeta_n)$ are Galois extensions of $\qq$ with abelian Galois groups. If $\qq(\zeta_n)$ were to contain $\qq(\sqrt[3]{2})$ it would have to contain its Galois closure over $\qq$, which is the splitting field of $x^{3} -2$, but that is an extension with Galois group isomorphic to $S_3$. Therefore by the Fundamental Theorem of Galois Theory, this would imply that the abelian group $Gal(\qq(\zeta_n)/\qq)$ contains a subgroup isomorphic to $S_3$, which is a contradiction!
\end{proof}

\vspace*{15pt}

\begin{problem}{14.5.11}
    Prove that the primitive $n^{th}$ roots of unity form a basis over $\qq$ for the cyclotomic field of $n^{th}$ roots of unity if and only if $n$ is squarefree (i.e., $n$ is not divisible by the square of any prime). 
\end{problem}
\begin{proof}
    Let $p$ be a prime, and suppose that $p^{2}\mid n$. We have then that $\zeta_n^{n/p}$ is a primitive $p^{th}$ root of unity. Which gives us,
    \begin{align*}
        \sum_{i = 0}^{p -1}\zeta_n\zeta_n^{n i/p} = \zeta_n\lrp{\sum_{i = 0}^{p-1}\zeta_n^{n i/p}} = \zeta_n 0 = 0
    \end{align*}
    and that $\zeta_n\zeta_n^{n i/p} = \zeta_n^{1 + n i/p}$ are primitive $n^{th}$ roots of unity for all $0 \leq i < p$ since the prime factors of $n$ are factors of $n/p$.  Therefor there are linear dependencies over $\qq$ between the primitive $n^{th}$ roots of unity, so they can't form a basis. 

    Now suppose the conclusion hold for product of less than $x$ primes and let $n = mp$ for prime $p$, and $m$ the product of $x-1$ distinct primes. By induction $\set{\zeta_p^{i}\mid 1 \leq i \leq p, \ (i,p) =1}$ is a basis of $\qq(\zeta_p)$ and $\set{\zeta_m^{j} \mid 1 \leq j \leq m, (j,m) =1}$ is a basis of $\qq(\zeta_m)$. Because $\qq(\zeta_p)\cap \qq(\zeta_m) = \qq$ we have that a basis $\qq(\zeta_m)\qq(\zeta_p) = \qq(\zeta_m, \zeta_p) = \qq(\zeta_n)$ which is \[\set{\zeta_p^{i}\zeta_m^{j} \mid 1 \leq i \leq p, 1 \leq j \leq m, (j,m) =1, (i,p) = 1}.\]
    Then by taking mod $m$ and mod $p$ of $mi + pj$ we have that the exponents of $mi + pj$ are relatively prime top $n$ so this basis consist of primitive $n^{th}$ roots of unity. Taking the mods we can again see all these exponents are distinct, so that there are $\phi(p)\phi(m) = \phi(n)$ elements in this basis, meaning it is composed of all the primitive $n^{th}$ roots of unity.   
\end{proof}

\vspace*{15pt}

\begin{problem}{14.5.12}
    Let $\sigma_p$ denote the Frobenius automorphism $x\mapsto x^{p}$ of the finite field $\ff_q$ of $q = p^{n}$ elements. Viewing $\ff_q$ as a vector space $V$ of dimension $n$ over $\ff_p$ we can consider $\sigma_p$ as a linear transformation $\sigma_p$ is diagonalizable over $\ff_p$ if and only if $n$ divides $p-1$, and is diagonalizable over the algebraic closure of $\ff_p$ if and only if $(n,p) = 1$.
\end{problem}
\begin{proof}
    Since for all $x\in \ff_{p^{n}}$, we have $x^{p^{n}} -x = 0$ we have that $\sigma_p$ satisfies $x^{n} --1$. Since this is a degree $n$ polynomials it is the characteristic polynomial. Now recall that $\sigma_p$ is diagonalizable if and only if the characteristic polynomial splits completely in $\ff_p$. 

    Now we observe that $\sigma_p$ is diagonalizable if and only if $\ff_p$ contains all the $n^{th}$ roots of unity, if and only if $\ff_p^{\times}$ contains a copy of $\zz/n\zz$. We have then by the Fundamental Theorem of Cyclic Groups this is the case if and only if $n\mid (p-1)$. 

    The linear transformation is diagonalizable over the closure of $\ff_p$ if and only if $x^{n}- 1$ is separable. This is true if and only if it is relatively prime to its derivative $nx^{n-1}$, but the this is only true if and only if $nx^{n-1}\neq 0$ and this is true if and only if $p \nmid n$. 
\end{proof}

\vspace*{15pt}

\begin{problem}{14.6.2}
    Determine the Galois groups of the following polynomials
    \begin{itemize}
        \item[(a)] $x^{3} -x^{2} -4$
        \item[(b)] $x^{3} -2x + 4$
        \item[(c)] $x^{3} -x + 1$
        \item[(d)] $x^{3} + x^{2} -2x -1$
    \end{itemize}
\end{problem}
\begin{proof}
    We know from the textbook that a reducible cubic has trivial Galois group if it is factored as three linear components and has Galois group $\zz_2$ if it is factored as a cubic and a linear polynomial. An irreducible cubic polynomial has Galois group either $A_3$ or $S_3$ and it is $A_3$ if and only if the discriminant $D= a^{2}b^{2} -4b^{3} - 4a^{3}c -27c^{2} + 18abc$ is a square. 

    (a) We have that $x^{3} -x^{2} - 4 = (x^{2} + x + 2)(x-2)$ and applying the quadratic formula we see the quadratic has complex roots, so its Galois group is $\zz_2$.

    (b) We have $x^{3} -2x + 4 = (x^{2} -2x + 2)(x+2)$ and the quadratic polynomials has complex roots by the quadratic formula, so like before, its Galois group is $\zz_2$. 

    (c) The polynomial $x^{3} -x+1$ is irreducible in $\qq$. This is because for $a,b \in \zz$ where $b\neq 0$ and $(a,b) = 1$ and
    \[\dfrac{a^{3}}{b^{3}} -\dfrac{a}{b} + 1 = 0\implies a^{3} = (a-b)b^{2}\] meaning that $b^{2}\mid a^{3}$ which contradicts $(a,b) = 1$. We see the discriminant of $x^{3} - x + 1$ is $4-27 = -23$  which is not a square, so its Galois group is $S_3$. 

    (d) We have $x^{3} + x^{2} -2x -1$ to be irreducible in $\qq$ since as before if $(a,b) = 1$ we see if,
    \[\dfrac{a^{3}}{b^{3}} + \dfrac{a^{2}}{b^{2}} -\dfrac{2a}{b} -1 = 0\]
    this would imply $a^{3} = (-a^{2} + 2ab +b^{2})$, which means $b\mid a^{3}$ which goes against the assumption that $a$ and $b$ are relatively prime. 

    The discriminant of $x^{3} + x^{2} -2x -1$ is $4+32 +4 -27 +36 = 7^{2}$, which is a square, therefore its Galois group is $A_3$
\end{proof}

\vspace*{15pt}

\begin{problem}{14.6.5}
    Determine the Galois group of $x^{4} + 4$
\end{problem}
\begin{proof}
    Let $p(x) = x^{4} + 4$ we see that it can be factored into,
    \begin{align*}
        p(x) = x^{4} + 4 = (x^{2} -2x + 4)(x^{2} + 2x + 2)
    \end{align*}
    which shows us that the roots of $p(x)$ are $\pm 1, \pm i$. Meaning the splitting field is $\qq(i)$, which is of degree 2 over $\qq$. This gives us then that the Galois group of $p(x)$ is cyclic of order 2, which is $\zz_2$.
\end{proof}
\vspace*{15pt}

\begin{problem}{14.6.10}
    Determine the Galois group of $x^{5} + x -1$
\end{problem}
\begin{proof}
    Let $p(x) = x^{5} + x -1$. We see that $p(x)$ can be factored as,
    \begin{align*}
        p(x) = x^{5} + x -1 = (x^{3} + x^{2} - 1)(x^{2} - x + 1) 
    \end{align*}
    We see that the discriminant of $x^{2} -x + 1$ is $-3$ giving us that it is irreducible and its Galois group is simply $\zz_2$. Next we see that $x^{3} +x^{2 }-1$ is irreducible through mod 2, and its discriminant is $-23$ so its Galois group is $S_3$. 

    Now let $K$ and $E$ be the splitting field of $x^{2} - x + 1$ and $x^{3}+x^{2} -1$ respectively. Now suppose the intersection between $K$ and $E$ is non-trivial. Because $[K: \qq ] = 2$ and $[E : \qq] = 6$ the intersection begin non-trivial would imply $K < E$ and therefore $E$ is an extension of degree $3$ on $K$. This gives us that $Gal(E/K)$ is some subgroup of $Gal(E/\qq) \cong S_3$ of order 3, and there is a unique subgroup satisfying this, $A_3$. Giving us $Gal(E/K) \cong A_3$. This is only possible though if the discriminant of $x^{3} + x^{2} -1$ is a square in $K = \qq(i \sqrt{3})$.  

    Now let $a,b,c,d \in \zz$ and $b,d \neq 0$, suppose then that 
    \[\lrp{\dfrac{a}{b} + \dfrac{c}{d}i\sqrt{3}}^{2} = -23\] for some $\dfrac{a}{b}, \dfrac{c}{d} \in \qq(i\sqrt{3})$ in lowest terms. This would give us,
    \begin{align*}
        (ad)^{2} -3(cb)^{2} + abcd2i\sqrt{3} = -23
    \end{align*}
    so either $a = 0$ or $c = 0$, but both lead to contradiction. 

    Therefore we have that $K\cap L$ must be trivial, giving us the Galois group to be $\zz_2 \times S_3$. 
\end{proof}

\vspace*{15pt}

\begin{problem}{14.6.11}
    Let $F$ be an extension of $\qq$ of degree 4 that is not Galois over $\qq$. Prove that the Galois closure of $F$ has Galois group either $S_4$, $A_4$ or the dihedral group $D_8$ of order 8. Prove that the Galois group is dihedral if and only if $F$ contains a quadratic extension of $\qq$.
\end{problem}
\begin{proof}
    Say that $E/\qq = \overline{F}$. Now for some $\alpha\in F$ that is a root, we can say that $F = \qq(\alpha)$, and so $E$ is the splitting field of the minimal polynomial of $\alpha$. We know this polynomial is of degree 4, we know then that $G = Gal(E/\qq)$ is a subgroup of $S_4$. 

    Because $E$ has a subfield that is 4th degree in $\qq$, $G$ must have a subgroup of index 4. Since $F$ is given to not be Galois over $\qq$, we have that $\abs{G} > 4$. So we have then that the order of $G$ must be 8, 12, or 24.

    If we have the order to be $8$, we have $G = D_8$, the only group of order 8 that has a subgroup which is not normal and therefore corresponds to $F$. If the order were to be 24 we'd have $G$ to be $S_4$ itself. If the order were to be 12 it is just the only index 2 subgroup of $S_4, A_4$. 

    $F$ contains a quadratic extension of $\qq$ if and only if each index 4 subgroup of $G$ is contained in an index 2 subgroup. Notice though that $S_4$ and $A_4$ fail this, but each element of $D_8$ having order 2 is contained in a subgroup of order 4. 
\end{proof}

\vspace*{15pt}

\end{document}