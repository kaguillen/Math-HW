\documentclass[11pt]{article}
 
\usepackage[top=0.75in, bottom=1.25in, left=1.5in, right=1.5in]{geometry} 
\usepackage{amsmath,amsthm,amssymb} %this is THE math package
\usepackage{mathtools}
\usepackage{tikz}
\usepackage{graphicx}
\usepackage{fancybox}
\usepackage{hyperref}
\usepackage{varwidth}
\usepackage{mdframed}
\usepackage{mathrsfs}
\usepackage[most]{tcolorbox}
\usepackage{polynom}
%------------------------
%Fonts I use, uncomment if you like to use them.
%The first is the general font, and the second a math font
\usepackage{mathpazo}
\usepackage{eulervm}
%------------------------
%This is so that we have standard fonts for the double-stroked symbols
%for reals, naturals etc. regardless of what font you use.
%Don't comment
\AtBeginDocument{
  \DeclareSymbolFont{AMSb}{U}{msb}{m}{n}
  \DeclareSymbolFontAlphabet{\mathbb}{AMSb}}
%------------------------

%----------------------------------------------
%User-defined environments
%Commented because we're not using them in this document
%The only uncommented ones are the Problem and Solution environment

% \newenvironment{theorem}[2][Theorem]{\begin{trivlist}
% \item[\hskip \labelsep {\bfseries #1}\hskip \labelsep {\bfseries #2.}]}{\end{trivlist}}
% \newenvironment{lemma}[2][Lemma]{\begin{trivlist}
% \item[\hskip \labelsep {\bfseries #1}\hskip \labelsep {\bfseries #2.}]}{\end{trivlist}}
% \newenvironment{exercise}[2][Exercise]{\begin{trivlist}
% \item[\hskip \labelsep {\bfseries #1}\hskip \labelsep {\bfseries #2.}]}{\end{trivlist}}
% \newenvironment{question}[2][Question]{\begin{trivlist}
% \item[\hskip \labelsep {\bfseries #1}\hskip \labelsep {\bfseries #2.}]}{\end{trivlist}}
% \newenvironment{corollary}[2][Corollary]{\begin{trivlist}
% \item[\hskip \labelsep {\bfseries #1}\hskip \labelsep {\bfseries #2.}]}{\end{trivlist}}
\newenvironment{problem}[2][Problem\!]{\begin{tcolorbox}\begin{trivlist}
\item[\hskip \labelsep {\bfseries #1}\hskip \labelsep {\bfseries #2}]}{\end{trivlist}\end{tcolorbox}}
%\newenvironment{sub-problem}[2][]{\begin{trivlist}
%\item[\hskip \labelsep {\bfseries #1}\hskip \labelsep {\bfseries #2}]}{\end{trivlist}}
\newenvironment{solution}{\begin{proof}[\textbf{\textit{Solution}}] }{\end{proof}}
%----------------------------------------------

%----------------------------
%User-defined notations
\newcommand{\zz}{\mathbb Z}   %blackboard bold Z
\newcommand{\qq}{\mathbb Q}   %blackboard bold Q
\newcommand{\ff}{\mathbb F}   %blackboard bold F
\newcommand{\rr}{\mathbb R}   %blackboard bold R
\newcommand{\nn}{\mathbb N}   %blackboard bold N
\newcommand{\cc}{\mathbb C}   %blackboard bold C
\newcommand{\af}{\mathbb A}   %blackboard bold A
\newcommand{\pp}{\mathbb P}   %blackboard bold P
\newcommand{\id}{\operatorname{id}} %for identity map
\newcommand{\im}{\operatorname{im}} %for image of a function
\newcommand{\dom}{\operatorname{dom}} %for domain of a function
\newcommand{\cat}[1]{\mathscr{#1}}   %calligraphic category
\newcommand{\abs}[1]{\left\lvert#1\right\rvert} %for absolute value
\newcommand{\norm}[1]{\left\lVert#1\right\rVert} %for norm
\newcommand{\modar}[1]{\text{ mod }{#1}} %for modular arithmetic
\newcommand{\set}[1]{\left\{#1\right\}} %for set
\newcommand{\setp}[2]{\left\{#1\ \middle|\ #2\right\}} %for set with a property
\newcommand{\card}[1]{\#\,{#1}} %for cardinality of a set
\newcommand\m[1]{\begin{pmatrix}#1\end{pmatrix}} 
\newcommand{\lrp}[1]{\left(#1\right)}
\newcommand{\lrb}[1]{\left[#1\right]}
\newcommand{\lrc}[1]{\left\{#1\right\}}

%Re-defined notations
\renewcommand{\epsilon}{\varepsilon}
\renewcommand{\phi}{\varphi}
\renewcommand{\emptyset}{\varnothing}
\renewcommand{\geq}{\geqslant}
\renewcommand{\leq}{\leqslant}
\renewcommand{\Re}{\operatorname{Re}}
\renewcommand{\Im}{\operatorname{Im}}
%----------------------------

\allowdisplaybreaks
 
 
\begin{document}
 
\title{Homework 1}
\author{Kevin Guillen\\[0.5em]
MATH 202 | Algebra III | Spring 2022}
\date{} 
\maketitle

%Use \[...\] instead of $$...$$

\begin{problem} {13.1.1}
    Show that $p(x) = x^{3} + 9x + 6$ is irreducible in $\qq[x]$. Let $\theta$ be a root of $p(x)$. Find the inverse of $1+\theta$ in $\qq(\theta)$
\end{problem}
\begin{proof}
    First to show that $p(x)$ is indeed irreducible we will use Eisenstein's Irreducibility Criterion, which we learned in Math 200, to show it is irreducible over $\zz[x]$ and by the Gauss Lemma irreducible over $\qq[x]$. We can see that with $p = 3$ we have that 3 divides 6 and 9, 3 doesn't divide 1, and finally that $3^{2}$ doesn't divide 6. Meaning then that $p(x)$ is irreducible.

    Now if $\theta$ is a root of $p(x)$ to find the inverse of $1 + \theta$ we will first perform division with remainder on $p(x)$ by $(1 + x)$.

    \[ \polylongdiv{x^{3} + 9x + 6}{x + 1}\]
    To get that $p(x) = (x + 1)(x^{2 } - x + 10) - 4$. We were given that $\theta$ is a root of $p(x)$, so it must be that case then that,
    \[(1+\theta)(\theta^{2} - \theta + 10) = 4\]
    implying,
    \[(1 + \theta)^{-1} = \dfrac{(\theta^{2} - \theta + 10)}{4}\]
    as desired.

\end{proof}
\newpage
\begin{problem}{13.1.2}
    Show that $x^{3} -2x -2$ is irreducible over $\qq$ and let $\theta$ be a root. Compute $(1 + \theta)(1 + \theta + \theta^{2})$ and $\dfrac{1 + \theta}{1 + \theta + \theta^{2}}$ in $\qq(\theta)$.
\end{problem}
\begin{proof}
    Like the previous problem we will use Eisenstein's Irreducibility Criterion again and apply the Gauss Lemma, but in this case $p = 2$. We see that $2$ divides $-2$, $2$ doesn't divide $1$, and that $2^{2}$ doesn't divide $-2$. Therefore $x^{3} - 2x -2$ is irreducible. 

    Computing $(1 + \theta)(1 + \theta + \theta^{2})$ we get, 
    \begin{align}
        \theta^{3} + 2\theta^{2} + 2\theta + 1
    \end{align}
    Recall though $\theta$ being a root of $x^{3} - 2x -2$ means $\theta^{3} - 2\theta -2 = 0$ and therefore,
    \[\theta^{3} = 2\theta + 2\]
    so plugging back into (1) we get,
    \begin{align*}
         2\theta^{2} + 4\theta + 3 
    \end{align*}

    Now we compute $\dfrac{1 + \theta}{1+ \theta + \theta^{2}}$, so first we need to obtain $(1 + \theta + \theta^{2})^{-1}$ which we do by performing division with remainder on $(x^{3} -2x -2)$ by $(x^{2} + x + 1)$,
    \[\polylongdiv{x^{3} -2x -2}{ x^{2} + x + 1}\]
    continuing we get,
    \[\polylongdiv{x^{2} + x + 1}{-2x-1}\]
    Giving us,
    \begin{align*}
        x^{3} -2x -2 &= (x^{2} + x + 1)(x-1) + (-2x - 1) \\
        x^{2} + x + 1 &= (-2x -1)(\dfrac{1}{2}x - \dfrac{1}{4}) + \dfrac{3}{4}
    \end{align*}
    Solving for the remainder in both these equations we get,
    \begin{align}
        (-2x - 1) & = (x^{3} -2x -2) - (x^{2} + x + 1)(x-1) \\
        \dfrac{3}{4} &= (x^{2} + x + 1) -(-2x -1)(\dfrac{1}{2}x - \dfrac{1}{4}) 
    \end{align}
    Now we multiply equation (3) by $\dfrac{4}{3}$ and plug in equation (2) into it to get,
    \begin{align*}
        1 = \dfrac{4}{3}(x^{2} + x + 1) - \dfrac{4}{3}((x^{3} -2x -2) - (x^{2} + x + 1)(x-1))(\dfrac{1}{2}x - \dfrac{1}{4}) 
    \end{align*}
    which works out to be,
    \begin{align*}
        1 = (-\dfrac{2}{3}x^{2} + \dfrac{1}{3}x + \dfrac{5}{3})(x^{2} + x + 1) + (\dfrac{2}{3}x + \dfrac{1}{3})(x^{3} - 2x -2)
    \end{align*}
    Meaning if we evaluate the equation at $\theta$ we get that,
    \begin{align*}
        1 = (-\dfrac{2}{3}\theta^{2} + \dfrac{1}{3}\theta + \dfrac{5}{3})(\theta^{2} + \theta + 1)
    \end{align*}
    therefore $(\theta^{2} + \theta + 1)^{-1} = (-\dfrac{2}{3}\theta^{2} + \dfrac{1}{3}\theta + \dfrac{5}{3})$. Now we can compute,
    \begin{align*}
        \dfrac{1 + \theta}{1 + \theta + \theta^{2}} &= (1+\theta)(-\dfrac{2}{3}\theta^{2} + \dfrac{1}{3}\theta + \dfrac{5}{3}) \\
        &= -\dfrac{2}{3}\theta^{2} + \dfrac{1}{3}\theta + \dfrac{5}{3} -\dfrac{2}{3}\theta^{3} + \dfrac{1}{3}\theta^{2} + \dfrac{5}{3}\theta \\
        &= -\dfrac{2}{3}\theta^{3} - \dfrac{1}{3}\theta^{2} + \dfrac{6}{3}\theta + \dfrac{5}{3} \\
        &= -\dfrac{4}{3}\theta - \dfrac{4}{3} - \dfrac{1}{3}\theta^{2}+ \dfrac{6}{3}\theta + \dfrac{5}{3} \\
        &= -\dfrac{1}{3}\theta^{2} + \dfrac{2}{3}\theta + \dfrac{1}{3}
    \end{align*}
    as desired.

\end{proof}

\begin{problem}{13.1.3}
    Show that $x^{3} + x + 1$ is irreducible over $\ff_2$ and let $\theta$ be a root. Compute the powers of $\theta$ in $\ff_2(\theta)$.
\end{problem}
\begin{proof}
    We see the given polynomial is of degree 3, therefore it will have to have a linear factor in order to be reducible. So it is enough to show that it has no roots, and because we are in $\ff_2$ the only roots it could possibly have are 0 and 1. We see though,
    \begin{align*}
        1^{3} + 1 + 1 &= 1 \\
        0 + 0 + 1 & = 1 
    \end{align*}
    therefore $x^{3} + x + 1$ is irreducible over $\ff_2$. Now obtaining the powers of $\theta$ we get,
    \begin{align*}
        \theta^{0} &= 1 \\
        \theta^{1} &= \theta \\
        \theta^{2} &= \theta^{2} \\ 
    \end{align*}
    we pause here to note that since $\theta$ is a root of the given polynomial we have,
    \[\theta^{3} + \theta + 1 = 0 \Rightarrow \theta^{3} = \theta + 1\]
    continuing,
    \begin{align*}
        \theta^{4} &= \theta^{3}\theta = (\theta + 1)\theta = \theta^{2} + \theta \\
        \theta^{5} &= (\theta^{2} + \theta)\theta = \theta^{3} +\theta^{2} = \theta^{2} + \theta + 1\\
        \theta^{6} &= (\theta^{2} + \theta + 1)\theta = \theta^{3} + \theta^{2} + \theta = \theta + 1 +\theta^{2} + \theta = \theta^{2} + 1 \\
        \theta^{7} &= (\theta^{2} + 1)\theta = \theta^{3} + \theta = \theta + 1 +\theta = 1 && \text{cycles back}
    \end{align*} 
    Therefore $\theta^{i}$ is unique for $0 \leq i \leq 6$, giving us all the powers of $\theta$, as desired.
 \end{proof}

\begin{problem}{13.1.4}
    Prove directly that the map $a + b\sqrt{2} \mapsto a - b\sqrt{2}$ is an isomorphism of $\qq(\sqrt2)$ with itself 
\end{problem}
\begin{proof}
    Let us denote the given map as $\pi$. The first thing we must do is show that $\pi$ is a homomorphism. First we see the additive property, 
    \begin{align*}
        \pi(a + b\sqrt2 + c + d\sqrt2) &= \pi((a+c) + (b + d)\sqrt{2}) \\
        &= a+c -b\sqrt2 - d\sqrt2 \\
        &= a - b\sqrt2 +c - d\sqrt2  \\ 
        &= \pi( a + b\sqrt2) + \pi(c + d\sqrt2)
    \end{align*}
    now the multiplicative property,
    \begin{align*}
        \pi((a+b\sqrt2)\cdot (c+d\sqrt2)) &= \pi(ac + 2bd + (ad+bc)\sqrt2) \\
        &= ac +2bd -ad\sqrt2 -bc\sqrt2 \\ 
        &= (a-b\sqrt2)(c-d\sqrt2) \\
        &= \pi(a+b\sqrt2)\cdot \pi(c + d\sqrt2)
    \end{align*}
    meaning $\pi$ is a homomorphism. 

    Now we must show that $\pi$ is injective,
    \begin{align*}
        \pi(a +b \sqrt2) = \pi(c + d\sqrt2) \Rightarrow a-b\sqrt2 = c -d \sqrt2
    \end{align*}
    and because $\sqrt2$ is irrational so therefore not in the field of rational numbers, we have that
    \begin{align*}
        a = b \text{ and } c = d
    \end{align*}
    therefore $\pi$ is injective. Now we show that it is surjective, consider any $a+ b\sqrt2 \in \qq(\sqrt2)$, we have then that,
    \begin{align*}
        \pi(a + (-b)\sqrt2) = a+b \sqrt2
    \end{align*}
    therefore $\pi$ is surjective. All this together means that $\pi$ is an isomorphism of $\qq(\sqrt2)$ with itself. 
\end{proof}

\begin{problem}{13.1.5}
    Suppose $\alpha$ is a rational root of a monic polynomial in $\zz[x]$. Prove that $\alpha$ is an integer. 
\end{problem}
\begin{proof}
    We will do a proof by contradiction and let's assume $\alpha = \dfrac{c}{d}$ where $c$ and $d$ are relatively prime, and $d \neq \pm 1$. We are given that $\alpha$ is a root of some monic polynomial $p(x)\in \zz[x]$ so,
    \begin{align*}
        0 &= \alpha^{n} + a_{n-1}\alpha^{n-1} + \dots + a_1 \alpha + a_0  \\
        0&= \lrp{\dfrac{c}{d}}^{n} + a_{n-1}\lrp{\dfrac{c}{d}}^{n-1} + \dots + a_1 \lrp{\dfrac{c}{d}} + \alpha_0 \\
        -\dfrac{c^{n}}{d^{n}} &= a_{n-1}\dfrac{c^{n-1}}{d^{n-1}} + \dots + a_1\dfrac{c}{d} + a_0 \\
        -c^{n} &= d^{n}(a_{n-1}\dfrac{c^{n-1}}{d^{n-1}} + \dots + a_1\dfrac{c}{d} + a_0) \\
        -c^{n} &= d(a_{n-1}c^{n-1} + \dots + a_1cd^{n-2} + a_0d^{n-1}) 
    \end{align*}
    Meaning that any prime that divides $d$ must also divide $c^{n}$ and therefore divide $c$, but recall $c$ and $d$ were relatively prime, so there can't be a prime dividing $d$, but that means $d = \pm 1$ which is a contradiction. Therefore $\alpha$ must be an integer. 
\end{proof}

\begin{problem}{13.2.3}
    Determine the minimal polynomial over $\qq$ for the element $1 + i$
\end{problem}
\begin{proof}
    It is clear that the minimal polynomial of the given element has to be at least degree 2 since $1 + i$ is not in the field of rational numbers. We see through conjugation that,
    \begin{align*}
        (x- (1+i))(x - (1 -i)) &= (x-1 - i)(x -1 +i) \\
        &= x^{2} -x -xi -x +1 +i +xi -i + 1\\
        &= x^{2} -2x +2
    \end{align*}
    Then like in previous problems we apply Eisenstein's Irreducibility Criterion with $p =2$, we see that $2$ divides 2 and -2, doesn't divide 1, and 4 doesn't divide 2, so it is irreducible. Meaning the minimal polynomial over $\qq$ for the given element is,
    \[x^{2} -2x + 2.\]
\end{proof}
\newpage
\begin{problem}{13.2.5}
    Let $F = \qq(i)$. Prove that $x^{3} -2$ and $x^{3} -3$ are irreducible over $F$.
\end{problem}
\begin{proof}
    Since $x^{3}-2$ is of degree 3, if we it assume it to be reducible we would have that it can be factored by a linear factor, and therefore have at least one root in $F$. In other words, 
    \begin{align*}
        x^{3}-2 = (x-\alpha)p(x)
    \end{align*}
    where $p(x)$ is a monic quadratic polynomial and $\alpha \in F$. Now let $\zeta = \dfrac{1}{2} + \dfrac{\sqrt3}{2}i$, we have that the roots of $x^{3} - 2$ to be $\sqrt[3]{2}, \sqrt[3]{2}\zeta$, and $\sqrt[3]{2}(\overline{\zeta})$. Note though that elements of $F$ are of the form $a+bi$ where $a,b\in \qq$, we see that none of these roots are of this form, therefore $x^{3}-2$ is irreducible over $F$.

    We proceed similarly to show the same for $x^{3} - 3$. If it were to be reducible over $F$ we would have the same story as above and the roots to be $\sqrt[3]{3}, \sqrt[3]{3}\zeta$, and $\sqrt[3]{3}(\overline{\zeta})$, but none of them are in $F$, meaning $x^{3} - 3$ is irreducible over $F$.
\end{proof}

\begin{problem}{13.2.13}
    Suppose $F = \qq(\alpha_1, \alpha_2, \dots, \alpha_n)$ where $\alpha_i ^{2}\in \qq$ for $i = 1,2,\dots,n$. Prove that $\sqrt[3]{2} \notin F$.
\end{problem}
\begin{proof}
    This will be a proof by contradiction. We see that ,
    \[\lrb{\qq(\alpha_1, \dots ,\alpha_i): \qq(\alpha_1, \dots, \alpha_{i-1})} \in \set{1,2}\]
    for $i = 1, \dots ,n$. So $[F:\qq] = 2^{k}$ for $k \in \nn$. Now assume that $\sqrt[3]{2}\in F$, we would have then that,
    \[\qq \subset \qq(\sqrt[3]{2}) \subset F\]
    and therefore $[\qq(\sqrt[3]{2}) : \qq]$ must divide  $[F: \qq]$, but that means $3$ divides $2^{k}$ which is a contradiction as desired. Meaning then that $\sqrt[3]{2}\in F$
\end{proof}

\begin{problem}{13.2.15}
    A field $F$ is said to be formally real if $-1$ is not expressible as a sum of squares in $F$. Let $F$ be a formally real field, let $f(x) \in F[x]$ be an irreducible polynomial of odd degree and let $\alpha$ be a root of $f(x)$. Prove that $F(\alpha)$ is also formally real. [Pick $\alpha$ a counterexample of minimal degree. Show that $-1 + f(x)g(x) = (p_1(x))^{2}+ \dots + (p_m(x))^{2}$ for some $p_i(x), g(x)\in F[x]$ where $g(x)$ has odd degree $<$ deg ($f$). Show that some root $\beta$ of $g$ has odd degree over $F$ and $F(\beta)$ is not formally real, violating the minimality of $\alpha$.]
\end{problem}
\begin{proof}
    Let $\alpha$ be of minimal degree so that $F(\alpha)$ is NOT formally real and $\alpha$ having minimal polynomial $f$ which is of odd degree. Meaning we can express the degree of said $f$ as,
    \begin{align*}
        deg(f) = 2k +1, \ k \in \nn
    \end{align*}
    As given in the problem statement $-1$ can be expressed as a sum of squares in $F(\alpha)$, and we have that $F(\alpha)$ is isomorphic to $F[x]/(f(x))$. Then we have that there exists polynomials $p_1(x), \dots , p_m(x)$, and $g(x)$ so that,
    \begin{align}
        (p_1(x))^{2} + \dots + (p_m(x))^{2} = -1 + f(x)g(x)
    \end{align}
    We know that elements of $F[x]/(f(x))$ can be expressed as a polynomial in $\alpha$ with degree $<$ deg$(f)$. Meaning we have then that, deg$p_i < 2k + 1$ for all $i$. This means that the degree on the LHS of (4) is less than $4k + 1$, so the degree of $g$ is also less than $2k + 1$. We want to show then that the degree of $g$ is odd because then that would imply the degree of the LHS of (4) must be even. 

    So now we let $d$ be the maximal degree over $p_i$ for all $i$. We see that $x^{2d}$ is a sum of squares. Because $F$ is formally real, we have then that $0$ cannot be expressed as a sum of squares in $F$, therefore $x^{2d} \neq 0$. Meaning the degree of the LHS of (4) must be $2d$, and thus the degree of $g$ is odd. Meaning $g$ contains an irreducible factor of odd degree which we will denote $r(x)$, and because the degree of $g$ is less than the degree of $f$ we have,
    \[deg(r) < deg(g) < deg(f)\]
    So let $\beta$ be a root of $r(x)$ (therefore a root of $g(x)$), then,
    \begin{align*}
        (p_1(x))^{2} + \dots + (p_m(x))^{2}  = -1 r(x)\dfrac{f(x)g(x)}{r(x)}
    \end{align*}
    meaning $-1$ is a square in $F[x]/(h(x))$ which is isomorphic to $F(\beta)$. Giving to us that $F(\beta)$ is not formally real. Implying that $\beta$ is a root of $r$ such that $F(\beta)$ is not formally real, but $deg(r) < deg(f)$ which violates the minimality of $\alpha$, as desired. 
\end{proof}

\begin{problem}{13.2.16}
    Let $K/F$ be an algebraic extension and let $R$ be a ring contained in $K$ and containing $F$. Show that $R$ is a subfield of $K$ containing $F$. 
\end{problem}
\begin{proof}
    All we have to show is that $R$ contains multiplicative inverses. So let $r \in R$ and $r \neq 0$. We have that $r$ is algebraic over $F$ meaning that there exists an irreducible polynomial $p(x) = x^{n} + a_{n-1}x^{n-1} + \dots + a_1x + a_0$ in $F[x]$ such that $r$ is a root. Because $p$ is irreducible we have that the constant term of $p(x)$ must be non-zero. We know that,
    \begin{align}
        r^{-1} = -a_0^{-1}(r^{n-1} + \dots + a_1)
    \end{align}
    because $a_i \in F$ and $F$ is contained in $R$, and $r$ was an element of $R$, we have that $r^{-1} \in R$. Meaning $R$ has multiplicative inverse, making it a subfield of $K$ which contains $F$, as desired. 
\end{proof}

\end{document}