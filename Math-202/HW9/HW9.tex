\documentclass[11pt]{article}
 
\usepackage[top=0.75in, bottom=1.25in, left=1.25in, right=1.25in]{geometry} 
\usepackage{amsmath,amsthm,amssymb} %this is THE math package
\usepackage{mathtools}
\usepackage{tikz}
\usepackage{graphicx}
\usepackage{fancybox}
\usepackage{enumitem}
\usepackage{hyperref}
\usepackage{varwidth}
\usepackage{mdframed}
\usepackage{mathrsfs}
\usepackage[most]{tcolorbox}
%------------------------
%Fonts I use, uncomment if you like to use them.
%The first is the general font, and the second a math font
\usepackage{mathpazo}
\usepackage{eulervm}
%------------------------
%This is so that we have standard fonts for the double-stroked symbols
%for reals, naturals etc. regardless of what font you use.
%Don't comment
\AtBeginDocument{
  \DeclareSymbolFont{AMSb}{U}{msb}{m}{n}
  \DeclareSymbolFontAlphabet{\mathbb}{AMSb}}
%------------------------

%----------------------------------------------
%User-defined environments
%Commented because we're not using them in this document
%The only uncommented ones are the Problem and Solution environment

% \newenvironment{theorem}[2][Theorem]{\begin{trivlist}
% \item[\hskip \labelsep {\bfseries #1}\hskip \labelsep {\bfseries #2.}]}{\end{trivlist}}
% \newenvironment{lemma}[2][Lemma]{\begin{trivlist}
% \item[\hskip \labelsep {\bfseries #1}\hskip \labelsep {\bfseries #2.}]}{\end{trivlist}}
% \newenvironment{exercise}[2][Exercise]{\begin{trivlist}
% \item[\hskip \labelsep {\bfseries #1}\hskip \labelsep {\bfseries #2.}]}{\end{trivlist}}
% \newenvironment{question}[2][Question]{\begin{trivlist}
% \item[\hskip \labelsep {\bfseries #1}\hskip \labelsep {\bfseries #2.}]}{\end{trivlist}}
% \newenvironment{corollary}[2][Corollary]{\begin{trivlist}
% \item[\hskip \labelsep {\bfseries #1}\hskip \labelsep {\bfseries #2.}]}{\end{trivlist}}
\newenvironment{problem}[2][Problem\!]{\begin{tcolorbox}\begin{trivlist}
\item[\hskip \labelsep {\bfseries #1}\hskip \labelsep {\bfseries #2}]}{\end{trivlist}\end{tcolorbox}}
%\newenvironment{sub-problem}[2][]{\begin{trivlist}
%\item[\hskip \labelsep {\bfseries #1}\hskip \labelsep {\bfseries #2}]}{\end{trivlist}}
\newenvironment{solution}{\begin{proof}[\textbf{\textit{Solution}}] }{\end{proof}}
%----------------------------------------------

%----------------------------
%User-defined notations
\newcommand{\zz}{\mathbb Z}   %blackboard bold Z
\newcommand{\qq}{\mathbb Q}   %blackboard bold Q
\newcommand{\ff}{\mathbb F}   %blackboard bold F
\newcommand{\rr}{\mathbb R}   %blackboard bold R
\newcommand{\nn}{\mathbb N}   %blackboard bold N
\newcommand{\cc}{\mathbb C}   %blackboard bold C
\newcommand{\af}{\mathbb A}   %blackboard bold A
\newcommand{\pp}{\mathbb P}   %blackboard bold P
\newcommand{\id}{\operatorname{id}} %for identity map
\newcommand{\im}{\operatorname{im}} %for image of a function
\newcommand{\dom}{\operatorname{dom}} %for domain of a function
\newcommand{\cat}[1]{\mathscr{#1}}   %calligraphic category
\newcommand{\abs}[1]{\left\lvert#1\right\rvert} %for absolute value
\newcommand{\norm}[1]{\left\lVert#1\right\rVert} %for norm
\newcommand{\modar}[1]{\text{ mod }{#1}} %for modular arithmetic
\newcommand{\set}[1]{\left\{#1\right\}} %for set
\newcommand{\setp}[2]{\left\{#1\ \middle|\ #2\right\}} %for set with a property
\newcommand{\card}[1]{\#\,{#1}} %for cardinality of a set
\newcommand\m[1]{\begin{pmatrix}#1\end{pmatrix}} 

%Re-defined notations
\renewcommand{\epsilon}{\varepsilon}
\renewcommand{\phi}{\varphi}
\renewcommand{\emptyset}{\varnothing}
\renewcommand{\geq}{\geqslant}
\renewcommand{\leq}{\leqslant}
\renewcommand{\Re}{\operatorname{Re}}
\renewcommand{\Im}{\operatorname{Im}}
%----------------------------

\allowdisplaybreaks

\newcommand{\tcr}[1]{\textcolor{red}{#1}}
\newcommand{\tcb}[1]{\textcolor{blue}{#1}}
\newcommand{\tco}[1]{\textcolor{orange}{#1}}

\newcommand{\lrp}[1]{\left(#1\right)}
\newcommand{\lrb}[1]{\left[#1\right]}
\newcommand{\lrc}[1]{\left\{#1\right\}}
\newcommand{\lrw}[1]{\left<#1\right>}
 
 
\begin{document}
 
\title{Homework 9}
\author{Kevin Guillen\\[0.5em]
MATH 202 | Algebra III | Spring 2022}
\date{} 
\maketitle

%Use \[...\] instead of $$...$$

\begin{problem} {11.5.1}
    Prove that if $M$ is a cyclic $R-$module then $\mathcal{T}(M) = \mathcal{S}(M)$, i.e., the tensor algebra $\mathcal{T}(M)$ is commutative.
\end{problem}
\begin{proof}
    Because $M$ is cyclic, we have without loss of generality that $M = R/I$ for some ideal of $R$. Then $\mathcal{S}(M) = \mathcal{T}(M)/J$ where $J$ is the 2-sided ideal generated by elements of the form 
    \[m_1 \otimes m_2 - m_2 \otimes m_1\]
    Writing $m_i = r_i(1 + I)$, we have $m_1 \otimes m_2 - m_2 \otimes m_1 = 0$ therefore $\mathcal{S}(M) = \mathcal{T}(M)$.  
\end{proof}

\vspace*{15pt}

\begin{problem} {11.5.4}
    Prove that $m\wedge n_1 \wedge n_2 \wedge \dots \wedge n_k = (-1)^{k}(n_1 \wedge n_2 \wedge \dots \wedge n_k \wedge m)$. In particular, $x\wedge (y \wedge z) = (y\wedge z)\wedge x$ for all $x,y,z \in M$.
\end{problem}
\begin{proof}
    We have $R$ to be commutative with a 1, now let $M$ be an $(R,R)$ bimodule such that for all $r \in R$ and $m \in M$ we have,
    \[rm = mr.\]

    Now for all $x,y,z \in M$ we have,
    \[x\wedge (y \wedge z) = (y \wedge z) \wedge x\]

    Now we will do induction on $n$. It is clear for $n = 0,1$ the result holds, now we assume it holds for all $n \leq k$. 

    Now we prove for $n = k +1$,
    \begin{align*}
        m \wedge n_1 \wedge n_2 \wedge \dots \wedge n_k \wedge n_{k + 1} &=  (m \wedge n_1 \wedge n_2 \wedge \dots \wedge n_k) \wedge n_{k + 1} \\
        &= (-1)^{k}(n_1 \wedge n_2 \wedge \dots \wedge n_k \wedge m) \wedge n_{k+1} \\
        &= (-1)^{k} (n_1 \wedge n_2 \wedge \dots \wedge n_{k+1}\wedge m) \\
        &= (-1)^{k}(n_1 \wedge n_2 \wedge \dots \wedge n_{k+1} \wedge m).
    \end{align*}
    Giving us the desired equality,
    \[m\wedge n_1 \wedge n_2 \wedge \dots \wedge n_k = (-1)^{k}(n_1 \wedge n_2 \wedge \dots \wedge n_k \wedge m).\]
\end{proof}

\vspace*{15pt}

\begin{problem} {11.5.6}
    If $A$ is an $R-$algebra in which $a^{2} = 0$ for all $a \in A$ and $\phi:M\to A$ is an $R-$module homomorphism, prove there is a unique $R-$algebra homomorphism $\Phi:\wedge(M) \to A$ such that $\Phi\mid_M = \phi$
\end{problem}
\begin{proof}
    As before we have $R$ to be commutative with a 1 and $M$ an $(R,R)$ bimodule, such that for all $r\in R$ and $m \in M$ such that,
    \[rm = mr.\]

    Through the properties of a tensor algebra there exists a unique $R-$module homomorphism $\overline{\Phi}:\overline{I}(M)\to A, \ \overline{\Phi}_M = \phi $

    $A(M)$ is ideal of $\mathcal{T}(M)$ is generated by simple tensor of the form $m \otimes m$.
    \begin{align*}
        \overline{\Phi}(m \otimes m) &= \overline{\Phi}(m^{2}) \\
        &= 0 \in A\\
        A(M) &\subseteq \ker \overline{\Phi}
    \end{align*}
    then by the 1st isomorphism theorem for $R-$algebras, 
    \begin{align*}
        \Phi&: \wedge(M) \to A, \text{ defined by } \Phi(\overline{t}) = \overline{\Phi}(t), \Phi\mid_M = \phi \\
        \psi &: \wedge(M) \to A
    \end{align*}
    then,
    \begin{align*}
        \psi(\wedge m_i) &= \prod \psi(m_i) \\
        &= \prod \phi(m_i) \\
        &= \prod \Phi(m_i) \\
        &= \Phi(\wedge m_i)
    \end{align*}
    so $\Phi$ is unique $R-$algebra homomorphism.
\end{proof}

\vspace*{15pt}

\begin{problem} {11.5.8}
    Let $R$ be an integral domain and let $F$ be its field of fractions
    \begin{itemize}
        \item[(a)] Considering $F$ as an $R-$module, prove that $\wedge^{2}F = 0$
        \item[(b)]  Let $I$ be any $R-$submodule of $F$ (for example, any ideal in $R$). Prove that $\wedge^{i}I$ is a torsion $R$-module for $i \geq 2$(i.e., for every $x \in \wedge^{i}I$ there is some nonzero $r\in R$ with $rx = 0$)
        \item[(c)] Give an example of an integral domain $R$ and an $R-$module $I$ in $F$ with $\wedge^{i}I \neq 0$ for every $i \geq 0$ (cf. the example following corollary 37)
    \end{itemize}
\end{problem}
\begin{itemize}
    \item[(a)]
    \begin{proof}
        Let $F$ be an $R-$module, we have,
        \begin{align*}
            \dfrac{a}{b} \otimes \dfrac{c}{d} &\in \mathcal{T}^{2}(F) \\
            \dfrac{a}{b} \otimes \dfrac{c}{d} &= \dfrac{ad}{bd} \otimes \dfrac{cb}{bd} \\
            &= abcd\lrp{\dfrac{1}{bd} \otimes \dfrac{1}{bd}} \\
            \dfrac{a}{b}\wedge \dfrac{c}{d} &= 0 \in \wedge^{2}(F)
        \end{align*}
    \end{proof} 
    \item[(b)]
    \begin{proof}
        Let $I$ be any $R-$submodule of $F$, we have,
        \begin{align*}
            \dfrac{a_1}{b_1} \wedge \dfrac{a_2}{b_2} \wedge \dots \wedge \dfrac{a_k}{b_k} \in \wedge^{2}(I)
        \end{align*}
        then $a_i \neq 0$ and $b-i\neq 0$, $a_1a_2b_1b_2 \neq 0 \in R$,
        \begin{align*}
            a_1a_2b_1b_2\lrp{\dfrac{a_1}{b_1}\wedge \dfrac{a_2}{b_2} \wedge \dots \wedge \dfrac{a_k}{b_k}} = \dfrac{a_1a_2}{1}\wedge \dfrac{a_1a_2}{1} \wedge \dots \wedge \dfrac{a_k}{b_k}
        \end{align*}
        every element of $\wedge^{K}(I)$ is torsion as desired. 
    \end{proof} 
    \item[(c)]  
    \begin{proof}
        Let us consider $R = Z[x_1, x_2, \dots , x_n]$ and $I = (x-1, x_2, \dots, x_n )$

        Now we consider some $j$ and let,
        \[\alpha_j x_j - \beta_j x_j = \sum_{i \neq j}(\beta_i -\alpha_i )x_i\]
        Since $R$ is a domain we have $x_j$ divides the right hand side,
        \[\sum_{i \neq j} (\beta_i - \alpha_i)x_i = x_jh_j\]
        Here, $h_i \in I$ such that $\alpha_i -\beta_i = h_i$

        Now we consider $\prod(I)$ as column vectors that means $\sum\alpha_kx_k$ as $[\alpha_1,\alpha_2,\dots,\alpha_n]^{T}$. Tow column vectors $A$ and $B$ represent the same element $I$, there exists a third column vector $H$, such that $A = B + H$. 

        Now consider the elements of $R^{k}$ as square matrix. The determinant of such matrix $A$, as an element of $R$ reduced mod $I$. If the matrix $A$ and $B$ represent the same elements of $I^{i}$ then matrix $H$ is such $A = B + H$. Now consider determinants of both sides mod $I$, which we compute using the combinatorial formula,
        \[det(B+H) = \sum_{\sigma \in \sigma_n}\epsilon(\sigma)\prod(\beta_{\sigma(i),j} + h_{\sigma(i),j})\]
        $h_{i,j}$ is divisible by some $x_i$ and hence goes to the quotient $R/I$ and so,
        \begin{align*}
            det(A) = det(B+H) \equiv det(B)\mod I
        \end{align*}
        Thus the map $det:I^{i} \to R/I$ is well defined alternating bilinear map. This map is nontrivial since \[det(x_1\otimes \dots \otimes x_n) = 1\]
        therefor for all $i$, $\wedge^{i}(I) \neq 0 $
    \end{proof} 
\end{itemize}

\vspace*{15pt}

\begin{problem} {11.5.9}
    Let $R = \zz[G]$ be the group ring of the group $G = \set{1,\sigma}$ of order 2. Let $M = \zz e_1 + \zz e_2$ be the free $\zz$-module of rank 2 with basis $e_1$ and $e_2$. Define $\sigma(e_1) = e_1 + 2 e_2$ and $\sigma(e_2) = -e_2$. Prove that this makes $M$ into an $R-$module and that the $R-$module $\wedge^{2}M$ is a group of order 2 with $e_1\wedge e_2$ as generator. 
\end{problem}
\begin{proof}
    We have the mapping $\phi M \to M$ by $\phi(e_1) = e_1 + 2e_2$ and $\phi(e_2) = -e_2$. By using this mapping we make $M$ into an $R-$module and compute the exterior power $\wedge^{2}(M)$ over $R$. 

    We have $\phi$ to be an endomorphism of order 2, and $\sigma^{2} = 1$. Now we define the following,
    \begin{align*}
        R\times M &\to M \\
        (a \cdot 1 + b\sigma) \cdot m &= am + b\phi(m)
    \end{align*}

    Now note that for any group $G$, ring $R$, and $S-$module $M$, If,
    \begin{align*}
        \alpha: R \to End_S(M) && \beta:G\to Aut_S(M)
    \end{align*}
    such that 
    \[\alpha \subseteq C_{End_S(M)}(\im \beta)\]
    then the induced map given by,
    \[\gamma: R[G]\to End_S(M)\]
    given by $\gamma(\sum r_i g_i) = \sum \alpha(r_i)\circ \beta(g_i)$ is a well defined ring homomorphism. So we have $M$ to be a $Z[G]$ module. As $R$ is commutative, so  $M$ is an $(R,R)$-bimodule such that $rm = mr$.
    Giving us
    \begin{align*}
        -(e_1 \wedge e_2) &= e_1 \wedge (-e_2) \\
        &= e_1 \wedge \sigma \cdot e_2 \\
        &= \sigma \cdot e_1 \wedge e_2 \\
        &= e_1 + 2e_2 \wedge e_2 \\
        &= e_1 \wedge e_2 \\
        2(e_1 \wedge e_2) &= 0 \\
        \sigma(e_1 \wedge e_2) &= 0
    \end{align*}

    here, $\sigma^{2}(M)$ is generated by $e_1\wedge e_2$ therefore $R(e_1\wedge e_2) = \set{0, e_1\wedge e_2}$

    Now we consider the mapping,
    \begin{align*}
        \det: M^{2} &\to \zz/2\zz \\
        (ae_1 + be_2, ce_1 + de_2) &\mapsto ad - bc \mod 2
    \end{align*}
    Therefore $\det$ is an alternating $\zz-$billinear form. To show that is is $R-$bilinear though we show that $\det(v,\sigma w) = \det(v\sigma, w)$,
    \begin{align*}
        \det(ae_1+ be_2, \sigma(ce_1 + de_2)) &= \det(ae_1 + be_2, ce_1 + (2c -d)e_2) \\
        &= a(2c - d) - bc \\
        &= ad - c(2a-b) \\
        &= \det(ae_1 + (2a-b)e_2, ce_1 + de_2) 
    \end{align*}
    and $\det(e_1,e_2) = 1\neq 0$ so $e_1\wedge e_2$ is nonzero in $\wedge^{2}(M)$ therefore $\wedge^{2}(M) \cong \zz/2\zz$
    
\end{proof}

\vspace*{15pt}

\begin{problem} {11.5.10}
    Prove that $z-(1/k!)Alt(z) = (1/k!)\sum_{\sigma \in S_k}(z -\epsilon(\sigma)\sigma z)$ for any $k-$tensor $z$ and use this to prove that the kernel of the $R$-module homomorphism $(1/k!)Alt$ in proposition 40 is $\mathcal{A}^{k}(M)$.
\end{problem}
\begin{proof}
    Let $z \in T^{k}(M)$ then,
    \begin{align*}
        z - \dfrac{1}{k!}Alt(z) &= z - \dfrac{1}{k!}\sum_{\sigma\in S_k}\epsilon (\sigma )\sigma z   \\
        &= \dfrac{1}{k!}\lrp{zk! - \sum_{\sigma \epsilon S_k} \epsilon (\sigma)\sigma z} \\
        &= \dfrac{1}{k!}\lrp{\sum_{\sigma \in S_k}z - \sum_{\sigma \in z_k} \epsilon (\sigma) \sigma z} \\
        &= \dfrac{1}{k!}\lrp{\sum_{\sigma \in S_k}(z - \epsilon(\sigma) \sigma z)}
    \end{align*}
    thus $z -\dfrac{1}{k!}Alt(z) = \dfrac{1}{k!}\lrp{\sum_{\sigma \in S_k}(z - \epsilon(\sigma)\sigma z)}$ as needed. 

    Now we for the latter statement. Let $z \in A^{k}(M)$ and suppose that $i$ and $i+j$ components of $z$ are equal, we have,
    \[\sigma z = \sigma(1 i + 1)z\]
    moreover,
    \[\epsilon(\sigma)\sigma z + \epsilon(\sigma(ij + 1))\sigma(ij+1)z = 0\]
    Now we consider the following equation,
    \[Alt(z) = \sum_{\sigma \in S_k}\epsilon(\sigma)\sigma z\]
    the RHS of this can be broke up into a summation over the cosets of $<(ij + 1)>$ each of which is zero, giving us
    \begin{align*}
        \dfrac{1}{k!}Alt(z) = 0 
    \end{align*}
    therefore 
    \begin{align*}
        A^{K}(M) \subseteq \ker \dfrac{1}{k!}Alt
    \end{align*}
    Now let $z \in \ker \dfrac{1}{k!} Alt$ then we have,
    \begin{align*}
        \dfrac{1}{k!} \sum_{\sigma \in S_k}(z - \epsilon(\sigma)\sigma z) = z
    \end{align*}
    for each $\sigma$, $z -\epsilon(\sigma)\sigma z \in A^{k}(M)$. Therefore $z \in A^{k}(M)$ and thus $\ker\dfrac{1}{k!} Alt = A^{k}(M)$ as desired. 
\end{proof}

\vspace*{15pt}

\begin{problem} {11.5.11}
    Prove that the image of $Alt_k$ is the unique largest subspace of $\mathcal{T}^{k}(V)$ on which each permutation $\sigma$ in the symmetric group $S_k$ acts as multiplication by the scalar $\epsilon (\sigma)$. 
\end{problem}
\begin{proof}
    We have $V$ to be an $F-$vector space. Now $S_k$ acts on the tensor power $T^{k}(V)$ by permuting the components. Let $k!$ be a unit in the ring $R$ and $M$ an $R-$module. The map $(1/k!)Alt$ induces and $R-$module isomorphism between the $k^{th}$ exterior power of $M$ and the $R-$sub module of alternating $k-$tensors:
    \begin{align*}
        \dfrac{1}{k!}Alt: \wedge^{k}M \cong \set{\text{alternating} k-\text{tensors}}
    \end{align*}
    and that $Alt_k$ is defined on $T^{k}(V)$ by the following,
    \begin{align*}
        Alt_K(z) = \sum_{\sigma \in S_k}\epsilon(\sigma)\sigma z
    \end{align*}

    Now let $z \in T^{k}(V)$ such that for all $\sigma \in S_k$ we have,
    \[\sigma z = \epsilon(\sigma)z\]
    then,
    \begin{align*}
        Alt_k(z) &= \dfrac{1}{k!}\sum_{\sigma \in S_k}\epsilon(\sigma)\sigma z \\ 
        &= \dfrac{1}{k!} \sum_{\sigma \in S_k}\epsilon(\sigma)\epsilon(\sigma)z \\
        &= \dfrac{1}{k!}\sum_{\sigma \in S_k}z \\
        &= z
    \end{align*}
    Therefore $z \in \im Alt_k$. Specifically any subspace of $T^{k}(V)$ upon which every permutation $\sigma\in S_k$ acts as scalar multiplication by $\epsilon(\sigma)$ is in $\im Alt_k$

    So it can be seen that $\sigma \in S_k$ acts on $\im Alt_k$ as multiplication by $\epsilon(\sigma)$ as $\im Alt_k \cong_F \wedge^{k}(V)$. Therefore $\im Alt_k$ is the unique largest subspace of $T^{k}(V)$ on which each permutation $\sigma$ in the symmetric group $S_k$ acts as multiplication by the scalar $\epsilon (\sigma)$. 
\end{proof}

\vspace*{15pt}

\begin{problem} {11.5.13}
    Let $F$ be any field in which $-1 \neq 1$ and let $V$ be a vector space over $F.$ Prove that $V \oplus_F V = \mathcal{S}^{2}(V)\oplus \wedge^{2}(V)$ i.e., that every 2-tensor may be written uniquely as a sum of a symmetric and an alternating tensor. 
\end{problem}
\begin{proof}
    We note that $\dim\mathcal{S}^{2}(V) = \dfrac{n(n+1)}{2}$ and that $\dim \wedge^{2}(V) = \dfrac{n(n+1)}{2}$. Therefore, $\mathcal{S}^{2}(V)\oplus \wedge^{2}(V) = n^{2} = \dim V \otimes_F V$. We get the desired result by prove both these spaces intersect trivially, so assume that $v \in S^{2}(V)\cap \wedge^{2}(V)$, then we have,
    \begin{align*}
        \begin{cases}
            \sigma v = v \\
            \sigma v = sgn(\sigma)v
        \end{cases} \iff v = sgn(\sigma) v \iff v(1-sgn(\sigma)) = 0
    \end{align*}
    we have though that $sgn(\sigma) =1$ since we are in the symmetric group $S_2$. The above equation implies that $v(1 - 1) - 0$, but we assumed $-1 \neq 1$ so that can't be, therefore $v$ is forced to be 0, giving us the desired result. 
\end{proof}

\end{document}















