\documentclass[11pt]{article}
 
\usepackage[top=0.75in, bottom=1.25in, left=1in, right=1in]{geometry} 
\usepackage{amsmath,amsthm,amssymb} %this is THE math package
\usepackage{mathtools}
\usepackage{tikz}
\usepackage{graphicx}
\usepackage{fancybox}
\usepackage{enumitem}
\usepackage{hyperref}
\usepackage{varwidth}
\usepackage{mdframed}
\usepackage{mathrsfs}
\usepackage[most]{tcolorbox}
%------------------------
%Fonts I use, uncomment if you like to use them.
%The first is the general font, and the second a math font
\usepackage{mathpazo}
\usepackage{eulervm}
%------------------------
%This is so that we have standard fonts for the double-stroked symbols
%for reals, naturals etc. regardless of what font you use.
%Don't comment
\AtBeginDocument{
  \DeclareSymbolFont{AMSb}{U}{msb}{m}{n}
  \DeclareSymbolFontAlphabet{\mathbb}{AMSb}}
%------------------------

%----------------------------------------------
%User-defined environments
%Commented because we're not using them in this document
%The only uncommented ones are the Problem and Solution environment

% \newenvironment{theorem}[2][Theorem]{\begin{trivlist}
% \item[\hskip \labelsep {\bfseries #1}\hskip \labelsep {\bfseries #2.}]}{\end{trivlist}}
% \newenvironment{lemma}[2][Lemma]{\begin{trivlist}
% \item[\hskip \labelsep {\bfseries #1}\hskip \labelsep {\bfseries #2.}]}{\end{trivlist}}
% \newenvironment{exercise}[2][Exercise]{\begin{trivlist}
% \item[\hskip \labelsep {\bfseries #1}\hskip \labelsep {\bfseries #2.}]}{\end{trivlist}}
% \newenvironment{question}[2][Question]{\begin{trivlist}
% \item[\hskip \labelsep {\bfseries #1}\hskip \labelsep {\bfseries #2.}]}{\end{trivlist}}
% \newenvironment{corollary}[2][Corollary]{\begin{trivlist}
% \item[\hskip \labelsep {\bfseries #1}\hskip \labelsep {\bfseries #2.}]}{\end{trivlist}}
\newenvironment{problem}[2][Problem\!]{\begin{trivlist}
\item[\hskip \labelsep {\bfseries #1}\hskip \labelsep {\bfseries #2}]}{\end{trivlist}}
%\newenvironment{sub-problem}[2][]{\begin{trivlist}
%\item[\hskip \labelsep {\bfseries #1}\hskip \labelsep {\bfseries #2}]}{\end{trivlist}}
\newenvironment{solution}{\begin{proof}[\textbf{\textit{Solution}}] }{\end{proof}}
%----------------------------------------------

%----------------------------
%User-defined notations
\newcommand{\zz}{\mathbb Z}   %blackboard bold Z
\newcommand{\qq}{\mathbb Q}   %blackboard bold Q
\newcommand{\ff}{\mathbb F}   %blackboard bold F
\newcommand{\rr}{\mathbb R}   %blackboard bold R
\newcommand{\nn}{\mathbb N}   %blackboard bold N
\newcommand{\cc}{\mathbb C}   %blackboard bold C
\newcommand{\af}{\mathbb A}   %blackboard bold A
\newcommand{\pp}{\mathbb P}   %blackboard bold P
\newcommand{\id}{\operatorname{id}} %for identity map
\newcommand{\im}{\operatorname{im}} %for image of a function
\newcommand{\dom}{\operatorname{dom}} %for domain of a function
\newcommand{\cat}[1]{\mathscr{#1}}   %calligraphic category
\newcommand{\abs}[1]{\left\lvert#1\right\rvert} %for absolute value
\newcommand{\norm}[1]{\left\lVert#1\right\rVert} %for norm
\newcommand{\modar}[1]{\text{ mod }{#1}} %for modular arithmetic
\newcommand{\set}[1]{\left\{#1\right\}} %for set
\newcommand{\setp}[2]{\left\{#1\ \middle|\ #2\right\}} %for set with a property
\newcommand{\card}[1]{\#\,{#1}} %for cardinality of a set
\newcommand\m[1]{\begin{pmatrix}#1\end{pmatrix}} 
\newcommand{\plog}{\operatorname{Log}}
\newcommand{\parg}{\operatorname{Arg}}

%Re-defined notations
\renewcommand{\epsilon}{\varepsilon}
\renewcommand{\phi}{\varphi}
\renewcommand{\emptyset}{\varnothing}
\renewcommand{\geq}{\geqslant}
\renewcommand{\leq}{\leqslant}
\renewcommand{\Re}{\operatorname{Re}}
\renewcommand{\Im}{\operatorname{Im}}
%----------------------------

\allowdisplaybreaks

\newcommand{\tcr}[1]{\textcolor{red}{#1}}
\newcommand{\tcb}[1]{\textcolor{blue}{#1}}
\newcommand{\tco}[1]{\textcolor{orange}{#1}}

\newcommand{\lrp}[1]{\left(#1\right)}
\newcommand{\lrb}[1]{\left[#1\right]}
\newcommand{\lrc}[1]{\left\{#1\right\}}
 
 
\begin{document}
 
\title{Assignment}
\author{Kevin Guillen\\[0.5em]
MATH  | Class | Quarter}
\date{} 
\maketitle

%Use \[...\] instead of $$...$$

\begin{problem}{5.1}
Writing $z = re^{i\parg z}$, show that, where $n \in \zz_{>0}$
\[\log(z^{1/n}) = \frac{1}{n}\ln r + i\left(\frac{\parg z + 2(pn + k)\pi}{n}\right),\quad p\in\zz,\ k = 0,\ldots,n-1.\]
Now, after writing 
\[\frac{1}{n}\log z = \frac{1}{n}\ln r + i\left(\frac{\parg z + 2q\pi}{n}\right),\quad q \in \zz,\]
show that we have equality of sets
\[\log(z^{1/n}) = \frac{1}{n}\log z\]
\end{problem}
%----------------------------------------
\begin{proof}
    First recall from Example 9.9 that $z^{1/n} = e^{log(z)/n}$. So expanding we have,
    \begin{align*}
        \log(z^{1/n}) &= \log(e^{\log(z) / n}) && \text{Apply Prop. 9.3(2)} \\
        &= \dfrac{\log z}{n } + 2p\pi i  && p\in \zz \\
        &= \dfrac{1}{n}\ln r + \dfrac{i \parg z + 2k\pi i }{n} + 2p\pi i  && k\in \zz \\
        &= \dfrac{1}{n}\ln r + \dfrac{i\parg z + 2k\pi i }{n} + \dfrac{2p \pi i n}{n} \\
        &= \dfrac{1}{n}\ln r + i\lrp{\dfrac{\parg z + 2(pn + k) \pi}{n}} && p\in \zz, k\in \lrc{0, \dots , n-1}
    \end{align*}
    giving us the desired expression. 

\end{proof}
%----------------------------------------

\newpage  %Do not delete

\begin{problem}{5.2}\hfill
\begin{itemize}[itemsep=3em]
\item[(a)] Find real valued functions $u,\, v : \rr^2 \to \rr$ such that $\cos z = u(x, y) + iv(x, y)$.
%----------------------------------------
\begin{solution}
%Uncomment and WRITE YOUR SOLUTION HERE
\end{solution}
%----------------------------------------

\item[(b)] Show that $e^{\overline{z}} = \overline{e^z}$ and $\cos\overline{z} = \overline{\cos z}$.
%----------------------------------------
\begin{solution}
%Uncomment and WRITE YOUR SOLUTION HERE
\end{solution}
%----------------------------------------

\item[(c)] Show that $e^{iz} = \cos z + i\sin z$, and prove $\sin\overline{z} = \overline{\sin z}$.
%----------------------------------------
\begin{solution}
%Uncomment and WRITE YOUR SOLUTION HERE
\end{solution}
%----------------------------------------

\end{itemize}
\end{problem}

\newpage  %Do not delete

\begin{problem}{5.3}\hfill
\begin{itemize}[itemsep=3em]
\item[(a)] Verify that $(z^\alpha)^n = z^{n\alpha}$ for $z \neq 0$ and $n \in \zz$.
%----------------------------------------
\begin{solution}
%Uncomment and WRITE YOUR SOLUTION HERE
\end{solution}
%----------------------------------------

\item[(b)] Find a counterexample to the statement: $(z^{\alpha})^\beta = z^{\alpha\beta}$, where $z \neq 0$ and $\alpha,\beta \in \cc$.
%----------------------------------------
\begin{solution}
%Uncomment and WRITE YOUR SOLUTION HERE
\end{solution}
%----------------------------------------

\end{itemize}
\end{problem}

\newpage  %Do not delete

\begin{problem}{5.4}
Determine the points at which the following functions are holomorphic. Find its derivative at those points.
\begin{itemize}[itemsep=3em]
\item[(a)] $f(z) = e^{\overline{z}}$.
%----------------------------------------
\begin{solution}
%Uncomment and WRITE YOUR SOLUTION HERE
\end{solution}
%----------------------------------------

\item[(b)] $g(z) = \cos\overline{z}$.
%----------------------------------------
\begin{solution}
%Uncomment and WRITE YOUR SOLUTION HERE
\end{solution}
%----------------------------------------

\item[(c)] $a(z) = \dfrac{\plog(2z - i)}{z^2 + 1}$.
%----------------------------------------
\begin{solution}
%Uncomment and WRITE YOUR SOLUTION HERE
\end{solution}
%----------------------------------------

\end{itemize}
\end{problem}

\newpage  %Do not delete

\begin{problem}{5.5}
Find all complex values $z$ satisfying the given equation.
\begin{itemize}[itemsep=3em]
\item[(a)] $\cos z = 4$
%----------------------------------------
\begin{solution}
%Uncomment and WRITE YOUR SOLUTION HERE
\end{solution}
%----------------------------------------

\item[(b)] $\cos z = i\,\sin z$
%----------------------------------------
\begin{solution}
%Uncomment and WRITE YOUR SOLUTION HERE
\end{solution}
%----------------------------------------

\end{itemize}
\end{problem}

\newpage %do not delete

\begin{problem}{5.6}
Let $z \in \cc$.
\begin{itemize}[itemsep=3em]
\item[(a)] Prove that $\abs{1^z}$ is single-valued if and only if $\Im z = 0$.
%----------------------------------------
\begin{solution}
%Uncomment and WRITE YOUR SOLUTION HERE
\end{solution}
%----------------------------------------

\item[(b)] Find a necessary and sufficient condition for $\abs{i^{iz}}$ to be single-valued.
%----------------------------------------
\begin{solution}
%Uncomment and WRITE YOUR SOLUTION HERE
\end{solution}
%----------------------------------------

%----------------------------------------
%Delete if not attempted
%EXTRA CREDIT. OPTIONAL
\item[(c)] Show by counterexample that the statement is false: \[\text{$1^z$ is single-valued if and only if $\Im z = 0$.}\]
%----------------------------------------
\begin{solution}
%Uncomment and WRITE YOUR SOLUTION HERE
\end{solution}
%EXTRA CREDIT. OPTIONAL
%Delete if not attempted
%----------------------------------------

\end{itemize}
\end{problem}



%----------------------------------------
%Delete if nothing to add
\newpage  %Do not delete

\begin{center}
\textbf{Collaborators:}
%List your peers with whom you discussed the Problem Set
\end{center}
\vfill 

\begin{center}
\textbf{References:}
%List any book/website/notes that you used to write your solutions
\end{center}
\begin{itemize}
\item[$\bullet$] [Book(s): Title, Author]
\item[$\bullet$] [Online: \href{http://example.com/}{\color{blue}Link}]
\item[$\bullet$] [Notes: \href{http://example.com/}{\color{blue}Link}]
\end{itemize}

\vfill
\begin{center}
Fin.
\end{center}
\vfill
%----------------------------------------

\end{document}