\documentclass[11pt]{article}
 
\usepackage[top=0.75in, bottom=1.25in, left=1.25in, right=1.25in]{geometry} 
\usepackage{amsmath,amsthm,amssymb} %this is THE math package
\usepackage{mathtools}
\usepackage{tikz}
\usepackage{graphicx}
\usepackage{fancybox}
\usepackage{enumitem}
\usepackage{hyperref}
\usepackage{varwidth}
\usepackage{mdframed}
\usepackage{mathrsfs}
\usepackage[most]{tcolorbox}
%------------------------
%Fonts I use, uncomment if you like to use them.
%The first is the general font, and the second a math font
\usepackage{mathpazo}
\usepackage{eulervm}
%------------------------
%This is so that we have standard fonts for the double-stroked symbols
%for reals, naturals etc. regardless of what font you use.
%Don't comment
\AtBeginDocument{
  \DeclareSymbolFont{AMSb}{U}{msb}{m}{n}
  \DeclareSymbolFontAlphabet{\mathbb}{AMSb}}
%------------------------

%----------------------------------------------
%User-defined environments
%Commented because we're not using them in this document
%The only uncommented ones are the Problem and Solution environment

% \newenvironment{theorem}[2][Theorem]{\begin{trivlist}
% \item[\hskip \labelsep {\bfseries #1}\hskip \labelsep {\bfseries #2.}]}{\end{trivlist}}
% \newenvironment{lemma}[2][Lemma]{\begin{trivlist}
% \item[\hskip \labelsep {\bfseries #1}\hskip \labelsep {\bfseries #2.}]}{\end{trivlist}}
% \newenvironment{exercise}[2][Exercise]{\begin{trivlist}
% \item[\hskip \labelsep {\bfseries #1}\hskip \labelsep {\bfseries #2.}]}{\end{trivlist}}
% \newenvironment{question}[2][Question]{\begin{trivlist}
% \item[\hskip \labelsep {\bfseries #1}\hskip \labelsep {\bfseries #2.}]}{\end{trivlist}}
% \newenvironment{corollary}[2][Corollary]{\begin{trivlist}
% \item[\hskip \labelsep {\bfseries #1}\hskip \labelsep {\bfseries #2.}]}{\end{trivlist}}
\newenvironment{problem}[2][Problem\!]{\begin{trivlist}
\item[\hskip \labelsep {\bfseries #1}\hskip \labelsep {\bfseries #2}]}{\end{trivlist}}
%\newenvironment{sub-problem}[2][]{\begin{trivlist}
%\item[\hskip \labelsep {\bfseries #1}\hskip \labelsep {\bfseries #2}]}{\end{trivlist}}
\newenvironment{solution}{\begin{proof}[\textbf{\textit{Solution}}] }{\end{proof}}
%----------------------------------------------

%----------------------------
%User-defined notations
\newcommand{\zz}{\mathbb Z}   %blackboard bold Z
\newcommand{\qq}{\mathbb Q}   %blackboard bold Q
\newcommand{\ff}{\mathbb F}   %blackboard bold F
\newcommand{\rr}{\mathbb R}   %blackboard bold R
\newcommand{\nn}{\mathbb N}   %blackboard bold N
\newcommand{\cc}{\mathbb C}   %blackboard bold C
\newcommand{\af}{\mathbb A}   %blackboard bold A
\newcommand{\pp}{\mathbb P}   %blackboard bold P
\newcommand{\id}{\operatorname{id}} %for identity map
\newcommand{\im}{\operatorname{im}} %for image of a function
\newcommand{\dom}{\operatorname{dom}} %for domain of a function
\newcommand{\cat}[1]{\mathscr{#1}}   %calligraphic category
\newcommand{\abs}[1]{\left\lvert#1\right\rvert} %for absolute value
\newcommand{\norm}[1]{\left\lVert#1\right\rVert} %for norm
\newcommand{\modar}[1]{\text{ mod }{#1}} %for modular arithmetic
\newcommand{\set}[1]{\left\{#1\right\}} %for set
\newcommand{\setp}[2]{\left\{#1\ \middle|\ #2\right\}} %for set with a property
\newcommand{\card}[1]{\#\,{#1}} %for cardinality of a set
\newcommand\m[1]{\begin{pmatrix}#1\end{pmatrix}} 
\newcommand{\plog}{\operatorname{Log}}
\newcommand{\parg}{\operatorname{Arg}}
\newcommand{\conj}[1]{\overline{#1}}

%Re-defined notations
\renewcommand{\epsilon}{\varepsilon}
\renewcommand{\phi}{\varphi}
\renewcommand{\emptyset}{\varnothing}
\renewcommand{\geq}{\geqslant}
\renewcommand{\leq}{\leqslant}
\renewcommand{\Re}{\operatorname{Re}}
\renewcommand{\Im}{\operatorname{Im}}
%----------------------------

\allowdisplaybreaks

\newcommand{\tcr}[1]{\textcolor{red}{#1}}
\newcommand{\tcb}[1]{\textcolor{blue}{#1}}
\newcommand{\tco}[1]{\textcolor{orange}{#1}}

\newcommand{\lrp}[1]{\left(#1\right)}
\newcommand{\lrb}[1]{\left[#1\right]}
\newcommand{\lrc}[1]{\left\{#1\right\}}
 
 
\begin{document}
 
\title{Homework 5}
\author{Kevin Guillen\\[0.5em]
MATH 103A | Complex Analysis | Spring 2022}
\date{} 
\maketitle

%Use \[...\] instead of $$...$$

\begin{problem}{5.1}
Writing $z = re^{i\parg z}$, show that, where $n \in \zz_{>0}$
\[\log(z^{1/n}) = \frac{1}{n}\ln r + i\left(\frac{\parg z + 2(pn + k)\pi}{n}\right),\quad p\in\zz,\ k = 0,\ldots,n-1.\]
Now, after writing 
\[\frac{1}{n}\log z = \frac{1}{n}\ln r + i\left(\frac{\parg z + 2q\pi}{n}\right),\quad q \in \zz,\]
show that we have equality of sets
\[\log(z^{1/n}) = \frac{1}{n}\log z\]
\end{problem}
%----------------------------------------
\begin{proof}
    First recall from Example 9.9 that $z^{1/n} = e^{log(z)/n}$. So expanding we have,
    \begin{align*}
        \log(z^{1/n}) &= \log(e^{\log(z) / n}) && \text{Apply Prop. 9.3(2)} \\
        &= \dfrac{\log z}{n } + 2p\pi i  && p\in \zz \\
        &= \dfrac{1}{n}\ln r + \dfrac{i \parg z + 2k\pi i }{n} + 2p\pi i  && k\in \zz \\
        &= \dfrac{1}{n}\ln r + \dfrac{i\parg z + 2k\pi i }{n} + \dfrac{2p \pi i n}{n} \\
        &= \dfrac{1}{n}\ln r + i\lrp{\dfrac{\parg z + 2(pn + k) \pi}{n}} && p\in \zz, k\in \lrc{0, \dots , n-1}
    \end{align*}
    giving us the desired expression. 

    Now we work out the second expression,
    \begin{align*}
      \dfrac{1}{n}\log z &= \dfrac{1}{n} \lrp{\log r + i \parg z + 2q \pi i }  && q \in \zz\\
      &= \dfrac{1}{n}\ln r + i\lrp{\dfrac{\parg z + 2q\pi}{n}}.
    \end{align*}

    It is obvious that these two expressions are the exact same from the fact that we have division with remainder in $\zz$. Let the first set and second set be referred to as $A$ and $B$ respectively. We input a $z$ to generate the sets for some choice of $n$. We see that $q$ is any integer, now we argue that $pn +k$ can equal any $q$. If $q = 0$ then it is clear we have $pn + k = 0$ when  $p = 0$ and $k = 0$. In the case $q > 0$ then we can consider the set,
    \[P = \set{p\in \zz \mid pn \leq q}\] 
    we know that there is a maximal element in $P$, let us refer to it as $\alpha$. If $\alpha n = q$ then we are done, if not, we know that $\alpha n + k = q$, where $k = q- \alpha n  $. We know we have such a $k$ since it ranged from $0, \dots, n-1$, because if not that would mean $q -\alpha n > n-1$, but that would contradict the maximality of $\alpha$ since it was supposed to be the integer that obtains the greatest multiple of $n$ that is less than or equal to $q$. The case where $q < 0$ follows similarly except we would choose the minimal element and reverse the requirement for generating $P$. 
    
    So we have then that $pn + k = q$ where $q \in \zz$, therefore these two sets must be equal since there expression is the same. 
    
    

\end{proof}
%----------------------------------------

\newpage  %Do not delete

\begin{problem}{5.2}\hfill
\begin{itemize}[itemsep=3em]
\item[(a)] Find real valued functions $u,\, v : \rr^2 \to \rr$ such that $\cos z = u(x, y) + iv(x, y)$.
%----------------------------------------
\begin{solution}
  We know $z = x + iy$, now using the trigonometric identities given to us in class we have,
  \begin{align*}
    \cos(z) &= \cos(x + i y) \\
    &= \cos(x)\cos(i y) - \sin(x)\sin(i y)
  \end{align*}
  Now we can use Definition 10.5 and see,
  \begin{align*}
    \cos(i y) &= \dfrac{e^{i i y} + e^{- i i y}}{2}   & \sin(i y) &= \dfrac{e^{i i y} - e^{- i i y}}{2i} \\
    &= \dfrac{e^{-y} + e^{y}}{2} & &= -1 \dfrac{1}{i}\dfrac{e^{y} - e^{-y}}{2} \\
    &= \dfrac{e^{y} + e^{-y}}{2} & &= i \dfrac{e^{y} - e^{-y}}{2} \\
    &= \cosh(y) & &= i\sinh(y).
  \end{align*}
  All together now, we have that,
  \begin{align*}
    \cos(z) = \cos(x)\cosh(y) - i \sin(x)\sinh(y)
  \end{align*}
  meaning $u(x,y) = \cos(x)\cosh(y)$ and $v(x,y) = -\sin(x)\sinh(y)$, as desired.
\end{solution}
%----------------------------------------

\item[(b)] Show that $e^{\overline{z}} = \overline{e^z}$ and $\cos\overline{z} = \overline{\cos z}$.
%----------------------------------------
\begin{solution}
  We know $z = x + i y$. Now applying definitions from class we see,
  \begin{align*}
    \conj{e^{z}} = \conj{e^{x}}\cdot \conj{e^{i y}} &= e^{x}\conj{\cos y - i \sin y} \\
    &= e^{x}(\cos y + i\sin y)
  \end{align*}
  and 
  \begin{align*}
    e^{\conj z} = e^{x - i y} &= e^{x} \cdot e^{ -i y} \\
    &= e^{x} (\cos(-y) - i \sin(-y)) \\
    &= e^{x}( \cos y + i\sin y)
  \end{align*}
  giving us that $e^{\conj z} = \conj{e^{z}}$.

  Let $z$ be expressed as before, applying definitions we see,
  \begin{align*}
    \cos (\conj z) = \cos(\conj{x + i y}) &= \cos(x - i y) \\
    &= \cos(x)\cos(-i y) - \sin(x)\sin(-i y)\\
    &= \cos(x)\cos(i y) + \sin(x)\sin(i y)  && \text{apply (a)} \\
    &= \cos(x)\cosh(i y) + i\sin(x)\sinh(y)
  \end{align*}
  and
  \begin{align*}
    \conj{\cos z} = \conj{\cos(x + i y)} &= \conj{\cos(x)\cos(i y) - \sin(x)\sin(i y)} \\
    &= \conj{\cos(x)\cosh(y)-i\sin(x)\sinh(y)} \\
    &= \cos(x)\cosh(y) + i\sin(x)\sinh(y)
  \end{align*}
  giving us that $\cos(\conj z) = \conj{\cos z}$, as desired.
\end{solution}
%----------------------------------------

\item[(c)] Show that $e^{iz} = \cos z + i\sin z$, and prove $\sin\overline{z} = \overline{\sin z}$.
%----------------------------------------
\begin{solution}
  Let us apply the definitions to the RHS of the first equation,
  \begin{align*}
    \cos z + i \sin z &= \dfrac{e^{iz} + e^{-iz}}{2 } + i \dfrac{e^{i z} - e^{-i z}}{2i} \\
    &= \dfrac{e^{iz} +e^{-iz} +e^{iz} -e^{iz}}{2} \\
    &= \dfrac{e^{iz} + e^{iz}}{2} \\
    &= e^{i z}
  \end{align*}
  giving us the desired equality.

  Let $z = x + i y$, applying the definitions from class we see,
  \begin{align*}
    \sin \conj z = \sin(\conj{x + i y}) = \sin(x - i y) &= \sin(x)\cos(-i y ) + \cos(x)\sin(-i y) \\
    &= \sin(x)\cos(i y)- \cos(x)\sin(i y) \\
    &= \sin(x)\cosh( y) - i\cos(x)\sinh( y )
  \end{align*}
  and 
  \begin{align*}
    \conj{\sin z} = \conj{\sin(x + iy)} &= \conj{sin(x)\cos(i y) + \cos(x) \sin(i y)} \\
    &= \conj{\sin(x)\cosh(y) + i\cos(x)\sinh(y)} \\
    &= \sin(x)\cosh(y) - i\cos(x)\sinh(y)
  \end{align*}
  giving us that $\sin\conj z = \conj{\sin z}$.
\end{solution}
%----------------------------------------

\end{itemize}
\end{problem}

\newpage  %Do not delete

\begin{problem}{5.3}\hfill
\begin{itemize}[itemsep=3em]
\item[(a)] Verify that $(z^\alpha)^n = z^{n\alpha}$ for $z \neq 0$ and $n \in \zz$.
%----------------------------------------
\begin{solution}
  Using definitions given to use we have,
  \begin{align*}
    \lrp{z^{\alpha}}^{n} = \lrp{e^{\alpha \log z}}^{n} &= exp({n\log(e^{\alpha\log z})}) \\
    &= e^{n\alpha \log(z) + 2kn\pi i } && e \text{ is periodic} \\
    &= e^{n\alpha \log z} \\
    &= z^{n\alpha}
  \end{align*}
\end{solution}
%----------------------------------------

\item[(b)] Find a counterexample to the statement: $(z^{\alpha})^\beta = z^{\alpha\beta}$, where $z \neq 0$ and $\alpha,\beta \in \cc$.
%----------------------------------------
\begin{solution}
  Expanding the LHS,
  \begin{align*}
    \lrp{z^{\alpha}}^{\beta} = \lrp{e^{\alpha\log z}}^{\beta} &= e^{\beta\lrp{\alpha \log z + 2k\pi i }} \\
    &= e^{\beta\alpha\log z + \beta2k\pi i } \\
    &= e^{\beta\alpha \log z}e^{\beta 2 k \pi i } 
  \end{align*}
  expanding the RHS,
  \begin{align*}
    z^{\alpha \beta} = e^{\alpha\beta \log z}
  \end{align*}

  Now that we have them expanded out, let $\beta = (1 +i)$ and $\alpha = (1-i)$ and $z = i$. We see that $\beta\alpha = 2$, and $\log i = \ln 1 + i \arg i = i\frac{\pi}{2}$, giving us,
  \begin{align*}
    \lrp{z^{\alpha}}^{\beta} &= e^{i\pi}e^{(1 +i)2 k \pi  i } & z^{\alpha\beta} &= e^{i\pi } \\
    &= e^{i \pi}e^{-2\pi k} & &=e^{i\pi}
  \end{align*}
  we see that the only integer solution where $e^{-2\pi k} = 1$ is when $k = 0$, but for all other $k$, say specifically $k = 1$ we see,
  \[\lrp{z^{\alpha}}^{\beta} = e^{i\pi}e^{-2\pi} \neq e^{i\pi} = z^{\alpha\beta}\]
\end{solution}
%----------------------------------------

\end{itemize}
\end{problem}

\newpage  %Do not delete

\begin{problem}{5.4}
Determine the points at which the following functions are holomorphic. Find its derivative at those points.
\begin{itemize}[itemsep=3em]
\item[(a)] $f(z) = e^{\overline{z}}$.
%----------------------------------------
\begin{solution}
  As we have seen previously we have $e^{\conj z} = \conj{e^{z}}$. Let $z = x + i y$ and use use Definition 8.10 to expand and see,
  \begin{align*}
    \conj{e^{z}} = \conj{e^{x + i y}} &= \conj{e^{x}\cos y +i e^{x}  \sin y} \\
    &= \underbrace{e^{x}\cos y}_{u(x,y)} - i \underbrace{e^{x} \sin y}_{v(x,y)} .
  \end{align*}
  Now if $f(z)$ where to be holomorphic at some $z$ then it would need to satisfy the Cauchy-Riemann equations, but we see,
  \begin{align*}
    u_x = e^{x}\cos y \neq -e^{x}\cos y = v_y
  \end{align*}
  therefore $f$ is holomorphic nowhere. 

\end{solution}
%----------------------------------------

\item[(b)] $g(z) = \cos\overline{z}$.
%----------------------------------------
\begin{solution}
  As before we recall that we showed $\cos \conj z = \conj{\cos z}$. Now using what we know from problem 5.2 (a) we have,
  \begin{align*}
    \conj{\cos z} = \conj{\cos (x + i y)} &= \conj{\cos(x)\cosh(y) - i\sin(x)\sinh(y)} \\
    &=\cos(x)\cosh(y) + i\sin(x)\sinh(y).
  \end{align*}
  If $g(z)$ were to be holomorphic at some $z$, then it must satisfy the Cauchy-Riemann equations, we see though,
  \begin{align*}
    u_x = -\sin(x)\cosh(y) \neq \sin(x)\cosh(y) = v_y
  \end{align*}
  therefore $g$ is holomorphic nowhere. 
\end{solution}
%----------------------------------------

\item[(c)] $a(z) = \dfrac{\plog(2z - i)}{z^2 + 1}$.
%----------------------------------------
\begin{solution}
  Since this is the quotient of two functions, we can determine where it is holomorphic by determining where the numerator and denominator are holomorphic. We see for numerator is holomorphic everywhere except when $2z - i = -1$, so $z = \frac{1}{2} + i \frac{1}{2}$, we know from the notes that polynomials are entire, but since it is in the denominator we have to avoid $z^{2} + 1 = 0$, meaning $z = i$. All together we have that $a$ is holomorphic everywhere except $z = -1$ and $z = i$. Using the definitions given in the notes we have the derivative to be,
  \begin{align*}
    a(z)' = \lrp{\dfrac{1}{2z - i}(z^{2} + 1) - 2z\plog(2z -i)}\dfrac{1}{z^{4}+ 2z^{2} + 1}
  \end{align*}
\end{solution}
%----------------------------------------

\end{itemize}
\end{problem}

\newpage  %Do not delete

\begin{problem}{5.5}
Find all complex values $z$ satisfying the given equation.
\begin{itemize}[itemsep=3em]
\item[(a)] $\cos z = 4$
%----------------------------------------
\begin{solution}
  Using the trigonometric identity given to us we have,
  \begin{align*}
    \dfrac{e^{i z} + e^{-i z }}{2} &= 4 \\
    e^{i z} + e^{- i z} &= 8 \\
    \lrp{e^{i z}}^{2} -8e^{i z} + 1  &= 0
  \end{align*}
  applying the quadratic formula we derived in homework 1 we get,
  \[e^{iz} = \dfrac{8 \pm \sqrt{64 - 4}}{2} = \dfrac{8 \pm \sqrt{60}}{2} = \dfrac{8 \pm 2\sqrt{15}}{2} = 4 \pm \sqrt{15}\]
  so we have,
  \begin{align*}
    i z &= \ln(4 \pm \sqrt{15}) + 2k\pi i  && k\in \zz \\
    z &= 2k\pi i + i\ln(4\mp\sqrt{15} )
  \end{align*}
\end{solution}
%----------------------------------------

\item[(b)] $\cos z = i\,\sin z$
%----------------------------------------
\begin{solution}
  One of the given trigonometric identities is $\cos^{2}(z) + \sin^{2}(z) = 1$, now note the following,
  \begin{align*}
    \cos(z) &= i\sin(z) \\
    \cos^{2}(z) &= -\sin^2(z) \\
    \cos^{2}(z) +\sin^{2}(z) &= 0
  \end{align*}
  which yields a contradiction, meaning there are no solutions. 
\end{solution}
%----------------------------------------

\end{itemize}
\end{problem}

\newpage %do not delete

\begin{problem}{5.6}
Let $z \in \cc$.
\begin{itemize}[itemsep=3em]
\item[(a)] Prove that $\abs{1^z}$ is single-valued if and only if $\Im z = 0$.
%----------------------------------------
\begin{solution}
  ($\Rightarrow$) Assume that $\abs{1^{z}}$ is single valued, let us expand it out,
  \begin{align*}
    \abs{1^{z}} = \abs{e^{z\log 1}} &= \abs{e^{z(\ln 1 + i\arg 1)}} \\
    &= \abs{e^{z(\ln 1 + i2k\pi )}} \\
    &= \abs{e^{(x + i y) i2k\pi}} \\
    &= \abs{e^{-y2k\pi + i2k\pi x}} \\
    &= \abs{e^{-y2k\pi}}\cdot \underbrace{\abs{e^{i(2k\pi x)}}}_{\star}.
  \end{align*}
  Now let us apply 5.2 (c) to $\star$,
  \begin{align*}
    \abs{e^{i(2k\pi x)}} &= \abs{\cos(2k\pi x) + i \sin(2k\pi x)} \\
    &= \cos^{2}( 2k\pi x) + \sin^{2}( 2k\pi x) \\
    &= 1.
  \end{align*}
  So going back to our expansion we have,
  \begin{align*}
    \abs{1^{z}} = \abs{e^{- y2k\pi }} = e^{- y2k\pi }
  \end{align*}
  recall that $e$ is periodic only for $2k\pi i $, meaning as $k$ changes through $\zz$, $-y2k\pi$ will generate different values and therefore change $\abs{1^{z}}$, but our assumption is that $\abs{1^{z}}$ is single valued. Therefore it must be that $y = 0$, or $\Im z = 0$.

  ($\Leftarrow$) Assuming now that $\Im z = 0$, we have,
  \begin{align*}
    \abs{1^{z}} = \abs{e^{z \log 1}} &= \abs{e^{x(i2k\pi)}} \\
    &= \abs{e^{i2k\pi x}} \\
    &= \abs{\cos(2k\pi x) + i\sin(2k\pi x)} \\
    &= \cos^{2}(2k\pi x) + \sin^{2}(2k\pi x) \\
    &= 1
  \end{align*}
  show that $\abs{1^{z}}$ is single valued when $\Im z = 0$. 

  All together we have that $\abs{1^{z}}$ is single valued if and only if $\Im z = 0$.
\end{solution}
%----------------------------------------

\item[(b)] Find a necessary and sufficient condition for $\abs{i^{iz}}$ to be single-valued.
%----------------------------------------
\begin{solution}
  Let $z= x + i $, and now let us expand like before,
  \begin{align*}
    \abs{1^{z}} = \abs{e^{iz\log i }} &= \abs{e^{i z\lrp{i \frac{\pi}{2} + 2k\pi i}}} && k \in \zz \\
    &= \abs{e^{-z \frac{\pi}{2} -2k\pi z}} \\
    &= \abs{e^{-\frac{\pi}{2}x - i\frac{\pi}{2}y - 2k\pi x - i 2k\pi y}}  \\
    &= \abs{e^{-x (\frac{\pi}{2} + 2k\pi) -i(\frac{\pi}{2}y + 2k\pi y)}} \\
    &= \abs{e^{-x (\frac{\pi}{2} + 2k\pi)}}\cdot \underbrace{\abs{e^{i(-\frac{\pi}{2}y - 2k\pi y)}}}_{ 1} \\
    &= e^{-x (\frac{\pi}{2} + 2k\pi)}
  \end{align*}
  therefore $\abs{i^{iz}}$ is single valued only when $x = 0$, which is to say $\Re z = 0$
\end{solution}
%----------------------------------------

%----------------------------------------
%Delete if not attempted
%EXTRA CREDIT. OPTIONAL
\item[(c)] Show by counterexample that the statement is false: \[\text{$1^z$ is single-valued if and only if $\Im z = 0$.}\]
%----------------------------------------
\begin{solution}
  Let us expand as before,
  \begin{align*}
    1^{z} = e^{z\log 1} = e^{z2k\pi i }.
  \end{align*}
  Now let us choose $z = 1/2$, and notice that we get,
  \begin{align*}
    1^{z} = e^{k\pi i }
  \end{align*}
  and for $k = 3$ we get $1^{z} = -1$, while $k=4$ we get $1^{z} = 1$. Showing that $1^{z}$ is not necessarily single valued when $\Im z = 0$. 
\end{solution}
%EXTRA CREDIT. OPTIONAL
%Delete if not attempted
%----------------------------------------

\end{itemize}
\end{problem}



%----------------------------------------
%Delete if nothing to add
\newpage  %Do not delete

\begin{center}
\textbf{Collaborators:}
Peers at section on Wednesday. 
\end{center}
\vfill 


%----------------------------------------

\end{document}