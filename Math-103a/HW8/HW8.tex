\documentclass[11pt]{article}
 
\usepackage[top=0.75in, bottom=1.25in, left=1in, right=1in]{geometry} 
\usepackage{amsmath,amsthm,amssymb} %this is THE math package
\usepackage{mathtools}
\usepackage{tikz}
\usepackage{graphicx}
\usepackage{fancybox}
\usepackage{enumitem}
\usepackage{hyperref}
\usepackage{varwidth}
\usepackage{mdframed}
\usepackage{mathrsfs}
\usepackage[most]{tcolorbox}
%------------------------
%Fonts I use, uncomment if you like to use them.
%The first is the general font, and the second a math font
\usepackage{mathpazo}
\usepackage{eulervm}
%------------------------
%This is so that we have standard fonts for the double-stroked symbols
%for reals, naturals etc. regardless of what font you use.
%Don't comment
\AtBeginDocument{
  \DeclareSymbolFont{AMSb}{U}{msb}{m}{n}
  \DeclareSymbolFontAlphabet{\mathbb}{AMSb}}
%------------------------

%----------------------------------------------
%User-defined environments
%Commented because we're not using them in this document
%The only uncommented ones are the Problem and Solution environment

% \newenvironment{theorem}[2][Theorem]{\begin{trivlist}
% \item[\hskip \labelsep {\bfseries #1}\hskip \labelsep {\bfseries #2.}]}{\end{trivlist}}
% \newenvironment{lemma}[2][Lemma]{\begin{trivlist}
% \item[\hskip \labelsep {\bfseries #1}\hskip \labelsep {\bfseries #2.}]}{\end{trivlist}}
% \newenvironment{exercise}[2][Exercise]{\begin{trivlist}
% \item[\hskip \labelsep {\bfseries #1}\hskip \labelsep {\bfseries #2.}]}{\end{trivlist}}
% \newenvironment{question}[2][Question]{\begin{trivlist}
% \item[\hskip \labelsep {\bfseries #1}\hskip \labelsep {\bfseries #2.}]}{\end{trivlist}}
% \newenvironment{corollary}[2][Corollary]{\begin{trivlist}
% \item[\hskip \labelsep {\bfseries #1}\hskip \labelsep {\bfseries #2.}]}{\end{trivlist}}
\newenvironment{problem}[2][Problem\!]{\begin{trivlist}
\item[\hskip \labelsep {\bfseries #1}\hskip \labelsep {\bfseries #2}]}{\end{trivlist}}
%\newenvironment{sub-problem}[2][]{\begin{trivlist}
%\item[\hskip \labelsep {\bfseries #1}\hskip \labelsep {\bfseries #2}]}{\end{trivlist}}
\newenvironment{solution}{\begin{proof}[\textbf{\textit{Solution}}] }{\end{proof}}
%----------------------------------------------

%----------------------------
%User-defined notations
\newcommand{\zz}{\mathbb Z}   %blackboard bold Z
\newcommand{\qq}{\mathbb Q}   %blackboard bold Q
\newcommand{\ff}{\mathbb F}   %blackboard bold F
\newcommand{\rr}{\mathbb R}   %blackboard bold R
\newcommand{\nn}{\mathbb N}   %blackboard bold N
\newcommand{\cc}{\mathbb C}   %blackboard bold C
\newcommand{\af}{\mathbb A}   %blackboard bold A
\newcommand{\pp}{\mathbb P}   %blackboard bold P
\newcommand{\id}{\operatorname{id}} %for identity map
\newcommand{\im}{\operatorname{im}} %for image of a function
\newcommand{\dom}{\operatorname{dom}} %for domain of a function
\newcommand{\cat}[1]{\mathscr{#1}}   %calligraphic category
\newcommand{\abs}[1]{\left\lvert#1\right\rvert} %for absolute value
\newcommand{\norm}[1]{\left\lVert#1\right\rVert} %for norm
\newcommand{\modar}[1]{\text{ mod }{#1}} %for modular arithmetic
\newcommand{\set}[1]{\left\{#1\right\}} %for set
\newcommand{\setp}[2]{\left\{#1\ \middle|\ #2\right\}} %for set with a property
\newcommand{\card}[1]{\#\,{#1}} %for cardinality of a set
\newcommand\m[1]{\begin{pmatrix}#1\end{pmatrix}} 

%Re-defined notations
\renewcommand{\epsilon}{\varepsilon}
\renewcommand{\phi}{\varphi}
\renewcommand{\emptyset}{\varnothing}
\renewcommand{\geq}{\geqslant}
\renewcommand{\leq}{\leqslant}
\renewcommand{\Re}{\operatorname{Re}}
\renewcommand{\Im}{\operatorname{Im}}
%----------------------------

\allowdisplaybreaks

\newcommand{\tcr}[1]{\textcolor{red}{#1}}
\newcommand{\tcb}[1]{\textcolor{blue}{#1}}
\newcommand{\tco}[1]{\textcolor{orange}{#1}}

\newcommand{\lrp}[1]{\left(#1\right)}
\newcommand{\lrb}[1]{\left[#1\right]}
\newcommand{\lrc}[1]{\left\{#1\right\}}
\newcommand{\lrw}[1]{\left<#1\right>}
 
 
\begin{document}
 
\title{Homework 8}
\author{Kevin Guillen\\[0.5em]
MATH 103A | Complex Analysis | Spring 2022}
\date{} 
\maketitle

%Use \[...\] instead of $$...$$

\begin{problem}{8.1}\hfill
\begin{itemize}[itemsep=3em]
\item[(a)] Let $C$ denote the positively oriented boundary of the square whose sides lie along the lines $x = \pm 2$ and $y = \pm 2$. Compute
\[\int_C\,\frac{\cos z}{z(z^2 + 8)}\]
%----------------------------------------
\begin{solution}
    Let,
    \[f(z) = \dfrac{\cos z}{z^{2}+8}\] 
    we have then that,
    \begin{align*}
        \int_C\,\frac{\cos z}{z(z^2 + 8)}\, d z = \int_C\, \dfrac{f(z)}{z - 0}\, d z .
    \end{align*}
    Let $z_0 = 0$. Since $f(z)$ is holomorphic on $C$ and $z_0$ is in the interior of $C$, we have by Cauchy's Integral Formula that,
    \begin{align*}
        f(z_0) &= \dfrac{1}{2\pi i} \int_C \, \dfrac{f(z)}{z - z_0} \, d z \\
        \dfrac{1}{8} &= \dfrac{1}{2\pi i}\int_C \, \dfrac{f(z)}{z - z_0} \, d z  \\
        \dfrac{\pi i }{4} & = \int_C \, \dfrac{f(z)}{z - z_0} \, d z 
    \end{align*}
    therefore,
    \[\int_C\,\frac{\cos z}{z(z^2 + 8)} = \dfrac{\pi i}{4}.\]
\end{solution}
%----------------------------------------

\item[(b)] Let $C$ denote the circle centered at $i$ of radius $2$, positively oriented. Compute
\[\int_C\,\frac{1}{(z^2 + 4)^2}\]
%----------------------------------------
\begin{solution}
    Let us note that,
    \[(z^{2} + 4) = (z + 2i)(z - 2i).\]
    Now let $f(z) = \dfrac{1}{z + 2 i }$, we can rewrite the given integral as,
    \begin{align*}
        \int_C\,\frac{1}{(z^2 + 4)^2} &= \int_C\,\frac{1}{(z- 2 i)^{2} (2 +2 i)^{2}} \\
        &= \int_C\,\dfrac{f(z)}{(z - 2i)^{2}}\, d z .
    \end{align*}
    Let $z_0 = 2i$. As before, we know that $f(z)$ is holomorphic on the given contour since the place it is not holomorphic is when $z = -2i$ which is not in or part of the contour, and since $z_0$ is inside the contour, we can apply the generalization of Cauchy's Integral Formula to obtain,
    \begin{align}
        f'(z_0) &= \dfrac{1!}{2\pi i }\int_C\, \frac{f(z)}{(z - z_0)^{2}} 
    \end{align} 
    We obtain the derivative of $f$ to be,
    \begin{align*}
        f'(z) = -\dfrac{2}{(z + 2 i )^{3}}
    \end{align*}
    which lets us calculate the LHS of (1) to be,
    \begin{align*}
        f'(2i) = -\dfrac{2}{(4i)^{3}} = \dfrac{2}{64 i} = -\dfrac{i}{32}.
    \end{align*} 
    Solving the integral now in (1) we get,
    \begin{align*}
        -\dfrac{i}{32} &= \dfrac{1}{2\pi i}\int_C\, \frac{f(z)}{(z - z_0)^{2}} \\
        \dfrac{ \pi }{16} &= \int_C\, \frac{f(z)}{(z - z_0)^{2}}
    \end{align*}
    therefore,
    \begin{align*}
        \int_C\,\frac{1}{(z^2 + 4)^2} = \dfrac{\pi}{16}.
    \end{align*}
\end{solution}
%----------------------------------------

\end{itemize}
\end{problem}

\newpage  %Do not delete

\begin{problem}{8.2}
Let $C$ be the circle of radius $3$, positively oriented, centered at the origin. Show that if
\[g(w) = \int_C\, \frac{2z^2 - z - 2}{z - w}\,dz,\quad \abs{w} \neq 3,\]
then $g(2) = 8\pi i$. What is the value of $g(w)$ when $|w| > 3$?
\end{problem}
%----------------------------------------
\begin{solution}
    Let $f(z) = 2z^{2} -z - 2$, we rewrite $g$ now as,
    \begin{align*}
        g(w) = \int_C\, \frac{2z^2 - z - 2}{z - w}\,d z = \int_C\, \dfrac{f(z)}{z - w} \, d z 
    \end{align*}
    so evaluating at $w = 2$ we know $2$ is in the interior of $C$, and $f(z)$ is holomorphic on $C$, so we can applying Cauchy's Integral formula to obtain,
    \begin{align*}
        f(2) &= \dfrac{1}{2\pi i}\int_C \dfrac{2z^{2} - z -2}{z - 2 }\, d z  \\
        4 &= \dfrac{1}{2\pi i} \int_C \dfrac{2z^{2} - z -2}{z - 2 }\, d z \\
        8\pi i &=  \int_C \dfrac{2z^{2} - z -2}{z - 2 }\, d z
    \end{align*}.

    We have that $g(w)$ is holomorphic over $C$ for $\abs{w} > 3$, so by Cauchy-Goursat Theorem $g(w) = 0$ when $\abs{w} > 3$.
\end{solution}
%----------------------------------------

\newpage  %Do not delete

\begin{problem}{8.3}
Let $C$ be the unit circle parametrised as $z(t) = e^{it},\  -\pi \leq t \leq \pi$. First show that for any $a \in \rr$,
\[\int_C\, \frac{e^{az}}{z}\,dz = 2\pi i\]
Then write this integral in terms of $t$, using the definition of a contour integral, to derive the integration formula
\[\int_0^\pi\, e^{a\cos t}\cos(a\sin t)\,dt = \pi.\]
\end{problem}
%----------------------------------------
\begin{solution}
%Uncomment and WRITE YOUR SOLUTION HERE
\end{solution}
%----------------------------------------

\newpage  %Do not delete

\begin{problem}{8.4}
Let $f$ be an entire function such that there exists an $M > 0$ such that $\Re(f(z)) \geq M$ for all $z \in \cc$. Prove that $f$ is constant. 
\end{problem}
%----------------------------------------
\begin{solution}
%Uncomment and WRITE YOUR SOLUTION HERE
\end{solution}
%----------------------------------------

\newpage  %Do not delete

\begin{problem}{8.5}
Let $f$ be an entire function such that $\abs{f(z)} \leq A\abs{z}$ for all $z$, where $A$ is a fixed positive number. Show that $f(z) = \alpha z$, where $\alpha$ is a complex constant.
\end{problem}
%----------------------------------------
\begin{solution}
%Uncomment and WRITE YOUR SOLUTION HERE
\end{solution}
%----------------------------------------


%----------------------------------------
%Delete if nothing to add
\newpage  %Do not delete

\begin{center}
\textbf{Collaborators:}
%List your peers with whom you discussed the Problem Set
\end{center}
\vfill 

\begin{center}
\textbf{References:}
%List any book/website/notes that you used to write your solutions
\end{center}
\begin{itemize}
\item[$\bullet$] [Book(s): Title, Author]
\item[$\bullet$] [Online: \href{http://example.com/}{\color{blue}Link}]
\item[$\bullet$] [Notes: \href{http://example.com/}{\color{blue}Link}]
\end{itemize}

\vfill
\begin{center}
Fin.
\end{center}
\vfill

\end{document}