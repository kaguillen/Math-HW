\documentclass[11pt]{article}
%------------------------
%Packages
\usepackage[top=0.75in, bottom=1.25in, left=1in, right=1in]{geometry} 
\usepackage{amsmath,amsthm,amssymb} %this is THE math package
\usepackage{mathtools}
\usepackage{tikz}
\usepackage{graphicx}
\usepackage{fancybox}
\usepackage{enumitem}
\usepackage{hyperref}
\usepackage{varwidth}
\usepackage{mdframed}
\usepackage{mathrsfs}
\usepackage{lipsum}
\usepackage{xcolor}
%------------------------
%Fonts I use, uncomment if you like to use them.
%The first is the general font, and the second a math font
\usepackage{mathpazo}
\usepackage{eulervm}
\usepackage[most]{tcolorbox}
%------------------------
%This is so that we have standard fonts for the double-stroked symbols
%for reals, naturals etc. regardless of what font you use.
%Don't comment
\AtBeginDocument{
  \DeclareSymbolFont{AMSb}{U}{msb}{m}{n}
  \DeclareSymbolFontAlphabet{\mathbb}{AMSb}}
%------------------------

%----------------------------------------------
%User-defined environments
%Commented because we're not using them in this document
%The only uncommented ones are the Problem and Solution environment

% \newenvironment{theorem}[2][Theorem]{\begin{trivlist}
% \item[\hskip \labelsep {\bfseries #1}\hskip \labelsep {\bfseries #2.}]}{\end{trivlist}}
% \newenvironment{lemma}[2][Lemma]{\begin{trivlist}
% \item[\hskip \labelsep {\bfseries #1}\hskip \labelsep {\bfseries #2.}]}{\end{trivlist}}
% \newenvironment{exercise}[2][Exercise]{\begin{trivlist}
% \item[\hskip \labelsep {\bfseries #1}\hskip \labelsep {\bfseries #2.}]}{\end{trivlist}}
% \newenvironment{question}[2][Question]{\begin{trivlist}
% \item[\hskip \labelsep {\bfseries #1}\hskip \labelsep {\bfseries #2.}]}{\end{trivlist}}
% \newenvironment{corollary}[2][Corollary]{\begin{trivlist}
% \item[\hskip \labelsep {\bfseries #1}\hskip \labelsep {\bfseries #2.}]}{\end{trivlist}}
\newenvironment{problem}[2][Problem\!]{\begin{trivlist}
\item[\hskip \labelsep {\bfseries #1}\hskip \labelsep {\bfseries #2.}]}{\end{trivlist}}
%\newenvironment{sub-problem}[2][]{\begin{trivlist}
%\item[\hskip \labelsep {\bfseries #1}\hskip \labelsep {\bfseries #2}]}{\end{trivlist}}
\newenvironment{solution}{\begin{proof}[\textbf{\textit{Solution}}]}{\end{proof}}
%----------------------------------------------

%----------------------------
%User-defined notations
\newcommand{\zz}{\mathbb Z}   %blackboard bold Z
\newcommand{\qq}{\mathbb Q}   %blackboard bold Q
\newcommand{\ff}{\mathbb F}   %blackboard bold F
\newcommand{\rr}{\mathbb R}   %blackboard bold R
\newcommand{\nn}{\mathbb N}   %blackboard bold N
\newcommand{\cc}{\mathbb C}   %blackboard bold C
\newcommand{\af}{\mathbb A}   %blackboard bold A
\newcommand{\pp}{\mathbb P}   %blackboard bold P
\newcommand{\id}{\operatorname{id}} %for identity map
\newcommand{\im}{\operatorname{im}} %for image of a function
\newcommand{\dom}{\operatorname{dom}} %for domain of a function
\newcommand{\cat}[1]{\mathscr{#1}}   %calligraphic category
\newcommand{\abs}[1]{\left\lvert#1\right\rvert} %for absolute value
\newcommand{\norm}[1]{\left\lVert#1\right\rVert} %for norm
\newcommand{\modar}[1]{\text{ mod }{#1}} %for modular arithmetic
\newcommand{\set}[1]{\left\{#1\right\}} %for set
\newcommand{\setp}[2]{\left\{#1\ \middle|\ #2\right\}} %for set with a property
\newcommand{\card}[1]{\#\,{#1}} %for cardinality of a set
\newcommand{\lrp}[1]{\left(#1\right)}
\newcommand{\lrb}[1]{\left[#1\right]}
\newcommand{\lrc}[1]{\left\{#1\right\}}

%Re-defined notations
\renewcommand{\epsilon}{\varepsilon}
\renewcommand{\phi}{\varphi}
\renewcommand{\emptyset}{\varnothing}
\renewcommand{\geq}{\geqslant}
\renewcommand{\leq}{\leqslant}
\renewcommand{\Re}{\operatorname{Re}}
\renewcommand{\gcd}{\operatorname{GCD}}
\renewcommand{\Im}{\operatorname{Im}}
%----------------------------
\newtcolorbox[auto counter, number within=chapter]{example}[1][]{
    enhanced,
    breakable,
    left=0.5em, right=0pt, top=1pt, bottom=15pt,    
    attach boxed title to top left={yshift=-\tcboxedtitleheight},
     boxed title style={%
        empty,
        right=0pt,
        frame code={\draw[line width=2pt, gray] (frame.north west)--(frame.north east) --++ (0:1pt) ;}},
    before upper=\hspace{\tcboxedtitlewidth},
     colbacktitle=white,
    coltitle={white},
    colback={white},
    fonttitle={\bfseries},
    title={-},
    sharp corners,
    frame hidden,
    boxrule=0pt,
    borderline west={2pt}{0pt}{blue},
     overlay unbroken and last={%
        \draw[line width=2pt, red] (frame.south west)   -- ++(0:2cm);},
    #1
    }
\newcommand{\tcr}[1]{\textcolor{red}{#1}}
\newcommand{\tcb}[1]{\textcolor{blue}{#1}}
\newcommand{\tco}[1]{\textcolor{orange}{#1}}
    
\allowdisplaybreaks
 
\begin{document}
 
\title{Homework 1}
\author{Kevin Guillen\\[0.5em]
MATH 103A --- Complex Analysis --- Spring 2022}
\date{} 
\maketitle

%Use \[...\] instead of $$...$$

\begin{problem}{1.1}
Let $p(z) = az^2 + bz + c$ be a polynomial with complex coefficients ($a\neq 0$). 
\begin{itemize}[itemsep=2em]
\item[(a)] By completing the square, show that the solution to $p(z) = 0$ is
\[z = \frac{-b \pm \Delta^{1/2}}{2a},\]
where $\Delta \coloneqq b^2 - 4ac$ is called the discriminant.
\begin{example}
\begin{proof}
    First we want the leading coefficient to be 1, so we will divide by $a$ and continue from there.
    \begin{align*}
        z^{2} + \dfrac{b}{a}z + \dfrac{c}{a} &= 0 \\[.25em]
        z^{2} + \dfrac{b}{a}z + \lrp{\dfrac{b}{2a}}^{2} &= -\dfrac{c}{a} + \lrp{\dfrac{b}{2a}}^{2} \\[.25em]
        z^{2} + \dfrac{b}{a}z + \dfrac{b^{2}}{4a^{2}} &= -\dfrac{c}{a} + \dfrac{b^{2}}{4a^{2}} \\[.25em]
        \lrp{z + \dfrac{b}{2a}}^{2}&= \dfrac{b^{2} -4ac}{4a^{2}} = \dfrac{\Delta}{4a^{2}} \\
        \lrp{z + \dfrac{b}{2a}}^{2} - \lrp{\dfrac{\Delta^{1/2}}{2a}}^{2} &= 0 \\
        \lrp{z + \dfrac{b}{2a} - \dfrac{\Delta^{1/2}}{2a}} \lrp{z + \dfrac{b}{2a} + \dfrac{\Delta^{1/2}}{2a}} &= 0
    \end{align*}
    Now we have two solutions that we can solve for,
    \begin{align}
        \lrp{z + \dfrac{b - \Delta^{1/2}}{2a}} &= 0 \\
        \lrp{z +  \dfrac{b + \Delta^{1/2}}{2a}} &= 0
    \end{align}
    solving both (1) and (2) for $z$ we get,
    \begin{align*}
        z  & =  - \dfrac{b - \Delta^{1/2}}{2a} = \dfrac{-b + \Delta^{1/2}}{2a} \\
        z & = -  \dfrac{b + \Delta^{1/2}}{2a} =   \dfrac{-b - \Delta^{1/2}}{2a}
    \end{align*}
    which means the solution for $p(z) = 0$ is,
    \[z =  \dfrac{-b \pm \Delta^{1/2}}{2a}\]
    as desired. 
\end{proof} 

\end{example}

\vspace*{2em} %Do not delete


\item[(b)] Consider the polynomial $p(z) = iz^2 -1$
\begin{itemize}[itemsep=2em]
\item[(i)] Compute $\Delta$.
\begin{example}
    \begin{solution} 
        Given $p(z) = iz^{2} -1$ we see $a = i$, $b = 0$, and $c = -1$, plugging this in we get,
        \[\Delta = 0^{2} -4i(-1) = 4i\] 
    \end{solution}
\end{example}



\item[(ii)] For the $\Delta$ obtained in (b), compute $\Delta^{1/2}$ by solving a pair of simultaneous equations in $x$ and $y$ obtained by considering the equation \[(x + iy)^2 = \Delta.\]
\begin{example}     
    \begin{solution}
        Solving the provided equation we get,
        \begin{align*}
            (x + iy)^{2} &= \Delta \\
            (x+ iy)(x+iy) &= 4i \\ 
            x^{2} - y^{2} + 2xyi -4i &= 0 \\
            x^{2} -y^{2} + (2xy - 4)i &= 0 \\
        \end{align*}
        meaning we must have that,
        \begin{align}
            x^{2}-y^{2} &= 0 \\
            2xy - 4 & = 0
        \end{align}
        equation (3) has solutions $x = y$ and $x = -y$, but from (4) we see that their product must be positive, eliminating the $x=-y$ case. Now plugging in $x =y$ we get,
        \begin{align*}
            2y^{2} -4 &= 0 \\
            y^{2} &= 2 \\
            y &= \pm \sqrt2 
        \end{align*}
        so we have that $x = y = \pm \sqrt 2$.

    \end{solution} 
\end{example}
\item[(iii)] Finally, write down the roots of $p(z)$ in the form $u + iv$.
\begin{example}
    \begin{solution}
        The roots of $p(z)$ are $\sqrt2 + \sqrt2i$ and $-\sqrt2 - \sqrt2i$
    \end{solution} 
\end{example}
\end{itemize}
\end{itemize}
\end{problem}


\newpage
\begin{problem}{1.2}
Consider the set of matrices
\[X \coloneqq \setp{\begin{pmatrix}x & -y\\ y & x \end{pmatrix}}{x,y \in \rr}.\]
\begin{itemize}
\item[(a)] Let $\cc$ denote the set of complex numbers. Show that the map $\phi: X \to \cc$ defined by 
\[\phi:X \to \cc,\quad \begin{pmatrix}x & -y\\ y & x \end{pmatrix} \mapsto x + iy\]
is a bijection.
\begin{example}
    \begin{proof}
        First we will show this mapping is \textbf{injective}. Let $A,B\in X$, then they are of the form $\begin{pmatrix}
            a & -b \\ b & a
        \end{pmatrix}$ and $\begin{pmatrix}
            c & -d \\ d & c
        \end{pmatrix}$ respectively. So if their image under $\phi$ were to be equal, that would mean, 
        \begin{align*}
            \phi(A) &= \phi(B) \\
            \phi\lrp{\begin{pmatrix}
            a & -b \\ b & a
        \end{pmatrix}} &= \phi\lrp{\begin{pmatrix}
            c & -d \\ d & c
        \end{pmatrix}} \\
            a + bi &= c + di \\
            (a-c) + (b - d)i &= 0
        \end{align*}
        This can only be satisfied if $a = c$ and $b = d$, meaning $A = B$. Therefore $\phi$ is \textbf{injective}. 

        Now we will show $\phi$ is \textbf{surjective}. If we have any $z \in \cc$, it is of the form $a + bi$ where $a,b \in \rr$. Matrices in $X$ are matrices with real numbers as entries, so we have $\begin{pmatrix} a & -b \\ b & a\end{pmatrix}\in X$ and its image under $\phi$ is,
        \begin{align*}
            \phi\lrp{\begin{pmatrix} a & -b \\ b & a\end{pmatrix}} = a +bi = z 
        \end{align*}
        showing $\phi$ is \textbf{surjective}. All together then we have shown $\phi$ is \textbf{bijective}, as desired. 
    \end{proof}
\end{example} 

\item[(b)] Let $I$ be the identity matrix. Consider $A,B \in X$, show that $\phi$ has the following properties.
\begin{itemize}
\item[(i)] $\phi(A+B) = \phi(A)+\phi(B)$
\begin{example}
    \begin{proof}
        As before we know $A$ and $B$ are of the form $\begin{pmatrix}
            a & -b \\ b & a
        \end{pmatrix}$ and $\begin{pmatrix}
            c & -d \\ d & c
        \end{pmatrix}$ respectively. We see through the following that $\phi$ is additive,
        \begin{align*}
            \phi(A + B) &= \phi\lrp{\begin{pmatrix}
                a & -b \\ b & a
            \end{pmatrix} +\begin{pmatrix}
                c & -d \\ d & c
            \end{pmatrix}} \\
            &= \phi\lrp{\begin{pmatrix}
                a + c & -b - d \\ b + d & a + c
            \end{pmatrix}} \\
            &= (a+c) + (b+d)i \\
            &= a + bi + c + di \\
            &= \phi(A) + \phi(B).
        \end{align*}
    \end{proof}
\end{example} 

\item[(ii)] $\phi(AB) = \phi(A)\phi(B)$
\begin{example}
    \begin{proof}
        Let $A$ and $B$ be as before, we see through the following that $\phi$ is multiplicative,
        \begin{align*}
            \phi(AB) &= \phi\lrp{\begin{pmatrix}
                a & -b \\ b & a
            \end{pmatrix} \begin{pmatrix}
                c & -d \\ d & c
            \end{pmatrix}} \\
            &=\phi\lrp{\begin{pmatrix}ac -bd & -ad-bc \\ bc + ad & -bd + ac  \end{pmatrix}} \\
            &= (ac-bd) + (bc+ad)i \\
            &= (a+bi)(c+di) \\
            &= \phi(A)\phi(B).
        \end{align*}
    \end{proof}
\end{example} 

\item[(iii)] $\phi(I) = 1$
\begin{example}
    \begin{proof}
        Recall that $I$ is simply $\begin{pmatrix}
            1 & 0 \\ 0 & 1
        \end{pmatrix}$
        and we see through the following that the identity matrix is mapped to 1,
        \begin{align}
            \phi\lrp{\begin{pmatrix}1 & 0 \\ 0 & 1\end{pmatrix}} = 1 + 0i = 1.
        \end{align}
    \end{proof}
\end{example} 

\end{itemize}

\item[(c)] Find a matrix $J$ satisfying $J^2 = -I$ and show that $\phi(J) = i$.
\begin{example}
    \begin{proof}
        We need to find a matrix $J$ satisfying $J^{2} = -I$,so writing this out explicitly we get,
        \begin{align*}
            \begin{pmatrix}
                a & -b \\ b & a
            \end{pmatrix}\begin{pmatrix}
                a & -b \\ b & a
            \end{pmatrix} &= \begin{pmatrix}-1 & 0 \\ 0 & -1 \end{pmatrix} \\\begin{pmatrix}
                a^{2} -b^{2} & -2ab \\ 2ab & a^{2}-b^{2}
            \end{pmatrix} &=\begin{pmatrix}-1 & 0 \\ 0 & -1 \end{pmatrix}
        \end{align*}
        giving us that $b = 1$ and $a = 0$. Therefore $J = \begin{pmatrix}
            0 & -1 \\ 1 & 0
        \end{pmatrix}$ and we see
        \begin{align*}
            \phi(J) = 0 + 1i = i 
        \end{align*}
        as desired. 
    \end{proof}
\end{example}
\end{itemize}

\end{problem}


\newpage  %Do not delete

\begin{problem}{1.3}
Let $z,w \in \cc$. 
\begin{itemize}[itemsep=3em]
\item[(a)] Prove the formula
\[\abs{z+w}^2 = \abs{z}^2 + 2 \Re z\overline{w} + \abs{w}^2\]
%----------------------------------------
\begin{example}
    \begin{proof}
        Since $z,w$ are elements of $\cc$ they are each of the form $a+bi$ and $c+di$ respectively. First it would be useful to see what $\text{Re}(z\bar{w})$ is,
        \begin{align*}
            z\hat{w} &= (a+bi)(c-di) \\
            &= ac + bd - (ad)i +(cb)i  \\
            &= (ac + bd) + (cb -ad)i
        \end{align*}
        from this we see Re$(z\bar{w}) = ac + bd$. So let us work out the LHS,
        \begin{align*}
            \abs{z +w}^{2} &= \abs{a + bi + c + di}^{2} \\
            &= \sqrt{(a+c)^{2} + (b+d)^{2}}^{2} \\
            &= (a+c)^{2} + (b+d)^{2} \\
            &= a^{2} + c^{2} + 2ac + b^{2} + d^{2} + 2bd \\
            &= \tcb{a^{2}} + \tcb{b^{2}} + \tco{2ac} + \tco{2bd}+ \tcr{c^{2}} + \tcr{d^{2}}  \\
            &= \tcb{\abs{z}^{2}} + \tco{2(ac + bd)} + \tcr{\abs{w}^{2}} \\
            &= \abs{z}^{2} + 2\text{Re}(z\bar{w}) + \abs{w}^{2}
        \end{align*}
        which proves the formula. 
    \end{proof}
\end{example}
%----------------------------------------


\item[(b)] Use (a) to deduce the \emph{parallelogram law}
\[\abs{z+w}^2 + \abs{z-w}^2 = 2\abs{z}^2 + 2\abs{w}^2\]
Give a geometric interpretation of this formula. 
%----------------------------------------
\begin{example}
    \begin{proof}
        Let $z$ and $w$ be as before, now we will work out the LHS,
        \begin{align}
            \abs{z + w}^{2} + \abs{z + (-w)}^{2} = \abs{z}^{2} + 2\text{Re}(z\bar{w}) + \abs{w}^{2} + \abs{z}^{2} + 2\text{Re}(z(\overline{-w})) + \abs{-w}^{2}
        \end{align}
        We already worked out what Re$(z\bar{w})$ is, so let us work out Re$(z(\overline{-w}))$ now,
        \begin{align*}
            z(\bar{-w}) = (a+bi)(-c+di) &= -ac + (ad)i -(bc)i -bd \\
            &= (-ac -bd) + (ad -bc)i
        \end{align*}
        so Re$(z(\overline{-w})) = -(ac+bd)$. We also recall the fact from class that the modulus is invariant to the sign of the inputted element. All this together we can continue with (6),
        \begin{align*}
            \abs{z + w}^{2} + \abs{z + (-w)}^{2} &= \abs{z}^{2} + 2(ac + bd) + \abs{w}^{2} + \abs{z}^{2} -2(ac +bd) + \abs{w}^{2} \\
            &=\abs{z}^{2} + \abs{z}^{2} +2(ac+bd) -2(ac + bd) + \abs{w}^{2} + \abs{w}^{2} \\
            &= 2\abs{z}^{2} + 2\abs{w}^{2}
        \end{align*}
        which deduces the parallelogram law. 

        What this is saying geometrically is, if we sum the squares of the parallelogram's two diagonals length, it will be the same as if we square the lengths of the 4 sides and sum it all up. 
    \end{proof}
\end{example}
%----------------------------------------

\end{itemize}

\end{problem}

---------------
%Delete if nothing to add
\newpage  %Do not delete

\begin{center}
\textbf{Collaborators:}
 I don't recall their names, but I worked on 1.1 and 1.2 with my assigned group from section on Wednesday 4/06. 
\end{center}

%----------------------------------------

\end{document}