\documentclass[11pt]{article}
%------------------------
%Packages
\usepackage[top=0.75in, bottom=1.25in, left=1in, right=1in]{geometry} 
\usepackage{amsmath,amsthm,amssymb} %this is THE math package
\usepackage{mathtools}
\usepackage{tikz}
\usepackage{graphicx}
\usepackage{fancybox}
\usepackage{enumitem}
\usepackage{hyperref}
\usepackage{varwidth}
\usepackage{mdframed}
\usepackage{mathrsfs}
\usepackage{xcolor}
%------------------------
%Fonts I use, uncomment if you like to use them.
%The first is the general font, and the second a math font
\usepackage{mathpazo}
\definecolor{darkblue}{rgb}{0.0,0.0,.7}
%------------------------
%This is so that we have standard fonts for the doublestroked symbols
%for reals, naturals etc. regardless of what font you use.
%Don't comment
\AtBeginDocument{
  \DeclareSymbolFont{AMSb}{U}{msb}{m}{n}
  \DeclareSymbolFontAlphabet{\mathbb}{AMSb}}

%----------------------------------------------
%User-defined environments
%Commented because we're not using them in this document
%The only uncommented ones are the Problem and Solution environment

% \newenvironment{theorem}[2][Theorem]{\begin{trivlist}
% \item[\hskip \labelsep {\bfseries #1}\hskip \labelsep {\bfseries #2.}]}{\end{trivlist}}
% \newenvironment{lemma}[2][Lemma]{\begin{trivlist}
% \item[\hskip \labelsep {\bfseries #1}\hskip \labelsep {\bfseries #2.}]}{\end{trivlist}}
% \newenvironment{exercise}[2][Exercise]{\begin{trivlist}
% \item[\hskip \labelsep {\bfseries #1}\hskip \labelsep {\bfseries #2.}]}{\end{trivlist}}
% \newenvironment{question}[2][Question]{\begin{trivlist}
% \item[\hskip \labelsep {\bfseries #1}\hskip \labelsep {\bfseries #2.}]}{\end{trivlist}}
% \newenvironment{corollary}[2][Corollary]{\begin{trivlist}
% \item[\hskip \labelsep {\bfseries #1}\hskip \labelsep {\bfseries #2.}]}{\end{trivlist}}
\newenvironment{problem}[2][Problem\!]{\begin{trivlist}
\item[\hskip \labelsep {\bfseries #1}\hskip \labelsep {\bfseries #2.}]}{\end{trivlist}}
%\newenvironment{sub-problem}[2][]{\begin{trivlist}
%\item[\hskip \labelsep {\bfseries #1}\hskip \labelsep {\bfseries #2}]}{\end{trivlist}}
\newenvironment{solution}{\begin{proof}[\textbf{\textit{Solution}}]}{\end{proof}}
%----------------------------------------------

%----------------------------
%User-defined notations
\newcommand{\zz}{\mathbf Z}   %blackboard bold Z
\newcommand{\qq}{\mathbf Q}   %blackboard bold Q
\newcommand{\ff}{\mathbf F}   %blackboard bold F
\newcommand{\rr}{\mathbf R}   %blackboard bold R
\newcommand{\nn}{\mathbf N}   %blackboard bold N
\newcommand{\cc}{\mathbf C}   %blackboard bold C
\newcommand{\af}{\mathbf A}   %blackboard bold A
\newcommand{\pp}{\mathbf P}   %blackboard bold P
\newcommand{\id}{\operatorname{id}} %for identity map
\newcommand{\parg}{\operatorname{Arg}} %principal argument
\newcommand{\plog}{\operatorname{Log}} %principal log
\newcommand{\im}{\operatorname{im}} %for image of a function
\newcommand{\dom}{\operatorname{dom}} %for domain of a function
\newcommand{\cat}[1]{\mathscr{#1}}   %calligraphic category
\newcommand{\abs}[1]{\left\lvert#1\right\rvert} %for absolute value
\newcommand{\norm}[1]{\left\lVert#1\right\rVert} %for norm
\newcommand{\modar}[1]{\text{ mod }{#1}} %for modular arithmetic
\newcommand{\set}[1]{\left\{#1\right\}} %for set
\newcommand{\setp}[2]{\left\{#1\ \middle|\ #2\right\}} %for set with a property
\newcommand{\card}[1]{\#\,{#1}} %for cardinality of a set

%Re-defined notations
\renewcommand{\epsilon}{\varepsilon}
\renewcommand{\phi}{\varphi}
\renewcommand{\emptyset}{\varnothing}
\renewcommand{\geq}{\geqslant}
\renewcommand{\leq}{\leqslant}
\renewcommand{\Re}{\operatorname{Re}}
\renewcommand{\gcd}{\operatorname{GCD}}
\renewcommand{\Im}{\operatorname{Im}}
%----------------------------

\allowdisplaybreaks
 
\begin{document}
 
\title{Homework 6}
\author{[Your Full Name Here]\\[0.5em]
MATH 103A --- Complex Analysis --- Spring 2022}
\date{} 
\maketitle

%Use \[...\] instead of $$...$$

\begin{problem}{6.1}
Let $z : [a, b] \to \cc$ be a parameterization of a smooth arc $C$ and suppose $f(z)$ is holomorphic at a point $z_0 = z(t_0)$ on $\cc$. Show that
\[(f \circ z)'(t_0) = f'(z(t_0))\,z'(t_0)\]
\end{problem}
%----------------------------------------
\begin{solution}
%Uncomment and WRITE YOUR SOLUTION HERE
\end{solution}
%----------------------------------------

\newpage  %Do not delete

\begin{problem}{6.2}
Let $\alpha, \beta \in \rr$. Evaluate the following integral of real-valued functions
\[\int_0^\pi e^{\alpha x}\cos\beta x\ dx \quad \text{and} \quad \int_0^\pi  e^{\alpha x}\sin\beta x\ dx\]
simultaneously by computing a \emph{single} integral of a complex-valued function.
\end{problem}
%----------------------------------------
\begin{solution}
%Uncomment and WRITE YOUR SOLUTION HERE
\end{solution}
%----------------------------------------

\newpage  %Do not delete

\begin{problem}{6.3}
Let $z_1, z_2 \in \cc$. Compute the integral
\[\int_C dz = \int_C 1\,dz\]
where $C$ is any contour joining $z_1$ to $z_2$.
\end{problem}
%----------------------------------------
\begin{solution}
%Uncomment and WRITE YOUR SOLUTION HERE
\end{solution}
%----------------------------------------

\newpage  %Do not delete

\begin{problem}{6.4}
Let $C$ denote the unit circle with positive orientation. Compute the integral
\[\frac{1}{2\pi i}\int_C \frac{(1 + z)^n}{z^{k+1}}\,dz\]
for any integers $0 \leq k \leq n$.
\end{problem}
%----------------------------------------
\begin{solution}
%Uncomment and WRITE YOUR SOLUTION HERE
\end{solution}
%----------------------------------------

\newpage  %Do not delete

\begin{problem}{6.5}
Integrate the function $f(z) = \overline{z}$ over the following contours:
\begin{itemize}[itemsep=3em]
\item[(a)] $C_1$: the line segment joining $0$ to $1 + i$;
%----------------------------------------
\begin{solution}
%Uncomment and WRITE YOUR SOLUTION HERE
\end{solution}
%----------------------------------------

\item[(b)] $C_2$: the line segment joining $0$ to $1$, following by the line segment joining $1$ to $1 + i$.
%----------------------------------------
\begin{solution}
%Uncomment and WRITE YOUR SOLUTION HERE
\end{solution}
%----------------------------------------

\end{itemize}
\end{problem}


%----------------------------------------
%Delete if nothing to add
\newpage  %Do not delete

\begin{center}
\textbf{Collaborators:}
%List your peers with whom you discussed the Problem Set
\end{center}
\vfill 

\begin{center}
\textbf{References:}
%List any book/website/notes that you used to write your solutions
\end{center}
\begin{itemize}
\item[$\bullet$] [Book(s): Title, Author]
\item[$\bullet$] [Online: \href{http://example.com/}{\color{blue}Link}]
\item[$\bullet$] [Notes: \href{http://example.com/}{\color{blue}Link}]
\end{itemize}

\vfill
\begin{center}
Fin.
\end{center}
\vfill
%----------------------------------------

\end{document}