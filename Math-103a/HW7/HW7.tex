\documentclass[11pt]{article}
 
\usepackage[top=0.75in, bottom=1.25in, left=1in, right=1in]{geometry} 
\usepackage{amsmath,amsthm,amssymb} %this is THE math package
\usepackage{mathtools}
\usepackage{tikz}
\usepackage{graphicx}
\usepackage{fancybox}
\usepackage{enumitem}
\usepackage{hyperref}
\usepackage{varwidth}
\usepackage{mdframed}
\usepackage{mathrsfs}
\usepackage[most]{tcolorbox}
%------------------------
%Fonts I use, uncomment if you like to use them.
%The first is the general font, and the second a math font
\usepackage{mathpazo}
\usepackage{eulervm}
%------------------------
%This is so that we have standard fonts for the double-stroked symbols
%for reals, naturals etc. regardless of what font you use.
%Don't comment
\AtBeginDocument{
  \DeclareSymbolFont{AMSb}{U}{msb}{m}{n}
  \DeclareSymbolFontAlphabet{\mathbb}{AMSb}}
%------------------------

%----------------------------------------------
%User-defined environments
%Commented because we're not using them in this document
%The only uncommented ones are the Problem and Solution environment

% \newenvironment{theorem}[2][Theorem]{\begin{trivlist}
% \item[\hskip \labelsep {\bfseries #1}\hskip \labelsep {\bfseries #2.}]}{\end{trivlist}}
% \newenvironment{lemma}[2][Lemma]{\begin{trivlist}
% \item[\hskip \labelsep {\bfseries #1}\hskip \labelsep {\bfseries #2.}]}{\end{trivlist}}
% \newenvironment{exercise}[2][Exercise]{\begin{trivlist}
% \item[\hskip \labelsep {\bfseries #1}\hskip \labelsep {\bfseries #2.}]}{\end{trivlist}}
% \newenvironment{question}[2][Question]{\begin{trivlist}
% \item[\hskip \labelsep {\bfseries #1}\hskip \labelsep {\bfseries #2.}]}{\end{trivlist}}
% \newenvironment{corollary}[2][Corollary]{\begin{trivlist}
% \item[\hskip \labelsep {\bfseries #1}\hskip \labelsep {\bfseries #2.}]}{\end{trivlist}}
\newenvironment{problem}[2][Problem\!]{\begin{trivlist}
\item[\hskip \labelsep {\bfseries #1}\hskip \labelsep {\bfseries #2}]}{\end{trivlist}}
%\newenvironment{sub-problem}[2][]{\begin{trivlist}
%\item[\hskip \labelsep {\bfseries #1}\hskip \labelsep {\bfseries #2}]}{\end{trivlist}}
\newenvironment{solution}{\begin{proof}[\textbf{\textit{Solution}}] }{\end{proof}}
%----------------------------------------------

%----------------------------
%User-defined notations
\newcommand{\zz}{\mathbb Z}   %blackboard bold Z
\newcommand{\qq}{\mathbb Q}   %blackboard bold Q
\newcommand{\ff}{\mathbb F}   %blackboard bold F
\newcommand{\rr}{\mathbb R}   %blackboard bold R
\newcommand{\nn}{\mathbb N}   %blackboard bold N
\newcommand{\cc}{\mathbb C}   %blackboard bold C
\newcommand{\af}{\mathbb A}   %blackboard bold A
\newcommand{\pp}{\mathbb P}   %blackboard bold P
\newcommand{\id}{\operatorname{id}} %for identity map
\newcommand{\im}{\operatorname{im}} %for image of a function
\newcommand{\dom}{\operatorname{dom}} %for domain of a function
\newcommand{\cat}[1]{\mathscr{#1}}   %calligraphic category
\newcommand{\abs}[1]{\left\lvert#1\right\rvert} %for absolute value
\newcommand{\norm}[1]{\left\lVert#1\right\rVert} %for norm
\newcommand{\modar}[1]{\text{ mod }{#1}} %for modular arithmetic
\newcommand{\set}[1]{\left\{#1\right\}} %for set
\newcommand{\setp}[2]{\left\{#1\ \middle|\ #2\right\}} %for set with a property
\newcommand{\card}[1]{\#\,{#1}} %for cardinality of a set
\newcommand\m[1]{\begin{pmatrix}#1\end{pmatrix}} 

%Re-defined notations
\renewcommand{\epsilon}{\varepsilon}
\renewcommand{\phi}{\varphi}
\renewcommand{\emptyset}{\varnothing}
\renewcommand{\geq}{\geqslant}
\renewcommand{\leq}{\leqslant}
\renewcommand{\Re}{\operatorname{Re}}
\renewcommand{\Im}{\operatorname{Im}}
%----------------------------

\allowdisplaybreaks

\newcommand{\tcr}[1]{\textcolor{red}{#1}}
\newcommand{\tcb}[1]{\textcolor{blue}{#1}}
\newcommand{\tco}[1]{\textcolor{orange}{#1}}

\newcommand{\lrp}[1]{\left(#1\right)}
\newcommand{\lrb}[1]{\left[#1\right]}
\newcommand{\lrc}[1]{\left\{#1\right\}}
\newcommand{\lrw}[1]{\left<#1\right>}
 
 
\begin{document}
 
\title{Homework 7}
\author{Kevin Guillen\\[0.5em]
MATH 103A | Complex Analysis | Spring 2022}
\date{} 
\maketitle

%Use \[...\] instead of $$...$$

\begin{problem}{7.1}
Let 
\[f(z) = \frac{z^2+2}{(z^2+3)(z^2 + 2z + 1)}\]
and let $C_R$ denote the semicircle of radius $R$ parameterized by $z(t) = Re^{it}$ with $t \in [0,\pi]$. Show that
\[\lim_{R\to \infty} \int_{C_R}f(z) \, dz = 0.\]
\end{problem}
%----------------------------------------
\begin{solution}
  First note that for any $z \in C_R$ we have that $\abs{z} = R$. Now let us see a chain of inequalities for all the polynomials that make up $f(z)$. For $\abs{z^{2} + 2}$ we can apply triangle inequality to obtain,
  \begin{align*}
    \abs{z^{2} + 2}  \leq \abs{z}^{2} + 2= R^{2} + 2
  \end{align*}

  now for $\abs{z^{2} + 3}$ we can apply reverse triangle inequality to obtain,
  \begin{align*}
    \abs{z^{2} + 3} \geq \abs{\abs{z}^{2} - \abs{3}} = \abs{R^{2} -3} = R^{2} -3
  \end{align*}

  last but not least,
  \begin{align*}
    \abs{z^{2} + 2z + 1} = \abs{(z + 1)^{2}} &= \abs{z + 1}^{2} && \text{apply reverse triangle inequality}  \\
    &\geq \abs{\abs{z} + \abs{1}}^{2} \\
    &= \abs{R - 1}^{2} \\
    &= (R-1)^{2}. 
  \end{align*}

  We apply all of these to obtain,
  \begin{align*}
    \abs{\int_{C_R} \dfrac{z^{2} + 2}{(z^{2} + 3)(z^{2} + 2z + 1)}} \leq \dfrac{R^{2} + 2}{(R^{2} - 3)(R-1)^{2}}R\pi \to 0 \text{ as } R \to 0
  \end{align*}
  then by the Squeeze Theorem we have,
  \[\lim_{R\to \infty} \int_{C_R}f(z) \, dz = 0\]
  as desired. 
\end{solution}
%----------------------------------------

\newpage  %Do not delete

\begin{problem}{7.2}
Let $C$ be a positively oriented simply closed contour and let $R$ be the region consisting of $C$ and its interior.
\begin{itemize}[itemsep=3em]
\item[(a)] Show that the area $A$ of the region $R$ is given by the formula \[A = \frac{1}{2i} \int_C \overline{z} \, dz.\]
%----------------------------------------
\begin{proof}
  We know the area of the region $R$ to be $\int\int_R dx dy $ which is what want to show the formula gives. Now observe the following,
  \begin{align*}
    \dfrac{1}{2i}\int_C \overline{z}\ d z &= \dfrac{1}{2i}\int_C (x - i y)(dx + i d y )  \\
    &= \dfrac{1}{2i}\int_C x dx + i x d y - i y dx + y d y \\
    &= \dfrac{1}{2i}\int_C \underbrace{(x - i y)}_{P(x,y)}d x + \underbrace{(y + i x )}_{Q(x,y)}d y  
  \end{align*}
  we pause here to note that $P(x,y)$ and $Q(x,y)$ both have continuous partial derivatives, which lets us apply Greene's Theorem to get,
  \begin{align*}
    \dfrac{1}{2i}\int_C \overline{z}d z &= \dfrac{1}{2i}\int\int_R(Q_x - P_y)\ dA \\
    &= \dfrac{1}{2i}\int\int_R(i - -i)\ dx d y \\
    &= \dfrac{1}{2i}\int\int_R 2i\ dx d y \\
    &= \dfrac{2i}{2i} \int\int_R dx d y  \\
    &= \int\int_R dx d y
  \end{align*}
  giving us the area of the region $R$ as desired. 
\end{proof}
%----------------------------------------
\newpage
\item[(b)] Compute the area $A$ of the region enclosed by the \emph{cardioid} $C$ with parameterization
\[z(t) = \frac{1}{2} + e^{it} + \frac{1}{2}e^{2it},\quad 0 \leq t \leq 2\pi.\] 
%----------------------------------------
\begin{solution}
  Using the given parameterization we have,
  \[dz = ie^{it} + ie^{2it}\, dt.\]
  Now we use the formula derived above to compute the area,
  \begin{align*}
    A = \dfrac{1}{2i}\int_C \overline{z}\, d z &= \dfrac{1}{2i}\int_C  \lrp{\dfrac{1}{2} + e^{-it} + \dfrac{1}{2i}e^{-2it}}\lrp{ie^{it} + ie^{2i t}}\, d t \\
    &= \dfrac{1}{2i}\int_0^{2\pi} \dfrac{1}{2}ie^{it} + i + \dfrac{1}{2}i + \dfrac{1}{2}i e^{2 i t} + ie^{i t} + \dfrac{1}{2}i e^{i t} \, d t \\
    &= \dfrac{1}{2i}\int_0^{2\pi} \dfrac{1}{2}ie^{2it} + 2ie^{it} + \dfrac{3}{2}i \, d t \\
    &= \dfrac{1}{2i}\lrb{\dfrac{1}{4}e^{it} + 2e^{it} + \dfrac{3}{2}it }_0^{2\pi} \\
    &= \dfrac{1}{2i}\lrp{\dfrac{1}{4} + 2 + 3\pi i - \dfrac{1}{4} - 2 + 0 } \\
    &= \dfrac{1}{2i}(3\pi i) \\
    &= \dfrac{3\pi}{2}.
  \end{align*}

\end{solution}
%----------------------------------------

\end{itemize}
\end{problem}

\newpage  %Do not delete

\begin{problem}{7.3}
Let $C$ be a closed contour and let $z_0 \in \cc$ be a point not lying on $C$. The \emph{winding number} of $C$ about $z_0$ is defined by the integral
\[n(C,z_0) = \frac{1}{2\pi i}\int_C \frac{1}{z-z_0} \, dz.\]
\begin{itemize}[itemsep=3em]
\item[(a)] Compute $n(C_1,z_0)$ where $C_1$ is parameterized by \[z(t) = z_0 + Re^{it},\quad 0 \leq t \leq 2k\pi,\ k\in \zz,\ R>0.\] 
%----------------------------------------
\begin{solution}
  From the parameterization of $C_1$ we have,
  \[dz = Rie^{it} \ dt. \]
  Now computing the winding number as defined,
  \begin{align*}
    n(C_1, z_0) &= \dfrac{1}{2\pi i }\int_0^{2k \pi}\dfrac{1}{z_0 + Re^{it} -z_0} (Ri e^{it})\, d t \\
    &= \dfrac{1}{2\pi i }\int_0^{2\pi k }\dfrac{Ri e^{it}}{R e^{it}}\, d t \\
    &= \dfrac{1}{2\pi i}\int_0^{2\pi k}i \, d t \\
    &= \dfrac{1}{2\pi}\int_0^{2k\pi }d t \\
    &= \dfrac{1}{2\pi}\lrb{t}_0^{2k\pi} \\
    &= \dfrac{1}{2\pi}2k\pi \\
    &= k.
  \end{align*}
  We have then that the  winding number of $C_1$ about $z_0$ is $k$. 
\end{solution}
%----------------------------------------

\item[(b)] Compute $n(C_2,z_0)$, where $C_2$ is any circle and $z_0$ is any point not lying on or interior to $C_2$. 
%----------------------------------------
\begin{solution}
   Let $R$ be the interior of $C_2$ joined with $C_2$ itself. Now consider the function,
   \begin{align*}
     f: R &\to \cc \\
     z &\mapsto \dfrac{1}{z-z_0}
   \end{align*} 
   it is obvious from class and previous homework that $f$ is holomorphic everywhere on $R$ except when $z= z_0$, but $z_0$ is assumed to not lie in $R$, so that is not a worry. This then allows us to apply Cauchy-Goursat Theorem when computing the winding number to obtain,
   \begin{align*}
      n(C_2, z_0) = \dfrac{1}{2\pi i }\int_C \dfrac{1}{z-z_0}\, dz = \dfrac{1}{2\pi i }\int_C f(z)\, dz = 0 
   \end{align*} 
\end{solution}
%----------------------------------------

\item[(c)] Let $C_3$ be any closed contour and $z_0$ any point not lying on $C_3$, parameterized by $z:[a,b] \to \cc$. For any such contour, we can always find real-valued (piece-wise) differentiable functions $r,\theta :[a,b] \to \rr$ with $r(t)>0$ such that $z(t) = z_0 + r(t) e^{i\theta(t)}$. Compute $n(C_3,z_0)$.  
%----------------------------------------
\begin{solution}
  Given the parameterization of $z$ we have that,
  \begin{align*}
    dz = r'(t)e^{i\theta(t)}- i r(t) e^{i\theta(t)}\theta'(t) \, dt
  \end{align*}
  Now computing the winding number of $C_3$ about $z_0$,
  \begin{align*}
    n(C_3, z_0) &= \dfrac{1}{2\pi i}\int_a^{b}\dfrac{1}{z_0 + r(t)e^{i\theta(t)} - z_0}(r'(t)e^{i\theta(t)}- i r(t) e^{i\theta(t)}\theta'(t)) \, dt \\
    &= \dfrac{1}{2\pi i}\int_a^{b} \dfrac{r'(t)e^{i\theta(t)}- i r(t) e^{i\theta(t)}\theta'(t)}{r(t)e^{i\theta(t)}}\, dt  \\
    &= \dfrac{1}{2\pi i}\int_a^{b}\dfrac{r'(t)}{r(t)} - i\theta'(t) \, dt \\
    &= \dfrac{1}{2\pi i }\lrp{\int_a^{b}\dfrac{r'(t)}{r(t)}\, dt - i\int_a^{b}\theta'(t)\, dt} \\
    &= \dfrac{1}{2\pi i }\lrp{\lrb{\ln(r(t))}_a^{b} - i \lrb{\theta(t)}_a^{b}} \\
    &= \dfrac{1}{2\pi i }\lrp{\ln(r(b)) - \ln(r(a)) -i(\theta(b) - \theta(a))} \\
    &= \dfrac{1}{2\pi i }\lrp{\ln\lrp{\dfrac{r(b)}{r(a)}} - i(\theta(b)-\theta(a))} \\
    &= \dfrac{1}{2 \pi i}\lrp{\ln(1) -i(\theta(b)-\theta(a))} \\
    &= \dfrac{\theta(a)-\theta(b)}{2\pi  }.
  \end{align*}
  
\end{solution}
%----------------------------------------

\end{itemize}
\end{problem}

\newpage  %Do not delete

\begin{problem}{7.4}
Let $a,b \in \cc$ and let $C_R$ be the circle of radius $R$ centered at the origin, traversed once in the positive orientation. If $\abs{a} < R < \abs{b}$, show that
\[\int_{C_R} \frac{1}{(z-a)(z-b)}\, dz = \frac{2\pi i}{a-b}.\]
\end{problem}
%----------------------------------------
\begin{solution}
  First we can break down the integral using partial fractions. 
  \begin{align*}
    \dfrac{1}{(z-a)(z-b)} &= \dfrac{A}{(z-a)}+ \dfrac{B}{(z-b)} \\
    1 &= A(z-b) + B(z-a) && \text{let $z = b$} \\
    1&= A(b-b) + B(b-a) \\
    \dfrac{1}{b -a} &= B
  \end{align*}
  jumping back to the equation on the second line, 
  \begin{align*}
    1 &= A(z-b) + B(z-a) && \text{let $z = a$} \\
    1 &= A(a -b) + B(a - a) \\
    \dfrac{1}{a-b} &= A.
  \end{align*}
  Now we compute the given integral using everything we found,
  \begin{align*}
    \int_{C_R}\dfrac{1}{(z-a)(z-b)}\, d z &= \dfrac{1}{a-b}\underbrace{\int_{C_R}\dfrac{1}{z-a}\, dz}_{(1)} + \dfrac{1}{b-a}\underbrace{\int_{C_R}\dfrac{1}{z-b}\, dz}_{(2)} .
  \end{align*}
  Recall from the previous problem, that the winding number of a contour, $C_R$, about a point $a$ is,
  \[n(C_R, a) = \dfrac{1}{2\pi i}\int_{C_R}\dfrac{1}{z-a}\, d z\]
  it is clear though that the winding number of $C_R$ about $a$ should be 1 since $C_R$ is just a circle being traversed once with positive orientation and $\abs{a}< R$. This means that the integral $(1)$ should evaluate to $2\pi i$. Applying the same reasoning to $C_R$ about $b$, the winding number should work out to be 0, because we are given that $R < \abs{b}$, meaning the contour goes around $B$ 0 times. Which means the integral $(2)$ has to evaluate to $0$. Plugging this back in we get,
  \begin{align*}
    \int_{C_R}\dfrac{1}{(z-a)(z-b)}\, d z &= \dfrac{1}{a-b}2\pi i + \dfrac{1}{b-a} 0 \\
    &= \dfrac{2\pi i}{a-b}
  \end{align*}
  as desired. 
\end{solution}
%----------------------------------------

\newpage  %Do not delete

\begin{problem}{7.5}
Let \[f(z) = \frac{1}{z^2+1}.\]
Determine whether $f$ has an antiderivative on the given domain $G$. You must prove your claims. 
\begin{itemize}[itemsep=3em]
\item[(a)] $G = \cc \setminus \{i,-i\}$.
%----------------------------------------
\begin{solution}
  Using the fact that $z^{2} + 1 = (z+i)(z-i)$, let us use partial fraction decomposition as before to expand $f$,
  \begin{align*}
    \dfrac{1}{(z + i)(z - i)} &=\dfrac{A}{z+i} + \dfrac{B}{z - i} \\
    1 &= A(z-i) + B(z+i) && \text{choose $z = i$} \\
    1 &= A(i - i) + B(i + i) \\
    \dfrac{1}{2i} &= B  
  \end{align*}
  going back to the equation on the second line,
  \begin{align*}
    1 &= A(z-i) + B(z +i) && \text{choose $z = -i$} \\
    1 &= A(-i - i) + B(-i + i) \\
    -\dfrac{1}{2i} &= A.
  \end{align*}
  Now let us integrate $f$ over the contour $C$, where $C$ is a circle with radius 1 centered around $i$ traversed once with positive orientation. So we have,
  \begin{align*}
    \int_C f(z)\, d z = -\dfrac{1}{2i}\int_C \dfrac{1}{z + i}\, d z + \dfrac{1}{2i}\int_C \dfrac{1}{z-i} \, d z.
  \end{align*}
  We apply similar reasoning found in the previous problem, where these are the integrals used in calculating the winding number of our contour about $i$ and about $-i$. We know though that $-i$ is outside our contour so the first integral must evaluate to 0. We know that $i$ is the center of our contour, so the winding number should be 1, meaning the second integral evaluates to $2\pi i $. Plugging this back in we get,
  \begin{align*}
    \int_C f(z)\, d z &= \dfrac{1}{2i}0 + \dfrac{1}{2i} 2\pi i  \\
    &= \pi.  
  \end{align*}
  Which is non-zero meaning the antiderivative does not exist on the given domain $G$. 

  
\end{solution}
%----------------------------------------

\item[(b)] $G = \setp{z \in \cc}{\Re z > 0}$.
%----------------------------------------
\begin{solution}
  Due to time, I'm not sure how to rigorously prove this. So I'll just explain my intuition. I believe the antiderivative exists over this domain. This is because the only parts where the function can fail will be $\pm-i$ are not only non-existent in this domain, but it is impossible to create a closed contour containing them like we did previously. This is because there is no way to have a contour "wrap" around $\pm i$ because they are on the line $z = x + i y$ for $x = 0$, but this domain has $\Re z > 0$ which is to say $x > 0 $. This means then that we can't have a integral over a closed contour that evaluates to something non-zero. 
\end{solution}
%----------------------------------------

\end{itemize}
\end{problem}


%----------------------------------------
%Delete if nothing to add
\newpage  %Do not delete

\begin{center}
\textbf{Collaborators:}
%List your peers with whom you discussed the Problem Set
\end{center}
\vfill 

\begin{center}
\textbf{References:}
%List any book/website/notes that you used to write your solutions
\end{center}
\begin{itemize}
\item[$\bullet$] [Book(s): Title, Author]
\item[$\bullet$] [Online: \href{http://example.com/}{\color{blue}Link}]
\item[$\bullet$] [Notes: \href{http://example.com/}{\color{blue}Link}]
\end{itemize}

\vfill
\begin{center}
Fin.
\end{center}
\vfill

\end{document}