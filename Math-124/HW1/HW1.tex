\documentclass[12pt]{article}
%------------------------------- BEGIN PREAMBLE
% packages used
\usepackage{amssymb,amsmath,amsfonts,mathrsfs,pgffor,marvosym,amsthm}
% macros
\newcommand      {\Nm}         {{\mathbb N}}
\newcommand      {\Zm}         {{\mathbb Z}}
\newcommand      {\Qm}         {{\mathbb Q}}
\newcommand      {\Rm}         {{\mathbb R}}
\newcommand      {\Cm}         {{\mathbb C}}
\newcommand      {\vb}        {\mathbf}
\newcommand      {\PP}        {{\mathscr P}}
\newcommand {\lines}[1] {\foreach \n in {1,...,#1}{ \vspace{9mm} \hrule height 
0.2pt  }\vspace{2mm} }
% adjustment of page dimensions
\textwidth=7in
\textheight=9.8in
\topmargin= -0.8in
\oddsidemargin= -0.3in
\evensidemargin= 0.0in
\setlength{\parskip}{1ex plus0.5ex minus0.2ex}
\setlength{\jot}{10pt}
%-------------------------------- END PREAMBLE
\begin{document}
\begin{flushleft}
    Kevin Guillen\\ Shuqing Li \\ John Brzezicki 
\end{flushleft}
\begin{center}
    {\bf 124 Topology - Summer Session 1 - HW1 - Due: 6/26/21}
\end{center}

% STATEMENT OF PROBLEM 1
\noindent \textbf{ (1)} Prove $(2,3)$ is an open set.
\begin{proof}
    
    To be an open set every point in the set must have a $\delta_x$-neighborhood for $\delta_x > 0$ such that the $\delta_x$-neighbhorhood is contained in the set. 

    To satisfy this we will define $\delta_x$ as $\delta_x := \text{min}\{3-x,x-2\}$.

    Now to show any point $y$ in the $\delta_x$ neighborhood of $x$ is actually contained in the set $(2,3)$, we will have to examine the 2 cases where $y \leq x$ and $y > x$. 

    For $y \leq x$:
    \begin{align*}
        x - \delta_x <\ &y \leq x && \text{We know }2 < x < 3 \\
        x - \delta_x <\ &y \leq x < 3 && \text{also know } \delta_x \leq x-2 \rightarrow 2 \leq x-\delta_x \\
        2 \leq x - \delta_x <\ &y \leq x < 3 \\
        2 <\ &y < 3.
    \end{align*}

    For $y > x$:
    \begin{align*}
        x <\ &y < x+\delta_x && \text{we know } 2 < x < 3 \\
        2 < x <\ &y < x+\delta_x && \text{also know }  \delta_x \leq 3-x \rightarrow x + \delta_x \leq 3 \\
        2 < x <\ &y < x +\delta_x \leq 3 \\ 
        2 <\ &y < 3
    \end{align*} as desired.
\end{proof}

% STATEMENT OF PROBLEM 2
\noindent \textbf{ (2)} Prove [-1,1] is closed by proving its complement is open in $\Rm$
\begin{proof}
    Complement of [-1,1] in $\Rm$ is $(-\infty, -1)\cup (1,\infty)$. We know any union of open sets is open, so we need to show each of these sets is open. We do this by using the same definion as in problem 1. 

    Proving $(-\infty, -1)$ is open. For any point $x$ in this set we will set $\delta_x = (-1 - x)$. Now looking at arbitrary point $y$ in the $\delta_x$-neighborhood of $x$ we will show it is an element in $(-\infty, -1)$. For the first case that $y \leq x$,
    \begin{align*}
        x - \delta_x <\ &y \leq x && \text{we know } -\infty < x < -1 \\
        x - \delta_x <\ &y \leq x < -1 && \text{obvious that for any real $\delta_x$ that }-\infty < x - \delta_x \\
        -\infty <\ &y < -1
    \end{align*}
\newpage
    In the case that $y > x$:
    \begin{align*}
        x <\ &y < x + \delta_x && \text{we know } -\infty < x < -1 \\
        -\infty < x <\ &y < x + \delta_x && \text{plugging in for $\delta_x$} \\
        -\infty <\ &y < x + - 1 -x \\
        -\infty <\ &y < -1
    \end{align*}
    as desired.

    Proving $(1, \infty)$ is open. For any point $x$ in the this set we will set $\delta_x$ = (x-1). Now just like before looking at an arbitrary point $y$ in the $\delta_x$-neighborhood of $x$ we will show it is an element in $(1, \infty)$. For the first case that $y \leq x$:
    \begin{align*}
        x - \delta_x <\ &y \leq x && \text{we know $1 < x < \infty$} \\
        x - \delta)x <\ &y \leq x < \infty && \text{plugging in for $\delta_x$} \\
        x - (x-1) <\ &y < \infty \\
        1 <\ &y < \infty 
    \end{align*}
    Now in the case that $y > x$:
    \begin{align*}
        x <\ &y < x + \delta_x && \text{we know $1 < x < \infty$} \\
        1 < x <\ &y < x + \delta_x && \text{obvious that for any real $\delta_x$ that $x + \delta_x < \infty$} \\
        1 <\ &y < \infty 
    \end{align*}
    as desired.

    We have shown that the complement of the set is indeed open, thus the set $[-1,1]$ is closed. 
\end{proof}


\noindent \textbf{(2.4)(Q)} Prove that the subset Z $\subset$ R, consisting of all the integers, is closed.
\begin{proof}
    The complement of the set of integers in $\Rm$ is $\cup_{n\in\Zm}(n,n+1)$. All we need to show now is that for any integer $n$ that $(n,n+1)$ is open. We do this by choosing an arbitrary point $x$ in the set $(n,n+1)$ and showing the $\delta_x$ neighborhood is contained in the set. 
    For a point $x$, let $\delta_x = $min$\{n+1-x,x-n\}$. Now choosing a point $y\in (x-\delta_x, x + \delta_x)$, let's look at the case $y \leq x$
    \begin{align*}
        x-\delta_x <\ &y \leq x && x < n+1 \\
        x -\delta_x <\ &y \leq x < n+1 && \text{we know $\delta_x \leq x -n \rightarrow n \leq x-\delta_x$} \\
        n \leq x-\delta_x <\ &y < n+1 \\
        n <\ &y < n+1
    \end{align*}  
    Now in the case that $y > x$
    \begin{align*}
        x <\ &y < x + \delta_x && n < x \\
        n < x <\ &y < x+\delta_x && \text{we know $\delta_x \leq n+1-x \rightarrow \delta_x +x \leq n+1$} \\
        n <\ &y < x+\delta_x \leq n+1 \\
        n <\ &y < n+1
    \end{align*}
\end{proof}
as desired. Since $n$ was arbitrary and we showed $(n,n+1)$ is oepn, we know from class that any union of open sets is open. Thus the set of integers is closed. 

\begin{flushleft}
    \textbf{(2.7)(Q)}Let f : R → R be the function $f(x)=3 - 2x.$ \\
    – Prove that f is continuous using the $\epsilon$, $\delta$ definition of continuity. \\
    – Prove that f is continuous by showing that the preimage f$^{-1}$(S)
    of every open set S $\subset$ R is open.
    \end{flushleft}
    \textbf{(a)}\begin{proof}Using the $\epsilon$, $\delta$ definition of continuity, 
    that for any $\epsilon$ $>$ 0 there exists a  $\delta$ $>$ 0 such that $| f(x) - f(a) | < \epsilon$ whenever $|x - a | <  \delta$. 
    \\ \\ For $f(x)=3-2x$, let $ \epsilon > 0 $, and $\delta =  \frac{\epsilon}{2}$. Suppose $x_0, x\in \Rm$, 
    assume $| x-x_0 |< \delta$. Then, 
    \begin{align*}
        |f(x)-f(x_0)| &= |(3-2x)-(3-2x_0)| \\
        &= 2|x-x_0| < 2\delta && \text{where $\delta = \dfrac{\epsilon}{2}$} \\
        &= 2|x-x_0| < 2\dfrac{\epsilon}{2} \\
        &= 2|x-x_0| < \epsilon
    \end{align*}
    Thus $f$ is continous.    
\end{proof}
    

    \noindent\textbf{(b)}
    \begin{proof} : We'll prove by showing that the preimage f$^{-1}$(S)
    of every open set S $\subset$ R is open.
    Take any open interval (x,y) in R, with the inverse image
    \[y=3-2x \rightarrow x=3-2y \rightarrow y=\frac{3-x}{2}\] \\
    So we have $f^{-1}(x,y) = (\dfrac{3-x}{2},\dfrac{3-y}{2})$. Now take an arbitrary point $p$ in this set, let \[\delta_p = \text{min}\{\dfrac{3-y}{2} - p,p - \dfrac{3-x}{2}\}\]
    Now we will take a point $q$ in the $\delta_p$-neighborhood of $p$ and show it is contained in the set.
    In the case $q \leq p$
    \begin{align*}
        p-\delta_p <\ &q \leq p && p < \dfrac{3-y}{2} \\
        p-\delta_p <\ &q \leq p < \dfrac{3-y}{2} && \text{we know $\delta_p \leq p - \dfrac{3-x}{2} \rightarrow \dfrac{3-x}{2} \leq p-\delta_p$} \\
        \dfrac{3-x}{2}< p-\delta_p <\ &q < \dfrac{3-y}{2} \\
        \dfrac{3-x}{2} <\ &q < \dfrac{3-y}{2}
    \end{align*}
    \newpage
    Now for when $q > p$
    \begin{align*}
        p <\ &q < p + \delta_p && \dfrac{3-x}{2} < p \\
        \dfrac{3-x}{2} < p <\ &q < p +\delta_p && \text{we know $\delta_p \leq \dfrac{3-y}{2}-p \rightarrow p+ \delta_p\leq \dfrac{3-y}{2}  $} \\
        \dfrac{3-x}{2} <\ &q < p +\delta_p \leq  \dfrac{3-y}{2} \\
        \dfrac{3-x}{2} <\ &q < \dfrac{3-y}{2}
    \end{align*}
    as desired. 
    \\ Thus, the preimage $f^{-1}(S)$
    of every open set S $\subset$ R is open. Hence, $f$ is continuous. \\
    \end{proof}

    \begin{flushleft}
        \textbf{(3.2)(Q)} Let S = \{a,b,c,d\}. Which of the following lists of “open” sets forms a topology on S \\
        – $\phi, \{a\}, \{a,b\}, \{a,b,c,d\}$ \\
        – $\phi, \{a\}, \{b\}, \{a,b,c,d\}$ \\
        – $\phi, \{a,c\}, \{a,b,c\}, \{a,c,d\}, \{a,b,c,d\}$ \\
        – $\{a\}, \{a,b\}, \{a,b,c\}, \{a,b,c,d\}$ \\
        \end{flushleft}
        \textbf{(A)}
        S = \{a,b,c,d\}, let V be the collection of subsets of following sets. \\ \\
        $ \textbf{Is}$ – $\phi, \{a\}, \{a,b\}, \{a,b,c,d\}$ \\
        This forms a topology on S because  \\
        1)$\phi, \{a,b,c,d\} \in V_1$, so it satisfies axioms T1 and T2. \\
        2)T3 is satisfied because the arbitrary unions of "open" sets are all "open", for example, \\
        $\{a\} \cup \{a,b\} \in V_1$\\
        3)T4 is is satisfied because finite intersections of “open” sets are “open”, for example, \\
        $\{a\} \cap \{a,b\} \in V_1$\\ \\
        $ \textbf{Not}$ – $\phi, \{a\}, \{b\}, \{a,b,c,d\}$ \\
        This doesn't form a topology on S because  \\
        T3 is not satisfied that $\{a\} \cup \{b\} = \{a,b\} \notin V_2$\\ \\
        $ \textbf{Is}$ – $\phi, \{a,c\}, \{a,b,c\}, \{a,c,d\}, \{a,b,c,d\}$ \\
        This forms a topology on S because  \\
        1) $\phi, \{a,b,c,d\} \in V_3$, so it satisfies axioms T1 and T2. \\
        2)T3 is satisfied because the arbitrary unions of "open" sets are all "open", for example, \\
        $\{a,c\} \cup \{a,b,c\} \in V_3$\\
        $\{a,b,c\} \cup \{a,c,d\} \in V_3$\\
        3)T4 is is satisfied because finite intersections of “open” sets are “open”, for example, \\
        $\{a,c\} \cap \{a,b,c\} = \{a,c\} \in V_3$\\
        $\{a,b,c\} \cap \{a,b,c,d\} = \{a,b,c\} \in V_3$ \\ \\
        $ \textbf{Not}$ – $\{a\}, \{a,b\}, \{a,b,c\}, \{a,b,c,d\}$ \\
        This doesn't form a topology on S because  \\
        T2 is not satisfied that $\phi \notin V_4$.
        \begin{flushleft}
        \textbf{(3.3)(Q)} For each topology on {a,b,c,d} from Question 3.2, list the open sets
        in the subspace topology for the subset {a,b,c}
        \end{flushleft}
        \textbf{(A)}
        1)$\{\phi, \{a\}, \{a,b\}, \{a,b,c\}\}$ \\
        2)$\{\phi, \{a,c\}, \{a,b,c\}\}$ \\

\begin{flushleft}
    \textbf{(3.4)(Q)}Prove that a function $f : S \rightarrow T$ between two topological spaces is continuous if, and only if, $f^{-1}(C)$ is closed whenever $C \subset T$ is closed.
\end{flushleft}
\begin{proof}
    Let $C$ be a closed set int $T$. By definition this means $T-C$ is open in $T$. 
    \begin{align*}
        f^{-1}(T-C) &= f^{-1}(T) - f^{-1}(C) \\
        &= S - f^{-1}(C) && \text{We know this is open since $T-C$ was open}
    \end{align*}
    Since $S- f^{-1}(C)$ is open by definition that implies $f^{-1}(C)$ is closed in $S$

    Now for the contraposition. Let $X$ be an open set in $T$, then by definition $T-X$ is closed in $T$, and $f^{-1}(T-X)$ is closed
    \begin{align*}
        f^{-1}(T-X) &= f^{-1}(T) - f^{-1}(X) \\
        &= S - f^{-1}(X) 
    \end{align*}
    Since $S-f^{-1}(X)$ is closed, that means $f^{-1}(X)$ is open. 

    We see the statement and the contraposition of it holds therefore it is true, $f$ is continous if and only if $f^{-1}(C)$ is closed whenever $C \subset T$ is closed.
\end{proof}

\end{document}