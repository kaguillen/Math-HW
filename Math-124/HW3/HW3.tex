\documentclass[12pt]{article}
%------------------------------- BEGIN PREAMBLE
% packages used
\usepackage{amssymb,amsmath,amsfonts,mathrsfs,pgffor,marvosym,amsthm, mathrsfs, mathtools}
\DeclarePairedDelimiter\set\{\}
% macros
\newcommand      {\Nm}         {{\mathbb N}}
\newcommand      {\Zm}         {{\mathbb Z}}
\newcommand      {\Qm}         {{\mathbb Q}}
\newcommand      {\Rm}         {{\mathbb R}}
\newcommand      {\Cm}         {{\mathbb C}}
\newcommand      {\vb}        {\mathbf}
\newcommand      {\PP}        {{\mathscr P}}
\newcommand      {\BB}        {{\mathscr B}}
\newcommand {\lines}[1] {\foreach \n in {1,...,#1}{ \vspace{9mm} \hrule height 
0.2pt  }\vspace{2mm} }

% adjustment of page dimensions
\textwidth=7in
\textheight=9.8in
\topmargin= -0.8in
\oddsidemargin= -0.3in
\evensidemargin= 0.0in
\setlength{\parskip}{1ex plus0.5ex minus0.2ex}
\setlength{\jot}{10pt}
%-------------------------------- END PREAMBLE
\begin{document}
\begin{flushright}
    
    
\end{flushright}
\begin{center}
    {\bf Topology - Summer Session 1 - HW3 - 7/16/2021}
\end{center}

% STATEMENT OF PROBLEM 1
\noindent {\bf (5.2)}
Let $S=\set{a, b, c, d}$ with the discrete topology and let $T=\set{a, b, c, d}$ with the indiscrete topology. Define $f:S\to T$ to be the identitymap. Is $f$ a homeomorphism?
\begin{proof}
    By defintion the discrete topology on $S$ is simply, $\mathcal{T}_S = \mathcal{P}(S) $. While the indiscrete topology for $T$ is $\mathcal{T}_T = \set{\emptyset, T}$. Meaning the topology $\mathcal{T}_S$ is strictly finer than the topology $\mathcal{T}_T$. This is key since a homeomorphism requires a biijection which is a 1-1 correspondence between the two topologies, and since they are of different size no bijectiction can exist. 
\end{proof}

\vspace*{1.5cm}

% STATEMENT OF PROBLEM 2
\noindent{\bf(5.7)} Show that the annulus $A=\set{(x, y)\in R^2:1\leq x^2+y^2\leq 4}$ is homeomorphic to the cylinder $C=\set{(x, y, z)\in R^3:x^2+y^2=1,0
\leq z\leq 1}$

% STATEMENT OF PROBLEM 3


% STATEMENT OF PROBLEM 4

% STATEMENT OF PROBLEM 5

% STATEMENT OF PROBLEM 6

\end{document}