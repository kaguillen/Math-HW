\documentclass[12pt]{article}
%------------------------------- BEGIN PREAMBLE
% packages used
\usepackage{amssymb,amsmath,amsfonts,mathrsfs,pgffor,marvosym,amsthm, mathrsfs, mathtools}
\DeclarePairedDelimiter\set\{\}
% macros
\newcommand      {\Nm}         {{\mathbb N}}
\newcommand      {\Zm}         {{\mathbb Z}}
\newcommand      {\Qm}         {{\mathbb Q}}
\newcommand      {\Rm}         {{\mathbb R}}
\newcommand      {\Cm}         {{\mathbb C}}
\newcommand      {\vb}        {\mathbf}
\newcommand      {\PP}        {{\mathscr P}}
\newcommand      {\BB}        {{\mathscr B}}
\newcommand {\lines}[1] {\foreach \n in {1,...,#1}{ \vspace{9mm} \hrule height 
0.2pt  }\vspace{2mm} }

% adjustment of page dimensions
\textwidth=7in
\textheight=9.8in
\topmargin= -0.8in
\oddsidemargin= -0.3in
\evensidemargin= 0.0in
\setlength{\parskip}{1ex plus0.5ex minus0.2ex}
\setlength{\jot}{10pt}
%-------------------------------- END PREAMBLE
\begin{document}
\begin{flushright}
    Name:  Kevin Guillen \\ Mingji Chen 
    
\end{flushright}
\begin{center}
    {\bf Topology - Summer Session 1 - HW2 - 7/10/2021}
\end{center}

I wasn't sure wheather or not to include our 3rd memeber Gordon since we tried contacting, but they didn't respond to our discussion in our Canvas group page, and he didn't submit any of his work this was just Kevin and Mingji. 

% STATEMENT OF PROBLEM 1
\noindent {\bf (3.3)}  For each topology on $\{a, b, c, d\}$ from Question 3.2, list the open sets in the subspace topology for the 
subset $\{a, b, c\}$.
\begin{proof}
    Recalling from $3.2$ the only list of open sets that formed a topology were the following,
    \begin{center}
        $V_1 = \emptyset, \{a\}, \{a,b\}, \{a,b,c,d\}$ \\
        $V_2 = \emptyset, \{a,c\}, \{a,b,c\}, \{a,c,d\}, \{a,b,c,d\}.$ 
    \end{center}
    So taking the intersection of $\{a,b,c\}$ with every open set in each topology will give us each the list of open sets in the subspace for the subset $\{a,b,c\}$
    \begin{align*}
        \forall U \in V_1,\ U\cap \{a,b,c\} &= \emptyset, \{a\}, \{a,b\},\{a,b,c\} \\
        \forall U \in V_2,\ U \cap \{a,b,c\} &= \emptyset, \{a,c\},\{a,b,c\}.
    \end{align*}

\end{proof}

% STATEMENT OF PROBLEM 2
\noindent {\bf (3.8)} Let $f:\Rm \to \Zm$ be the floor function which rounds a real number $x$ down to the nearest integer:
\begin{align*}
    f(x) = n \text{ provided that } n \in \Zm \text{ and } n \leq x < n+1
\end{align*}
Determine whether $f$ is continuous. 

\begin{proof}
    We know in $\Zm$ that for any $n\in \Zm$, $\{n\}$ will be an open set. Taking the inverse image of this open set we see we get,
    \begin{align*}
        f^{-1}(n) = [n,n+1)
    \end{align*}
    which is not a closed set in $\Rm$, but neither is it open Thus, by definition the floor function is not continuous.
\end{proof}

% STATEMENT OF PROBLEM 3
\noindent {\bf (3.11)}Let $T$ be a set and $B$ a collection of subsets of $T$. Show that if every element of $T$ belongs to at least one subset in $B$ and $B$ is closed under finite intersections, then the collection of all unions of sets in $B$ forms a topology on $T$
\begin{proof}
    Assuming that if $\forall x \in T,\ \exists U \in B$ such that, $x \in U$, and that $B$ is closed under finite intersections, we want to show the collection of all unions of sets in $B$ forms a topology on $T$. Let's refer to this collection as $\BB$.
    
    The first thing we need to show is that the empty set is in $B'$. Since $\BB$ is defined as all unions of sets in $B$, let's take the union of sets in $B$ indexed by $I$, where $I = \emptyset$ to get
    \begin{align*}
        \bigcup_{i\in I}U_i = \emptyset, \text{ where $U_i \in B$}.
    \end{align*}
    Then we have that $\emptyset \in \BB$, satisfying the first requirement. 

    Now to show that $T$ is in $\BB$. 
    This is similar to as before in that $B$ is defined in that for any element in $T$ there is at least one subset in $B$ that contains that element. Since $B$ is also defined to be subsets of $T$ taking any union of these subsets can never equal anything more than $T$. Now $\BB$ is defined to be any     union of set in $B$, thus taking the union of all sets in $B$ will result in $T$.

    Thus, we have it that $T\in \BB$
    

    Now to show that the union of elements of any subcollection is closed. Taking a union of subcollections in $\BB$ to be $U$. In other words $U$ is obtained by,
    \begin{align*}
        U = \bigcup_{i \in I}U_i 
    \end{align*}
    where $I$ is some index set, and $U_i \in \BB$. We know $\BB$ to be defined as the collection of any unions of set in $B$, there each $U_i$ can be expressed as,
    \begin{align*}
        U_i = \bigcup_{j \in J}B_j
    \end{align*}
    where $J$ is some index set, and $B_j \in B$. Expanding this out we get,
    \begin{align*}
        U = \bigcup_{i\in I}\bigcup_{j \in J}B_{j,i}
    \end{align*}
    we see that this is simply a union of sets in $B$, and $\BB$ is defined to be the collection of any union of sets in $B$, thus $U \in \BB$.

    Now we need to show $\BB$ is closed under finite intersection. Take $U_1$, $U_2\in \BB$. They are defined as, \[U_1 = \bigcup_{i \in I}B_i\] \[U_2 =\bigcup_{j \in J}B_j\] for $I$ and $J$ as some index and $B_i,\ B_j\in B$.
    Taking the intersection of these two we get,
    \begin{align*}
        U_1 \cap U_2 =  \bigcup_{i \in I}B_i \cap \bigcup_{j \in J}B_j =  \bigcup_{i \in I}\bigcup_{j \in J}(B_i \cap B_j)
    \end{align*}
    Since $B_i$ and $B_j$ are elements of $B$, and $B$ was defined to be closed under intersection, this then just becomes a union of set in $B$, which by definition is then in $\BB$. It is obvious that this will hold for finite intersection through induction. 

    Thus, $\BB$ does indeed form a topology for $T$
    
\end{proof}

% STATEMENT OF PROBLEM 3
\noindent {\bf (4.1)} Let $S$ be a disconnected space say $S = U \cup V$ and $U \cap V = \emptyset$, and that $U,V$ are open and nonempty. If $x\in U$ and $y \in V$ prove there can be continous map $f:[0,1]\to S$ such that $f(0) = x$ and $f(1) = y$
\begin{proof}
    We know there exists $x$ in $U$ such that $f(0) = x$, therefore $f^{-1}(U) = U'$ and that 0 is in $U'$. The same is true that there exists some $y$ in $V$ such that $f(1) = y$ and therefore, $f^{-1}(V) = V'$ and $1$ is in $V'$ 

    We know since $U$ and $V$ were open in $S$ that $U'$ and $V'$ have to be open in $[0,1]$. And because $U\cup V = S$ that means $U'\cup V' = [0,1]$.and since $U \cap V = \emptyset$ then $U'\cap V'= \emptyset$. This means $[0,1]$ can be formed by a union of 2 open and disjoint sets, meaning it is disconnected, but we know from class this is not true, therefore there can be no continous map from $[0,1]$ to $S$


\end{proof}    

\noindent {\bf (4.2)} Let $T$ be the set $\{a,b,c\}$ for the following topologies on $T$ determine which are connected and which are Hausdorff. 

$1)\ \emptyset,\{a\},\{a, b\},\{a, b, c\}. \\
\indent 2)\ \emptyset,\{a\},\{b, c\},\{a, b, c\}. \\
\indent 3)\ \emptyset,\{a\},\{a, b\},\{a, c\},\{c\},\{a, b, c\}$
\begin{proof}
    1) is connected since there are no 2 disjoint sets such that their union forms $T$. However though it is not Hausdorff since for example take $a$ and $b$ there are no 2 disjoint open sets where 1 contains $a$ and the other contains $b$

    2) is disconnected since $\{a\} \cap \{b,c\} = \emptyset$, and $\{a\}\cup\{b,c\} = T$ and they are both open making this topology disconnected. This topology is not Hausdorff though since there are no 2 disjoint open sets that one contains $b$ and one contains $c$

    3) is disconnected since $\set{a,b} \cap \set{c} = \emptyset$ and $\set{a,b} \cup \set{c} = T$, and they are both open, making this topology disconnected. This topology is not Hausdorff though since there are no 2 disjoint open sets that one contains $a$ and the other contains $b$

\end{proof}

\noindent {\bf (4.3)} Prove that every continous map from $\Rm \to \Qm$ is constant. 

\begin{proof}
    To prove this we're going to show that $\Qm$ is actually a disconnected space. Take $a$ and $b$ in $\Qm$, we know that between any two rational numbers there exists an irrational number between them, let's refer to it as $q$, $a < q < b$. 
    
    Let's consider the sets formed using $q$, $U = (-\infty, q)$ and $V = (q, \infty)$. We now want to show that these two sets are indeed open, consider for any $x \in \Qm$ such that $x < q$. We know that for any $\delta-$neighborhood of $x$, where $\delta = q - x$ that $\delta-$neighborhood of $x$ will be contained in $(-\infty, q)$. 

    Similair reasoning will apply for $x > q$ simply setting $\delta = x - q$ meaning $U$ and $V$ are open. So we have $U \cup V = \Qm$ and $U \cap V = \emptyset$. Making $\Qm$ disconnecetd. 

    We know from class that $\Rm$ is a connected space, so assuming $f$ to be a continous map from $\Rm$ to $\Qm$. We know $\Qm = U \cup V$, therefore $\Rm = f^{-1}(U)\cup f^{-1}(V)$, we know these sets are open since they are open in $\Qm$, and that they are disjoint. But like stated before we know $\Rm$ is connected. Making this a contradiction. Thefore any continous map from $\Rm$ to $\Qm$ has to be constant since $\Qm$ is disconnected. 

    
\end{proof}

\noindent {\bf (4.5)} Is $[0,\frac{1}{2})$ compact, if not give an example of an unbounded function. 
\begin{proof}
    It is not compact. An example would be $f(x) = \frac{1}{x - \frac{1}{2}}$ on $[0,\dfrac{1}{2})$ since,
    \begin{align*}
        \lim_{x \to \frac{1}{2}}\dfrac{1}{x-\frac{1}{2}} = \infty  
    \end{align*}
\end{proof}

\noindent {\bf (4.7)} Show that $\Zm$ is Hausdorff without using proposition 4.35

\begin{proof}
    Consider for any integers $m$ and $n$ the ball of radius $\frac{1}{3}$ we know in the set of integers the $d(m,n) = |n-m|$ as a lowerbound of 1. Therefore since our raidus is $\frac{1}{3}$ which is less than 1, meaning for any two integers these sets will always be disjoint. $B_{\frac{1}{3}}(m)\cap \Zm = \set{m}$ and $B_{\frac{1}{3}}(n) \cap \Zm= \set{n}$, so we have $\set{m}$ and $\set{n}$ as open disjoint sets. Showing $\Zm$ actually has a discrete topology under the metric $d(n,m) = |n-m|$, and every discrete topology is Hausdorff
\end{proof}

\noindent {\bf (4.8)} Prove that if $T$ is a Hausdorff space and $x_1,...,x_n$ is a finite list of distinct points in $T$, then there are open sets $U_1,...,U_n$ each containing one, and only one, of the points $x_1,...,x_n$

\begin{proof}
    Since $T$ is Hausdorff for $x_i$ and $x_j$ ($i \neq j$) $\exists U_{i,j}$ and $U_{j,i}$ that are open and disjoint containing $x_i$ and $x_j$ respectively. Then for $x_i$ we can find, \[U_{i,1},\ U_{i,2},\ ...,\ U_{i,n} \rightarrow \bigcap_{j =1, j \neq i}^n (U{i,j})\] that is an open set and contains $x_i$ but doesn't contain any other $x_j (i\neq j)$. Hence, denote this set to be \[U_1 = \bigcap_{j = 1}^n(U{i,j})\] similairly we can find $U_2,\ ....,\ U_n$
\end{proof}

\noindent {\bf (4.9)} If the set of integers were given the indiscrete topology, would it be connected? Compact? Hausdorff?
\begin{proof}
    If $\Zm$ would be given an indiscrete topology we cannot seperate $m$ and $n$ with disjoint open sets, therfore $\Zm$ can't be Hausdorff.
    
    $\Zm$ is connected because there are no other open sets such that $Z = U \cup V$.

    $\Zm$ is copmpact because any open cover has a finite subcover. 
\end{proof}



\end{document}