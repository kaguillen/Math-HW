\documentclass[12pt]{article}
%------------------------------- BEGIN PREAMBLE
% packages used
\usepackage{amssymb,amsmath,amsfonts,mathrsfs,pgffor,marvosym,amsthm, mathrsfs, mathtools}
\DeclarePairedDelimiter\set\{\}
% macros
\newcommand      {\Nm}         {{\mathbb N}}
\newcommand      {\Zm}         {{\mathbb Z}}
\newcommand      {\Qm}         {{\mathbb Q}}
\newcommand      {\Rm}         {{\mathbb R}}
\newcommand      {\Cm}         {{\mathbb C}}
\newcommand      {\vb}        {\mathbf}
\newcommand      {\PP}        {{\mathscr P}}
\newcommand      {\BB}        {{\mathscr B}}
\newcommand {\lines}[1] {\foreach \n in {1,...,#1}{ \vspace{9mm} \hrule height 
0.2pt  }\vspace{2mm} }

% adjustment of page dimensions
\textwidth=7in
\textheight=9.8in
\topmargin= -0.8in
\oddsidemargin= -0.3in
\evensidemargin= 0.0in
\setlength{\parskip}{1ex plus0.5ex minus0.2ex}
\setlength{\jot}{10pt}
%-------------------------------- END PREAMBLE
\begin{document}
\begin{flushright}
    Names: Mingji Chen \\ Kevin Guillen
    
\end{flushright}
\begin{center}
    {\bf Topology - Summer Session 1 - HW4 - \today}
\end{center}
\noindent{\bf (6.1)}
Write down a homotopy equivalence between $(0,1)$ and $[0,1]$.
\begin{proof}
    First let's define 
    \begin{align*}
        f: [0,1] &\to (0,1) \\ x&\mapsto \dfrac{x+1}{3}
    \end{align*} Then we will define
    \begin{align*}
        g:(0,1)&\hookrightarrow [0,1] \\ x &\mapsto x.
    \end{align*}
    It's obvious that both of these are continuous functions, and we see $g\circ f \cong  Id_{[0,1]}$ by,
    \begin{align*}
        H: [0,1]\times [0,1] &\to [0,1]\\
        (x,t)&\mapsto x\cdot t + (1-t)\cdot\left(\frac{x+1}{3}\right)
    \end{align*}
    We can see $H(x,0) = g\circ f$ and $H(x,1) = x,\ \forall x \in [0,1]$.

    We can also see $f\circ g \cong Id_{(0,1)}$ by,
    \begin{align*}
        H:(0,1)\times [0,1] &\to (0,1) \\
        (x,t) &\mapsto x\cdot t + (1-t)\cdot\left(\frac{x+1}{3}\right)
    \end{align*}
    We can see $H(x,0) = f\circ g$ and $H(x,1) = x,\ \forall x \in (0,1)$.

    Thus $f$ and $g$ is a homotopy equivalence between $(0,1)$ and $[0,1]$
\end{proof}

%\noindent\textbf{(6.2)} List all homotopy classes of maps $(0, 1) \to (0, 1)$
%\begin{proof}
%   Take any two continuous functions $f,g:(0,1)\to (0,1)$. If we define, \[H: (0,1)\times[0,1] \to (0,1)\]
%    through $H(x,t) = t\cdot g(x) + (1-t)\cdot f(x)$. It is obvious that $H$ is continous since it is a composition of continuous functions. We also see $H(x,0) = f(x)$ and $H(x,1) = g(x)$ meaning $H$ is ahomtopy between $f$ and $g$. Note though $f$ and $g$ were arbitrary that means any two continuous functions on $(0,1)$ are homotopic. Since we showed any two functions are homotopic to one another through transitivity we know this forms a class, and thus the homotopy classes of maps is composed by 1 element. This element will just be a single point $\set{0}$ since we know any open interval $(a,b)$ is homotopy equivalent to $\set{0}$, since $(a,b)$ is homeomorphic with $(0,1)$
%\end{proof}

\noindent\textbf{(6.2)} List all homotopy classes of maps $(0, 1) \to (0, 1)$
\begin{proof}
    Since $(0,1)$ is contractible, then there exists a homotopy $H$ between the $Id$ and a constant point $\set{0}$. Now consider $f:(0,1) \to (0,1)$. Then $H\circ f:(0,1)\times I \to (0,1)$ is a homotopy between $f$ and the constant map. Since $f$ is homotopic to the constant map and the constant map is homotopic to the $Id$, $f\simeq  Id$, therefor there is only on homotopy class.
\end{proof}
\newpage
\noindent{\bf (6.3)}
Prove that a discrete space consisting of $m$ points in homotopy equivalent to a discrete space consisting of $n$ points if, and only if, $m=n$.
\begin{proof}
    If $m\neq n$, denote these points as $X = \set{x_1, x_2, ..., x_m}$ and $Y = \set{y_1, y_2, ...., y_n}$. Assume then that $m > n$ then at least two points of $X$ maps to one point of $Y$. Without loss of generality we have,
    \begin{alignat*}{3}
        f: X &\to Y \qquad g:&Y \to X \\
        x_1 &\mapsto y_1 \qquad &y_1 \mapsto x_1 \\
        x_m &\mapsto y_1 \qquad &y_1 \mapsto x_m \\
        x_i &\mapsto y_i \qquad &y_j \mapsto x_i 
    \end{alignat*}
    Then we have,
    \begin{align*}
        H:X\times I &\to X \\
        (x,0) &\mapsto g\circ f(x) \\
        (x,1) &\mapsto x
    \end{align*}
    then we have $g\circ f(x_1) = g\circ f(x_1) \ g(y_1) $.

    For a fixed $x\in X$ the path $r:I\to X$, $t\mapsto H(x,t)$ is continuous. Since $I$ is connected and $X$ is discrete then $r$ must be constant. 

    We have then $x_1 = r(0) = r(1) = x_m$, each point in $Y$ has only one preimage and vice versa, therefore $m = n$.

    On the other hand if $m =n$ then $X = \set{x_1, x_2,...x_n}$ and $Y = \set{y_1, y_2,...y_n}$. Let $f:X\to Y$, $x_i\mapsto y_i$ and $g:Y\to X$, $y_i \mapsto x_i$. Then it is obvious that $f\circ g \simeq Id_Y$ and $g\circ f \simeq Id_X$
        
\end{proof}


\noindent{\bf(6.5)}
Show that the map $f:S^1\rightarrow S^1$ given by $f(x,y)=(-x,-y)$ is homotopic to the identity map.
\begin{proof}
    For $f: S^1 \to S^1$, $(x,y)\mapsto (-x,-y)$. We can consider the homotopy $H$ as follows,
    \begin{align*}
        H: S^1\times[0,1] &\to S^1 \\
        ((cos(\theta),sin(\theta)),t) &\mapsto (cos(\theta + (1-t)\pi), sin(\theta+(1-t)\pi))
    \end{align*}

    We can see that $H((cos(\theta),sin(\theta)),0) = (cos(\theta + \pi),sin(\theta + \pi))$, this is rotating the point around the circle 180 degrees, which means it is sending it to it's antipodal point, therefore it is equal to $f$. Now for $H((cos(\theta), sin(\theta)),1) = (cos(\theta + 0), sin(\theta + 0))$ which is just the identity map. Therefore $f\simeq Id_{S^1}$
\end{proof}


\end{document}