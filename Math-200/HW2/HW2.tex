\documentclass[10pt]{article}
 
\usepackage[top=0.75in, bottom=1.25in, left=1in, right=1in]{geometry} 
\usepackage{amsmath,amsthm,amssymb} %this is THE math package
\usepackage{mathtools}
\usepackage{tikz}
\usepackage{graphicx}
\usepackage{fancybox}
\usepackage{hyperref}
\usepackage{varwidth}
\usepackage{mdframed}
\usepackage{mathrsfs}
\usepackage{xcolor}
\usepackage{pagecolor}
\usepackage{lipsum}  
\usepackage[most]{tcolorbox}
%------------------------
%Fonts I use
\usepackage{mathpazo}
\usepackage{eulervm}

%------------------------
%   Inverting Page color. 
\pagecolor{black}
\color{white}

%------------------------
%This is so that we have standard fonts for the doubletracked symbols
%for reals, naturals etc. regardless of what font you use.
%Don't comment
\AtBeginDocument{
  \DeclareSymbolFont{AMSb}{U}{msb}{m}{n}
  \DeclareSymbolFontAlphabet{\mathbb}{AMSb}}
%------------------------

%----------------------------------------------
%User-defined environments
%Commented because we're not using them in this document
%The only uncommented ones are the Problem and Solution environment

% \newenvironment{theorem}[2][Theorem]{\begin{trivlist}
% \item[\hskip \labelsep {\bfseries #1}\hskip \labelsep {\bfseries #2.}]}{\end{trivlist}}
% \newenvironment{lemma}[2][Lemma]{\begin{trivlist}
% \item[\hskip \labelsep {\bfseries #1}\hskip \labelsep {\bfseries #2.}]}{\end{trivlist}}
% \newenvironment{exercise}[2][Exercise]{\begin{trivlist}
% \item[\hskip \labelsep {\bfseries #1}\hskip \labelsep {\bfseries #2.}]}{\end{trivlist}}
% \newenvironment{question}[2][Question]{\begin{trivlist}
% \item[\hskip \labelsep {\bfseries #1}\hskip \labelsep {\bfseries #2.}]}{\end{trivlist}}
% \newenvironment{corollary}[2][Corollary]{\begin{trivlist}
% \item[\hskip \labelsep {\bfseries #1}\hskip \labelsep {\bfseries #2.}]}{\end{trivlist}}
\newenvironment{problem}[2][Problem\!]{\begin{trivlist}
\item[\hskip \labelsep {\bfseries #1}\hskip \labelsep {\bfseries #2.}]}{\end{trivlist}}
%\newenvironment{sub-problem}[2][]{\begin{trivlist}
%\item[\hskip \labelsep {\bfseries #1}\hskip \labelsep {\bfseries #2}]}{\end{trivlist}}
\newenvironment{solution}{\begin{proof}[\textbf{\textit{Solution}}]}{\end{proof}}
%----------------------------------------------

%----------------------------
%User-defined notations
\newcommand{\zz}{\mathbb Z}   %blackboard bold Z
\newcommand{\qq}{\mathbb Q}   %blackboard bold Q
\newcommand{\ff}{\mathbb F}   %blackboard bold F
\newcommand{\rr}{\mathbb R}   %blackboard bold R
\newcommand{\nn}{\mathbb N}   %blackboard bold N
\newcommand{\cc}{\mathbb C}   %blackboard bold C
\newcommand{\af}{\mathbb A}   %blackboard bold A
\newcommand{\pp}{\mathbb P}   %blackboard bold P
\newcommand{\id}{\operatorname{id}} %for identity map
\newcommand{\im}{\operatorname{im}} %for image of a function
\newcommand{\dom}{\operatorname{dom}} %for domain of a function
\newcommand{\cat}[1]{\mathscr{#1}}   %calligraphic category
\newcommand{\abs}[1]{\left\lvert#1\right\rvert} %for absolute value
\newcommand{\norm}[1]{\left\lVert#1\right\rVert} %for norm
\newcommand{\modar}[1]{\text{ mod }{#1}} %for modular arithmetic
\newcommand{\set}[1]{\left\{#1\right\}} %for set
\newcommand{\setp}[2]{\left\{#1\ \middle|\ #2\right\}} %for set with a property
\newcommand{\card}[1]{\#\,{#1}} %for cardinality of a set

%Re-defined notations
\renewcommand{\epsilon}{\varepsilon}
\renewcommand{\phi}{\varphi}
\renewcommand{\emptyset}{\varnothing}
\renewcommand{\geq}{\geqslant}
\renewcommand{\leq}{\leqslant}
\renewcommand{\Re}{\operatorname{Re}}
\renewcommand{\Im}{\operatorname{Im}}
%----------------------------

\allowdisplaybreaks
 
\begin{document}
 
\title{Homework 2}
\author{Kevin Guillen\\[0.5cm]
MATH 200 | Algebra I | Fall 2021}
\date{} 
\maketitle

%-------------------------------PROBLEM 1 --------------------------
\begin{tcolorbox}
    \begin{problem}{2.6}
        Show that for any non-empty subset $X$ of a group $G$, the normalizer of $X$, $N_G(X)$ and the centralizer of $X$, $C_G(X)$ is again a subgroup of $G$. Show also that $C_G(X)$ is contained in $N_G(X)$.
    \end{problem}
\end{tcolorbox}

\begin{proof}
    \textbf{Normalizer} We know the normalizer of a subset $X$ is defined as the following, \[N_G(X) = \set{g \in G \mid gXg^{-1} = X}.\]

    So consider $x,y \in N_G(x)$. Let $z = xy$, we want to show that $z \in N_G(X)$. In other words we want to show $zXz^{-1} = X$, based on the above. We can see through the following that this is indeed true,
    \begin{align*}
        zXz^{-1} &= (xy)X(xy)^{-1} && (xy)^{-1} = y^{-1}x^{-1} \\
        &= xyXy^{-1}x^{-1} && y \in N_G(x) \\
        &= xXx^{-1} && x \in N_G(x) \\
        &= X.
    \end{align*}
    Meaning $N_G(X)$ is closed under group operation. 

    Let $y\in N_G(X)$, based on the definition of the normalizer though,
    \begin{align*}
        yXy^{-1} &= X && \text{taking $y$ on the right} \\
        yX &= Xy && \text{taking $y^{-1}$ on the left} \\
        X &= y^{-1}Xy &&y = (y^{-1})^{-1} \\
        X &= y^{-1}X(y^{-1})^{-1}
    \end{align*}
    that $y^{-1}$ is indeed in $N_G(X).$
    Thus by the subgroup criterion, $N_G(X)$ is indeed a subgroup. 
    
    \textbf{Centralizer:} We know the definition of the centralizer of a subset $X$ is the following, \[C_G(X) = \set{g\in G \mid gxg^{-1} = x, \forall x \in X}.\]

    So consider $a,b \in C_G(X)$. Let $z = ab$, we want to show that $z \in C_G(X)$. In other words we want to show $zxz^{-1} = x$ for all $x\in X$. We see through the following that this does indeed hold.
    \begin{align*}
        zxz^{-1} &= (ab)x(ab)^{-1} && (ab)^{-1} = b^{-1}a^{-1} \\
        &= (ab)x(b^{-1}a^{-1}) && \text{We know associativity holds in $G$} \\
        &= a(bx^{-1})a^{-1}  && b \in C_G(X) \\
        & = axa^{-1} && a \in C_G(X) \\
        &= x
    \end{align*}
    Meaning $C_G(X)$ is closed under group operation. 

    Let $y \in C_G(X)$. By definition that means for all $x\in X$, $yxy^{-1} = x$, but consider the following,
    \begin{align*}
        yxy^{-1} &= x && \text{taking $y^{-1}$ on the left} \\
        xy^{-1} &= y^{-1}x && \text{taking $y$ on the right} \\
        x &= y^{-1}xy && y = (y^{-1})^{-1} \\
        x &= y^{-1}x(y^{-1})^{-1}.  
    \end{align*}
    This means that for any $y\in C_G(X)$, that $y^{-1}$ is also in $C_G(X)$. Thus $C_G(X)$ is a subgroup. 
    
    Now we want to show that the centralizer is contained in the normalizer. Expanding on the definition of the normalizer $gXg^{-1} = X \rightarrow gX = Xg$. This means there exists some $s,t\in X$ such that $gs = tg$. What we see though is that this is simply a weaker property when compared to the centralizer definition. Expanding on the definition of the centralizer, for all $x\in X$ we have $gxg^{-1} = x \rightarrow gx = xg$. Meaning any $g\in C_G(X)$ has the property that $gs = tg$ where $t = s = x$, which means it is also in $N_G(X)$, thus $C_G(X) \subset N_G(X)$
\end{proof}

%--------------------PROBLEM 2------------------------
\begin{tcolorbox}
    \begin{problem}{2.7}
        Let $f:G\rightarrow H$ be a group homomorphism. 
        \begin{itemize}
            \item[(a)] If $U \leq G$ then $f(U)\leq H$.
            \item[(b)] If V $\leq$ H then $f^{-1}(V) = \set{g\in G \mid f(g) \in V}$ is a subgroup of $G$.
            \item[(c)] Show that $f$ is injective if and only if ker$(f)$ = $\set{1}$ 
        \end{itemize}
    \end{problem}
\end{tcolorbox}
\begin{proof}
    \begin{itemize}
        \item[(a)] Let $x,y \in f(U)$, and let $z = xy$, we want to show $z\in f(U)$. Since $x,y \in f(U)$, that means there exists $x',y' \in U$ such that $f(x') = x$ and $f(y') = y$. Giving us,
        \begin{align*}
            z &= xy \\
            &= f(x')f(y') &&\text{$f$ is a homomorphism so,} \\
            &= f(x'y') 
        \end{align*} 
        Because $U$ is a subgroup then $x'y' \in U$, meaning $z = f(x'y')\in f(U)$, thus $f(U)$ is closed under group operation.
        
        Given $x\in f(U)$, we want to show $x^{-1}\in f(U)$. By $x\in f(U)$ that means there exists $x'\in U$ such that $x = f(x')$. Since $U$ is a subgroup there exists $x'^{-1} \in U$, meaning $f(x'^{-1})\in f(U)$. Recall though $f$ is a homomorphism that means it respects inverses, thus $f(x'^{-1}) = f(x')^{-1}$, which will be $x^{-1}$. We verify through the following,
        \begin{align*}
            xx^{-1} &= f(x')f(x')^{-1} \\
            &= f(x'x'^{-1}) \\
            &= f(1) \\
            &= 1
        \end{align*}

    \end{itemize}
\end{proof}

\newpage
new page
\end{document}
