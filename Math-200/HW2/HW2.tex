\documentclass[10pt]{article}
 
\usepackage[top=0.75in, bottom=1.25in, left=1in, right=1in]{geometry} 
\usepackage{amsmath,amsthm,amssymb} %this is THE math package
\usepackage{mathtools}
\usepackage{tikz}
\usepackage{graphicx}
\usepackage{fancybox}
\usepackage{hyperref}
\usepackage{varwidth}
\usepackage{mdframed}
\usepackage{mathrsfs}
\usepackage{xcolor}
\usepackage{pagecolor}
\usepackage{lipsum}  
\usepackage[most]{tcolorbox}
%------------------------
%Fonts I use
\usepackage{mathpazo}
\usepackage{eulervm}

%------------------------
%   Inverting Page color. 


%------------------------
%This is so that we have standard fonts for the doubletracked symbols
%for reals, naturals etc. regardless of what font you use.
%Don't comment
\AtBeginDocument{
  \DeclareSymbolFont{AMSb}{U}{msb}{m}{n}
  \DeclareSymbolFontAlphabet{\mathbb}{AMSb}}
%------------------------

%----------------------------------------------
%User-defined environments
%Commented because we're not using them in this document
%The only uncommented ones are the Problem and Solution environment

% \newenvironment{theorem}[2][Theorem]{\begin{trivlist}
% \item[\hskip \labelsep {\bfseries #1}\hskip \labelsep {\bfseries #2.}]}{\end{trivlist}}
% \newenvironment{lemma}[2][Lemma]{\begin{trivlist}
% \item[\hskip \labelsep {\bfseries #1}\hskip \labelsep {\bfseries #2.}]}{\end{trivlist}}
% \newenvironment{exercise}[2][Exercise]{\begin{trivlist}
% \item[\hskip \labelsep {\bfseries #1}\hskip \labelsep {\bfseries #2.}]}{\end{trivlist}}
% \newenvironment{question}[2][Question]{\begin{trivlist}
% \item[\hskip \labelsep {\bfseries #1}\hskip \labelsep {\bfseries #2.}]}{\end{trivlist}}
% \newenvironment{corollary}[2][Corollary]{\begin{trivlist}
% \item[\hskip \labelsep {\bfseries #1}\hskip \labelsep {\bfseries #2.}]}{\end{trivlist}}
\newenvironment{problem}[2][Problem\!]{\begin{trivlist}
\item[\hskip \labelsep {\bfseries #1}\hskip \labelsep {\bfseries #2.}]}{\end{trivlist}}
%\newenvironment{sub-problem}[2][]{\begin{trivlist}
%\item[\hskip \labelsep {\bfseries #1}\hskip \labelsep {\bfseries #2}]}{\end{trivlist}}
\newenvironment{solution}{\begin{proof}[\textbf{\textit{Solution}}]}{\end{proof}}
%----------------------------------------------

%----------------------------
%User-defined notations
\newcommand{\zz}{\mathbb Z}   %blackboard bold Z
\newcommand{\qq}{\mathbb Q}   %blackboard bold Q
\newcommand{\ff}{\mathbb F}   %blackboard bold F
\newcommand{\rr}{\mathbb R}   %blackboard bold R
\newcommand{\nn}{\mathbb N}   %blackboard bold N
\newcommand{\cc}{\mathbb C}   %blackboard bold C
\newcommand{\af}{\mathbb A}   %blackboard bold A
\newcommand{\pp}{\mathbb P}   %blackboard bold P
\newcommand{\id}{\operatorname{id}} %for identity map
\newcommand{\im}{\operatorname{im}} %for image of a function
\newcommand{\dom}{\operatorname{dom}} %for domain of a function
\newcommand{\cat}[1]{\mathscr{#1}}   %calligraphic category
\newcommand{\abs}[1]{\left\lvert#1\right\rvert} %for absolute value
\newcommand{\norm}[1]{\left\lVert#1\right\rVert} %for norm
\newcommand{\modar}[1]{\text{ mod }{#1}} %for modular arithmetic
\newcommand{\set}[1]{\left\{#1\right\}} %for set
\newcommand{\setp}[2]{\left\{#1\ \middle|\ #2\right\}} %for set with a property
\newcommand{\card}[1]{\#\,{#1}} %for cardinality of a set

%Re-defined notations
\renewcommand{\epsilon}{\varepsilon}
\renewcommand{\phi}{\varphi}
\renewcommand{\emptyset}{\varnothing}
\renewcommand{\geq}{\geqslant}
\renewcommand{\leq}{\leqslant}
\renewcommand{\Re}{\operatorname{Re}}
\renewcommand{\Im}{\operatorname{Im}}
%----------------------------

\allowdisplaybreaks
 
\begin{document}
 
\title{Homework 2}
\author{Kevin Guillen\\[0.5cm]
MATH 200 | Algebra I | Fall 2021}
\date{} 
\maketitle

I'd like my proof for problem 3.2 (It's the last one) to be graded please, thank you. 

%-------------------------------PROBLEM 1 --------------------------
\begin{tcolorbox}
    \begin{problem}{2.6}
        Show that for any non-empty subset $X$ of a group $G$, the normalizer of $X$, $N_G(X)$ and the centralizer of $X$, $C_G(X)$ is again a subgroup of $G$. Show also that $C_G(X)$ is contained in $N_G(X)$.
    \end{problem}
\end{tcolorbox}

\begin{proof}
    \textbf{Normalizer} We know the normalizer of a subset $X$ is defined as the following, \[N_G(X) = \set{g \in G \mid gXg^{-1} = X}.\]

    So consider $x,y \in N_G(x)$. Let $z = xy$, we want to show that $z \in N_G(X)$. In other words we want to show $zXz^{-1} = X$, based on the above. We can see through the following that this is indeed true,
    \begin{align*}
        zXz^{-1} &= (xy)X(xy)^{-1} && (xy)^{-1} = y^{-1}x^{-1} \\
        &= xyXy^{-1}x^{-1} && y \in N_G(x) \\
        &= xXx^{-1} && x \in N_G(x) \\
        &= X.
    \end{align*}
    Meaning $N_G(X)$ is closed under group operation. 

    Let $y\in N_G(X)$, based on the definition of the normalizer though,
    \begin{align*}
        yXy^{-1} &= X && \text{taking $y$ on the right} \\
        yX &= Xy && \text{taking $y^{-1}$ on the left} \\
        X &= y^{-1}Xy &&y = (y^{-1})^{-1} \\
        X &= y^{-1}X(y^{-1})^{-1}
    \end{align*}
    that $y^{-1}$ is indeed in $N_G(X).$
    Thus by the subgroup criterion, $N_G(X)$ is indeed a subgroup. 
    
    \textbf{Centralizer:} We know the definition of the centralizer of a subset $X$ is the following, \[C_G(X) = \set{g\in G \mid gxg^{-1} = x, \forall x \in X}.\]

    So consider $a,b \in C_G(X)$. Let $z = ab$, we want to show that $z \in C_G(X)$. In other words we want to show $zxz^{-1} = x$ for all $x\in X$. We see through the following that this does indeed hold.
    \begin{align*}
        zxz^{-1} &= (ab)x(ab)^{-1} && (ab)^{-1} = b^{-1}a^{-1} \\
        &= (ab)x(b^{-1}a^{-1}) && \text{We know associativity holds in $G$} \\
        &= a(bx^{-1})a^{-1}  && b \in C_G(X) \\
        & = axa^{-1} && a \in C_G(X) \\
        &= x
    \end{align*}
    Meaning $C_G(X)$ is closed under group operation. 

    Let $y \in C_G(X)$. By definition that means for all $x\in X$, $yxy^{-1} = x$, but consider the following,
    \begin{align*}
        yxy^{-1} &= x && \text{taking $y^{-1}$ on the left} \\
        xy^{-1} &= y^{-1}x && \text{taking $y$ on the right} \\
        x &= y^{-1}xy && y = (y^{-1})^{-1} \\
        x &= y^{-1}x(y^{-1})^{-1}.  
    \end{align*}
    This means that for any $y\in C_G(X)$, that $y^{-1}$ is also in $C_G(X)$. Thus $C_G(X)$ is a subgroup. 
    
    Now we want to show that the centralizer is contained in the normalizer. Expanding on the definition of the normalizer $gXg^{-1} = X \rightarrow gX = Xg$. This means there exists some $s,t\in X$ such that $gs = tg$. What we see though is that this is simply a weaker property when compared to the centralizer definition. Expanding on the definition of the centralizer, for all $x\in X$ we have $gxg^{-1} = x \rightarrow gx = xg$. Meaning any $g\in C_G(X)$ has the property that $gs = tg$ where $t = s = x$, which means it is also in $N_G(X)$, thus $C_G(X) \subset N_G(X)$
\end{proof}

%--------------------PROBLEM 2------------------------
\begin{tcolorbox}
    \begin{problem}{2.7}
        Let $f:G\rightarrow H$ be a group homomorphism. 
        \begin{itemize}
            \item[(a)] If $U \leq G$ then $f(U)\leq H$.
            \item[(b)] If V $\leq$ H then $f^{-1}(V) = \set{g\in G \mid f(g) \in V}$ is a subgroup of $G$.
            \item[(c)] Show that $f$ is injective if and only if ker$(f)$ = $\set{1}$ 
        \end{itemize}
    \end{problem}
\end{tcolorbox}

    \begin{itemize}
        \item[(a)]
        \begin{proof} 
            Let $x,y \in f(U)$, and let $z = xy$, we want to show $z\in f(U)$. Since $x,y \in f(U)$, that means there exists $x',y' \in U$ such that $f(x') = x$ and $f(y') = y$. Giving us,
            \begin{align*}
                z &= xy \\
                &= f(x')f(y') &&\text{$f$ is a homomorphism so,} \\
                &= f(x'y') 
            \end{align*} 
            Because $U$ is a subgroup then $x'y' \in U$, meaning $z = f(x'y')\in f(U)$, thus $f(U)$ is closed under group operation.
            
            Given $x\in f(U)$, we want to show $x^{-1}\in f(U)$. By $x\in f(U)$ that means there exists $x'\in U$ such that $x = f(x')$. Since $U$ is a subgroup there exists $x'^{-1} \in U$, meaning $f(x'^{-1})\in f(U)$. Recall though $f$ is a homomorphism that means it respects inverses, thus $f(x'^{-1}) = f(x')^{-1}$, which will be $x^{-1}$. We verify through the following,
            \begin{align*}
                xx^{-1} &= f(x')f(x')^{-1} \\
                &= f(x'x'^{-1}) \\
                &= f(1) \\
                &= 1.
            \end{align*}
            Thus we have that if $U\leq G$ then $f(U) \leq H$. 
        \end{proof}
        \item[(b)] 
        \begin{proof}
            Let $x,y \in f^{-1}(V)$, that means there exists $x',y' \in V$ such that $f(x) = x'$ and $f(y) = y'$. Recall though $V$ is a subgroup so $x'y' \in V$, but $x'y' = f(x)f(y)$ and $f$ is a homomorphism so $f(x)f(y) = f(xy) \in V$ which means $xy \in f^{-1}(V)$. 
            
            
            Let $x\in f^{-1}(V)$ then $f(x) \in V$, and because $V$ is a subgroup we have $f(x)^{-1} \in V$. Recall though $f$ is a homomorphism, and so $f(x)^{-1} = f(x^{-1}) \in V$ and thus $x^{-1} \in f^{-1}(V)$. 
        \end{proof}
        \item[(c)] 
        \begin{proof}
            $\Rightarrow$ Given that $f$ is injective and a group homomorphism, that means it respects the identity element, meaning $f(1_G) = 1_H$. That also means whenever $f(x) = f(y) \rightarrow x = y$. Take an element $x\in $ ker$(f)$, by definition that means $f(x) = 1_H$, recall though $f(1_G) = 1_H$. So we have $f(x) = f(1_G)$, but by definition that means $x = 1_G$. Therefore if $f$ is injective, the kernel of $f$ is $\set{1}$

            $\Leftarrow$ Given that $f$ is a group homomorphism and that $\ker(f) = \set{1}$. We want to show that $f$ is injective. Consider $x,y \in G$ such that $f(x) = f(y)$. Now consider the following,
            \begin{align*}
                f(xy^{-1}) = f(x)f(y^{-1}) \\
                &= f(x)f(y)^{-1} && f(x) = f(y) \\
                &= f(x)f(x)^{-1} \\
                &= 1_H.
            \end{align*}
            Recall though $\ker(f) = \set{1_G}$, and we see $f(xy^{-1}) = 1_H$ that means $x = y$, and thus $f$ is injective.  
        \end{proof}
    \end{itemize}


\begin{tcolorbox}
    \begin{problem}{2.9}
        Let $G$ and $A$ be groups and assume that $A$ is abelian. Show that the set Hom$(G,A)$ of group homomorphisms from $G$ to $A$ is again an abelian group under the multiplication defined by
        \begin{align*}
            (f_1 \cdot f_2)(g):= f_1(g)f_2(g) && \text{for } f_1,f_2 \in \text{Hom}(G,A) \text{ and } g\in G 
        \end{align*}
    \end{problem}
\end{tcolorbox}
\begin{proof} From this point forward let $H = $ Hom$(G,A)$

    \textbf{Closure.} Let $f_1,f_2 \in H$, let $f_3 = f_1 \cdot f_2$ we want to show that $f_3 \in H$. Now let $a,b \in G$, we have the following,
    \begin{align*}
        f_3(ab) = (f_1f_2)(ab) &= f_1(ab)f_2(ab) && \text{Recall though $f_1,f_2$ are group homomorphisms} \\
        &= f_1(a)f_1(b)f_2(a)f_2(b) && \text{All these elements are in $A$, and $A$ is abelian} \\
        &= f_1(a)f_2(a)f_1(b)f_2(b)  \\
        &= (f_1f_2)(a)(f_1f_2)(b) \\
        &= f_3(a)f_3(b).
    \end{align*} 
    We see then that $f_3$ is a group homomorphism meaning it is also in $H$. Thus $H$ is closed under the multiplication.

    \textbf{Identity.} Simply let the the be the identity element be, $f_e: G\to A$, $g \mapsto 1_A$. It is obvious that this is a homomorphism, and is in $H$. We see through the following that it does indeed serve the role of the identity element. Let $f_1\in H, g \in G$
    \begin{align*}
        (f_ef_1f_e)(g) &= f_e(g)f_1(g)f_e(g) \\
        &= 1_A f_1(g) 1_A && \text{Recall $f_1(g)$ is an element of $A$} \\
        &= f_1(g)
    \end{align*}

    \textbf{Inverse.} Let $f_1\in H$. We see the inverse is simply $f_1^{-1}\in H$, we verify through the following where $g\in G$,
    \begin{align*}
        (f_1f_1^{-1})(g) &= f_1(g)f_1^{-1}(g) \\
        &= f_1(g)f_1(g^{-1}) \\
        &= f_1(gg^{-1}) \\
        &= f_1(1_G) \\
        &= 1_A \\
        &= f_e(g)
    \end{align*}
    \newpage

    \textbf{Associativity.} Let $f_1,f_2,f_2 \in H$. We see through the following that associativity holds. Let $g\in G$
    \begin{align*}
        ((f_1f_2)f_3)(g) &= ((f_1f_2)(g)f_3(g) \\
        &= (f_1(g)f_2(g))f_3(g) && \text{Recall these are element in $A$, and $A$ is a group} \\
        &= f_1(g)(f_2(g)f_2(g)) \\
        &= f_1(g)(f_2f_3)(g) \\
        &= (f_1(f_2f_3))(g) 
    \end{align*}
    as we can see associativity does indeed hold.

    \textbf{Commutativity.} Let $f_1,f_2 \in H$. We see through the following commutativity holds. Let $g\in G$,
    \begin{align*}
        (f_1f_2)(g) = f_1(g)f_2(g) && \text{These are elements of $A$ and $A$ is abelian} \\
        = f_2(g)f_1(g)  \\
        = (f_2f_1)(g)
    \end{align*}
    as we can see commutativity does indeed hold.

    With all this together that means Hom$(G,A)$ is indeed an abelian group, as desired. 
\end{proof}

\begin{tcolorbox}
    \begin{problem}{3.1}
        Let $M$ and $N$ be normal subgroups of a group $G$. Show that also $M \cap N$ and $MN$ are normal subgroups of $G$. 
    \end{problem}
\end{tcolorbox}
\begin{proof}
    To begin we know from 2.11 example (c) that the intersection of any collection of subgroups of a group is again a subgroup. Meaning $M\cap N$ is a subgroup of $G$. All that is left to show now is that it is a normal subgroup of $G$. We see immediately though that for all $k \in M\cap N$ and for all $g\in G$ that $gkg^{-1} \in M$, and $gkg^{-1} \in N$. This is because any element that is in the intersection of $M$ and $N$ must be in both those subgroups, and those subgroups were said to be normal. This means for all $k\in M\cap N$ and for all $g\in G$ that $gkg^{-1} \in M\cap N$. Therefore proving that $M\cap N$ is indeed a normal subgroup of $G$. 
\end{proof}

\begin{tcolorbox}
    \begin{problem}{3.2}
        Let $G$ be a group and let $X$ be a subset of $G$. Show that $C_G(X) \unlhd N_G(X)$
    \end{problem}
\end{tcolorbox}
\begin{proof}
    In this same homework we worked out from problem 2.6 that $C_G(X)$ is indeed contained in $N_G(X)$. So to solve this problem we just need to show that it is indeed a subgroup and then that it is normal. Let's begin with showing that it is a subgroup.

    \textbf{Closure.} Let $a,b \in C_G(X)$, we want to show $(ab) \in C_G(X)$. For all $x\in X$ we see through the following,
    \begin{align*}
        (ab)x(ab)^{-1} &= (ab)x(b^{-1}a^{-1}) && \text{we know associativity holds} \\
        &= a(bxb^{-1})a^{-1} && b \in C_G(X) \\
        &= axa^{-1} && a \in C_G(X) \\
        &= x
    \end{align*}
    that $C_G(X)$ is indeed closed under group operation. 

    \textbf{Inverse.} Now for any $a\in C_G(X)$ we will show that $a^{-1}$ is also in $C_G(X)$. By definition of $a$ being in $C_G(X)$ we have for all $x\in X$,
    \begin{align*}
        axa^{-1} &= x \\
        a^{-1}axa^{-1}&=a^{-1}x \\
        xa^{-1}a  &= a^{-1}x a \\
        x &= a^{-1}xa. 
    \end{align*}
    Meaning $a^{-1}$ is also in $C_G(X)$. Therefore $C_G(X) \leq N_G(X)$.

    Now to show that is normal. We will use Theorem 3.1 (iii) to prove that this subgroup is indeed normal. Consider the map $f: N_G(X) \to \text{Aut}(X)$ where $n \mapsto (x \mapsto nxn^{-1})$. We know $(x\mapsto axa^{-1})$ is indeed an automorphism based on Example 2.6 (c). So first we want to show that $f$ is a homomorphism. 
    \begin{align*}
        f(a)f(b) &= (x\mapsto axa^{-1})(x \mapsto bxb^{-1}) \\
                &= (x\mapsto (abxb^{-1}a^{-1}))&& (ab)^{-1} = b^{-1}a^{-1} \\
                &= (x\mapsto (ab)x(ab)^{-1}) \\
                &= f(ab)
    \end{align*}

    Now with that out of the way to apply the theorem stated earlier we need to show that $\ker(f) = C_G(X)$. We will do this by considering an element $a\in \ker(f)$, and for all $x\in X$, we see through the following,
    \begin{align*}
        x &= axa^{-1} \\
        xa &= ax
    \end{align*}
    that $a\in C_G(X)$ and thus $\ker(f) = C_G(X)$. Since we see by being in the kernel of $f$ an element must satisfy the definition of being in the centralizer of the subset $X$. 

\end{proof}

\end{document}
