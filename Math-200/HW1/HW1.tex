\documentclass[12pt]{article}
%------------------------------- BEGIN PREAMBLE
% packages used
\usepackage{amssymb,amsmath,amsfonts,mathrsfs,pgffor,marvosym,amsthm, mathrsfs, mathtools}
\DeclarePairedDelimiter\set\{\}
% macros
\newcommand      {\Nm}         {{\mathbb N}}
\newcommand      {\Zm}         {{\mathbb Z}}
\newcommand      {\Qm}         {{\mathbb Q}}
\newcommand      {\Rm}         {{\mathbb R}}
\newcommand      {\Cm}         {{\mathbb C}}
\newcommand      {\vb}        {\mathbf}
\newcommand      {\PP}        {{\mathscr P}}
\newcommand      {\BB}        {{\mathscr B}}
\newcommand {\lines}[1] {\foreach \n in {1,...,#1}{ \vspace{9mm} \hrule height 
0.2pt  }\vspace{2mm} }

% adjustment of page dimensions
\textwidth=7in
\textheight=9.8in
\topmargin= -0.8in
\oddsidemargin= -0.3in
\evensidemargin= 0.0in
\setlength{\parskip}{1ex plus0.5ex minus0.2ex}
\setlength{\jot}{10pt}
%-------------------------------- END PREAMBLE
\begin{document}
\begin{flushright}
    Name: Kevin Guillen \\*
    Student ID: 1747199
\end{flushright}
\begin{center}
    {\bf Class - Quarter - Assignment - Due Date}
\end{center}

% STATEMENT OF PROBLEM 1
\noindent {\bf (1)} Determine the invertible elements of the monoids among the examples in 1.2.
\begin{itemize}
    \item{A)} For the monoid $(\Nm_0, +)$ the only invertible element will be 0 since we see $0 + 0 = 0$
    \\ For $(\Nm_0, \cdot)$ we see the only invertible element will be 1 since we can see $1 \cdot 1 = 0$

    \item{B)} For the monoid $(\mathcal{P}(X), \cup)$ we can see the set of 
\end{itemize}



\end{document}