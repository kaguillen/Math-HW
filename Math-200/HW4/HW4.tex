\documentclass[11pt]{article}
 
\usepackage[top=0.75in, bottom=1.25in, left=1in, right=1in]{geometry} 
\usepackage{amsmath,amsthm,amssymb} %this is THE math package
\usepackage{mathtools}
\usepackage{tikz}
\usepackage{graphicx}
\usepackage{fancybox}
\usepackage{hyperref}
\usepackage{varwidth}
\usepackage{mdframed}
\usepackage{mathrsfs}
\usepackage[most]{tcolorbox}
%------------------------
%Fonts I use, uncomment if you like to use them.
%The first is the general font, and the second a math font
\usepackage{mathpazo}
\usepackage{eulervm}

%------------------------
%This is so that we have standard fonts for the double-stroked symbols
%for reals, naturals etc. regardless of what font you use.
%Don't comment
\AtBeginDocument{
  \DeclareSymbolFont{AMSb}{U}{msb}{m}{n}
  \DeclareSymbolFontAlphabet{\mathbb}{AMSb}}
%------------------------

%----------------------------------------------
%User-defined environments
%Commented because we're not using them in this document
%The only uncommented ones are the Problem and Solution environment

% \newenvironment{theorem}[2][Theorem]{\begin{trivlist}
% \item[\hskip \labelsep {\bfseries #1}\hskip \labelsep {\bfseries #2.}]}{\end{trivlist}}
% \newenvironment{lemma}[2][Lemma]{\begin{trivlist}
% \item[\hskip \labelsep {\bfseries #1}\hskip \labelsep {\bfseries #2.}]}{\end{trivlist}}
% \newenvironment{exercise}[2][Exercise]{\begin{trivlist}
% \item[\hskip \labelsep {\bfseries #1}\hskip \labelsep {\bfseries #2.}]}{\end{trivlist}}
% \newenvironment{question}[2][Question]{\begin{trivlist}
% \item[\hskip \labelsep {\bfseries #1}\hskip \labelsep {\bfseries #2.}]}{\end{trivlist}}
% \newenvironment{corollary}[2][Corollary]{\begin{trivlist}
% \item[\hskip \labelsep {\bfseries #1}\hskip \labelsep {\bfseries #2.}]}{\end{trivlist}}
\newenvironment{problem}[2][Problem\!]{\begin{trivlist}
\item[\hskip \labelsep {\bfseries #1}\hskip \labelsep {\bfseries #2}]}{\end{trivlist}}
%\newenvironment{sub-problem}[2][]{\begin{trivlist}
%\item[\hskip \labelsep {\bfseries #1}\hskip \labelsep {\bfseries #2}]}{\end{trivlist}}
\newenvironment{solution}{\begin{proof}[\textbf{\textit{Solution}}] }{\end{proof}}
%----------------------------------------------

%----------------------------
%User-defined notations
\newcommand{\zz}{\mathbb Z}   %blackboard bold Z
\newcommand{\qq}{\mathbb Q}   %blackboard bold Q
\newcommand{\ff}{\mathbb F}   %blackboard bold F
\newcommand{\rr}{\mathbb R}   %blackboard bold R
\newcommand{\nn}{\mathbb N}   %blackboard bold N
\newcommand{\cc}{\mathbb C}   %blackboard bold C
\newcommand{\af}{\mathbb A}   %blackboard bold A
\newcommand{\pp}{\mathbb P}   %blackboard bold P
\newcommand{\id}{\operatorname{id}} %for identity map
\newcommand{\im}{\operatorname{im}} %for image of a function
\newcommand{\dom}{\operatorname{dom}} %for domain of a function
\newcommand{\cat}[1]{\mathscr{#1}}   %calligraphic category
\newcommand{\abs}[1]{\left\lvert#1\right\rvert} %for absolute value
\newcommand{\norm}[1]{\left\lVert#1\right\rVert} %for norm
\newcommand{\modar}[1]{\text{ mod }{#1}} %for modular arithmetic
\newcommand{\set}[1]{\left\{#1\right\}} %for set
\newcommand{\setp}[2]{\left\{#1\ \middle|\ #2\right\}} %for set with a property
\newcommand{\card}[1]{\#\,{#1}} %for cardinality of a set
\newcommand{\sy}[1]{\text{Sym}(#1)}
\newcommand{\al}[1]{\text{Alt}(#1)}
\newcommand\m[1]{\begin{pmatrix}#1\end{pmatrix}} 

%Re-defined notations
\renewcommand{\epsilon}{\varepsilon}
\renewcommand{\phi}{\varphi}
\renewcommand{\emptyset}{\varnothing}
\renewcommand{\geq}{\geqslant}
\renewcommand{\leq}{\leqslant}
\renewcommand{\Re}{\operatorname{Re}}
\renewcommand{\Im}{\operatorname{Im}}
%----------------------------

\allowdisplaybreaks
 
 
\begin{document}
 
\title{Homework 4}
\author{Kevin Guillen\\[0.5em]
MATH 200 | Algebra I | Fall 2021}
\date{} 
\maketitle

May I please have my proof for 6.1 graded, thank you.

%Use \[...\] instead of $$...$$
\begin{tcolorbox}
    \begin{problem}{4.1}
        \begin{itemize}
            \item[(a)] Show that Alt$(4)$ is the derived subgroup of Sym$(4)$.
            \item[(b)] Find all composition series of Sym$(4)$.
            \item[(c)] Determine the higher derived subgroups of Sym$(4)$.
        \end{itemize}
    \end{problem}    
\end{tcolorbox}
\begin{solution}
    \begin{itemize}
        \item[(a)] Well by definition given to us, the derived subgroup of any group is the smallest normal subgroup such that the factor group will be abelian. We know from class that $\text{Sym}(4)/\text{Alt}(4)$ is commutative. This indicates to us that the derived subgroup will be contained in Alt$(4)$. To show that the derived subgroup of $\sy{4}$ is indeed $\al{4}$ look at the following.
        \begin{align*}
            (ab)(ad)(ab)(ad) = (abd)
        \end{align*} 
        We see that every 3-cycle is a commutator. We know there are 8 3-cycles in $\sy{4}$, meaning the derived subgroup of it will be of size at least 9. Therefore its derived subgroup must be $\al{4}$. This is because the order of $\sy{4}$ is 24, so the only possible subgroups will have to divide 24, and $\al4$ is of order 12, and the derived subgroup is contained in $\al{4}$
        \item[(b)] All composition series for $\sy4$ are the following, \[\sy4 \rhd \al4 \rhd V \rhd <a>\rhd \set{1}\] where $a$ in $\set{1, a}$ is simply any non identity element in the Klein 4 group $V$. This is because any non identity in that group has order 2, and any subgroup of $V$ is normal since $V$ is abelian. We already know from class that the composition factors of this series is indeed simple. 
        \item[(c)]  Let $G = \sy4$. From part (a) we know that $G' = \al(4)$. So, $G'' = \al4'$. To find the derived subgroup of $\al4$ we will use the fact that we know $V$ (the Klein 4 group) is normal in $\al4$ because conjugation in $\sy4$ does not change cycle structure. Because of this we know that $al4' \leq V$. We see though that every element in $V$ is a commutator,
        \begin{align*}
            (14)(23) = (124)(134)(142)(143) = [(142),(143)] \\
            (13)(24) = (123)(143)(132)(134) = [(132),(134)] \\
            (12)(34) = (132)(142)(123)(124) = [(123),(124)]
        \end{align*} 
        Meaning any element in $V$ is also an element in $\al4'$. Giving us $\al4' = V$. Therefore $G'' = V$. 

        Now to get $G''' = V'$. Recall though the derived subgroup of an abelian group is simply $\set{1}$. Therefore $G''' = \set{1}$. 

        Putting all this together we have the following, 
        \begin{align*}
            G^0 &= \sy4 \\
            G^1 &= \al4 \\
            G^2 &= V\ (\text{Klein 4 group})\\
            G^3 &= \set{1}
        \end{align*}
    \end{itemize}
\end{solution}

\begin{tcolorbox}
    \begin{problem} {6.1}
        Show that for every cycle $(a_1, \dots , a_k)$ in $\sy n$ and every $\sigma \in \sy n$ one has,
        \[\sigma (a_1, \dots, a_k)\sigma^{-1} = (\sigma(a_1),\dots, \sigma(a_k))\]
    \end{problem}
\end{tcolorbox}
\begin{proof}
    This can be proved by showing that $\forall t \in \set{1,\dots,n}$ the following holds,
    \begin{align*}
        (\sigma (a_1, \dots, a_k)\sigma^{-1})(t) = (\sigma(a_1),\dots, \sigma(a_k))(t).
    \end{align*}

    This gives us 2 cases. The first is that $t$ is equal to $\sigma(a_i)$ for some $i\in \set{1,\dots, k}$. This means,
    \begin{align*}
        (\sigma \circ (a_1, \dots, a_k)\sigma^{-1})(t) &= (\sigma \circ (a_1, \dots, a_k))\sigma^{-1}(t) \\
        &= (\sigma(a_1, \dots, a_k))\sigma^{-1}(\sigma(a_i)) \\
        &= \sigma(a_1, \dots, a_k)(a_i) && (*) \\
        &= \sigma(a_{i+1}) 
    \end{align*}
    (*) Recognizing the fact that if $i$ were to be $k$, then $i +1$ would actually be 1 and not literally $i+1$.

    Now the last case is that, for any $i \in \set{1,\dots, k}$, $t \neq \sigma(a_i)$ which also implies $\sigma^{-1}(t) \neq a_i$. This means that the cycle $(a_1,\dots, a_k)$ has no effect on $\sigma^{-1}(t)$. This gives us the following,
    \begin{align*}
        (\sigma \circ (a_1, \dots, a_k)\sigma^{-1})(t) &= \sigma(a_1, \dots, a_k)(\sigma^{-1}(t)) \\
        &= \sigma(\sigma^{-1}(t)) \\
        &= t
    \end{align*}

    Meaning for any value of $t$ not in $(\sigma(a_1),\dots, \sigma(a_k))$, $\sigma(a_1,\dots, a_k)\sigma^{-1}$ leaves $t$ fixed. All this together then means,  
    \[\sigma  (a_1, \dots, a_k)\sigma^{-1} = (\sigma(a_1),\dots, \sigma(a_k))\]
    as desired.
\end{proof}



\end{document}
