\documentclass[11pt]{article}
 
\usepackage[top=0.75in, bottom=1.25in, left=1in, right=1in]{geometry} 
\usepackage{amsmath,amsthm,amssymb} %this is THE math package
\usepackage{mathtools}
\usepackage{tikz}
\usepackage{graphicx}
\usepackage{fancybox}
\usepackage{hyperref}
\usepackage{varwidth}
\usepackage{mdframed}
\usepackage{mathrsfs}
\usepackage[most]{tcolorbox}
%------------------------
%Fonts I use, uncomment if you like to use them.
%The first is the general font, and the second a math font
\usepackage{mathpazo}
\usepackage{eulervm}
%------------------------
%This is so that we have standard fonts for the double-stroked symbols
%for reals, naturals etc. regardless of what font you use.
%Don't comment
\AtBeginDocument{
  \DeclareSymbolFont{AMSb}{U}{msb}{m}{n}
  \DeclareSymbolFontAlphabet{\mathbb}{AMSb}}
%------------------------

%----------------------------------------------
%User-defined environments
%Commented because we're not using them in this document
%The only uncommented ones are the Problem and Solution environment

% \newenvironment{theorem}[2][Theorem]{\begin{trivlist}
% \item[\hskip \labelsep {\bfseries #1}\hskip \labelsep {\bfseries #2.}]}{\end{trivlist}}
% \newenvironment{lemma}[2][Lemma]{\begin{trivlist}
% \item[\hskip \labelsep {\bfseries #1}\hskip \labelsep {\bfseries #2.}]}{\end{trivlist}}
% \newenvironment{exercise}[2][Exercise]{\begin{trivlist}
% \item[\hskip \labelsep {\bfseries #1}\hskip \labelsep {\bfseries #2.}]}{\end{trivlist}}
% \newenvironment{question}[2][Question]{\begin{trivlist}
% \item[\hskip \labelsep {\bfseries #1}\hskip \labelsep {\bfseries #2.}]}{\end{trivlist}}
% \newenvironment{corollary}[2][Corollary]{\begin{trivlist}
% \item[\hskip \labelsep {\bfseries #1}\hskip \labelsep {\bfseries #2.}]}{\end{trivlist}}
\newenvironment{problem}[2][Problem\!]{\begin{trivlist}
\item[\hskip \labelsep {\bfseries #1}\hskip \labelsep {\bfseries #2}]}{\end{trivlist}}
%\newenvironment{sub-problem}[2][]{\begin{trivlist}
%\item[\hskip \labelsep {\bfseries #1}\hskip \labelsep {\bfseries #2}]}{\end{trivlist}}
\newenvironment{solution}{\begin{proof}[\textbf{\textit{Solution}}] }{\end{proof}}
%----------------------------------------------

%----------------------------
%User-defined notations
\newcommand{\zz}{\mathbb Z}   %blackboard bold Z
\newcommand{\qq}{\mathbb Q}   %blackboard bold Q
\newcommand{\ff}{\mathbb F}   %blackboard bold F
\newcommand{\rr}{\mathbb R}   %blackboard bold R
\newcommand{\nn}{\mathbb N}   %blackboard bold N
\newcommand{\cc}{\mathbb C}   %blackboard bold C
\newcommand{\af}{\mathbb A}   %blackboard bold A
\newcommand{\pp}{\mathbb P}   %blackboard bold P
\newcommand{\id}{\operatorname{id}} %for identity map
\newcommand{\im}{\operatorname{im}} %for image of a function
\newcommand{\dom}{\operatorname{dom}} %for domain of a function
\newcommand{\cat}[1]{\mathscr{#1}}   %calligraphic category
\newcommand{\abs}[1]{\left\lvert#1\right\rvert} %for absolute value
\newcommand{\norm}[1]{\left\lVert#1\right\rVert} %for norm
\newcommand{\modar}[1]{\text{ mod }{#1}} %for modular arithmetic
\newcommand{\set}[1]{\left\{#1\right\}} %for set
\newcommand{\setp}[2]{\left\{#1\ \middle|\ #2\right\}} %for set with a property
\newcommand{\card}[1]{\#\,{#1}} %for cardinality of a set
\newcommand\m[1]{\begin{pmatrix}#1\end{pmatrix}} 

%Re-defined notations
\renewcommand{\epsilon}{\varepsilon}
\renewcommand{\phi}{\varphi}
\renewcommand{\emptyset}{\varnothing}
\renewcommand{\geq}{\geqslant}
\renewcommand{\leq}{\leqslant}
\renewcommand{\Re}{\operatorname{Re}}
\renewcommand{\Im}{\operatorname{Im}}
%----------------------------

\allowdisplaybreaks
 
 
\begin{document}
 
\title{Homework 8}
\author{Kevin Guillen\\[0.5em]
MATH 200 | Algebra I | Fall 2021}
\date{} 
\maketitle

May I have my proof for 11.3 graded please, thank you.

\begin{tcolorbox}
  \begin{problem} {11.1} Prove the statements in Remark 11.2.
  \end{problem}
\end{tcolorbox}
\begin{proof}
    \begin{itemize}
        \item[(a)] $a \mid a$ and $a \mid 0$.
        \begin{proof}
            Given $a \in R$. We can consider the multiplicative identity $1\in R$, we know that it satisfies $a1 = a$. Which by definition means $a \mid a$

            Now consider $0 \in R$, by definition it satisfies for $a \in R$, that $a0 = 0$. Meaning $a \mid 0$.
        \end{proof} 
        \item[(b)] $0 \mid a$ if and only if $a = 0$.
        \begin{proof}
            If $0 \mid a$ that means there exists $c \in R$ such that $0c = a$, but by definition $a$ must be 0. 

            If $a = 0$ then from (a) we know $0 \mid 0$. 
        \end{proof} 
        \item[(c)] $u \mid a$.
        \begin{proof}
            Because $u$ is a unit of $R$ that means there exists $u^{-1}\in R$. We also know $R$ is closed under multiplication therefore $u^{-1}a \in R$. Now consider the following,
            \begin{align*}
                u(u^{-1}a) = (uu^{-1})a = 1a = a
            \end{align*} 
            therefore $u\mid a$.
        \end{proof} 
        \item[(d)] $a \mid u$ if and only if $a \in R^{\times}$.
        \begin{proof}
            Given $a\mid u$, it means there exists $b \in R$ such that $ab = u$. Since $u$ is a unit there exists $u^{-1}\in R$ such that $uu^{-1} = 1$. Now consider the following,
            \begin{align*}
                uu^{-1} &= 1 \\
                (ab)(ab)^{-1} &= 1 \\
                (ab)(b^{-1}a^{-1}f &= 1 \\
                aa^{-1} &=1 \\
                1 &= 1. 
            \end{align*}
            Therefore if $a \mid u$ then $a \in R^{\times}$. 

            Given $a \in R^{\times}$, then it follows from (c) that $a \mid u$.
        \end{proof} 
        \item[(e)] If $a \mid b$ and $b \mid c$ then $a \mid c$.
        \begin{proof}
            Given that $a \mid b$ that means there exists $a' \in R$ such that $aa' = b$. Now given $b\mid c$ that means there exists $b' \in R$ such that $bb' = c$. Now consider the following,
            \begin{align*}
                bb' &= c \\
                aa'b' &= c\\
                a(a'b')&= c && a'b' \in R
            \end{align*}
            therefore $a \mid c$.
        \end{proof} 
        \item[(f)] If $a \mid b_1, \dots, a \mid b_n$ then $a\mid r_1b_1 + \dots + r_n b_n$.
        \begin{proof}
            Given that $a \mid b_1, \dots, a\mid b_n$, then we know there exists $c_i\in R$ for $i \in \set{1, \dots, n}$ such that $ac_i = b_i$. Therfore,
            \begin{align*}
                 r_1b_1 + \dots + r_nb_n = r_1(ac_1) + \dots + r_n(ac_n) 
            \end{align*} 
            and by definition of being in a ring we have distribution so therefore,
            \begin{align*}
                r_1b_1 + \dots + r_nb_n =a(r_1c_1 + \dots +r_nc_n). 
            \end{align*}
            Because $(r_1c_1 + \dots + r_nc_n) \in R$, that means $a \mid r_1b_1 + \dots + r_nb_n$
        \end{proof} 
        \item[(g)] $a\mid b$ if and only if $bR \subseteq aR$.
        \begin{proof}
            Given that $a\mid b$, that means there exists $c \in R$ such that $ac = b$. Therefore for all $r \in R$ we have,
            \begin{align*}
                br = (ac)r &= a(cr) && cr = r' \in R \\
                &= ar'\in aR
            \end{align*}
            Therefore $bR \subseteq aR$.

            Given that $bR \subseteq aR$, it means that for any $r \in R$ then there exists some $r' \in R$ such that $br = ar'$. Now consider the case where $r =1$, then $b = ar'$ which means $a\mid b$.
        \end{proof} 
        \item[(h)] $\thicksim$ is an equivalance relation on $R$.
        \begin{proof}
            First we see if $a \thicksim a$ that means $a\mid a$ and $a \mid a$ which follows from  (a).

            Next we see if $a \thicksim b $ and $b \thicksim c$ that means $a\mid b$ and $b \mid c$. From (e) we know then that $a \mid c$. Next we know it means that $b \mid a$ and $c \mid b$, but it also follows from (e) that $c \mid a$. Meaning $a \thicksim c$.

            Finally the symmetric follows trivially since $a \thicksim b$ implies $a\mid b$ AND $b \mid a$ which means $b \thicksim a$.
        \end{proof} 
    \end{itemize}
\end{proof}
\newpage
\begin{tcolorbox}
    \begin{problem} {11.3}
        Show that the ideal $(X)$ of $\zz[X]$ is a prime ideal but not a maximal ideal. 
    \end{problem}
\end{tcolorbox}
\begin{proof}
    Every $f(x) \in \zz[X]$ is of the form $\sum_{k = 0}^{n}a_kx^{k}$ where $a_0$ is the constant term which is an integer. Now consider the following map,
    \begin{align*}
        \pi: \zz[X] &\to \zz \\
        f(X) &\mapsto f(0) = a_0
    \end{align*}
    This simply takes a polynomial and evaluates it at 0 which we know then is a ring homomorphism. This just leaves the constant term, $a_0$. Now we want to show that this map is surjective so let $b \in \zz$ be an arbitrary integer. We want to show that there exists a polynomial $f\in \zz[X]$ such that $\pi(f) = f(0) = b$. We know this $f$ exists simply consider $f(X) = X + b$, 
    \begin{align*}
        f(0) = 0 + b = b.     
    \end{align*}
    This shows that $\pi$ is surective meaning $\im(\pi) = \zz$.
    Now if we consider the kernel of $\pi$ it is simply \[\ker(\pi) = \set{f \mid f\in \zz[X], f(0) = 0}\]

    If $f \in \ker(\pi)$ that means then $f \in (X)$. As a matter of fact $\ker(\pi) = (X)$. This is because $(X)$ is all polynomials with integer coefficients where every polynomial's constant term is 0. Therefore by the fundamental theorem of ring homomorphisms $\zz[X]/\ker(\pi)=\zz[X]/(X) \cong \im(\pi) = \zz$. Note though that $\zz$ is an integeral domain, and therefore so is $\zz[X]/(X)$ meaning $(X)$ is a prime ideal. The reason it is not maximal is because $\zz$ is not a field, meaning $\zz[X]/(X)$ is not a field, so $(X)$ cannot be maximal as we've proved in class.
\end{proof}

\begin{tcolorbox}
    \begin{problem} {12.6}
        Let $R =\zz[i] = \set{a + bi \mid a,b\in \zz}$. Computer the greatest common divisor of $\alpha = 10$ and $\beta = 1 - 5i$.
    \end{problem}
\end{tcolorbox}
\begin{solution}
    Let's consider the following,
    \begin{align*}
        10 &= (2i)(1-5i) -2i && N(-2i) = 4 < N(1-5i) = 26\\
        1-5i & = 2(-2i) + 1-i && N(1-i) = 2 < N(-2i) = 4\\
        -2i &= (1-i)(1-i) + 0
    \end{align*}

    Thus $1-i$ and its associates are the GCD of $\alpha$ and $\beta$
\end{solution}

\begin{tcolorbox}
    \begin{problem} {12.9} 
        Show that the ring $R$ = $\zz[\sqrt2] = \set{a +b \sqrt2 \mid a,b \zz}$ is a Euclidean domain. 
    \end{problem}
\end{tcolorbox}

\begin{proof}
    We already know that this ring forms an integral domain. Now let us consider the map given to us by,
    \begin{align*}
        N: R &\to \nn_0 \\
        a +b\sqrt2&\mapsto \abs{a^{2} - 2b^{2}} 
    \end{align*}

    First we see that for $0 \in R$, $N(0) = 0^{2} -2\cdot0^{2} = 0$. Now we want to show that this norm has a divison algorithm.  Let $\alpha$, $\beta \in \zz[\sqrt2]$, where $\beta \neq 0$. Then they are of the form $\alpha = x + y \sqrt2$, $\beta = w + z\sqrt2$. Consider the following,
    \begin{align*}
        \frac{\alpha}{\beta} = \frac{x + y\sqrt2}{w + z \sqrt2} = \frac{xw - 2yz + (yw - xz)\sqrt2}{w^{2} - 2z^{2}}.
    \end{align*}

    Now let $j,k \in \qq$, where $j = \frac{xw-2yz}{w^{2}-2z^{2}}$ and $k = \frac{(yw - xz)}{w^{2}-2z^{2}}$. Now let $n,m\in \zz$ be the smallest integers such that,
    \begin{align*}
        \abs{j -n} \leq \frac{1}{2} \\
        \abs{k -m} \leq \frac{1}{2}
    \end{align*}
    Now let $\gamma$ be defined as the following,
    \begin{align*}
        \gamma = (j - n) + (k -m)\sqrt2 = j + k\sqrt2 -n -m\sqrt2 = \frac{\alpha}{\beta} - (n+m\sqrt2)
    \end{align*}.

    We then get the following,
    \begin{align*}
        \gamma &= \frac{\alpha}{\beta} - (n+m\sqrt2) \\
        \beta\gamma &= \alpha - \beta(n+m\sqrt2) \\
        \beta\gamma + \beta(n+m\sqrt2) &= \alpha
    \end{align*}
    where $(n+m\sqrt2)$ and $\beta\gamma \in \zz[\sqrt2]$.


    Now consider the norm of $\gamma$, which will be $\abs{(j-n)^{2} -2(k-m)^{2}}$, using the triangle inequality and how we chose $n$ and $m$ we get the following,
    \begin{align*}
        \abs{(j-n)^{2} -2(k-m)^{2}} \leq \abs{j-n}^{2} + 2\abs{k-m}^{2} \leq \frac{1}{4} + 2\frac{1}{4} = \frac{3}{4}
    \end{align*}

    This is useuful to us because by definition we need $N(\beta\gamma) < N(\beta)$, where $\beta\gamma$ is the remainder when dividing $\alpha$ by $\beta$. We see this is true because consider the following, 
    \begin{align*}
        N(\beta\gamma) = N(\beta)N(\gamma) \leq N(\beta)\frac{3}{4}
    \end{align*}
    which is obviously less than $N(\beta)$. This means then that $\zz\sqrt2$ is indeed a Euclidean Domain since all together we have $\alpha = \beta(n+m\sqrt2) + \beta\gamma$ where $N(\beta\gamma) < N(\beta)$, and $(n+m\sqrt2),\beta\gamma \in \zz[\sqrt2]$ 
\end{proof}

\end{document}