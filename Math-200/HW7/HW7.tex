\documentclass[11pt]{article}
 
\usepackage[top=0.75in, bottom=1.25in, left=1in, right=1in]{geometry} 
\usepackage{amsmath,amsthm,amssymb} %this is THE math package
\usepackage{mathtools}
\usepackage{tikz}
\usepackage{graphicx}
\usepackage{fancybox}
\usepackage{hyperref}
\usepackage{varwidth}
\usepackage{mdframed}
\usepackage{mathrsfs}
\usepackage[most]{tcolorbox}
%------------------------
%Fonts I use, uncomment if you like to use them.
%The first is the general font, and the second a math font
\usepackage{mathpazo}
\usepackage{eulervm}
%------------------------
%This is so that we have standard fonts for the double-stroked symbols
%for reals, naturals etc. regardless of what font you use.
%Don't comment
\AtBeginDocument{
  \DeclareSymbolFont{AMSb}{U}{msb}{m}{n}
  \DeclareSymbolFontAlphabet{\mathbb}{AMSb}}
%------------------------

%----------------------------------------------
%User-defined environments
%Commented because we're not using them in this document
%The only uncommented ones are the Problem and Solution environment

% \newenvironment{theorem}[2][Theorem]{\begin{trivlist}
% \item[\hskip \labelsep {\bfseries #1}\hskip \labelsep {\bfseries #2.}]}{\end{trivlist}}
% \newenvironment{lemma}[2][Lemma]{\begin{trivlist}
% \item[\hskip \labelsep {\bfseries #1}\hskip \labelsep {\bfseries #2.}]}{\end{trivlist}}
% \newenvironment{exercise}[2][Exercise]{\begin{trivlist}
% \item[\hskip \labelsep {\bfseries #1}\hskip \labelsep {\bfseries #2.}]}{\end{trivlist}}
% \newenvironment{question}[2][Question]{\begin{trivlist}
% \item[\hskip \labelsep {\bfseries #1}\hskip \labelsep {\bfseries #2.}]}{\end{trivlist}}
% \newenvironment{corollary}[2][Corollary]{\begin{trivlist}
% \item[\hskip \labelsep {\bfseries #1}\hskip \labelsep {\bfseries #2.}]}{\end{trivlist}}
\newenvironment{problem}[2][Problem\!]{\begin{trivlist}
\item[\hskip \labelsep {\bfseries #1}\hskip \labelsep {\bfseries #2}]}{\end{trivlist}}
%\newenvironment{sub-problem}[2][]{\begin{trivlist}
%\item[\hskip \labelsep {\bfseries #1}\hskip \labelsep {\bfseries #2}]}{\end{trivlist}}
\newenvironment{solution}{\begin{proof}[\textbf{\textit{Solution}}] }{\end{proof}}
%----------------------------------------------

%----------------------------
%User-defined notations
\newcommand{\zz}{\mathbb Z}   %blackboard bold Z
\newcommand{\qq}{\mathbb Q}   %blackboard bold Q
\newcommand{\ff}{\mathbb F}   %blackboard bold F
\newcommand{\rr}{\mathbb R}   %blackboard bold R
\newcommand{\nn}{\mathbb N}   %blackboard bold N
\newcommand{\cc}{\mathbb C}   %blackboard bold C
\newcommand{\af}{\mathbb A}   %blackboard bold A
\newcommand{\pp}{\mathbb P}   %blackboard bold P
\newcommand{\id}{\operatorname{id}} %for identity map
\newcommand{\im}{\operatorname{im}} %for image of a function
\newcommand{\dom}{\operatorname{dom}} %for domain of a function
\newcommand{\cat}[1]{\mathscr{#1}}   %calligraphic category
\newcommand{\abs}[1]{\left\lvert#1\right\rvert} %for absolute value
\newcommand{\norm}[1]{\left\lVert#1\right\rVert} %for norm
\newcommand{\modar}[1]{\text{ mod }{#1}} %for modular arithmetic
\newcommand{\set}[1]{\left\{#1\right\}} %for set
\newcommand{\setp}[2]{\left\{#1\ \middle|\ #2\right\}} %for set with a property
\newcommand{\card}[1]{\#\,{#1}} %for cardinality of a set
\newcommand\m[1]{\begin{pmatrix}#1\end{pmatrix}} 

%Re-defined notations
\renewcommand{\epsilon}{\varepsilon}
\renewcommand{\phi}{\varphi}
\renewcommand{\emptyset}{\varnothing}
\renewcommand{\geq}{\geqslant}
\renewcommand{\leq}{\leqslant}
\renewcommand{\Re}{\operatorname{Re}}
\renewcommand{\Im}{\operatorname{Im}}
%----------------------------

\allowdisplaybreaks
 
 
\begin{document}
 
\title{Homework 7}
\author{Kevin Guillen\\[0.5em]
MATH 200 | Algebra I | Fall 2021}
\date{} 
\maketitle

%Use \[...\] instead of $$...$$

May I please have my proof for problem 10.3 graded, thank you.

\begin{tcolorbox}
    \begin{problem} {7.2}
        \begin{itemize}
            \item[(a)] Let $G$ be a cyclic group of prime order $p$. Show that Aut$(G)$ has order $p-1$.
            \item[(b)] Let $G$ be a group of order $pq$ with primes $p < q$ such that $p \nmid q-1$. Show that $G$ is cyclic
        \end{itemize}
    \end{problem}
\end{tcolorbox}
\begin{itemize}
    \item[(a)]
    \begin{proof}
        We know that $G$ is cyclic and therefore G = $\set{g, g^{2}, \dots, g^{p} = e} = <g>$. In other words $g$ is a generator of $G$. We know if we have an automorphism $f$, that $f(g)$ has to map to a generator of $G$. Note thoug that all elements of $G$ have order $p$ except $e$, meaning all elements of $G$ are generators except $e$. Therefore there are only $p-1$ choices, meaning there are only $p-1$ autmorphisms for $G$.
    \end{proof} 
    \item[(b)]
    \begin{proof}
        Let $H \in $ Syl$_p(G)$ and $H' \in $ Syl$_q(G)$. We know the number of Sylow $p-$subgroups of $G$ is $1+np$, and has to divde $pq$. We know though that $1 +np$ cannot divide $p$ and therefore must divide $q$. Recall though we are given that $q$ is another prime, therefore $1 +np = 1, q$. But consider the following,
        \begin{align*}
            1 +np &= q \\
            np &= q-1 \\
            \rightarrow p &\mid q-1  
        \end{align*}
        which can't be because we were given that $p \nmid q-1$. Therfore $1 +np =1$, which means $|\text{Syl}_p(G)| = 1$, and by the same reasoning $|\text{Syl}_q(G)| = 1$. From here we know then that $H \cap H' = e$, and therefore when considring the union of these 2 subgroups we know there will be $p +q -1$ elements. Note though that \[p +q -1 < 2q \leq pq\]
        which means there exists and element $e \neq a \in G$, that is neither in $H$ or $H'$, and o$(a) = pq$. Which means that $G$ is indeed cyclic.
    \end{proof} 
\end{itemize}
\begin{tcolorbox}
  \begin{problem} {9.2}
    \begin{itemize}
        \item[(a)] Let $R$ be a finite integral domain. Show that $R$ is a field.
        \item[(b)] Let $R$ be a division ring. Show that $Z(R)$ is a field. 
    \end{itemize}
  \end{problem}
\end{tcolorbox}
\begin{itemize}
    \item[(a)]
    \begin{proof}
        Given that $R$ is a finite integral domain, all that is missing to show it is a field is that every element has a multiplicative inverse. To show this let us consider a non-zero element $a\in R$. We want to show this element has an inverse by showing there is some element $r \in R$ such that $ar = 1_R$. We know because $R$ is finite we can consider all its elements as $\set{r_1, r_2, \dots, r_n}$ for some $n \in \nn$. 

        Next we know the set $\set{ar_i | r_i \in R, i = 1, \dots, n}$ must be the same size as $R$. This is because for two elements $r_i, r_j \in R$, $i \neq j$, we would have $ar_i = ar_j$, but we are in an integral domain so this implies $r_i = r_j$ which would be a contradiction. Therefore there must be some $i \in \set{1, \dots, n}$ such that $ar_i = 1_R$. Meaning for any nonzero element $a \in R$, it has a multiplicative inverse in $R$, making $R$ a field.
    \end{proof} 
    \item[(b)]
    \begin{proof}
        Given that $R$ is a division ring, we know that every element in $Z(R)$ commutes with every element in $R$. Consider an element $x \in Z(R)$, we want to show that $x^{-1}\in Z(R)$. Let $r \in R$, we know that $xr = rx$ and that $(xr) \in R$, meaning there exists $(xr)^{-1}$ because $R$ is a division ring. We see though,
        \begin{align*}
            (xr)^{-1} &= (rx)^{-1} \\
            r^{-1}x^{-1} &= x^{-1}r^{-1}.            
        \end{align*}

        Notice though that $r$ was arbitrary in $R$, and therefore $x^{-1}$ commutes with every element in $R$, meaing $x^{-1} \in Z(R)$. Therefore $Z(R)$ is indeed a field. 
    \end{proof} 
\end{itemize}

\begin{tcolorbox}
    \begin{problem} {10.2}
        \begin{itemize}
            \item[(a)] Show that $\set{0}$ and $D$ are the only ideals of $D$.
            \item[(b)] Let $R$ be a non-trivial ring and let $f: D\mapsto R$ be a ring homomorphism. Show that $f$ is injective. 
        \end{itemize}
    \end{problem}
\end{tcolorbox}
\begin{itemize}
    \item[(a)]
    \begin{proof}
        Let $I$ be an ideal of $D$ such that $I \neq \set{0}$. This means there is some element $a \in I$, and because $I \subseteq D$, $a \in D$. Now let $b$ be any element in $D$. We know $D$ is closed under multiplication so $ba^{-1} \in D$. We also know that $(ba^{-1})a \in I$, because $a \in I$. Observe though,
        \begin{align*}
            (ba^{-1})a &\in I \\
            b(a^{-1}a) &\in I \\
            b1_D &\in I \\
            b &\in I.
        \end{align*}
        We said $b$ to be any element in $D$, thus if $I$ as any non-zero element in it, $D \subseteq I$.Wwe also had though that $I \subseteq D$, therfore $I = D$ if $I \neq \set{0}$, given that $D$ is a division ring.
    \end{proof} 
    \item[(b)]
    \begin{proof}
        In class we defined ring homomorphimsms to respect multiplicative identites between rings. This is key because consider $a \in ker(f)$ and assume $a \neq 0$. That means $f(a) = 0$, we also know $\exists a^{-1} \in D$, so consider the following,
        \begin{align*}
            f(1) &= f(aa^{-1}) \\
            &= f(a)f(a^{-1}) \\
            &= 0f(a^{-1}) \\
            &= 0.
        \end{align*}
        This can't be though since a proper ring homomorphimsm as we defined in class we must have $f(1) = 1$. Therefore the kernel of $f$ must be trivial which means, $f$ is indeed injective.
    \end{proof} 
\end{itemize}

\begin{tcolorbox}
    \begin{problem} {10.3}
        \begin{itemize}
            \item[(a)] Let $F$ be a field. Show that the characteristic of $F$ is either a prime number or 0.
            \item[(b)] Let $p$ be a prime and let $R$ be a ring with $p$ elements. Show that $R \cong \zz/p\zz$. 
        \end{itemize}
    \end{problem}
\end{tcolorbox}
\begin{itemize}
    \item[(a)]
    \begin{proof}
        If char$(F) = 0$ then we are done. We know that char$(F) \neq 1$ because that would imply $1 = 0$ which means $F$ is not a field. Now if char$(F) = n$, assume $n$ to be composite, meaning there exists 2 natural numbers, $k,l$, where $1 < k, l < n$ such that $n = kl$. Consider the following,
        \begin{align*}
            (k \cdot 1)(l \cdot 1) &= kl \cdot 1 \\
            &= n \cdot 1 \\
            &= 0.
        \end{align*}
        Recall though a field is an integral domain meaning there are no zero divisors, therefore either $(k \cdot 1) = 0$ or $(l \cdot 1) = 0$. Also recall though $n$ is supposed to be the smallest non-negative integer such that $n \cdot 1 = 0$. So if either of the two cases were to be true it would contradict the minimality of $n$. Therefore if char$(F) = n$, $n$ has to be a prime number. All together we have show that char$(F) = 0$ or char$(F) = p$ where $p$ is a prime. 
    \end{proof} 
    \item[(b)]
    \begin{proof}
        Consider the unique homomorphism from the ring of integers to any ring $R$,
        \begin{align*}
            f: \zz &\to R \\
            z & \mapsto z1_R.
        \end{align*}

        We already know this is indeed a ring homomorphism. So by definition we know the image of $f$ will be mapped to a subring of $R$. The only two options is $\set{0}$ and $R$ itself since $R$ has $p$ elements and therefore the order of the subring must divide $p$. We know it cant be $\set{0}$ though since ring homomorphism respect multiplicative identities, meaning $f(1_\zz) = 1_R$. Therfore $f$ is surjective meaning im$(f) = R$. 

        Now according to the fundamental theorem of homomorphisms $\zz/\ker(f) \cong \text{im}(f)$. We already know im$(f) = R$. We also know that the only subrings of $\zz$ are of the form $n\zz$, meaning $\ker(f) = n\zz$ for some $n \in \nn_0$. All this together gives us $\zz/n\zz \cong R$. Isomorphisms though are 1-1 meaning the two rings must be of the same order, $R$ has $p$ elements, so $\zz/n\zz$ must have $p$, elements, but this can only be true if $n =p$. Therefore $\zz/p\zz \cong R$, as desired. 

    \end{proof} 
\end{itemize}

\begin{tcolorbox}
    \begin{problem} {10.4}
        Let $R$ be a ring. An element $r \in R$ is called $\textit{nilpotent}$ if there exists $n \in \nn$ such that $r^{n} = 0$.
        \begin{itemize}
            \item[(a)] Show that if $r\in R$ is nilpotent then $1-r$ is a unit of $R$. 
            \item[(b)] Show that if $R$ is commutative then the nilpotent elements of $R$ form an ideal $N$ of $R$.
            \item[(c)] Show that if $R$ is commutative and $N$ is the ideal of nilpotent elements then 0 is the only nilpotent element of $R/N$  
        \end{itemize}
    \end{problem}
\end{tcolorbox}
\begin{itemize}
    \item[(a)]
    \begin{proof}
        Given that $r$ is nilpotent, consider the following,
        \begin{align*}
            1 = (1 - 0) = (1 - r^{n}) = (1-r)(1 + r + r^{2} + \dots + r^{n-1}).
        \end{align*}
        This means that the inverse of $1-r$ is simply $(1 + r + \dots + r^{n})$
    \end{proof} 
    \item[(b)]
    \begin{proof}
        Let $N$ be the set of all nilpotent elements of $R$. It is clear that $0$ is in this set because $0^{1} = 0$. 

        Let $x,y \in N$, we want to now show that $(x-y) \in N$. We will use what we know about binomial expansion to show there exists and $n \in \nn$ to show that $(x-y)^{n} = 0$. We already know there exists a $n_1$ and $n_2$ such that $x^{n_1} = 0$ and $y^{n_2} = 0$. That means for all $n_1' > n_1$ and $n_2' > n_2$ we have $x^{n_1'} = 0$ and $y^{n_2'} = 0$. 

        That means if we let $b = \text{max}(n_1, n_2)$, $x^{b} = 0 = y^{b}$.

        This is important to us because, ignoring the coefficents for a moment, we know that $(x-y)^{n}$ expanded is,
        \[ax^{n}y^{0} + bx^{n-1}y^{1} + \dots + zx^{0}y^{n}.\]
        
        This means we can find a $n$ sufficently large enough such that for each term in the expansion $x^{p}y^{q}$ either $p$ or $q$ will be greater than $b$ resulting in the term being 0. It's obvious in this case $n = 2b$, this way for every term $x^{p}y^{q}$ either $p \geq b$ or $q \geq b$. Meaning every term will then be 0, implying $(x -y)^{n} = 0$, therefore $(x-y) \in N$

        Finally we want to show that for any $a \in R$ and for any $x \in N$ that $ax \in N$. We know there exists some $n \in \nn$ such that $x^{n} = 0$. So consider the following,
        \begin{align*}
            (ax)^{n} &= a^{n}x^{n} \\
            &=a^{n}0 \\
            &= 0
        \end{align*}
        therfore $(ax) \in N$. Altogether we have that $N$ is indeed an ideal of $R$, given that $R$ is commutative.
    \end{proof}
    \item[(c)]
    \begin{proof}
        Let $r\in R$, and let us denote $r + N$ as $x$. If $x$ is nilpotent that means there is a $n \in \nn$ such that,
        \begin{align*}
            x^{n} = (r + N)^{n} = r^{n} + N = 0 + N = N
        \end{align*}
        Which means that $r^{n} \in N$, and therfore exists an $k \in \nn$ such that $(r^{n})^{k} = 0$. This means $r$ is a nilpotent element of $R$, meaning $x = N$, which is the zero of $R/N$.
    \end{proof} 
\end{itemize}

\end{document}