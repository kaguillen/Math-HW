\documentclass[11pt]{article}
 
\usepackage[top=0.75in, bottom=1.25in, left=1in, right=1in]{geometry} 
\usepackage{amsmath,amsthm,amssymb} %this is THE math package
\usepackage{mathtools}
\usepackage{tikz}
\usepackage{graphicx}
\usepackage{fancybox}
\usepackage{hyperref}
\usepackage{varwidth}
\usepackage{mdframed}
\usepackage{mathrsfs}
\usepackage[most]{tcolorbox}
%------------------------
%Fonts I use, uncomment if you like to use them.
%The first is the general font, and the second a math font
\usepackage{mathpazo}
\usepackage{eulervm}
%------------------------
%This is so that we have standard fonts for the double-stroked symbols
%for reals, naturals etc. regardless of what font you use.
%Don't comment
\AtBeginDocument{
  \DeclareSymbolFont{AMSb}{U}{msb}{m}{n}
  \DeclareSymbolFontAlphabet{\mathbb}{AMSb}}
%------------------------

%----------------------------------------------
%User-defined environments
%Commented because we're not using them in this document
%The only uncommented ones are the Problem and Solution environment

% \newenvironment{theorem}[2][Theorem]{\begin{trivlist}
% \item[\hskip \labelsep {\bfseries #1}\hskip \labelsep {\bfseries #2.}]}{\end{trivlist}}
% \newenvironment{lemma}[2][Lemma]{\begin{trivlist}
% \item[\hskip \labelsep {\bfseries #1}\hskip \labelsep {\bfseries #2.}]}{\end{trivlist}}
% \newenvironment{exercise}[2][Exercise]{\begin{trivlist}
% \item[\hskip \labelsep {\bfseries #1}\hskip \labelsep {\bfseries #2.}]}{\end{trivlist}}
% \newenvironment{question}[2][Question]{\begin{trivlist}
% \item[\hskip \labelsep {\bfseries #1}\hskip \labelsep {\bfseries #2.}]}{\end{trivlist}}
% \newenvironment{corollary}[2][Corollary]{\begin{trivlist}
% \item[\hskip \labelsep {\bfseries #1}\hskip \labelsep {\bfseries #2.}]}{\end{trivlist}}
\newenvironment{problem}[2][Problem\!]{\begin{trivlist}
\item[\hskip \labelsep {\bfseries #1}\hskip \labelsep {\bfseries #2}]}{\end{trivlist}}
%\newenvironment{sub-problem}[2][]{\begin{trivlist}
%\item[\hskip \labelsep {\bfseries #1}\hskip \labelsep {\bfseries #2}]}{\end{trivlist}}
\newenvironment{solution}{\begin{proof}[\textbf{\textit{Solution}}] }{\end{proof}}
%----------------------------------------------

%----------------------------
%User-defined notations
\newcommand{\zz}{\mathbb Z}   %blackboard bold Z
\newcommand{\qq}{\mathbb Q}   %blackboard bold Q
\newcommand{\ff}{\mathbb F}   %blackboard bold F
\newcommand{\rr}{\mathbb R}   %blackboard bold R
\newcommand{\nn}{\mathbb N}   %blackboard bold N
\newcommand{\cc}{\mathbb C}   %blackboard bold C
\newcommand{\af}{\mathbb A}   %blackboard bold A
\newcommand{\pp}{\mathbb P}   %blackboard bold P
\newcommand{\id}{\operatorname{id}} %for identity map
\newcommand{\im}{\operatorname{im}} %for image of a function
\newcommand{\dom}{\operatorname{dom}} %for domain of a function
\newcommand{\cat}[1]{\mathscr{#1}}   %calligraphic category
\newcommand{\abs}[1]{\left\lvert#1\right\rvert} %for absolute value
\newcommand{\norm}[1]{\left\lVert#1\right\rVert} %for norm
\newcommand{\modar}[1]{\text{ mod }{#1}} %for modular arithmetic
\newcommand{\set}[1]{\left\{#1\right\}} %for set
\newcommand{\setp}[2]{\left\{#1\ \middle|\ #2\right\}} %for set with a property
\newcommand{\card}[1]{\#\,{#1}} %for cardinality of a set
\newcommand\m[1]{\begin{pmatrix}#1\end{pmatrix}} 

%Re-defined notations
\renewcommand{\epsilon}{\varepsilon}
\renewcommand{\phi}{\varphi}
\renewcommand{\emptyset}{\varnothing}
\renewcommand{\geq}{\geqslant}
\renewcommand{\leq}{\leqslant}
\renewcommand{\Re}{\operatorname{Re}}
\renewcommand{\Im}{\operatorname{Im}}
%----------------------------

\allowdisplaybreaks
 
 
\begin{document}
 
\title{Homework 6}
\author{Kevin Guillen\\[0.5em]
MATH 200 | Algebra I | Fall 2021}
\date{} 
\maketitle

%Use \[...\] instead of $$...$$
Can I please have my 9.1 graded, thank you.

\begin{tcolorbox}
    \begin{problem}{3.3}
        Let $G$ be a group. Show that $Z(G)$ and $G'$ are characteristic subgroups of $G$. 
    \end{problem}
\end{tcolorbox}
\begin{proof}
  \begin{itemize}
      \item[(a)] Showing the center of $G$ is characteristic subgroup of $G$. First let $\phi \in \text{Aut}(G)$. We need to show $\phi(Z(G)) = Z(G)$. Let $z\in Z(G)$, and $x \in G$. Then we know there exists some $y\in G$ such that $x = \phi(y)$. This gives us the following,
      \begin{align*}
          \phi(z)x &= \phi(z)\phi(y) \\
          &= \phi(zy) \\
          &= \phi(yz) \\
          &= \phi(y)\phi(z) = x\phi(z)
      \end{align*} 
      We can see then that $\phi(z)$ commutes with all elements of $G$. Therefore $\phi(Z(G)) \subset Z(G)$. Applying the same reasoning to $\phi^{-1}$ we get that $\phi^{-1}(Z(G)) = \subset Z(G)$ which gives us that $Z(G) \subset \phi(Z(G))$ and thus $Z(G) = \phi(Z(G))$ as desired.
      \item[(b)] Show the commutator subgroup of $G$ is characteristic. Consider any $\phi \in \text{Aut}(G)$. We want to show that $\phi(G') = G'$. Let $x$ be a commutator then $x = a^{-1}b^{-1}ab$. We have then the following,
      \begin{align*}
          \phi(x) &= \phi(a^{-1}b^{-1}ab) \\
          &= \phi(a^{-1})\phi(b^{-1})\phi(a)\phi(b) \\
          &= \phi(a)^{-1}\phi(b)^{-1}\phi(a)\phi(b) 
      \end{align*} 
      that $\phi(x)$ is indeed a commutator as well. 
      Now let $g \in G'$ then we know for some commutators $x_i$ for $1 \leq i \leq n$ that $g = x_1 \dots x_n$. Now with the following,
      \begin{align*}
          \phi(g) &= \phi(x_1 \dots x_n) \\
          &= \phi(x_1)\dots \phi(x_n)
      \end{align*}
      we see that $\phi(g)$ must be in $G'$ because every $\phi(x_i)$ is in $G'$, and $G'$ is a subgroup. Therefore $\phi(G') = G'$ meaning $G'$ is characteristic. 
  \end{itemize}  
\end{proof}
\newpage 
\begin{tcolorbox}
  \begin{problem} {9.1} (a) Determine the unit group of the ring $\zz[i] = \set{a + bi \mid a,b \in \zz}$, the ring of Gaussian integers.
    
    (b) Show that the ring $\zz[\sqrt{2}] = \set{a + b \sqrt{2} \mid a,b \in \zz}$ has infintely many units and find all units of finite order. 
  \end{problem}
\end{tcolorbox}
\begin{itemize}
    \item[(a)]
    \begin{proof}
        If $x \in \zz[i]$ is a unit that means there exists $y\in \zz[i]$ such that $xy = 1$. We know that the Guassian integers are a subring of $\cc$. Meaning we have consider the norm squared of these values. Where the norm of is defined as the following,
        \begin{align*}
            N(a + bi) = \sqrt{(a +bi)(a-bi)} = \sqrt{a^{2} + b^{2}}
        \end{align*}

        When two values are equal we know their norm will be equal. Therfore we have, $N(xy)^{2} = N(1)^{2} = 1$. Recall though both $x$ and $y$ are of the form $a + bi$ and $c + di$ respectively. This gives us the following,
        \begin{align*}
            N(xy)^{2} &= 1  && \text{Norm is multiplicative}\\
            N(x)^{2}N(y)^{2} &= 1 \\
            N(a+bi)^{2}N(c + di)^{2} &= 1 \\
            (a^{2} + b^{2})(c^{2} + d^{2}) &= 1
        \end{align*}

        Thus $(a^{2} + b^{2})$ and $(c^{2} + d^{2})$ must both be non negative integers and must be equal to 1. In other words if $a^{2} + b^{2} =1$ then $a^{2}$ and $b^{2}$ must be less than or equal to 1. This gvies us the following solutions as $\pm1 + 0i$ and $0 \pm 1i$. Thus the only units in the Guassian integers are \[\set{1, i , -1, -i}\]
 
    \end{proof}
    \item[(b)]
    \begin{proof}
        If $x\in \zz[\sqrt{2}]$ is a unit that means there exists a $y \in \zz[\sqrt{2}]$ such that $xy = 1$. From 111b we know the norm of this ring is simply,
        \begin{align*}
            N: \zz[\sqrt{2}] &\rightarrow \nn \\
            a +b\sqrt{2} &\mapsto a^{2} - 2b^{2}
        \end{align*}
        Next we will use the fact that $N(x) =1$ if and only if $x$ is unit. The reasoning for this is simple in that if $x$ is a unit there there exists a $y$ such that $xy = 1$. Taking the norm then implies that $N(xy) = N(x)N(y) =1$ which is only possible if $N(x) =1$ and $N(y)$. Then if $N(x) =1 $ that means $x \cdot \bar{x} = 1$ (where $\bar{x}$ is the conjugate of $x$). Using this to our advantage we can see that $3 + 2\sqrt{2}$ must be a unit because $9 -2\cdot4 = 1$. Let us consider the power of $(3 + 2\sqrt2)$ now though. We see through the following,
        \begin{align*}
            N((3 + 2\sqrt{2})^{n}) &= ? \\
            N(\underbrace{(3 + 2\sqrt{2})\cdot \dots \cdot(3 + 2\sqrt(2))}_{n \text{ times}}) &=  &&\text{Norm is multiplicative}\\
            \underbrace{N(3 + 2 \sqrt{2})\cdot \dots \cdot N(3 +2\sqrt{2})}_{n \text{ times}} &= && \\
            1 \cdot \dots \cdot 1 &= 1
        \end{align*}
        Thus we have all powers of $(3+2\sqrt{2})$ are units. From here we must take a look at the order of $(3+ 2\sqrt{2})$. It is obvious it does not have a finite order since $1 < (3 + 2 \sqrt{2})$ and when taking powers of $(3 + 2 \sqrt{2})$ we are simply multiplying and adding positive numbers which means it will only be increasing and can never decrease to be $1$. As a result the order of $(3 + 2\sqrt{2})$ is infinite, and we showed every power of it is a unit, meaning there is an infinite number of units in $\zz[\sqrt{2}]$. Therefore the only units of finite order must be $-1$ and $1$. 
    \end{proof} 


\end{itemize}

\end{document}