\documentclass[12pt]{article}
%------------------------------- BEGIN PREAMBLE
% packages used
\usepackage{amssymb,amsmath,amsfonts,mathrsfs,pgffor,marvosym,amsthm, mathrsfs, mathtools}
\DeclarePairedDelimiter\set\{\}
% macros
\newcommand      {\Nm}         {{\mathbb N}}
\newcommand      {\Zm}         {{\mathbb Z}}
\newcommand      {\Qm}         {{\mathbb Q}}
\newcommand      {\Rm}         {{\mathbb R}}
\newcommand      {\Cm}         {{\mathbb C}}
\newcommand      {\vb}        {\mathbf}
\newcommand      {\PP}        {{\mathscr P}}
\newcommand      {\BB}        {{\mathscr B}}
\newcommand {\lines}[1] {\foreach \n in {1,...,#1}{ \vspace{9mm} \hrule height 
0.2pt  }\vspace{2mm} }

% adjustment of page dimensions
\textwidth=7in
\textheight=9.8in
\topmargin= -0.8in
\oddsidemargin= -0.3in
\evensidemargin= 0.0in
\setlength{\parskip}{1ex plus0.5ex minus0.2ex}
\setlength{\jot}{10pt}
%-------------------------------- END PREAMBLE
\begin{document}
\begin{flushright}
    Name: Kevin Guillen \\*
    Student ID: 1747199
\end{flushright}
\begin{center}
    {\bf Math 117 - SS2 - HW 1 - August 6th}
\end{center}

\begin{itemize}

    \item[$\textbf{[1]}$]
    Confirm that the following form a group. Furthermore, determine which are Abelian. 
    \begin{itemize}
    
    \vspace{.3cm}
    \item[(a)]
    The cyclic group $\langle g \rangle = \{ e,g,g^2,g^3,\dots,g^{n-1} \}$ of order $n$ is defined to be the collection of powers of $g$ under the restrictions that $g^n = e$ for $e = g^0$ representing the identity element and $g^i = g^j$ if and only if $i = j$. 
    \begin{proof}
        Identity: For this $e$ serves as the identity element, and we see that for any $g^j \in \langle g \rangle $ that \[e + g^j = g^0 + g^j = g^{0+j} = g^j = g^{j+0} = g^j + g^0 = g^j + e\]
    
        Inverses: We see for any $g^j \in \langle g \rangle$ there exists $g^{n-j}\in \langle g \rangle $ such that \[g^j + g^{n-j} = g^{j + n - j} = g^n = e\]
    
        Associativity: For any $g^i, g^j, \text{ and }g^k \in \langle g \rangle $, we have,
        \[g^i + (g^j + g^k) = g^i + g^{j + k} = g^{i + (j+k)} = g^{(i+j) +k} = g^{i+j} + g^k = (g^i + g^j) +g^k\]
    
        Commutativity: For any $g^i, g^j \in \langle g \rangle $ we see,
        \[g^i + g^j = g^{i+j} = g^{j+i} = g^j + g^i\]
    
        Therefore $\langle g \rangle $ is indeed a group and abelian. 
    \end{proof}
    
    \vspace{.3cm}
    \item[(b)]
    Let $\mathcal{S} = \{a,b\}$ be a collection of two distinct symbols. The \textit{free group} on two generators, denoted by $\text{Free}(\mathcal{S})$, is defined to be the collection of all finite strings that can be formed from the four symbols $a$, $a^{-1}$, $b$, and $b^{-1}$ such that no $a$ appears directly next to an $a^{-1}$ and no $b$ appears directly next to a $b^{-1}$. This collection comes attached with the operation of concatenation of strings. 
    \begin{proof}
        Identity: Since Free$(\mathcal{S})$ is the collection of all finite strings that can be formed with elements in $\mathcal{S}$. We can take string of length 0 to be our identity $e$. From here we see for any string $\overline{w}\in \text{Free}(\mathcal{S})$, \[e + \overline{w} = \overline{w} = \overline{w} + e.\]
    
        Associativity: Let $\overline{w}, \overline{v}, \text{ and } \overline{z}$ be arbitrary strings from Free$(\mathcal{S})$, we can see,
        \[\overline{w} + (\overline{v} + \overline{z}) = \overline{w} + \overline{vz} = \overline{wvz} = \overline{wv} + \overline{z} = (\overline{w} + \overline{v}) + \overline{z}\]
    
        Inverses: Let $\overline{w}$ be a string from Free$(\mathcal{S})$. The inverse of $\overline{w}$ will simply be the inverse of each character $(a \to a^{-1})$ in reverse order. 
    
        $\overline{w}$ is composed of characters, we can write it out as \[\overline{w} = w_0 w_1 \dots w_n.\] Meaning the inverse of $\overline{w}$ will be of the form \[w_n^{-1}w_{n-1}^{-1}\dots w_0^{-1}.\] Thus,
        \begin{align*}
            \overline{w} + \overline{w}^{-1}  &= w_0 w_1 \dots w_n + w_n^{-1}w_{n-1}^{-1}\dots w_0^{-1} \\
            &= w_0 w_1 \dots w_n w_n^{-1}w_{n-1}^{-1}\dots w_0^{-1} \\
            &= w_0 w_1 \dots w_{n-1}w_{n-1}^{-1}\dots w_0^{-1} \\ 
            &\vdots \\
            &= w_0 w_0^{-1} \\
            &= e
        \end{align*}
    
        We know this inverse exists since Free$(\mathcal{S})$ is the collection of all finite strings from $\mathcal{S}$
    \end{proof}
    
    \end{itemize}
    
    \vspace{.5cm}
    
    \item[$\textbf{[2]}$]
    Confirm that the following form a field.
    \begin{itemize}
    
    \vspace{.3cm}
    \item[(a)]
    Let $\mathbb{Z}/p\mathbb{Z}$ for $p$ a prime represent the collection of equivalence classes formed out of the equivalence relation on $\mathbb{Z}$ where $n \sim m$ if $n \equiv m \pmod{p}$. Addition and multiplication are defined by:
    \begin{equation*}
    [n] + [m] = [n + m] \hspace{.3cm} \text{and} \hspace{.3cm} [n] \cdot [m] = [n \cdot m]
    \end{equation*}
    You may assume that $\mathbb{Z}$ has all the standard properties such as associativity, commutativity, etc...
    
    \vspace{.3cm}
    \item[(b)]
    Consider the collection $\mathbb{Q}(\sqrt{2}) = \{a + b\sqrt{2} \in \mathbb{R} \hspace{.1cm} | \hspace{.1cm} a,b \in \mathbb{Q}\}$ that comes attached with the binary operations:
    \begin{equation*}
    \begin{split}
    (a_1 + b_1\sqrt{2}) + (a_2 + b_2\sqrt{2}) &= (a_1 + a_2) + (b_1 + b_2)\sqrt{2} \\
    (a_1 + b_1\sqrt{2}) \cdot (a_2 + b_2\sqrt{2}) &= (a_1a_2 + 2b_1b_2) + (a_1b_2 + a_2b_1)\sqrt{2}
    \end{split}
    \end{equation*}
    You may assume that $\mathbb{Q}$ has all of the standard properties of a field. 
    
    \end{itemize}
    
    \vspace{.5cm}
    
    \item[$\textbf{[3]}$]
    The fact that $\mathbb{Z}/p\mathbb{Z}$ (where $p$ is a prime) is a field shows that not quite all the laws of elementary arithmetic hold in fields; in $\mathbb{Z}/2\mathbb{Z}$, for instance, $1 + 1 = 0$. Prove that if $\mathbb{F}$ is a field, then either the result of repeatedly adding $1$ to itself is always different from $0$, or else the first time that it is equal to $0$ occurs when the number of summands is a prime. (The \textit{characteristic} of the field $\mathbb{F}$, denoted by $\text{char}(\mathbb{F})$, is defined to be $0$ in the first case and the crucial prime in the second.) 
    
    \vspace{.5cm}
    
    \item[$\textbf{[4]}$]
    Let $\mathbb{R}^2 = \{ (x,y) \hspace{.1cm} | \hspace{.1cm} x,y \in \mathbb{R} \}$. 
    \begin{itemize}
    
    \vspace{.3cm}
    \item[(a)]
    If addition and multiplication are defined by:
    \begin{equation*}
    (x,y) + (z,w) = (x + z,y + w) \hspace{.3cm} \text{and} \hspace{.3cm} (x,y) \cdot (z,w) = (x \cdot z,y \cdot w)
    \end{equation*}
    does $\mathbb{R}^2$ become a field?
    
    \vspace{.3cm}
    \item[(b)]
    If addition and multiplication are defined by:
    \begin{equation*}
    (x,y) + (z,w) = (x + z,y + w) \hspace{.3cm} \text{and} \hspace{.3cm} (x,y) \cdot (z,w) = (x \cdot z - y \cdot w,x \cdot w + y \cdot z)
    \end{equation*}
    is $\mathbb{R}^2$ a field then?
    
    \end{itemize}
    
    \vspace{.5cm}
    
    \item[$\textbf{[5]}$]
    Show that for any field $\mathbb{F}$ the set $\mathbb{F}^n = \{ (x_1,\dots,x_n) \hspace{.1cm} | \hspace{.1cm} x_1,\dots,x_n \in \mathbb{F} \}$ forms a vector space over the field $\mathbb{F}$ where addition of vectors is taken componentwise. If $\mathbb{F} = \mathbb{Z}/p\mathbb{Z}$ for $p$ a prime, how many vectors are there in $\mathbb{F}^n$?
    
    
    
    \item[$\textbf{[6]}$]
    Consider the $\mathbb{C}$-vector space $\mathbb{C}^3$. For each of the following determine whether the subsets form a vector subspace:
    \begin{itemize}
    
    \vspace{.3cm}
    \item[(a)]
    $U_1 = \{ (z_1,z_2,z_3) \in \mathbb{C}^3 \hspace{.1cm} | \hspace{.1cm} z_1 \in \mathbb{R} \}$
    
    \vspace{.3cm}
    \item[(b)]
    $U_2 = \{ (z_1,z_2,z_3) \in \mathbb{C}^3 \hspace{.1cm} | \hspace{.1cm} z_1 = 0 \}$
    
    \vspace{.3cm}
    \item[(c)]
    $U_3 = \{ (z_1,z_2,z_3) \in \mathbb{C}^3 \hspace{.1cm} | \hspace{.1cm} z_1 = 0 \hspace{.2cm} \text{or} \hspace{.2cm} z_2 = 0 \}$
    
    \vspace{.3cm}
    \item[(d)]
    $U_4 = \{ (z_1,z_2,z_3) \in \mathbb{C}^3 \hspace{.1cm} | \hspace{.1cm} z_1 + z_2 = 0 \}$
    
    \vspace{.3cm}
    \item[(e)]
    $U_5 = \{ (z_1,z_2,z_3) \in \mathbb{C}^3 \hspace{.1cm} | \hspace{.1cm} z_1 + z_2 = 1 \}$
    
    \end{itemize}
    
    \vspace{.5cm}
    
    \item[$\textbf{[7]}$]
    \begin{itemize}
    
    \item[(a)]
    Under what conditions on the scalar $\xi \in \mathbb{C}$ are the vectors $(1 + \xi, 1 - \xi)$ and $(1 - \xi, 1 + \xi)$ in $\mathbb{C}^2$ (over the field $\mathbb{C}$) linearly dependent?
    
    \vspace{.3cm}
    \item[(b)]
    Under what conditions on the scalar $\xi \in \mathbb{R}$ are the vectors $(\xi,1,0)$, $(1,\xi,1)$, and $(0,1,\xi)$ in $\mathbb{R}^3$ (over the field $\mathbb{R}$) linearly dependent?
    
    \vspace{.3cm}
    \item[(c)]
    What is the answer for (b) for $\mathbb{Q}^3$ (over the field $\mathbb{Q}$) in place of $\mathbb{R}^3$ (over the field $\mathbb{R}$). 
    
    \end{itemize}
    
    \vspace{.5cm}
    
    \item[$\textbf{[8]}$]
    For any field $\mathbb{F}$ let $\mathbb{F}[x] = \{ a_0 + a_1x + \dots + a_nx^n \hspace{.1cm} | \hspace{.1cm} a_0,a_1,\dots,a_n \in \mathbb{F} \}$ where $x^i = x^j$ if and only if $i = j$. 
    \begin{itemize}
    
    \vspace{.3cm}
    \item[(a)]
    If the addition of polynomials is given by the standard procedure of combining like powers of $x$ show that $\mathbb{F}[x]$ forms a vector space over $\mathbb{F}$.
    
    \vspace{.3cm}
    \item[(b)]
    A polynomial $p(x) \in \mathbb{F}[x]$ is called \textit{even} if $p(-x) = p(x)$ and \textit{odd} if $p(-x) = -p(x)$ identically in $x$. Let $\mathcal{E}$ and $\mathcal{O}$ represent the subsets of $\mathbb{F}[x]$ that consist of strictly even and odd polynomials, respectively. Show that $\mathcal{E}$ and $\mathcal{O}$ form vector subspaces of $\mathbb{F}[x]$. 
    
    \vspace{.3cm}
    \item[(c)]
    Show that $\mathbb{F}[x] = \mathcal{E} \oplus \mathcal{O}$. You may assume that $\text{char}(\mathbb{F}) \neq 2$. 
    
    \end{itemize} 
    
    \vspace{.5cm}
    
    \item[$\textbf{[9]}$]
    \begin{itemize}
    
    \item[(a)]
    Show that if both $U$ and $W$ are three-dimensional vector subspaces of a five-dimensional $\mathbb{F}$-vector space $V$, then $U$ and $W$ are not disjoint. 
    
    \vspace{.3cm}
    \item[(b)]
    Show that if $U$ and $W$ are finite-dimensional vector subspaces of a $\mathbb{F}$-vector space $V$, then:
    \begin{equation*}
    \dim(U) + \dim(W) = \dim(U + W) - \dim(U \cap W)
    \end{equation*}
    This is the analogue of the \textit{Inclusion-Exclusion Principle} for sets adapted to vector spaces. In a certain sense the dimension for vector spaces plays the same role cardinality has with respect to sets. 
    
    \end{itemize}
    
    \vspace{.5cm}
    
    \item[$\textbf{[10]}$]
    Let $V$ be a finite-dimensional $\mathbb{F}$-vector space with dual $V^*$. If $y \in V^*$ is non-zero and $\alpha \in \mathbb{F}$ is arbitrary, does there necessarily exist a vector $x \in V$ such that $[x,y] = \alpha$, or equivalently $y(x) = \alpha$?
    
    \end{itemize}


\end{document}