\documentclass[12pt]{article}
%------------------------------- BEGIN PREAMBLE
% packages used
\usepackage{amssymb,amsmath,amsfonts,mathrsfs,pgffor,marvosym,amsthm, mathrsfs, mathtools}
\DeclarePairedDelimiter\set\{\}
% macros
\newcommand      {\Nm}         {{\mathbb N}}
\newcommand      {\Zm}         {{\mathbb Z}}
\newcommand      {\Qm}         {{\mathbb Q}}
\newcommand      {\Rm}         {{\mathbb R}}
\newcommand      {\Cm}         {{\mathbb C}}
\newcommand      {\vb}        {\mathbf}
\newcommand      {\PP}        {{\mathscr P}}
\newcommand      {\BB}        {{\mathscr B}}
\newcommand      {\Fm}         {{\mathbb F}}
\newcommand {\lines}[1] {\foreach \n in {1,...,#1}{ \vspace{9mm} \hrule height 
0.2pt  }\vspace{2mm} }

% adjustment of page dimensions
\textwidth=7in
\textheight=9.8in
\topmargin= -0.8in
\oddsidemargin= -0.3in
\evensidemargin= 0.0in
\setlength{\parskip}{1ex plus0.5ex minus0.2ex}
\setlength{\jot}{12pt}
%-------------------------------- END PREAMBLE
\begin{document}
\begin{flushright}
    Name: Kevin Guillen \\*
    Student ID: 1747199
\end{flushright}
\begin{center}
    {\bf Math 117 - SS2 - Mastery Problems 1 - \today}
\end{center}

% STATEMENT OF PROBLEM 1
\textbf{Disclaimer:} Sorry if I took too much abstract algebra properties as granted. A lot of this stuff popped up in 134 and 111b, so I solved it from what I remember from there and what's in Dummit and Foote's Abstract Algebra

\noindent \textbf{3a) } $\mathbb{F} = \Zm/2\Zm$. Let $f\in \mathbb{F}[x]$, where $f(x) = x^2 + x + 1$. Show $f$ is irreducible.
\begin{proof}
    First we will begin by showing $\Fm$ is a field, more generally that $\Zm/p\Zm$ is a field where $p$ is a prime integer. We know from class that for any integer $n$, $\Zm/n\Zm$ will be a commutative ring. All we need to show now is that, when $n$ is prime, every non-zero element in it will have multiplicative inverses. 

    By definition a prime number, $p$, will share no common divisors except 1 with another integer $n$ ($n\neq p$). Thus take any non-zero element $n \in \Zm/p\Zm$. $n$ represents a congruence class of elements which are by definition not multiples of $p$. Thus, gcd$(n,p) = 1$. 

    From here we know from elementary number theory that there exists $u,v\in\Zm$ such that \[u\cdot n + v\cdot p = 1.\] Bringing this into $\Zm/p\Zm$ we have
    \begin{align*}
        \overline{u}\cdot \overline{n} + \overline{0} \equiv \overline{1} \\
        \overline{u}\cdot \overline{n} \equiv \overline{1}
    \end{align*}

    That means for any non-zero $n\in \Zm/p\Zm$, where $p$ is prime, that there exists a $u\in \Zm/p\Zm$ such that $\overline{u}\cdot \overline{n}= 1 $. Which means that every non-zero element has a multiplicative inverse. Thus satisfying the criteria to be a field. 

    We know that since $\Fm$ is a field, $f\in \Fm[x]$ will have a factor of degree one if and only if $f$ has a root in $\Fm$. In other words $a\in \Fm$, $f(a) = 0$ 
    
    A quick proof of this is as follows. If $f(x)$ has a factor of degree one, and because $\Fm$ is a field, we can assume the factor to be a monic. Meaning for $a\in\Fm$ it will have the form $(x-a)$, but $f(a) = 0$. The converse direction is as follows, assuming $f(a) = 0$. We can use the division algorithm in $\Fm[x]$ to get $f(x) = q(x)(x-a) + r$. But we assumed $f(a) = 0$ that means $r$ must be 0, therefore $f(x)$ will have $(x-a)$ as a factor. It follows from here that any polynomial of degree 2 or 3 in $\Fm[x]$ will be reducible if and only if it has a root in $\Fm$. Since a polynomial of degree 2 or 3 is reducible if and only if it has at least 1 linear factor. 

    Finally, we know the elements of $\Zm/2\Zm$ are $\set{\overline{0},\overline{1}}$. Plugging this into $f(x) = x^2 + x + 1$ we get
    \begin{align*}
        f(0) &= 0 + 0 + 1 \equiv \overline{1} \\
        f(1) &= 1 + 1 + 1 \equiv 3 \equiv \overline{1}
    \end{align*}
    We see neither are $0$, thus $f$ cannot be reducible, meaning it is irreducible. 
\end{proof}
\newpage

\noindent\textbf{3b) } Following the setup of part (a) let $(x^2 + x + 1) =$ Span$\set{x^2 + x + 1}$. Show that dim$_\Fm(\Fm[x]/(x^2 + x + 1)) = 2$ and $|F[x]/(x^2 + x + 1)|= 4$

\begin{proof}
    By definition the span of vectors is just the set of all linear combination of said vectors. Since our only choices are  $\overline{1},\overline{0}\in\Zm/2\Zm$ then the set is simply $f$. First we will show $|F[x]/(x^2 + x + 1)|= 4$. We know that the complete set of representatives of the congruence classes of $\Fm[x]$ modulo $f$ will be of degree $<$ 2, since deg$(f) = 2$. Since these polynomials are restricted to their degree being less than 2 and their coefficients being in $\Zm/2\Zm$ this becomes an easy counting problem. $F[x]/(x^2 + x + 1) = \set{ax + b:\ a,b \in \Zm/2\Zm}$. As stated before there are only 2 elements in $\Fm$, thus there are only $2\cdot 2 = 4$ possible polynomials to choose from in this set of representatives. Hence, $|F[x]/(x^2 + x + 1)|= 4$.

    From class, we know that the dimension of a finite dimensional vector space is the number of elements in a basis of said vector space. We also know from class that any basis of a finite dimensional vector space will be of the same dimension. So all we need to show is 2 vectors in $F[x]/(x^2 + x + 1)$ that can be a basis to show the dimension is 2. 

    $F[x]/(x^2 + x + 1) = \set{0,1,x,x+1}$. Let the basis $U = \set{x,1}$ we can see for $a,b\in \Fm$,
    \begin{align*}
        x+1 &= 1 \cdot(x) + 1\cdot(1) \\
        x &= 1 \cdot(x) + 0 \cdot (1) \\
        1 &= 0 \cdot (x) + 1\cdot(1)\\
        0 &= 0 \cdot(x) + 0 \cdot(1)
    \end{align*}

    We can see $|U| = 2$. Restating as before we know the number of elements in any bass of a finite dimensional vector space is the same as in any other basis, thus dim$_\Fm(\Fm[x]/(x^2 + x + 1)) = 2$
\end{proof}

\noindent\textbf{3c) } Show $E = F[x]/(x^2 + x + 1)$ forms a field with precisely four elements and of characteristic 2. 
\begin{proof}
    We already saw that $E$ contains only 4 elements since the polynomials are restricted to degree less than 2 and coefficients in $\Zm/2\Zm$ meaning $E = \set{ax + b :\ a,b \in \Zm/2\Zm}$ which means there are 4 representatives. 

    To show it is a field though we will prove something more general in that if $\Fm$ is a field and $f(x)\in \Fm[x]$ irreducible then $\Fm[x]/f(x)$ is a field. We already know from abstract algebra that this does indeed form a commutative ring. All that is left is to show it has multiplicative inverses. 
    
    First we let $p(x)\in \Fm[x]$ with $p(x) + (f(x)) \neq 0 + (f(x))$. Meaning $f(x)\nmid p(x)$. Now we need to show there exists $u(x) \in \Fm[x]$ such that $p(x)u(x) \equiv 1 \text{ mod} f(x)$. We know though that $f(x)$ is irreducible and because $p(x)$ is not a multiple of $f(x)$ it means that every common divisor of $f(x)$ and $p(x)$ must be of degree 0. Meaning the constant polynomial 1 is the greatest common divisor of both $f(x)$ and $p(x)$. We know by polynomial division with remainder that there exists polynomials $u(x),v(x)\in\Fm[x]$ such that \[u(x)\cdot p(x) + v(x)\cdot f(x) = 1.\] Implying that $u(x)p(x) \equiv 1 \text{ mod}f(x)$, meaning $p(x) + (f(x))$ is invertible in $\Fm[x]/(f(x))$ as desired. 

    Thus, $F[x]/(x^2 + x + 1)$ does indeed form a field. 

    We know that since this is a field and thereby a ring that the characteristic is simply the minimum number of times we must take the multiplicative identity in a sum to get the additive identity. Since the coefficients are restricted to $\Zm/2\Zm$ this is simply \[\overline{1} + \overline{1} = \overline{2} \equiv \overline{0}\]
    Thus the characteristic is 2. 
\end{proof}



\end{document}